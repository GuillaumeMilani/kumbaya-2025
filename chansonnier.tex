\documentclass[12pt]{article}
\usepackage[chorded]{songs}

\songcolumns{2}
% \columnsep=0.4cm

\noversenumbers
\setlength{\songnumwidth}{0.5cm}
%\renewcommand{\lyricfont}{\sffamily\small}
\renewcommand{\chorusfont}{\it}
% \renewcommand{\snumbgcolor}{red}
\renewcommand{\printchord}[1]{\rmfamily\bf#1}
\versesep=12pt plus 2pt minus 2pt

\setlength{\cbarwidth}{0pt}
\setlength{\sbarheight}{0pt}

%\setlength{\versesep}{0pt}
%\setlength{\chordsep}{0pt}

\usepackage{geometry}
\geometry{
	a5paper,
	% total={170mm,257mm},
	left=5mm,
	top=10mm,
	right=5mm,
	bottom=3mm
}

\baselineadj=-10pt plus 1pt minus 0pt
%\renewcommand{\clineparams}{
%	\baselineskip=15pt
%	\lineskiplimit=5pt
%	\lineskip=1pt
%}

\begin{document}
	Titre
	\newpage
	\newpage
	Test how does is work ?
	\newpage
	\begin{songs}{}
		\beginsong{Il faut que je m’en aille}[by={Graeme Allwright}]
		\beginverse
		Le temps est \[Do]loin de nos 20 ans
		Des coups de \[Fa]poings, des coups de \[Do]sang
		Mais qu’à cela n’\[Fa]tienne, c’est pas \[Do]fini
		On peut chan\[Lam]ter quand le \[Fa]verre est \[(Sol7)]bien \[Do]rempli
		\endverse
		\beginchorus
		Buvons en\[Fa]core une dernière \[Sol]fois
		À l’ami\[Fa]tié, l’a\[Sol]mour, la \[Do]joie
		On a fê\[Fa]té nos retrou\[Do]vailles
		Ça m’fait d’la \[Lam]peine, mais il \[Fa]faut que \[Sol]je m’en \[Do]aille
		\endchorus
		\beginverse
		Et souviens-toi de cet été
		La première fois qu’on s’est saoulé
		Tu m’as ramené à la maison
		En chantant, on marchait à reculons
		\endverse
		\beginchorus
		Refrain
		\endchorus
		\beginverse
		Je suis parti changer d’étoile
		Sur un navire, j’ai mis la voile
		Pour n’être plus qu’un étranger
		Ne sachant plus très bien où il allait
		\endverse
		\beginchorus
		Buvons encore une dernière fois
		À l’amitié, l’amour, la joie
		On a fêté nos retrouvailles
		Je m’ennuie pas, mais il faut que je m’en aille
		\endchorus
		\beginverse
		J’t’ai raconté mon mariage
		À la mairie d’un p’tit village
		Je rigolais dans mon plastron
		Quand le maire essayait d’prononcer mon nom
		\endverse
		\beginchorus
		Refrain
		\endchorus
		\beginverse
		J’n’ai pas écrit toutes ces années
		Et toi aussi, t’es marié
		T’as trois enfants à faire manger
		Mais j’en ai cinq, si ça peut te consoler
		\endverse
		\endsong
		
	\beginsong{Les Copains d’abord}[by={Georges Brassens}]
	\beginverse
		Non, ce n’était pas le radeau de la Méduse, ce bateau
		Qu’on se le dise au fond des ports, dise au fond des ports
		Il naviguait en pèr’peinard sur la grand-mare des canards
		Et s’app’lait les Copains d’abord, les Copains d’abord
		\endverse
		\beginverse
		Ses "fluctuat nec mergitur", c’était pas d’la littérature
		N’en déplaise aux jeteurs de sort, aux jeteurs de sort
		Son capitaine et ses mat’lots, n’étaient pas des enfants 
		d’salauds
		Mais des amis franco de port, des copains d’abord
			\endverse
		\beginverse
		C’étaient pas des amis de luxe, des petits Castor et Pollux
		Des gens de Sodome et Gomorrhe, Sodome et Gomorrhe
		C’étaient pas des amis choisis, par Montaigne et La Boétie
		Sur le ventre ils se tapaient fort, les copains d’abord
			\endverse
		\beginverse
		C’étaient pas des anges non plus, l’Évangile, ils l’avaient pas lu
		Mais ils s’aimaient toutes voiles dehors toutes voiles dehors
		Jean, Pierre, Paul et compagnie, c’était leur seule litanie
		Leur credo, leur confiteor, aux copains d’abord
			\endverse
		\beginverse
		Au moindre coup de Trafalgar, c’est l’amitié qui prenait l’quart
		C’est elle qui leur montrait le nord, leur montrait le nord
		Et quand ils étaient en détresse qu’leurs bras lançaient des 
		S.O.S.
		On aurait dit des sémaphores, les copains d’abord
			\endverse
		\beginverse
		Au rendez-vous des bons copains, y’avait pas souvent de lapins
		Quand l’un d’entre eux manquait à bord, c’est qu’il était mort
		Oui, mais jamais, au grand jamais, son trou dans l’eau n’se 
		refermait
		Cent ans après, coquin de sort, il manquait encore
			\endverse
		\beginverse
		Des bateaux j’en ai pris beaucoup mais le seul qui ait tenu le 
		coup
		Qui n’ai jamais viré de bord, mais viré de bord
		Naviguait en père peinard sur la grand-mare des canards
		Et s’app’lait les Copains d’abord, les Copains d’abord
		\endverse
	\endsong
	\beginsong{À nos souvenirs}[by={Trois Cafés Gourmands}]
	\beginverse
	Regardez-\[Sol]nous, à \[Do]quinze ans sur cette \[Ré]photo\\
	On fait les \[Sol]fous, tous les \[Do]deux sur le même vé\[Ré]lo\\
	On s'est connus, \[Sol]toi et \[Do]moi bien avant nos \[Ré]mots\\
	On s'est per\[Sol]dus de \[Do]vue, trop \[Ré]tôt
	\endverse
	
	\beginchorus
	À nos souve\[Ré]nirs, à nos es\[Sol]poirs, à la \[Do]vie\\
	À nos souve\[Ré]nirs, à ces fous \[Sol]rires, à l'in\[Do]fini\\
	À l'amour, à \[Ré]nos nuits blanches, à nos dé\[Do]lires\\
	À nos souve\[Sol]nirs
	\endchorus
	\endsong
	\end{songs}
\end{document}

