\beginsong{1987}[by={Calogero (2017)}]

\beginverse
Tu t'souv\\[D]iens
Les couleurs sur les bask\\[Bm]ets
Les crayons dans les cass\\[A]ettes
Je rembobine
\endverse

\beginverse
Tu t'souviens
Tous ces rêves plein nos disquettes
À Paris c'était les States
1987
\endverse

	Refrain
Il y a certains jours où je reprends mon skate
Et je vais faire un tour en 1987
Il y a certains jours dans lesquels je me jette
Et je suis de retour en 1987
Tu sais de tous ces jours y'a rien que je regrette
Mais parfois je retourne en 1987, en 87

\beginverse
Tu t'souviens
Les survêts et les houppettes
Sabrina et 7 sur 7
Dans la cuisine c'était rien
Que 12 mois sur la planète
L'URSS, INXS
On chantait I want your sex
\endverse

	Refrain

\beginverse
Tu verras bien qu'un jour une chanson dans la tête
Tu l'auras à ton tour ton 1987
Tu verras bien qu'un jour une chanson dans la tête
Tu l'auras à ton tour ton 1987
C'est tout ce que je te souhaite
Tu l'auras à ton tour ton 1987
C'est tout ce que je te souhaite
Tu l'auras à ton tour ton 1987
C'est tout ce que je te souhaite
Tu l'auras à ton tour ton 1987
Tu te souviens
\endverse
\endsong
\beginsong{99 Luftballons}[by={Nena (1983)}]

\beginverse
H\\[Ré]ast du etwas Ze\\[Mim]it für mich?
Dann s\\[Sol]inge ich ein L\\[La]ied für dich
Von 9\\[Ré]9 L\\[Mim]uftballons
Auf i\\[Sol]hrem Weg zum H\\[La]orizont
D\\[Ré]enkst du vielleicht g\\[Mim]’rade an mich
Dann s\\[Sol]inge ich ein L\\[La]ied für dich
Von 9\\[Ré]9 L\\[Mim]uftballons
Und d\\[Sol]as sowas von s\\[La]owas kommt
\endverse

\beginverse
99 Luftballons
Auf ihrem Weg zum Horizont
Hielt man für UFOs aus dem All
Darum schickte ein General
Eine Fliegerstaffel hinterher
Alarm zu geben, wenn es so wäre
Dabei waren da am Horizont
Nur 99 Luftballons
\endverse

\beginverse
99 Düsenflieger
Jeder war ein großer Krieger
Hielten sich für Captain Kirk
Das gab ein großes Feuerwerk
Die Nachbarn haben nichts gerafft
Und fühlten sich gleich angemacht
Dabei schoss man am Horizont
Auf 99 Luftballons
\endverse

\beginverse
99 Kriegsminister
Streichholz und Benzinkanister
Hielten sich für schlaue Leute
Witterten schon fette Beute
Riefen Krieg und wollten Macht
Mann, wer hätte das gedacht
Dass es einmal soweit kommt
Wegen 99 Luftballons
\endverse

\beginverse
99 Jahre Krieg
Ließen keinen Platz für Sieger
Kriegsminister gibt es nicht mehr
Und auch keine Düsenflieger
Heute ziehe ich meine Runden
Sehe die Welt in Trümmern liegen
Habe einen Luftballon gefunden
Denke an dich und lasse ihn fliegen
\endverse

\beginsong{A nos souvenirs}[by={Trois cafés gourmands (2018)}]

\beginverse
Comment puis-je oublier
Ce coin de paradis?
Ce petit bout de terre
Où vit encore mon père
\endverse

\beginverse
Comment pourrais-je faire
Pour me séparer d'elle?
Oublier qu'on est frères
Belle Corrèze charnelle
Oublier ce matin que tu es parisien
Que t'as de l'eau dans le vin
Que tu es parti loin
\endverse

\beginverse
Ce n'était pas ma faute
On joue des fausses notes
On se trompe de chemin
Et on a du chagrin
On se joue tout un drame
On a des vagues à l'âme
Tu as du mal au cœur
Tu as peur du bonheur
\endverse

\beginverse
Acheter des tableaux
Et des vaches en photo
C'est tout ce que t'as trouvé
Pour te la rappeler
Vous me trouvez un peu con
N'aimez pas ma chanson
\endverse

\beginverse
Vous me croyez bizarre
Un peu patriotard
Le fruit de ma réflexion
Ne touchera personne
Si vos pas ne résonnent
Jamais dans ma région
\endverse

\beginverse
C'est pire qu'une religion
Au-delà d'une confession
Je l'aime à en mourir
Pour le meilleur et pour le pire
Et si je monte au ciel
Il y aura peut-être Joël
\endverse

\beginverse
Guillaume et Jeremy
Et mon cousin Piedri
Yoan sera en voyage
Dans un autre pays
Allez fais tes bagages
Viens rejoindre tes amis
\endverse

\beginverse
On veut du Clody Musette
À en perdre la tête
On veut un dernier Chabrol
Un petit coup de gnôle
Les yeux de nos grands-mères
La voix de nos grands-pères
L'odeur de cette terre
Vue sur les Monédières
\endverse

\beginverse
C'est pire qu'un testament
Au-delà d'une confidence
On est des petits enfants
De ce joli coin de France
Enterrez-nous vivants
Bâillonnés s'il le faut
Mais prenez soin avant
De remplir notre jabot
\endverse

\beginverse
La relève est pour toi
Notre petit Lucas
On te laisse en héritage la piste
Nous, on dégage
Le temps nous a gâtés
On en a bien profité
On a des souvenirs en tête
Ce soir, faisons la fête
\endverse

	Refrain
Acceptez ma rengaine
Elle veut juste dire "je t'aime"
Soyez sûr, j'en suis fier
J'ai la Corrèze dans le cathéter
D'être avec vous ce soir
J'ai le cœur qui pétille
Mimi, sers-nous à boire
On a les yeux qui brillent

\beginverse
Trois, deux, un
\endverse

\beginverse
Papayapapa, papayapapa, papayapapa
Paya, papayapapa \\[4x]
\endverse

	Refrain

\beginverse
On va se quitter comme ça
Lançons une dernière fois ensemble, on veut faire la fête
Pour faire la fête, on va tous se mettre à croupir, s'il vous plaît
Tout le monde, tout le monde
Tout le monde, tout le monde
Tout le monde, tout le monde
Tout le monde, tout le monde, tout le monde
(Ouais, ouais, ouais, ouais, ouais, ouais, ouais, ouais)
\endverse

\beginverse
Trois, deux, un
\endverse

\beginverse
Papayapapa, papayapapa, papayapapa
Paya, papayapapa \\[4x]
\endverse

	refrain

\beginverse
Du fond du cœur, merci
Merci beaucoup
\endverse

\beginsong{L’agriculteur}

\beginverse
Ridan (2003)
\endverse

\beginverse
J'allume mon poste de télé
Pour admirer ce qu'il s'y passe
Un milliardaire s'envoie en l'air
Toute l'atmosphère pour voir l'espace
J'troque son bol d'air et sa cuillère
Contre un p'tit verre sur ma terrasse
J'en ai ras-le-bol de tout ce béton
J'ai la folie des grands espaces
J'en ai ras-le-bol de tout ce béton
J'ai la folie des grands espaces
Mais qu'est-ce qui se passe dans nos p'tites têtes?
On s'entasse tous comme des sardines
Dans les grosses boîtes que l'on conserve
Le p'tit poisson doit suivre sa ligne
(bis 2 der)
Refrain
Et puis merde\! J'ai décidé de vivre loin sur la colline
Vivre seul dans une maison avec la vue sur ma raison
J'préfère vivre pauvre avec mon âme, que vivre riche avec la leur
Et si le blé m'file du bonheur, j'me ferais p'têt' agriculteur \\[bis]
\endverse

\beginverse
Y a trop d'feux rouges dans les grandes villes
J'ai préféré me mettre au vert
J'ai plus d'bonheur à vivre en paix
Que d'm'admirer au fond d'un verre
J'boirai l'eau saine de mon ruisseau
Plutôt qu'l'eau sale du fond de la Seine
Chargée en plomb et en histoire
Que la surface ne laisse plus voir
Chargée en plomb et en histoire
Que la surface ne laisse plus voir
J'ferai des bornes pour m'éloigner
Pour m'retrouver face au miroir
Juste une seconde de vérité
Pour qu'mon passé coule sous les ponts
J'ferai des bornes pour m'éclipser
Pour m'retrouver face à que dalle
Juste une seconde de vérité
Pour contempler ce qu'on est tous
\endverse

	Refrain

\beginverse
Ça fait longtemps que j'n'ai plus vu
Ce coin d'soleil à l'horizon
Ça fait longtemps que j'l'attendais
La petite lueur de la raison
Une petite chanson au clair de lune
Pour réchauffer le cœur de pierre
Le grand retour à l'essentiel
Le feu de bois éclaire le ciel
Le grand retour à l'essentiel
Le feu de bois éclaire le ciel
La mélodie de la nature
Reprend ses droits sur la folie
C'est toute la vie qui nous observe
Que l'on oublie au fil du temps
La mélodie, celle de la vie
Que l'on consume à chaque instant
Tous nos acquis s'écrasent au sol
Et j'ai choisi la clef des champs
Tous nos acquis s'écrasent au sol
Et j'ai choisi la clef des champs
Refrain \\[bis]
\endverse
\endsong
\beginsong{Ah les crocodiles \!}[by={Jacques Offenbach (1856), Julien Doré (2024)}]

\beginverse
U\\[Ré]n crocodile s'en allant à la g\\[La]uerre
D\\[Mi]isait au revoir à\\[Mi7] ses petits e\\[La]nfants
T\\[Ré]raînant ses pieds, ses pieds dans la poussi\\[La]ère
I\\[Mi]l s'en allait combattre les élép\\[La7]hants
\endverse

	Refrain
A\\[Ré]h' les crocrocros, les crocrocros, les crocod\\[La7]iles
Sur les bords du Nil, ils sont partis, n'en parlons p\\[Ré]lus
\\[bis]

\beginverse
Il fredonnait une marche militaire
Dont il mâchait les mots à grosses dents
Quand il ouvrait la gueule tout entière
On croyait voir ses ennemis dedans
\endverse

\beginverse
Il agitait sa grand queue à l'arrière
Comme s'il était d'avance triomphant
Les animaux devant sa mine altière
Dans les forêts s'enfuyaient tout tremblants.
\endverse

\beginverse
Un éléphant parut et sur la terre
Se prépara ce combat de géants
Mais près de là courait une rivière
Le crocodile s'y jeta subitement
\endverse

\beginverse
Et tout rempli d'une crainte salutaire
Il s'en retourna vers ses petits enfants
Notre éléphant, d'une trompe plus fière
Voulut alors accompagner ce chant.
\endverse

\beginsong{Aïcha}[by={Khaled (1996)}][by={Capo III}]

\\[Mim]Comme s\\[Do]i je n'exi\\[Sol]stais p\\[Ré]as
\\[Mim]Elle es\\[Do]t passée à côté\\[Sol] de m\\[Ré]oi
\\[Mim]Sans un r\\[Do]egard, r\\[Sol]eine de S\\[Ré]aba
\\[Mim]J'ai d\\[Do]it «Aïcha, prends, tout e\\[Sol]st pour to\\[Ré]i.»

\beginverse
Voici les perles, les bijoux 
Aussi l'or autour de ton cou
Les fruits, bien mûrs au goût de miel
Ma vie, Aïcha, si tu m'aimes
J'irai où ton souffle nous mène 
Dans les pays d'ivoire et d'ébène 
J'effacerai tes larmes, tes peines
Rien n'est trop beau pour une si belle, oh\!
\endverse

\beginverse
Refrain
\\[Mim]Aïch\\[Do]a, Aïcha, é\\[Sol]coute-mo\\[Ré]i
\\[Mim]Aïch\\[Do]a, Aïcha, t\\[Sol]’en va pa\\[Ré]s
\\[Mim]Aïch\\[Do]a, Aïcha, re\\[Sol]garde-mo\\[Ré]i
\\[Mim]Aïch\\[Do]a, Aïcha, ré\\[Sol]ponds-mo\\[Ré]i\!
\endverse

\beginverse
Je dirai les mots, les poèmes 
Je jouerai les musiques du ciel 
Je prendrai les rayons du soleil 
Pour éclairer tes yeux de reine, oh\!
\endverse

\beginverse
Refrain
\endverse

\\[Misus4]E\\[Lam]lle a dit, « Garde tes tr\\[Fa]ésors
M\\[Lam]oi, je vaux mieux que tout ça\\[Fa]
\\[Rém]Des barreaux sont des barreaux, même e\\[Sol]n or
Je veux les m\\[Misus4]êmes dro\\[Mi]its que to\\[Lam]i
\\[Fa]Et du respect pour chaque jo\\[Rém]ur
M\\[Rém]oi, je ne veux que de l'a\\[Misus4]mour. »\\[Mi]

\\[Mim]Comme s\\[Do]i je n'exi\\[Sol]stais p\\[Ré]as
Elle est passée à côté de moi
Sans un regard, reine de Saba 
J'ai dit « Aïcha, prends, tout est pour toi. »

\beginverse
Refrain
\endverse

\beginsong{L’aigle noir}[by={Barbara (1970)}]

\beginverse
U\\[Ré]n beau jour ou peut-être u\\[La]ne nuit
P\\[Mim]rès d'un lac je m'étais endormie
Quand soudain semblant crever le ciel 
Et venant de nulle part
Surgit un aigle noir
\endverse

\beginverse
Lentement les ailes déployées
Lentement je le vis tournoyer
Près de moi dans un bruissement d'ailes
Comme tombé du ciel
L'oiseau vint se poser
\endverse

\beginverse
Il avait les yeux couleur rubis
Et les plumes couleur de la nuit
A son front, brillant de mille feux
L'oiseau roi couronné
Portait un diamant bleu
\endverse

\beginverse
De son bec il a touché ma joue
Dans ma main il a glissé son cou
C'est alors que je l'ai reconnu
Surgissant du passé
Il m'était revenu
\endverse

\beginverse
Dis, l'oiseau, oh dis, emmène-moi
Retournons au pays d'autrefois
Comme avant dans mes rêves d'enfant
Pour cueillir en tremblant 
Des étoiles, des étoiles 
\endverse

\beginverse
Comme avant dans mes rêves d'enfant
Comme avant sur un nuage blanc 
Comme avant, allumer le soleil
Etre faiseur de pluie 
Et faire des merveilles
\endverse

\beginverse
L'aigle noir dans un bruissement d'ailes
Prit son vol pour regagner le ciel
\endverse

\beginverse
Un beau jour ou peut-être une nuit
Près d'un lac je m'étais endormie
Quand soudain semblant crever le ciel 
Et venant de nulle part
Surgit un aigle noir
\endverse

\beginsong{Allô le monde}[by={Pauline (2007)}]

\beginverse
Il paraît que les nouvelles ne sont pas si bonnes
Que le moral descend
Et que les forces t'abandonnent
J'entends
Tous les gens
Parler de tes histoires
Que l'avenir qui t'attend
Se joue sur le fil du rasoir
Qu'en est-il de l'amour?
Des larmes et de la peine?
De la vie de tous les jours?
Et de la paix sereine?
\endverse

\beginverse
Refrain
Allô le monde
Est-ce que tout va bien?
Allô le monde
Je n'y comprends plus rien
Allô le monde
Prends soin de toi
Allô le monde
Ne te laisse pas aller
Comme ça
Comme ça
\endverse

\beginverse
Quel est le nom du mal dont tu subis la fièvre
Les étranges idéaux, les hystéries funèbres?
Dis-moi ce que je peux faire de ma petite place
Quels sont les actes et les mots qui peuvent t'aider à faire face?
Pousser à la révolte
Pour faire le premier pas
Semer pour qu'on récolte
Pour crier mon effroi
\endverse

\beginverse
Allô le monde
Allô le monde
Allô le monde
Allô le monde
Allô le monde
\endverse

\beginverse
Allô le monde
Est-ce que tout va bien?
Allô le monde
Allô le monde
Prends soin de toi
Allô le monde
Le monde, le monde, le monde, le monde
Le monde, le monde, le monde, le monde
Allô le monde
Allô le monde, le monde
\endverse

\beginsong{L’alphabet scout}[by={Traditionnel}]

\beginverse
Un jo\\[Do]ur la troupe campa, A A\\[Sol] A 
La pluie se mit à tomber, B\\[Sol7] B B\\[Do]
L'orage a tout cassé, C C C\\[Sol] 
Faillit nous inond\\[Do]er, A B\\[Sol] C D\\[Do] 
\endverse

\beginverse
Le chef s'est écrié, E E E 
A son adjoint Joseph, F F F 
Fais-nous vite à manger, G G G 
Les scouts sont sous la bâche E F G H
\endverse

\beginverse
Les pinsons dans leur nid, I l I 
Les cerfs dans leur logis, J J J 
Chahutent, quel fracas, K K K 
Avec les hirondelles, I J K L
\endverse

\beginverse
Joseph nous fit d'la crème, M M M 
Et du lapin d'Garenne, N N N 
Et même du cacao, O O O
Mes amis quel souper, M N O P
\endverse

\beginverse
Soyez bien convaincus, QQ Q
Que la vie au grand air, R R R
Fortifie la jeunesse, S S S
Et lui rend la santé, Q R S T 
\endverse

\beginverse
Maint'nant qu'il ne pleut plus, U U U 
Les scouts peuvent se sauver, V V V 
Le temps est au beau fixe, X X X
Plus besoin qu'on les aide, U V XZ
\endverse

\beginsong{L’amour brille sous les étoiles}[by={Elton John, Tim Rice \- Le Roi Lion (1994)}]

\beginverse
C'est terrible c'est affreux
Quoi?
Et ils se moquent de tout
Qui?
L'amour s'amène et nous pauvres pouilleux
Ils nous jettent tous les deux
Où?
Sous les diamants des étoiles
Quel magique univers
Mais
Dans cette romantique atmosphère
Ça sent mauvais dans l'air
\endverse

\beginverse
L'amour brille sous les étoiles
D'une étrange lumière
La terre entière
En parfaite harmonie
Vit un moment royal
\endverse

\beginverse
Je voudrais lui dire je t'aime
Mais comment lui avouer
Mon secret mes problèmes
Impossible
Elle serait trop blessée
\endverse

\beginverse
Quel lourd secret cache-t-il
Derrière tant de ranceur
Moi je sais qu'il est ce roi en exil
Qui règne dans mon coeur
\endverse

\beginverse
L'amour brille sous les étoiles
D'une étrange lumière
La terre entière
En parfaite harmonie
Vit sa plus belle histoire
\endverse

\beginverse
L'amour brille sous les étoiles
Illuminant leurs coeurs
Sa lumière éclaire à l'infini
Un sublime espoir
\endverse

\beginverse
S'ils s'enfuient vers
Leur rêve ce soir
Dans leur folle ronde
Si notre ami nous dit au revoir
Nous serons seuls au monde
\endverse

\beginsong{Amour censure\*}[by={Hoshi (2020)}]

\beginverse
Au placard mes sentiments
Surtout ne rien dire, et faire semblant
Être à part, un peu penchant
Au bout du navire, je coule doucement
\endverse

\beginverse
Maman désolée, j'vais pas te mentir
C'est dur d'effacer tout ce qui m'attire
Un peu dépassée par tous mes désirs
Papa c'est vrai, j'ai poussé de travers
J'suis une fleur qui se bat entre deux pierres
J'ai un cœur niqué par les bonnes manières
\endverse

\beginverse
Refrain
Est-ce qu'on va un jour en finir
Avec la haine et les injures
Est-ce que quelqu'un viendra leur dire
Qu'on s'aime et que c'est pas impur
Pour pas que j'pense à en finir
Vos coups m'ont donné de l'allure
Pour le meilleur et pour le pire
J'prendrai d'sa main un jour c'est sûr
\endverse

\beginverse
Il n'y a pas d'amour censure
Il n'y a que d'l'amour sincère
Il n'y a pas d'amour censure
Il n'y a que d'l'amour sincère
\endverse

\beginverse
Travestir qui je suis vraiment
Faire taire la rumeur
Les mots sont tranchants
Se mentir à s'arracher les dents
Ils cherchent un docteur
On souffre sans être souffrants
\endverse

\beginverse
Maman désolée, j'ai pris tes calmants
C'est pas que j'voulais partir, mais c'est violent
J'voulais juste dormir un peu plus longtemps
Papa t'inquiète j'ai appris à courir
Moi aussi j'veux une famille à nourrir
On s'en fout près de qui j'vais m'endormir
\endverse

	Refrain\\[2x]

\beginverse
Il n'y a pas d'amour censure
Il n'y a que d'l'amour sincère
(Les enfants, c'est pour un homme et une femme)
Il n'y a pas d'amour censure
(Ce n'est absolument pas pour des homosexuels)
Il n'y a que d'l'amour sincère
\endverse

\beginsong{Amsterdam\*}[by={Jacques Brel (1964)}][by={Capo IX}]

\beginverse
Dans le po\\[Lam]rt d'Amsterdam, y a des ma\\[Mim]rins qui chantent 
Les rêv\\[Fa]es qui les hantent, au larg\\[Mim7]e d'Amsterdam
Dans le po\\[Lam]rt d'Amsterdam, y a des mar\\[Mim]ins qui dorment 
Comme de\\[Fa]s orif\\[Mi7]lammes le long de\\[Lam]s berges mornes 
Dans le po\\[Do]rt d'Amsterdam, y a des mar\\[Sol7]ins qui meurent
Pleins de biè\\[Lam]re et de drames aux prem\\[Mi7]ières lueurs
Mais dans le po\\[Fa]rt d'Amsterdam, y a des mar\\[Mim]ins qui naissent 
Dans la ch\\[Fa]aleur épa\\[Mi7]isse des langu\\[Lam]eurs océanes
\endverse

\beginverse
Dans le port d'Amsterdam, y a des marins qui mangent 
Sur des nappes trop blanches des poissons ruisselants
Ils vous montrent des dents à croquer la fortune
A décroisser la Lune, à bouffer des haubans 
Et ça sent la morue jusque dans le cœur des frites 
Que leurs grosses mains invitent à revenir en plus
Puis se lèvent en riant, dans un bruit de tempête
Referment leur braguette et sortent en rotant
\endverse

\beginverse
Dans le port d'Amsterdam, y a des marins qui dansent 
En se frottant la panse sur la panse des femmes 
Et ils tournent et ils dansent comme des soleils crachés 
Dans le son déchiré d'un accordéon rance
Ils se tordent le cou pour mieux s'entendre rire
Jusqu'à ce que tout à coup l'accordéon expire
Alors, le geste grave, alors le regard fier 
Ils ramènent leur Batave jusqu'en pleine lumière
\endverse

\beginverse
Dans le port d'Amsterdam, y a des marins qui boivent 
Et qui boivent et reboivent, et qui reboivent encore
Ils boivent à la santé des putains d'Amsterdam 
De Hambourg ou d'ailleurs, enfin ils boivent aux dames
Qui leur donnent leur joli corps, qui leur donnent leur vertu
Pour une pièce en or, et quand ils ont bien bu
Se plantent le nez au ciel, se mouchent dans les étoiles 
Et ils pissent comme je pleure sur les femmes infidèles 
Dans le po\\[lam]rt d'Amsterdam, dans le po\\[Mi7]rt d'Amsterdam(Fa-Mi7-Lam).
\endverse

\beginsong{Andalouse}[by={Kendji Girac (2014)}]

\beginverse
Tu viens le soir
Danser sur des airs de guitare
Et puis tu bouges
Tes cheveux noirs, tes lèvres rouges
Tu te balances
Le reste n'a pas d'importance
Comme un soleil
Tu me brûles et me réveilles
Tu as dans les yeux, le sud et le feu
Je t'ai dans la peau
Baila, baila, oh
\endverse

\beginverse
Refrain
Toi, toi, ma belle Andalouse
Aussi belle que jalouse
Quand tu danses, le temps s'arrête
Je perds le nord, je perds la tête
Toi, ma belle Espagnole
Quand tu bouges tes épaules
Je n'vois plus le monde autour
C'est peut-être ça, l'amour
\endverse

\beginverse
Des airs d'orient \\[baila]
Le sourire et le cœur brûlant
Regard ébène
J'aime te voir bouger comme une reine
Ton corps ondule \\[baila]
Déjà mes pensées se bousculent
Comme la lumière \\[baila]
Il n'y a que toi qui m'éclaire
Tu as dans la voix
Le chaud et le froid
Je t'ai dans la peau
Baila, baila, oh
\endverse

	Refrain

\beginverse
Oh-yé-yé-yé, oh-oh, oh-oh
Oh-yé-yé-yé, oh-oh
Oh-yé-yé-yé, oh-oh, oh-oh (ma belle Andalouse)
Oh-yé-yé-yé, oh-oh
(Dos, tres, baila)
\endverse

	Refrain\\[bis]

\beginsong{L’arc-en-ciel}[by={Pellisier-Moulin}]

	Refrain
Viens mélanger tes couleurs avec moi
Réveiller le bonheur qui dort au fond de toi
Faire jaillir la lumière de nos vies
Improviser la fête au plein cœur de la nuit

\beginverse
Tu connais la misère qui condamne au silence :
Prend la main de tes frères, invente un pas danse \!
\endverse

\beginverse
Tu refoules tes larmes dans ta gorge nouée :
Oublie le bruit des armes, réapprend à chanter \!
\endverse

\beginverse
Tu t'opposeras à la force qui tue la liberté :
Entrouvre ton écorce au soleil de l'été \!
\endverse

\beginverse
Tu rêves d'innocence au lieu d'hypocrisie :
Retrouve ton enfance, recommence ta vie \!
\endverse

\beginverse
Tu pardonnes sa haine au frère qui t'a blessé :
Laisse tomber ta gêne, donne-lui un baiser \!
\endverse

\beginsong{Aux arbres citoyens}[by={Yannick Noah (2006)}]

\beginverse
Le ciment dans les plaines coule jusqu'aux montagnes
Poison dans les fontaines, dans nos campagnes
De cyclones en rafales, notre histoire prend l'eau
Reste notre idéal, "faire les beaux"
\endverse

\beginverse
S'acheter de l'air en barre, remplir la balance
Quelques pétrodollars, contre l'existence
De l'Équateur aux pôles, ce poids sur nos épaules
De squatteurs éphémères, maintenant, c'est plus drôle
\endverse

	Refrain
Puisqu'il faut changer les choses
Aux arbres citoyens
Il est grand temps qu'on propose
Un monde pour demain

\beginverse
Aux arbres citoyens, quelques baffes à prendre
La veille est pour demain, des baffes à rendre
Faire tenir debout une armée de roseaux
Plus personne à genoux, fais passer le mot
\endverse

\beginverse
C'est vrai la Terre est ronde, mais qui viendra nous dire
Qu'elle l'est pour tout le monde et les autres à venir
\endverse

\beginverse
Refrain \\[bis]
\endverse

\beginverse
Plus le temps de savoir à qui la faute
De compter sur la chance ou les autres
Maintenant, on se bat
Avec toi, moi, j'y crois
\endverse

	Refrain

\beginsong{L’aventurier}[by={Indochine (1982)}]

\beginverse
É\\[Mim]garé dans la v\\[Do]allée infernale
Le h\\[Sol]éros s'appelle B\\[Lam]ob Morane 
À\\[Mim] la recherche de l'O\\[Do]mbre Jaune 
Le b\\[Sol]andit s'appelle Mister K\\[Lam]ali Jones 
A\\[Mim]vec l'ami Bill B\\[Do]allantine 
S\\[Sol]auvé de justesse des c\\[Lam]rocodiles
S\\[Mim]top au trafic des C\\[Do]araïbes 
E\\[Sol]scale dans l'opération N\\[Lam]adawieb 
\endverse

\beginverse
L\\[Mim]a lalalaaa la la\\[Do]lalaa, L\\[Sol]a lalala la l\\[Lam]ala-lalaa \\[bis]
\endverse

\beginverse
Le cœur tendre dans le lit de Miss Clark 
Prisonnière du Sultan de Jarawak 
En pleine terreur à Manicouagan 
Isolé dans la jungle birmane
Emprisonnant les flibustiers
L'ennemi est démasqué
On a volé le collier de Civa
Le Maharadjah en répondra
\endverse

\beginverse
La lalalaaa la lalalaa, La lalala la lala-lalaa \\[bis]
\endverse

	Refrain
E\\[Lam]t soudain surgit f\\[Mim]ace au vent
Le v\\[Sol]rai héros de t\\[Ré]ous les temps
B\\[Lam]ob Morane contre t\\[Mim]out chacal
L'av\\[Sol]enturier contre t\\[Ré]out guerrier
B\\[Lam]ob Morane contre t\\[Mim]out chacal
L'av\\[Sol]enturier contre t\\[Ré]out guerrier

\beginverse
Dérivant à bord du Sampang
L'aventure au parfum d'Ylalang
Son surnom, Samouraï du Soleil 
En démantelant le gang de l'Archipel 
L'otage des guerriers du Doc Xhatan 
Il s'en sortira toujours à temps
Tel l'aventurier solitaire
Bob Morane est le roi de la terre
\endverse

\beginverse
La lalalaaa la lalalaa, La lalala la lala-lalaa \\[bis]
\endverse

	Refrain

\beginsong{Balade jurassienne\*}[by={Christophe Meyer (2003)}]

\beginverse
Il pleut toujours à Pleujouse          
Quand le temps sʼéclaire à Réclère 
Jʼai couru à Courroux 
Mʼémerveiller à Mervelier
Si jʼai failli manquer Fahy   
Je ne vais pas rater Les Genevez 
Jʼfais aussi un saut à Saulcy   
\endverse

\beginverse
Jʼai visité mes amies dʼAlle
Vic à Vicques puis celle de Lucelle     
Assez près dʼelle à Séprais
Elle mʼa dit vers ma ferme à Vermes 
Quʼelle était soûle à Soulce
Elle voulait que je la plaigne à Pleigne 
Les cheveux dans lʼvent à Damvant 
\endverse

\beginverse
Chatouilleux à Châtillon
Jʼai lu le coran à Corban
Perdu mon savon à Courchavon 
Cʼétait dans une rose maison – jʼai vu 
Le bon et la folle de Bonfol      
Les soldats sʼaligner à Lugnez
Se faire couper les courtes mèches 
\endverse

\beginverse
Quand vous montiez à Montignez     
Jʼai embrassé la joue du rʼgard
À Charmoille sur un char de paille Mis ma ceinture à St-Ursanne 
Je recours à Rocourt
Car cʼest aux Bois quʼon boit     
Le taux dʼalcool à Montenol 
\endverse

\beginverse
Jʼavais lʼair affreux à Damphreux     
Au court de la beuverie dʼOcourt     
Même pas eu lʼtemps dʼcuver à Coeuve   
Pour sʼbourrer lʼgnon à Bourrignon    
Quand les copains burent à Bure
Beurré au Raisin dʼBeurnevésin
Cʼétait soit hier à Soyhières
\endverse

\beginverse
Un peu maraud à Muriaux
Jʼai volé des pommes aux Pommerats 
Mangé mon melon à Montmelon 
Quʼavait un goût moite à Goumois
Jʼai soupé à Soubey
Sans faire de bruit à Buix
Jʼmʼendors sur la roche de Roche DʼOr
\endverse

\beginsong{La ballade des gens heureux}[by={Gérard Lenorman (1975)}][by={Capo V}]

\beginverse
Notre v\\[Do]ieille terre est une étoile
Où toi aussi tu b\\[Rém]rilles un p\\[Sol]eu
Je viens te c\\[Rém]hanter l\\[Sol7]a ball\\[Do]ade
La balla\\[Rém]de des g\\[Sol7]ens heur\\[Do]eux
(bis 2 der)
\endverse

\beginverse
Tu n'as pas de titre ni de grade
Mais tu dis tu quand tu parles à Dieu
Je viens te chanter la ballade
La ballade des gens heureux
(bis 2 der)
\endverse

\beginverse
Journaliste pour ta première page
Tu peux écrire tout ce que tu veux
Je t'offre un titre formidable
La ballade des gens heureux
(bis 2 der)
\endverse

\beginverse
Toi qui a planté cet arbre
Dans ton petit jardin de banlieue
Je viens te chanter la ballade
La ballade des gens heureux.
(bis 2 der)
\endverse

\beginverse
Il s'endort et tu le regardes
Comme un enfant, il te ressemble un peu
On vient lui chanter la ballade
La ballade des gens heureux.
(bis 2 der)
\endverse

\beginverse
Toi la star du haut de ta vague
Descends vers nous tu nous verras mieux
On vient te chanter la ballade
La ballade des gens heureux.
(bis 2 der)
\endverse

\beginverse
Roi de la drague et de la rigolade
Rouleur, flambeur ou gentil petit vieux
On vient te chanter la ballade
La ballade des gens heureux.
(bis 2 der)
\endverse

\beginverse
Comme un choeur dans une cathédrale
Comme un oiseau qui fait ce qu'il veut
On vient te chanter la ballade
La ballade des gens heureux.
(bis 2 der)
\endverse

\beginsong{La ballade nord-irlandaise\*}[by={Renaud (1991)}]

\beginverse
J'ai voulu plante\\[Sol]r un o\\[Do]range\\[Sol]r
Là où la chans\\[Mim]on n'en v\\[Lam]erra jam\\[Ré]ais
Là où les \\[Sol]arbres n'ont jamais donn\\[Mim]é
Que d\\[Do]es gren\\[Sol]ades d\\[Do]égoupill\\[Sol]ées
\endverse

\beginverse
Jusqu'à Derry ma bien aimée
Sur mon bateau j'ai navigué
J'ai dit aux hommes qui se battaient
Je viens planter un oranger
\endverse

\beginverse
Buvons un verre, allons pêcher
Pas une guerre ne pourra durer
Lorsque la bière et l'amitié
Et la musique nous feront chanter
\endverse

\beginverse
Tuez vos dieux à tout jamais
Sous aucune croix l'amour ne se plaît
Ce sont les hommes, pas les curés 
Qui font pousser les orangers
\endverse

\beginverse
Je voulais planter un oranger
Là où la chanson n'en verra jamais
Il a fleuri et il a donné
Les fruits sucrés de la liberté
\endverse

\beginsong{La batelière}[by={Traditionnel}]

\beginverse
Gentille batelière, laisse-là ton bateau
Préfère à ta chaumière les honneurs du château
J'irai cueillir la fleur nouvelle
Chaque matin pour toi
Tu choisiras rubis, dentelles
Blanche viens avec moi \!
\endverse

\beginverse
Refrain	
Non, non, J'aime mieux mon p'tit bateau
Ma rame flexible sur l'onde paisible 
Et ma chaumière au bord de l'eau
Tra la la la la la la
\endverse

\beginverse
Belle enfant qu'au rivage on entend chaque soir
Malgré les vents, l'orage, dire des chants d'espoir \!
Tu reverras dans la vallée
Tes chalets et tes bois
Tu ne seras plus isolée
Blanche viens avec moi \!
\endverse

\beginverse
Rien ne trouble ton âme, rien ne trouble ton cœur, 
Tu doûtes de ma flamme, tu ris de ma douleur
Que te faut-il enfant cruelle Pour vaincre ton dédain,
Te faire oublier ta nacelle ?
Veux-tu mon cœur, ma main ?
\endverse

\beginverse
Ah \! Ah \! Cette foi, mon seigneur
Tra la la la
Je veux bien vous donner mon cœur
Tra la la la
\endverse

\beginsong{Le beau tambour}[by={Henri Dès (1986)}]

\beginverse
J'ai reçu, plan, plan
J'ai reçu, plan, plan
J'ai reçu un beau tambour
Et je joue, plan, plan
Et je joue, plan, plan
Et je joue quand il fait jour
Et quand il fait nuit, et le mercredi
Et quand papa dort encore
Et pour les voisins, le dimanche matin
Je vais dans le corridor
\\[6x]
\endverse

\beginsong{Bella ciao\*}[by={Auteur anonyme (1944)}][by={Capo V}]

\beginverse
Una matt\\[Lam]ina mi son svegl\\[Lam]iato
O bella c\\[Lam]iao, bella ciao, bella c\\[Mi7]iao ciao ciao
Una matt\\[Rém]ina mi son svegl\\[Lam]iato
Eo ho trov\\[Mi7]ato l'invas\\[Lam]or
\endverse

\beginverse
O partigiano portami via
O bella ciao, bella ciao, bella ciao ciao ciao
O partigiano portami via
Che mi sento di morir
\endverse

\beginverse
E se io muoio da partigiano
O bella ciao, bella ciao, bella ciao ciao ciao
E se io muoio da partigiano
Tu mi devi seppellir
\endverse

\beginverse
Mi seppellire lassù in montagna
O bella ciao, bella ciao, bella ciao ciao ciao
Mi seppellire lassù in montagna
Sotto l'ombra di un bel fiore
\endverse

\beginverse
E la gente che passeranno
O bella ciao, bella ciao, bella ciao ciao ciao
E la genta che passeranno
Mi diranno: «Che bel fior»
\endverse

\beginverse
È questo il fiore del partigiano
O bella ciao, bella ciao, bella ciao ciao ciao
È questo il fiore del partigiano
Morto per la libertà
\endverse

\beginsong{Berceuse tchèque}[by={Traditionnel}]

\beginverse
E\\[Do]coute la prière
Qui du camp monte vers To\\[Sol7]i
Vers la grande lumièr\\[Do]e
V\\[Rém]ers la pai\\[Sol7]x et vers la jo\\[Do]ie
(bis 2 der)
\endverse

\beginverse
Illumine la route
Où le monde nous attend
Que suivant la loi scoute
Nous aidions les pauvres gens
(bis 2 der)
\endverse

\beginverse
Donne à notre patrie
Divisée en ses frontières
La paix qui fut promise
A ceux qui s'aiment en frères
(bis 2 der)
\endverse

\beginverse
mmmmm...
mmmmm…
\endverse

# 

\beginsong{Les bêtises}[by={Henri Dès (1991)}]

\beginverse
Refrain
Zoum zoum zoum-zoum-zoum
Zoum zouzoum zoum-zoum
C’est à l’école, tagadagada
Qu’on apprend les bêtises
C’est à l’école, tagadagada
Qu’on apprend les bêtises
\endverse

\beginverse
Le grand Dédé — poil poil au nez
Devant toute la classe
Monte au tableau — poil poil au dos
Pour faire des grimaces
\endverse

\beginverse
Refrain
\endverse

\beginverse
Quand le Julien — poil poil aux mains
Raconte ses histoires
Elles sont si bêtes — poil aux chaussettes
Qu’on pleure dans nos mouchoirs
\endverse

\beginverse
Refrain
\endverse

\beginverse
Et la maîtresse — poil poil aux tresses
Qui pousse des soupirs
Quand Marion — poil au menton
Attrape le fou rire
\endverse

\beginverse
Refrain
\endverse

\beginverse
Et puis y’a moi — poil poil au doigt
Qui marche à quatre pattes
Pour chatouiller — poil au mollet
Ma voisine de droite
\endverse

\beginverse
Refrain
\endverse

\beginverse
Y'a la Thérèse — poil à la chaise
C’est la plus rigolote
Quand elle s’asseye — poil aux orteils
On lui voit sa culotte
\endverse

\beginverse
Refrain
\endverse

\beginverse
Heureusement — poil poil aux dents
Quand vient la sonnerie
Tout le monde s’arrête — poil aux baskets
Par ici la sortie
\endverse

\beginverse
Refrain
\endverse

\beginverse
Zoum zoum zoum-zoum-zoum
Zoum zouzoum zoum-zoum
\endverse

\beginsong{Le bleu lumière}[by={Cerise Calixte \- Vaïana (2016)}]

\beginverse
Le bleu du ciel n'est pas le bleu de la mer
Ce bleu que moi je préfère
Sans vraiment savoir pourquoi
J'aimerai tant rester fidèle à ma terre
Oublier le vent éphémère
J'ai essayé tant de fois
J'ai beau dire "je reste, je ne partirai pas"
Chacun de mes gestes, chacun de mes pas
Me ramènent sans cesse
Malgré les promesses
Vers ce bleu lumière
L'horizon où la mer touche le ciel
Et m'appelle
Cache un trésor
Que tous ignorent
C'est le vent, doucement qui se lève
Et me révèle
Le bleu de l'eau
Si je pars j'irai plus loin et toujours plus haut
Il faut aimer mon île et son histoire
Pour ceux qui veulent encore y croire
Oublier le temps qui passe
Il faut aimer mon île et son histoire
Et garder encore l'espoir
Un jour je trouverai ma place
Je peux les guider
Les rendre plus grands
Les accompagner
Je prendrai le temps
Mais cette voie cachée
Pense tout autrement
Je ne comprends pas
Le soleil vient danser sur la mer éternelle
Mais tous ignorent
Ces reflets d'or
Elle m'attend sous un tapis de lumière
La mer m'appelle
Moi je veux voir
Derrière les nuages de nouveaux rivages
L'horizon où la mer touche le ciel
Et m'appelle
Cache un trésor
Que tous ignorent
C'est le vent doucement qui se lève
Et me révèle
J'ai le droit
D'aller là-bas
\endverse

\beginsong{Blowin’ in the wind}[by={Bob Dylan (1962)}]

\beginverse
H\\[Ré]ow many r\\[Sol]oads must a m\\[La]an walk d\\[Ré]own
Before you c\\[Sol]all him a m\\[Ré]an?
H\\[Ré]ow many s\\[Sol]eas must a w\\[La]hite dove s\\[Ré]ail
Before she s\\[Sol]leeps in the s\\[La]and?
Yes, and h\\[Ré]ow many t\\[Sol]imes must the c\\[La]annonballs f\\[Ré]ly
Before they're f\\[Sol]orever b\\[Ré]anned?
\endverse

\beginverse
Refrain
T\\[Sol]he answer, m\\[La]y friend, i\\[Ré]s blowin' in the w\\[Sol]ind
The answer is b\\[La]lowin' in the w\\[Ré]ind
\endverse

\beginverse
Yes, and how many years must a mountain exist
Before it is washed to the sea?
Yes, and how many years can some people exist
Before they're allowed to be free?
Yes, and how many times can a man turn his head
And pretend that he just doesn't see?
\endverse

\beginverse
Refrain
\endverse

\beginverse
Yes, and how many times must a man look up
Before he can see the sky?
Yes, and how many ears must one man have
Before he can hear people cry?
Yes, and how many deaths will it take 'til he knows
That too many people have died?
\endverse

\beginverse
Refrain
\endverse

\beginsong{Le blues du businessman}[by={Claude Dubois, Starmania (1978)}]

\beginverse
J'ai du succès dans mes affaires
J'ai du succès dans mes amours
Je change souvent de secrétaire
J'ai mon bureau en haut d'une tour
D'où je vois la ville à l'envers
D'où je contrôle mon univers
\endverse

\beginverse
J'passe la moitié de ma vie en l'air
Entre New York et Singapour
Je voyage toujours en première
J'ai ma résidence secondaire
Dans tous les Hilton de la Terre
J'peux pas supporter la misère
\endverse

\beginverse
J'suis pas heureux mais j'en ai l'air
J'ai perdu le sens de l'humour
Depuis qu'j'ai le sens des affaires
J'ai réussi et j'en suis fier
Au fond je n'ai qu'un seul regret
J'fais pas ce que j'aurais voulu faire
\endverse

\beginverse
J'aurais voulu être un artiste
Pour pouvoir faire mon numéro
Quand l'avion se pose sur la piste
À Rotterdam ou à Rio
\endverse

\beginverse
J'aurais voulu être un chanteur
Pour pouvoir crier qui je suis
J'aurais voulu être un auteur
Pour pouvoir inventer ma vie
Pour pouvoir inventer ma vie
\endverse

\beginverse
J'aurais voulu être un acteur
Pour tous les jours changer de peau
Et pour pouvoir me trouver beau
Sur un grand écran en couleur
Sur un grand écran en couleur
\endverse

\beginverse
J'aurais voulu être un artiste
Pour avoir le monde à refaire
Pour pouvoir être un anarchiste
Et vivre comme un millionnaire
Et vivre comme un millionnaire
\endverse

\beginverse
J'aurais voulu être un artiste
Pour faire du laid, pour faire du beau
Pour pouvoir dire pourquoi j'existe
Oui, oui, oui
Merci beaucoup
\endverse

\beginsong{La bohème}[by={Charles Aznavour (1966)}][by={Capo III}]

\beginverse
J\\[Rém]e vous parle d'un temps
Que les moins de vingt ans
Ne peuvent pas conn\\[Lam]aître
Montmartre en ce temps-l\\[Rém]à
Accrochait ses lil\\[Lam]as
Jusque sous nos fenêtres
Et si l'humble g\\[Rém]arni
Qui nous servait de nid
Ne payait pas de min\\[Lam]e
C'est là qu'on s'est c\\[Rém]onnu
Moi qui criait f\\[Mi7]amine
Et toi qui posais nu\\[Lam]e
\endverse

	Refrain
La boh\\[Rém]ème, la boh\\[Lam]ème
Ça voulait d\\[Rém]ire
On e\\[Mi7]st heur\\[Lam]eux
La boh\\[Rém]ème, la boh\\[Lam]ème
Nous ne mang\\[Rém]ions qu'un j\\[Mi7]our sur d\\[Lam]eux

\beginverse
Dans les cafés voisins
Nous étions quelques-uns
Qui attendions la gloire
Et bien que miséreux
Avec le ventre creux
Nous ne cessions d'y croire
Et quand quelque bistro
Contre un bon repas chaud
Nous prenait une toile
Nous récitions des vers
Groupés autour du poêle
En oubliant l'hiver
\endverse

	Refrain
La bohème, la bohème
Ça voulait dire
Tu es jolie
La bohème, la bohème
Et nous avions tous du génie

\beginverse
Souvent il m'arrivait
Devant mon chevalet
De passer des nuits blanches
Retouchant le dessin
De la ligne d'un sein
du galbe d'une hanche
Et ce n'est qu'au matin
Qu'on s'asseyait enfin
Devant un café-crème
Épuisés mais ravis
Fallait-il que l'on s'aime
Et qu'on aime la vie
\endverse

	Refrain
La bohème, la bohème
Ça voulait dire
On a vingt ans
La bohème, la bohème
Et nous vivions de l'air du temps

\beginverse
Quand au hasard des jours
Je m'en vais faire un tour
À mon ancienne adresse
Je ne reconnais plus
Ni les murs, ni les rues
Qui ont vu ma jeunesse
En haut d'un escalier
Je cherche l'atelier
Dont plus rien ne subsiste
Dans son nouveau décor
Montmartre semble triste
Et les lilas sont morts
\endverse

	Refrain
La bohème, la bohème
On était jeunes
On était fous
La bohème, la bohème
Ça ne veut plus rien dire du tout

\beginsong{Le bon Dieu s’énervait}[by={Hugues Aufray (1966)}]

\beginverse
L\\[Mi]e Bon Dieu s'énervait dans son at\\[La]elier:
« Ça fa\\[Mi]it déjà trois ans que j'ai pla\\[Si7]nté cet arbre.
Et j'a\\[Mi]i beau l'arroser à lon\\[La]gueur de journées,
Il pousse en\\[Mi]core moins vit\\[Si7]'que ma bar\\[Mi]be. » Si7)
\endverse

	Refrain
Pour faire un a\\[Mi]rbre, Mon D\\[La]ieu que c'est long
Pour faire un a\\[Mi]rbre, Mon D\\[Si7]ieu que c'est l\\[Mi]ong
\\[Bis]

\beginverse
Le Bon Dieu s'énervait dans son atelier:
« Sur ce maudit baudet dix ans j'ai travaillé.
Je n'arrive pas à le faire avancer, 
Et encore moins à le faire reculer. »
\endverse

	Refrain
Pour faire un âne, Mon Dieu que c'est long (4 x)

\beginverse
Le Bon Dieu s'énervait dans son atelier
En regardant Adam marcher à quatre pattes:
« Et pourtant nom d'une pipe j'avais tout calculé
Oui pour qu'il marche sur ses deux pieds. »
\endverse

	Refrain
Pour faire un homme, Mon Dieu que c'est long. (4 x)

\beginverse
Le Bon Dieu s'énervait dans son atelier
En regardant le monde qu'il avait fab-riqué:
« Les gens se battent comme des chiffonniers
Et je n' peux mêm' plus dormir en paix. »
\endverse

	Refrain
Pour faire un arbre, Bon Dieu que c'est long.
Pour faire un âne, Bon Dieu que c'est long.
Pour faire un homme, Bon Dieu que c'est long.
Pour faire un monde, Bon Dieu que c'est long.

\beginsong{La brouette d'Echallens}[by={Traditionnel \- Romandie}]

\beginverse
Sur le Lausanne-Echallens,
Tout doux, tout doux, tout doucement 
J'ai entrepris un voyage d'agrément,
Tout doux, tout doucement
\endverse

\beginverse
Le train part au bout d'un moment ...
Pour s'arrêter immédiatement ...
\endverse

\beginverse
Charrette \! On avait oublié seulement ...
Trois ou quatr'compartiments ....
\endverse

\beginverse
On revient chercher ces tire-au-flanc ...
Et l'on repart vite pour rattraper le temps ...
\endverse

\beginverse
Toutes les vaches du pays romand ...
Regardaient passer le train en rigolant ...
\endverse

\beginverse
Les petits veaux disaient : «Maman ...
Regarde voir l'air bête qu'ils ont là-dedans
\endverse

\beginverse
Au bout d'un moment, le mécanicien descend ...
Pour satisfaire un petit besoin pressant ...
\endverse

\beginverse
On s'arrête en gare de Bottens ...
Pour attendre l'express de Boulens qui doit passer dans 30 ans ...
\endverse

\beginverse
Moi, j'ai attendu pendant 10 ans ...
Puis je suis rentré pédestrement ...
\endverse

\beginverse
Ma bourgeoise qui me croyait mort depuis longtemps ...
S'était remariée et avait eu 9 enfants ...
\endverse

\beginsong{C’est le printemps}[by={Henri Dès (1991)}]

\beginverse
J'suis content c'est l'printemps
Aujourd'hui j'ai rien à faire
Quelle aubaine turlutaine
Je marche le nez en l'air
\endverse

\beginverse
J'suis content c'est l'printemps
Les arbres sont en couleur dans les nids
Les petit's s'égosillent tous en choeur
\endverse

	Refrain
Le matin, le matin ne rime plus avec chagrin
A midi, à midi
Je n'aurai pas plus de soucis
A 4 heures, à 4 heures
Ça rime avec tartine au beurre
Et le soir, et le soir
Ça rime toujours avec espoir

\beginverse
J'suis content c'est l'printemps
Qui vient juste après l'hiver le voilà
Youp' lala c'est joli pis c'est pas cher
\endverse

\beginverse
J'suis content c'est l'printemps
C'est pour moi qu'elles butinent
Les abeilles dans l'soleil
Me préparent mes tartines
\endverse

	 refrain

\beginverse
J'suis content c'est l'printemps
Je compte les rossignols j'suis gâté
C'est congé je n'irai pas à l'école
\endverse

\beginverse
J'suis content c'est l'printemps
Poussent des petit's bourgeons dans les prés
Sur mon nez poussent des petit's boutons
\endverse

	refrain

\beginverse
J'suis content dans l'étang
Y'a de nouveau des grenouilles
Elles s'enlacent elles s'embrassent
Y'en a mêm' qui s'tripatouillent
J'suis content c'est l'printemps
J'aurai bientôt une p'tite soeur
C'est maman en chantant
Qui me l'a dit tout à l'heure
\endverse

	refrain

\beginverse
J'suis content c'est l'printemps
\endverse

\beginsong{C'est un pays\*}[by={Soldat Louis (1995)}]

	Refrain
C'est un pays, fallait qu'j't'en parle
Car j'l'ai dans l'coeur comme tu crois pas
Quand j'suis pas d'dans c'est pas normal
A croire que l'monde n'existe pas.

\beginverse
C'est pas fait pour les cons qui râlent
Après la pluie ou j'sais pas quoi
Moi j'l'aime mieux sous un ciel qui chiale
Balayé par un vent d'noroît.
\endverse

\beginverse
Là-bas c'est la mer qui donne
Et qui reprend quand ça lui plaît
Et ce putain d'glas qui résonne
Quand elle a r'pris tout l'monde le sait.
\endverse

\beginverse
Là-bas si c'est pas pour ta pomme
On te le f'ra savoir vit'fait
Ils en ont vu passer des tonnes
De colons et voire même d'Anglais.
\endverse

\beginverse
Parfois toute la violence
Qui fait lever l'poing sur la place
Qui rappelle qu'il y a méfiance
Après la langue on vise la race.
\endverse

\beginverse
Qu'elle s'est pas trop gênée la France
Pour lui mettre les pieds dans la crasse
Des fois qu'l'idée d'indépedance
Ne laiss'rait pas vraiment de glace.
\endverse

\beginverse
Car ça n'aime pas les conquérants
A la cupidité vénale
D'puis qu'une Duchesse encore enfant
S'est fait mettr' d'une manière royale.
\endverse

\beginverse
Sa liberté c'est l'océan
Qui la nuit va r'joindre les étoiles
Et sa terre qui a fait serment
D'être à jamais terre nationale.
\endverse

\beginverse
C'est aux cris des oiseaux de mer
Quand il reviennent près du rivage
Que j'ai compris qu'il y a l'enfer
Mais qu'ça vaut toujours mieux qu'une cage.
\endverse

\beginverse
Et même quand chaque jour est une guerre
Qui n'se lit que sur les visages
Ici on n'parle pas d'sa misère
Et encore moins de son courage.
\endverse

\beginverse
Si j'en rajoute un peu, tant pis
Au début j't'ai bien dit que j'l'aime
Dans tout c'merdier c'putain d'pays
M'tient plus chaud qu'la gonzesse que j'traîne.
\endverse

\beginverse
J'ai pas fini d'l'ouvrir pour lui
Pour lui j'fil'rais même des chataîgnes
Au premier salaud qui l'détruit
Ou qui voudrait lui r'mettre des chaînes
\endverse

	refrain \\[bis]

\beginsong{Ça fait rire les oiseaux}[by={La compagnie créole (1986)}]

	Refrain
Ça fait rire les oiseaux
Ça fait chanter les abeilles
Ça chasse les nuages
Et fait briller le soleil

\beginverse
Ça fait rire les oiseaux
Et danser les écureuils
Ça rajoute des couleurs
Aux couleurs de l'arc-en-ciel
\endverse

\beginverse
Ça fait rire les oiseaux
Oh, oh, oh, rire les oiseaux
\\[bis]
\endverse

\beginverse
Une chanson d'amour
C'est comme un looping en avion
Ça fait battre le cœur
Des filles et des garçons
\endverse

\beginverse
Une chanson d'amour
C'est d'l'oxygène dans la maison
Tes pieds touchent plus par terre
T'es en lévitation
\endverse

\beginverse
Si y a d'la pluie dans ta vie
Si le soir te fait peur
La musique est là pour ça
\endverse

\beginverse
Y a toujours une mélodie pour des jours meilleurs
Allez, tape dans tes mains
Ça porte bonheur
C'est magique, un refrain, qu'on reprend tous en chœur
\endverse

	refrain

\beginverse
Ça fait rire les oiseaux
Oh, oh, oh, rire les oiseaux
\endverse

\beginverse
T'es revenu chez toi
La tête pleine de souvenirs
Des soirs au clair de lune
Des moments de plaisir
\endverse

\beginverse
T'es revenu chez toi
Et tu veux déjà repartir
Pour trouver l'aventure
Qui n'aurait pas de finir
\endverse

\beginverse
Si y a du gris dans ta nuit
Des larmes dans ton cœur
La musique est là pour ça
\endverse

\beginverse
Y a toujours une mélodie pour des jours meilleurs
Allez, tape dans tes mains
Ça porte bonheur
C'est magique, un refrain, qu'on reprend tous en chœur
\endverse

	refrain

\beginverse
Ça fait rire les oiseaux
Oh, oh, oh, rire les oiseaux
\endverse

	refrain

\beginverse
Ça fait rire les oiseaux
Oh, oh, oh, rire les oiseaux
\endverse

	refrain

\beginverse
Ça fait rire les oiseaux
Oh, oh, oh, rire les oiseaux
\\[bis]
\endverse

\beginsong{Le café}[by={Oldelaf (2006)}]

\beginverse
Pour bien commencer ma petite journée
Et me réveiller, moi j'ai pris un café
Un arabica noir et bien corsé
J'enfile ma parka, ça y est je peux y aller
"Où est-ce que tu vas" me cri mon aimée
Prenons un kawa, je viens de me lever
Étant en avance et un peu forcé
Je change de sens et j'reprends un café
\endverse

\beginverse
À 8h moins l'quart, faut bien l'avouer
Les bureaux sont vides on pourrait s'ennuyer
Mais je reste calme, je sais m'adapter
Le temps qu'ils arrivent, j'ai l'temps pour un café
La journée s'emballe, tout l'monde peut bosser
Au moins jusqu'à l'heure de la pause café
Ma secrétaire entre "Fort comme vous l'aimez"
Ah mince, j'viens d'en prendre un, 'fin bon, maintenant qu'il est fait
\endverse

\beginverse
Un repas d'affaires tout près du sentier
Il fait un temps superbe mais je me sens stressé
Mes collègues se marrent "Détends-toi Renée
Prends un bon cigare et un petit café"
Une fois fini, mes collègues crevés appellent un taxi
Ho mais moi j'ai envie d'sauter
Je fais tout Paris, ahou, puis j'vois un troquet
J'commande un déca' mais encaféiné
\endverse

\beginverse
J'arrive au bureau, ma secrétaire me fait
"Vous êtes un peu en retard, je me suis inquiétée"
Mmh, j'la jette par la fenêtre, elle l'avait bien cherché
D'façon il faut qu'je rentre, mais avant, un café
Attendant l'métro, je me fais agresser
Une petite vieille me dit "Euh, vous avez l'heure s'il-vous-plaît"
Hmm, j'lui casse la tête, et j'la pousse sur le quai
Je file à la maison et puis j'me sers un, devinez
\endverse

"Papa, mon papa, en classe je suis premier"
Putain mais quoi, tu vas arrêter d'me faire chier
Ho mais qu'il est con ce gosse, en plus, il s'met à chialer
J'm'enferme dans la cuisine, il reste un peu d'café
Ça fait 14 jours que je suis enfermé
J'suis seul dans ma cuisine et je bois du café
Il faudrait bien qu'je dorme, les flics vont m'choper
Alors je clous les portes et j'reprends du café

\beginsong{Casatschok\*}[by={Rika Zaraï (1969)}][by={Capo II}]

\beginverse
Refrain 1
C\\[Lam]'est l'hiver qui frappe à notre p\\[Mim]orte
Mes amis, allumons un bon f\\[Lam]eu
C'est hiv\\[Do]er que le d\\[Sol]iable l'emp\\[Lam]orte
M\\[Rém]es am\\[Lam]is ce s\\[Mi]oir oublions-l\\[Lam]e
(bis 2 der)
\endverse

\beginverse
B\\[La]abouchka apporte les pains d\\[Ré]'orge
C\\[Mi]e qu'il y a de bon dans la m\\[La]aison 
L\\[La]a vodka qui brûle un peu la g\\[Ré]orge
M\\[Mi]ais qui nous laisse le cœur plein de chans\\[La]ons
(bis 2 der)
\endverse

	
Refrain2
C'est l'hiver qui frappe à notre porte
Mes amis, dansons comme le feu 
C'est l'hiver que le diable l'emporte
Mes amis, dansons comme le feu 
(bis 2 der)

\beginverse
Refrain 3
Dans les bois les loups font une ronde 
Sur la neige frissonnent les corbeaux
Oublions la tristesse du monde
Tous les loups et les vilains oiseaux
(bis 2 der)
\endverse

\beginverse
Pétrouchka, prends ta balalaika 
Et joue-moi un air à ta façon
Joue d'abord les bateliers de la Volga 
E\\[Mi]t quand tu auras fini nous danserons
\endverse

\beginverse
Refrain 1
\endverse

\beginsong{Ce rêve bleu}[by={Karine Kosta, Paolo Domingo \- Aladdin (1992)}]

\beginverse
Je vais t’offrir un monde
Au mille et une splendeurs
Dis-moi princesse
N’as-tu jamais laissé parler ton cœur
Je vais ouvrir tes yeux
Aux délices et aux merveilles
De ce voyage en plein ciel
Au pays du rêve bleu
Ce rêve bleu
C’est un nouveau monde en couleurs
Où personne ne nous dit
C’est interdit
De croire encore au bonheur
Ce rêve bleu
Je n’y crois pas c’est merveilleux
Pour moi c’est fabuleux
Quand dans les cieux
Nous partageons ce rêve bleu à deux
Nous faisons ce rêve bleu à deux
Sous le ciel de cristal
Je me sens si légère
Je vire délire et chavire
Dans un océan d’étoiles
Ce rêve bleu (ne ferme pas les yeux)
C’est un voyage fabuleux (et contemple ces merveilles)
Je suis montée trop haut, allée trop loin
Je ne peux plus retourner d’où je viens
Un rêve bleu (sur les chevaux du monde)
Vers les horizons du bonheur (dans la poussière d’étoiles)
Naviguons dans le temps
Infiniment
Et vivons ce rêve merveilleux
Ce rêve bleu
Aux mille nuits
Qui durera
Pour toi et moi
Toute la vie
\endverse

\beginsong{Céline\*}[by={Hugues Aufray (1966)}][by={Capo I}]

\beginverse
Dis-m\\[Mim]oi, Céline, les années ont passé
Pourquoi n'as-tu jamais pensé à t\\[Lam]e marier
\\[Ré]De tout's mes soeurs qui viv\\[Mim]aient ici
Tu es l\\[Lam]a seule sans mar\\[Rém]i.
\endverse

	Refrain
Non, non, non, ne rougis pas, non, ne rougis pas
Tu as, tu as toujours de beaux yeux
Non, non, non, ne rougis pas, non, ne rougis pas
Tu aurais pu rendre un homme heureux.

\beginverse
Dis-moi, Céline, toi qui es notre aînée
Toi qui fût notre mère, toi qui l'as remplacée
N'as-tu vécu que pour nous autrefois
Que sans jamais penser à toi.
\endverse

\beginverse
Dis-moi, Céline, qu'est-il donc devenu 
Ce gentil fiancé qu'on n'a jamais revu 
Est-c'pour ne pas nous abandonner
Que tu l'as laissé s'en aller.
\endverse

\beginverse
Mais non, Céline, ta vie n'est pas perdue
Nous sommes les enfants que tu n'as jamais eus.
Il y a longtemps que je le savais 
Et je ne l'oublierai jamais.
\endverse

	Refrain
Non, non, non, ne pleure pas, non, ne pleure pas
Tu as toujours les yeux d'autrefois
Non, non, non, ne pleure pas, non, ne pleure pas
Nous resterons toujours près de toi.

\beginsong{Cendrillon\*}[by={Téléphone (1982)}][by={Capo II}]

\beginverse
C\\[Sol]endrillon pour s\\[Ré]es vingt ans
Est l\\[Mim]a plus jolie d\\[Do]es enfants
Son b\\[Sol]el amant, le p\\[Ré]rince charmant
L\\[Mim]a prend sur son c\\[Do]heval blanc
Elle o\\[Ré]ublie le t\\[Sol]emps
Dans s\\[Ré]on palais d'arg\\[Mim]ent
Pour n\\[Lam]e pas voir qu'un nouveau jour se lève
Elle f\\[Do]erme les yeux et dans ses rêves
\endverse

\beginverse
Refrain
Elle p\\[Sol]art \\[Ré] \\[Mim]
\\[Do]Jolie petite hist\\[Sol]oire \\[Ré] \\[Mim]
\endverse

\beginverse
Cendrillon pour ses trente ans
Est la plus triste des mamans
Le prince charmant a foutu l'camp
Avec la Belle au bois dormant
Elle a vu cent chevaux blancs
Loin d'elle emmener ses enfants
Et elle commence à boire
A traîner dans les bars
Emmitouflée dans son cafard
Maintenant elle fait le trottoir
\endverse

\beginverse
Refrain
\endverse

\beginverse
Dix ans de cette vie ont suffi
A la changer en junkie 
Et dans un sommeil infini
Cendrillon voit finir sa vie
Les lumières dansent dans l'ambulance
Mais elle tue sa dernière chance
Tout ça n'a plus d'importance
\endverse

	Refrain
Elle part
Fin de l'histoire

\beginverse
Notre père qui êtes si vieux
As-tu vraiment fait de ton mieux
Car sur la terre et dans les cieux
Tes anges n'aiment pas devenir vieux
\endverse

\beginsong{Les Champs-Élysées}[by={Joe Dassin (1969)}][by={Capo IV}]

\beginverse
Je m'\\[Do]baladais sur l'a\\[Mi7]venue, 
Le cœ\\[Lam]ur ouvert à l'in\\[Do7]connu 
J'av\\[Fa]ais envie de dir\\[Do]e bonjour à n'i\\[Ré7]mporte qu\\[Sol]i, 
N'imp\\[Do]orte qui et ce\\[Mi7] fut toi, et je t\\[Lam]'ai dit n'im\\[Do7]porte quoi
Il s\\[Fa]uffisait de te\\[Do] parler pour t\\[Ré]'app\\[Sol7]rivois\\[Do]er
\endverse

	Refrain
A\\[Do]ux C\\[Mi7]hamps-Elysé\\[Lam]es\\[Do7], a\\[Fa]ux Ch\\[Do]amps-Elysé\\[Ré7]es\\[Sol] 
A\\[Do]u soleil, s\\[Mi7]ous la plus à\\[Lam] midi ou à\\[Do7] minuit
I\\[Fa]l y a tout c’que vo\\[Do]us voulez aux C\\[Sol]hamps\\[Sol7]-Elysé\\[Do]es.

\beginverse
Tu m'as dit : « J'ai rendez-vous
Dans un sous-soi avec des fous
Qui vivent la guitare à la main du soir au matin »
Alors je t'ai accompagnée, on a chanté, on a dansé
Et l'on n'a même pas pensé à s'embrasser
\endverse

	Refrain

\beginverse
Hier au soir deux inconnus, et ce matin sur l'avenue 
Deux amoureux tout étourdis par la longue nuit 
Et de l'Etoile à la Concorde, un orchestre à mille cordes
Tous les oiseaux du point du jour chantent l'amour
\endverse

\beginsong{Chanson pour l’Auvergnat}[by={Georges Brassens (1965)}][by={Capo II}]

\beginverse
E\\[Lam]lle est à toi cette\\[Mi7] chanson, 
Toi l'Auvergnat qui san\\[Lam]s façon,
M'as donné quatre bo\\[Mi7]uts de bois,
Quand da\\[Lam]ns ma vie il f\\[Sol7]aisait fro\\[Do]id.\\[Mi7]
T\\[Lam]oi qui m'as donné d\\[Mi7]u feu quand
Les croquantes et l\\[Lam]es croquants,
Tous les gens bien i\\[Mi7]ntentionnés
M\\[Lam]'avaient fermé la p\\[Sol7]orte au n\\[Do]ez.
C\\[Do7]e n'était r\\[Fa]ien qu\\[Sol7]'un peu de b\\[Do]ois,
M\\[Lam]ais il m'a\\[Rém]vait c\\[Mi7]hauffé le c\\[Lam]orps
E\\[Rém]t dans mon âme il brûle e\\[Lam]ncore,
A\\[Fa] la manière d'\\[Rém]un feu de j\\[Mi7]oie.
\endverse

	Refrain
T\\[Lam]oi l'Auvergnat quand t\\[Mi7]u mourras,
Quand le croqu'mort t'em\\[Lam]portera,
Qu'il te conduise\\[Ré], à travers ci\\[Sol]el,
A\\[Fa]u Père E\\[Mi7]terne\\[Lam]l.

\beginverse
Elle est à toi cette chanson, 
Toi l'hôtesse qui sans façon
M'as donné quatre bouts de pain,
Quand dans ma vie il faisait faim.
Toi qui m'ouvris ta huche quand
Les croquantes et les croquants,
Tous les gens bien intentionnés
S'amusaient à me voir jeûner.
Ce n'était rien qu'un peu de pain,
Mais il m'a réchauffé le corps
Et dans mon âme il brûle encore,
A la manière d'un grand festin.
\endverse

	Refrain
Toi l'hôtesse quand tu mourras,
Quand le croqu'mort t'emportera,
Qu'il te conduise à travers ciel,
Au Père Eternel.

\beginverse
Elle est à toi cette chanson,
Toi l'étranger qui sans façon
D'un air malheureux m'as souri,
Lorsque les gendarmes m'ont pris.
Toi qui n'as pas applaudi quand
Les croquantes et les croquants
Tous les gens bien intentionnés,
Riaient de me voir emmener.
Ce n'était rien qu'un peu de miel,
Mais il m'a réchauffé le corps
Et dans mon âme il brûle encore,
A la manière d'un grand soleil.
\endverse

	Refrain
Toi l'étranger quand tu mourras,
Quand le croqu'mort t'emportera,
Qu'il te conduise à travers ciel,
Au Père Eternel.

\beginsong{Chanson pour Pierrot}[by={Renaud (1979)}]

\beginverse
T'es pas né dans la rue
T'es pas né dans l'ruisseau
T'es pas un enfant perdu
Pas un enfant d'salaud
Vu que t'es né que dans ma tête
Et que tu vis dans ma peau
J'ai construit ta planète
Au fond de mon cerveau
\endverse

\beginverse
Refrain
Pierrot
Mon gosse, mon frangin, mon poteau
Mon copain, tu me tiens chaud
Pierrot
\endverse

\beginverse
Depuis l'temps que j'te rêve
Depuis l'temps que j't'invente
Ne pas te voir j'en crève
Mais j'te sens dans mon ventre
Le jour où tu t'ramènes
J'arrête de boire promis
Au moins toute une semaine
Ce sera dur, mais tant pis
\endverse

\beginverse
Refrain
\endverse

\beginverse
Que tu sois fils de princesse
Ou que tu sois fils de rien
Tu seras fils de tendresse
Tu seras pas orphelin
Et j'connais pas ta mère
Et je la cherche en vain
Je connais que la misère
D'être tout seul sur le chemin
\endverse

\beginverse
Refrain
\endverse

\beginverse
Dans un coin de ma tête
Y a déjà ton trousseau
Un jean, une mobylette
Une paire de Santiago
T'iras pas à l'école
J't'apprendrai des gros mots
On jouera au football
On ira au bistrot
\endverse

\beginverse
refrain
\endverse

\beginverse
Tu te laveras pas les pognes
Avant de venir à table
Et tu me traiteras d'ivrogne
Quand j'piquerai ton cartable
J't'apprendrai mes chansons
Tu les trouveras débiles
T'auras p't-être bien raison
Mais j'serai vexé quand même
\endverse

\beginverse
refrain
\endverse

\beginverse
Allez, viens, mon Pierrot
Tu seras le chef de ma bande
J'te refilerai mon couteau
J't'apprendrai la truande
Allez, viens, mon copain
J't'ai trouvé une maman
Tous les trois, ça sera bien
Allez, viens, je t'attends
\endverse

\beginverse
refrain
\endverse

\beginsong{Chanson sur ma drôle de vie}[by={Véronique Sanson (1972)}]

\beginverse
Tu m'as dit que j'étais faite
Pour une drôle de vie
J'ai des idées dans la tête
Et je fais ce que j'ai envie
Je t'emmène faire le tour
De ma drôle de vie
Je te verrai tous les jours
\endverse

\beginverse
Et si je te pose des questions (qu'est-ce que tu diras?)
Et si je te réponds (qu'est-ce que tu diras?)
Si on parle d'amour (qu'est-ce que tu diras?)
Si je sais que tu m'aimes
\endverse

\beginverse
La vie que tu aimes au fond de moi
Me donne tous ses emblèmes
Me touche quand même du bout de ses doigts
Même si tu as des problèmes
Tu sais que je t'aime, ça t'aidera
Laisse les autres totems
Tes drôles de poèmes et viens avec moi
\endverse

\beginverse
On est parti tous les deux
Pour une drôle de vie
On est toujours amoureux
Et on fait ce qu'on a envie
Tu as sûrement fait le tour
De ma drôle de vie
Je te demanderai toujours
\endverse

\beginverse
Et si je te pose des questions (qu'est-ce que tu diras?)
Et si je te réponds (qu'est-ce que tu diras?)
Si on parle d'amour (qu'est-ce que tu diras?)
Et si je sais que tu m'aimes
\endverse

\beginverse
La vie que tu aimes au fond de moi
Me donne tous ses emblèmes
Me touche quand même du bout de ses doigts
Même si tu as des problèmes
Tu sais que je t'aime, ça t'aidera
Laisse les autres totems
Tes drôles de poèmes et viens avec moi
\endverse

\beginverse
Même si je sais que tu m'aimes
La vie que tu aimes au fond de moi
Me donne tous ses emblèmes
Me touche quand même du bout de ses doigts
Même si tu as des problèmes
Tu sais que je t'aime, ça t'aidera
Laisse les autres totems
Tes drôles de poèmes et viens avec moi
\endverse

\beginsong{Chant des marais}[by={Johann Esser (1933)}]

\beginverse
L\\[Lam]oin vers l’infini s’étendent l\\[Rém]es grands p\\[Lam]rés m\\[Mi]arécag\\[Lam]eux \\[Sol]
P\\[Do]as un seul oiseau ne chante d\\[Rém]ans les a\\[Lam]rbres s\\[Mi]ecs et c\\[Lam]reux \\[Sol]
\endverse

\beginverse
Refrain
Ô t\\[Do]erre de détr\\[Sol7]esse, où n\\[Lam]ous devons sans c\\[Mi]esse
Pioch\\[Lam]er, pioch\\[Mi], pioch\\[Lam]er
\endverse

\beginverse
Dans ce camp morne et sauvage, entouré de murs de fer
Il nous semble vivre en cage au milieu d’un grand désert
\endverse

\beginverse
Refrain
\endverse

\beginverse
Bruit des pas et bruit des armes, sentinelles jour et nuit
Et du sang, des cris, des larmes, la mort pour celui qui fuit
\endverse

\beginverse
Refrain
\endverse

\beginverse
Mais un jour dans notre vie, le printemps refleurira
Liberté, liberté chérie, je dirai : tu es à moi \!
\endverse

\beginverse
Ô terre d’allégresse, où nous pourrons sans cesse aimer
Ô terre enfin libre, où nous pouvons revivre… aimer
\endverse

\beginsong{Le chant des sirènes}[by={Fréro Delavega (2014)}]

\beginverse
Enfants des parcs, gamins des plages
Le vent menace les châteaux de sable, façonnés de mes doigts
Le temps n'épargne personne, hélas
Les années passent, l'écho s'évade sur la Dune du Pyla
\endverse

\beginverse
Refrain
Au gré des saisons, des photomatons
Je m'abandonne à ces lueurs d'autrefois
Au gré des saisons, des décisions, je m'abandonne
\endverse

\beginverse
Refrain
Quand les souvenirs s'en mêlent, les larmes me viennent
Et le chant des sirènes me replonge en hiver
Oh, mélancolie cruelle, harmonie fluette, euphorie solitaire
\endverse

\beginverse
Ta-dada-dan, ta-dada-da
Ta-dada-dan, ta-dada-da
\endverse

\beginverse
Combien de farces, combien de frasques
Combien de traces, combien de masques
Avons-nous laissé là-bas
Poser les armes, prendre le large
Trouver le calme dans ce vacarme avant que je ne m'y noie
\endverse

\beginverse
Refrain
\endverse

\beginverse
Oh-oh
\endverse

\beginverse
Refrain
\endverse

\beginverse
Ta-dada-dan, ta-dada-da
Ta-dada-dan, ta-dada-da
Tadalalala, tadalalala
Tadalalala, tadalala
\endverse

# 

\beginsong{Le chanteur\*}[by={Daniel Balavoine (1978)}]

\beginverse
J\\[La]e m'présente je m'appelle Henri
J\\[La]'voudrais bien réussir ma vie, être aim\\[Mi]é
Etre b\\[Ré]eau, gagner de l'argent
Puis surt\\[Rém]out être intelligent
Mais pour t\\[La]out ça il f\\[Lam]audrait que je bosse à plein t\\[Ré]emps
\endverse

\beginverse
Je suis chanteur, je chante pour mes copains
J'veux faire des tubes et que ça tourne bien, tourne bien
J'veux écrire une chanson dans le vent
Un air gai, chic et entraînant
Pour faire danser dans les soirées de Monsieur Durand
\endverse

	Refrain
Et partout dans la r\\[Ré]ue j'veux qu'on parle de m(Fa\#m)oi
Que les filles soient n\\[Ré]ues qu'elles se jettent sur m(Fa\#)oi
Qu'elles m'admirent qu'elles me t\\[Dodim7]uent
Qu'elles s'arrachent ma vertu
Pour les anciennes de l'école devenir une idole
Je veux que toutes les nuits essoufflées dans leur lit
Elles trompent leur mari, dans leurs rêves maudits

\beginverse
Puis après je ferai des galas
Mon public se prosternera devant moi
Des concerts de cent mille personnes
Où même le Tout Paris s'étonne
Et se lève pour prolonger le combat
\endverse

	Refrain

\beginverse
Puis quand j'en aurai assez de rester leur idole
Je remonterai sur scène comme dans les années folles
Je ferai pleurer mes yeux je ferai mes adieux
Et puis l'année d'après je recommencerai
Et puis l'année d'après je recommencerai
Je me prostituerai, pour la postérité
\endverse

\beginverse
Les nouvelles de l'école diront que je suis pédé
Que mes yeux puent l'alcool, que je ferais bien d'arrêter
Brûleront mon auréole, saliront mon passé
\endverse

\beginverse
Alors je serai vieux et je pourrai crever
Je me chercherai un Dieu pour tout me pardonner
J'veux mourir malheureux pour ne rien regretter
J'veux mourir malheureux.
\endverse

\beginsong{Chevaliers de la Table ronde\*}[by={Les 4 Barbus (1956)}]

\beginverse
C\\[Do]hevaliers de la table ronde
Goûtons vo\\[Sol7]ir si le vin est b\\[Do]on
\\[bis]
\endverse

\beginverse
Goûtons vo\\[Fa]ir, oui, oui, oui
Goûtons vo\\[Do]ir, non, non, non
Goûtons vo\\[Sol7]ir si le vin est b\\[Do]on
\\[bis] (reprendre la dernière phrase du couplet)
\endverse

\beginverse
S'il est bon, s'il est agréable
J'en boirai jusqu'à mon plaisir
\\[bis]
\endverse

\beginverse
Si je meurs je veux qu'on m'enterre
Dans une cave où y a du bon vin
\\[bis]
\endverse

\beginverse
Les deux pieds contre la muraille
Et la tête sous le robinet
\\[bis]
\endverse

\beginverse
Et les quatre plus grands ivrognes
Porteront les quatre coins du drap
\\[bis]
\endverse

\beginverse
Pour donner le discours d'usage
On prendra le bistrot du coin
\\[bis]
\endverse

\beginverse
Et si le tonneau se débouche
J'en boirai jusqu'à mon plaisir
\\[bis]
\endverse

\beginverse
Et s'il en reste quelques gouttes
Ce sera pour me rafraîchir
\\[bis]
\endverse

\beginverse
Sur ma tombe je veux qu'on inscrive
Ici gît le roi des buveurs
\\[bis]
\endverse

\beginsong{Le colporteur}[by={Traditionnel}]

\beginverse
S\\[Mi]eul sur la steppe ar\\[Lam]ide
C'est ains\\[Mi]i que je vais, vagab\\[Lam]ond marchant toujours
T\\[Rém]ous mes paniers sont v\\[Lam]ides
Et mon c\\[Mi]œur depuis longtemps déj\\[Lam]à est sans amour
(bis 2 der)
\endverse

\beginverse
Refrain
Héia héia hé la la l\\[Mi]a la la la l\\[Lam]a la la la \\[bis]
La la la l\\[Rém]a… \\[Lam]\\[Mi]\\[Lam] \\[bis]
\endverse

\beginverse
Je n'ai plus rien à vendre
Ni mouchoir, ni collier, ni ruban, je n'ai plus rien
On a dû me les prendre
Ou sinon j'aurais donc tout perdu jusqu'à mon chien
(bis 2 der)
\endverse

	Refrain

\beginverse
La route immense et grise
Qui là-bas disparaît dans la nuit je ne sais où
Egaré je l'ai prise
Je suis un malheureux colporteur, un pauvre fou
(bis 2 der)
\endverse

	Refrain

\beginverse
Et si ma voix t'appelle
Ne fuis pas, pauvre enfant, mais écoute mon émoi
Et si ta soeur est belle
Conte-lui mon histoire et qu'elle ait pitié de moi
(bis 2 der)
\endverse

\beginsong{Comme des enfants}[by={Coeur de pirate (2008)}][by={Capo III}]

\beginverse
A\\[Sol]h, tu vois comme t\\[Si7]out se mêle
Et du c\\[Mim]œur à tes lèvres, je dev\\[Do]iens un casse-tête
Ton r\\[Sol]ire me crie, de t\\[Si7]e lâcher
Avant d\\[Mim]e perdre prise et d\\[Do]'abandonner
Car j\\[Sol]e ne t'en demanderai jam\\[Si7]ais autant
Déjà q\\[Mim]ue tu me traites, comme u\\[Do]n grand enfant
Et nous n\\[Sol]'avons plus rien, à\\[Si7] risquer
À part n\\[Mim]os vies qu'on laisse d\\[Do]e côté
\endverse

\beginverse
Refrain
Et il m\\[Do]'aime enc\\[Sol]ore, et moi j\\[Do]e t'aime un peu plus f\\[Sol]ort
Mais il m\\[Do]'aime enc\\[Sol]ore, et moi j\\[Do]e t'aime un peu plus f\\[Sol]ort
\endverse

\beginverse
S'en est assez de ces dédoublements
C'est plus dur à faire, qu'autrement
Car sans rire c'est plus facile de rêver
À ce qu'on ne pourra, jamais plus toucher
Et on se prend la main, comme des enfants
Le bonheur aux lèvres, un peu naïvement
Et on marche ensemble, d'un pas décidé
Alors que nos têtes nous crient de tout arrêter
\endverse

\beginverse
Il m'aime encore, et toi tu m'aimes un peu plus fort
Mais il m'aime encore, et moi je t'aime un peu plus fort
\endverse

\beginverse
Et malgré ça, il m'aime encore, et moi je t'aime un peu plus fort
Mais il m'aime encore, et moi je t'aime encore plus fort
\endverse

\beginverse
Et malgré ça, il m'aime encore, et moi je t'aime un peu plus fort
Mais il m'aime encore, et moi je t'aime un peu plus fort
\\[bis]
\endverse

\beginsong{Complainte de la Blanche biche}[by={Tri Yann (1974)}]

\beginverse
Celles qui vont au bois c'est la mère et la fille
La mère va chantant et sa fille soupire
Qu'a vous à soupirer, ma blanche Marguerite?
J'ai bien trop d'ire en moi et n'ose vous le dire
\endverse

\beginverse
Je suis fille le jour et la nuit blanche biche
La chasse est après moi, des barons et des princes
Et mon frère Renaud qui est encore le pire
Allez ma mère, allez, bien promptement lui dire
\endverse

\beginverse
Qu'il arrête ses chiens jusqu'à demain midi
Où sont tes chiens Renaud, et la chasse gentille?
Ils sont dedans le bois, à courre blanche biche
Arrête-les, Renaud, arrête, je t'en prie
\endverse

\beginverse
Trois fois les a cornés, de son cornet de cuivre
À la troisième fois, la blanche biche est prise
Mandons le dépouilleur, qu'il dépouille la biche
Celui qui la dépouille dit je ne sais que dire
\endverse

\beginverse
Elle a le cheveu blond et le sein d'une fille
A tiré son couteau, en quartiers il l'a mise
En on fait un dîner aux barons et aux princes
Nous voici tous sied, hors ma sœur Marguerite
\endverse

\beginverse
Vous n'avez qu'à manger, suis la première assise
Ma tête est dans le plat et mon cœur aux chevilles
Mon sang est répandu par toute la cuisine
Et sur vos noirs charbons mes pauvres os s'y grillent
\endverse

\beginverse
Celles qui vont au bois, c'est la mère et la fille
La mère va chantant et la fille soupire
Qu'a vous à soupirer, ma blanche Marguerite?
J'ai bien trop d'ire en moi et n'ose vous le dire
\endverse

\beginsong{Les copains d’abord}[by={Georges Brassens (1965)}]

\beginverse
Non ce n'é\\[Do])tait pas le radeau 
De la Méduse ce bateau
Qu'on se le d\\[Ré7]ise au fond des ports
Dise au fond des ports
Il navig\\[Fa]uait en père peinard
Sur la grand' ma\\[Mi]are des canards
Et s'appelait l\\[Lam]es Copains d'ab\\[Ré7]ord
Les Cop\\[Sol]ains d'ab\\[Do]ord\\[Sol]\\[Do]
\endverse

\beginverse
Ses « fluctuat nec mergitur »
C'était pas d' la littérature
N'en déplaise aux jeteurs de sorts
Aux jeteurs de sorts
Son capitaine et ses matelots
N'étaient pas des enfants d'salauds
Mais des amis franco de port
Des Copains d'abord
\endverse

\beginverse
C'étaient pas des amis de luxe
Des petits Castor et Pollux
Des gens de Sodome et Gomorrhe
Sodome et Gomorrhe
C'étaient pas des amis choisis
Par Montaigne et La Boétie
Sur le ventre ils se tapaient fort
Les Copains d'abord
\endverse

\beginverse
C'étaient pas des anges non plus
L'évangile ils l'avaient pas lu
Mais ils s'aimaient toutes voiles dehors
Toutes voiles dehors
Jean, Pierre, Paul et compagnie
C'était leur seule litanie
Leur Credo, leur Confiteor
Aux Copains d'abord.
\endverse

\beginverse
Au moindre coup de Trafalgar
C'est l'amitié qui prenait l'quart
C'est elle qui leur montrait le nord
Leur montrait le nord 
Et quand ils étaient en détresse
Qu'leurs bras lançaient des S.O.S.
On aurait dit des sémaphores
Les Copains d'abord
\endverse

\beginverse
Au rendez-vous des bons copains
Y avait pas souvent de lapins
Quand l'un d'entre eux manquait à bord
C'est qu'il était mort
Oui mais jamais, au grand jamais
Son trou dans l'eau n'se refermait
Cent ans après, coquin de sort
Il manquait encore
\endverse

\beginverse
Des bateaux j'en ai pris beaucoup
Mais le seul qui ait tenu le coup
Qui n'ait jamais viré de bord
Mais viré de bord
Naviguait en père peinard
Sur la grand mare des canards
Et s'appelait les Copains d'abord
Les Copains d'abord
\endverse

\beginverse
Non ce n'était pas le radeau
De la Méduse ce bateau
Qu'on se le dise au fond des ports
Dise au fond des ports
Il naviguait en père peinard
Sur la grand' mare des canards
Et s'appelait les Copains d'abord
Les Copains d'abord
\endverse

\beginsong{Les corons}[by={Pierre Bachelet (1982)}][by={Capo II}]

	Refrain
Au no\\[Sol]rd, c\\[Ré]'étaient les coro\\[Sol]ns,
La te\\[Ré]rre, c'était le charb\\[Mim]on
Le ci\\[Do]el, c'était l'h\\[Ré]oriz\\[Sol]on,
Les h\\[Ré]ommes, d\\[Si7]es mineurs de fo\\[Mim]nd

\beginverse
Nos fen\\[Mim]êtres donnaient sur des f'nêtres sembl\\[Ré]ables,
Et la pluie mouillait mon c\\[Mim]artable
Et mon père en rentrant avait des yeux si b\\[Ré]leus,
Que je croyais voir le ciel b\\[Sol]leu
J'apprenais mes leçons la joue contre son b\\[Ré]ras,
Je crois qu'il était f\\[Si7]ier de m\\[Mim]oi
Il était généreux comme ceux du pa\\[Ré]ys,
Et je lui dois ce que je su\\[Si7]is.
\endverse

\beginverse
Et c'était mon enfance et elle était heureuse,
Dans la buée des lessiveuses
Et j'avais des terrils à défaut de montagnes,
D'en haut je voyais la campagne
Mon père était « gueule noire » comme l'étaient ses parents
Ma mère avait les cheveux blancs,
Ils étaient de la fosse comme on est du pays
Grâce à eux je sais qui je suis.
\endverse

\beginverse
Y'avait à la mairie le jour de la kermesse,
Une photo de Jean Jaurès
Et chaque verre de vin était diamant rose,
Posé sur fond de silicose
Ils parlaient de « 36 » et des coups de grisou,
Des accidents du fond du trou
Ils aimaient leur métier comme on aime un pays,
C'est avec eux que j'ai compris.
\endverse

\beginsong{Les crapauds}[by={Marc Legrand (1897), Alain Souchon (2011)}]

\beginverse
La nuit e\\[Do]st limpide, l'étang est sans ride
Dans le ciel splendide luit le c\\[Sol]roissant d\\[Do]'or
O\\[Do]rme chêne ou tremble, nul arbre ne tremble
Au loin le bois semble un g\\[Sol]éant qui d\\[Do]ort
Chien ni loup ne q\\[Fa]uitte sa niche o\\[Sol]u son g\\[Do]îte
Aucun bruit n'ag\\[Fa]ite la terre a\\[Sol]u repos
Alors d\\[Do]ans la v\\[Fa]ase, ouvrant e\\[Sol]n ext\\[Do]ase
Leurs yeux d\\[Lam]e top\\[Rém]aze, chantent l\\[Sol]es crap\\[Do]auds.
\endverse

\beginverse
Ils disent nous sommes haïs par les hommes
Nous troublons leur somme de nos tristes chants
Pour nous point de fête, Dieu seul sur nos têtes
Sait qu'il nous fit bêtes et non point méchants
Notre peau terreuse se gonfle et se creuse
D'une bave affreuse nos flancs sont lavés
Et l'enfant qui passe loin de nous s'efface
Et pâle nous chasse à coups de pavés.
\endverse

\beginverse
Des saisons entières dans les fondrières
Un trou sous les pierres est notre réduit
Le serpent en boule près de nous s'enroule
Quand il pleut en foule nous sortons la nuit.
Et dans les salades, faisant des gambades
Pesants camarades, nous allons manger
Manger sans grimace cloporte ou limace
Ou vers qu'on ramasse dans le potager.
\endverse

\beginverse
Nous aimons la mare qu'un reflet chamarre
Où dort à l'amarre un canot pourri
Dans l'eau qu'elle souille sa chaîne se rouille
La verte grenouille y cherche un abri
Là la source épanche son écume blanche
Un vieux saule penche au milieu des joncs,
Et les libellules aux ailes de tulle
Font crever des bulles au nez des goujons.
\endverse

\beginsong{Le cuisinier de la troupe}[by={Traditionnel}]

\beginverse
Si j'ai pour la cuisine un goût très prononcé
C'est grâce à ma cousine qui m'a bien éduqué
C'est à ses connaissances que je dois de savoir
Avec un peu d'aisance faire un œuf au miroir \!
\endverse

	Refrain
Le rata, le rata, j'connais ça, j'connais ça
Je suis le cuisinier de la troupe
La maggi, le maggi, et le riz, tout pourri
Tout me sert à préparer la soupe
Je fricote la popote sans jamais m'lasser
Je boulotte les carottes sans les éplucher.

\beginverse
Pour faire les omelettes, j'm'y entends spécialement
Je mets dans une assiette trois douzaines d'œufs seul'ment
J'ajoute de la farine et une pincée de sel
Un peu de graisse surfine et de l'eau d'Romanel.
\endverse

\beginverse
Quand notre caisse est riche et l'caissier généreux
De l'épargne on s'en fiche et l'on fait du copieux
Une bonne friture servie dans un p'tit plat
Ou de la confiture avec du chocolat.
\endverse

\beginverse
Parfois j'ai d'la déveine cela peut arriver
Malgré toute ma peine, j'vois le rata brûler
Tous les zigots ronchonnent et veulent me dégommer
Sans m'en faire je chantonne afin d'les consoler.
\endverse

\beginsong{La danse des canards}[by={Jean-Jacques Lionel (1980)}]

\beginverse
C'est la danse des canards
Qui en sortant de la mare
Se secouent le bas des reins
Et font coin-coin
Fait's comme les petits canards
Et pour que tout l'monde se marre
Remuez le popotin
En f'sant coin-coin
À présent claquez du bec
En secouant vos plumes avec
Avec beaucoup plus d'entrain
Et des coin-coin
Allez mettez-en un coup
On s'amuse comme des petits fous
Maintenant pliez les genoux
Redressez-vous
\endverse

\beginverse
Refrain
Tournez, c'est la fête
Bras dessus-dessous
Comme des girouettes
C'est super chouette
C'est extra-fou
\endverse

\beginverse
C'est la danse des canards
Les gamins comme les loubards
Vont danser ce gai refrain
Dans tous les coins
Ne soyez pas en retard
Car la danse des canards
C'est le tube de demain
Coin-coin, coin-coin
Il suffit d'fermer son bec
En mettant ses plumes au sec
Pliez les genoux c'est bien
Et faites coin-coin
Ça y est vous avez compris
Attention c'n'est pas fini
Nous allons jusqu'au matin
Faire des coin-coin
\endverse

\beginverse
Refrain
\endverse

\beginverse
C'est la danse des canards
Qui en sortant de la mare
Se secouent le bas des reins
Et font coin-coin
À présent claquez du bec
En secouant vos plumes avec
Avec beaucoup d'entrain
Et des coin-coin
C'est la danse des canards
C'est dément et c'est bizarre
C'est terribilos comme tout
C'est dingue, c'est tout
Allez mettez-en un coup
On s'amuse comme des petits fous
Maintenant pliez les genoux
Redressez-vous
\endverse

\beginverse
Refrain
\endverse

\beginverse
C'est la danse des canards
Qui en sortant de la mare
Se secouent le bas des reins
Et font coin-coin
Fait's comme les petits canards
Et pour que tout l' monde se marre
Remuez le popotin
En f'sant coin-coin
C'est la danse des canards
Les gamins comme les loubards
Vont danser ce gai refrain
Dans tous les coins
Ne soyez pas en retard
Car la danse des canards
C'est le tube de demain
Coin-coin coin-coin
\endverse

\beginsong{Dans les prisons de Nantes}[by={Tri Yann (1972)}]

\beginverse
D\\[lam]ans les prisons de N\\[Sol]antes
dam dibidibidam dam dibidibidibidam
D\\[Lam]ans les prisons de Nantes
Y'avait un prisonnie\\[Sol]r
Y'avait un prisonnie\\[Lam]r\\[Mim]\\[Lam] 
\endverse

\beginverse
Personne ne le vint le voir
Que la fille du geôlier
\endverse

\beginverse
Elle lui apporte à boire
A boire et à manger
\endverse

\beginverse
Un jour il demande:
\- Et que dit-on de moi
\endverse

\- On dit de vous en ville
Que vous serez pendu

\- Et si fon doit me pendre
Déliez-moi les pieds

\beginverse
La fille encore jeunette
Lui délia les pieds
\endverse

\beginverse
Le prisonnier alerte
Dans la Loire s'est jeté
\endverse

\beginverse
A la première plonge
A failli se noyer
\endverse

\beginverse
A la deuxième plonge
La Loire a traversé
\endverse

\beginverse
Quand il fut sur la rive
Il s'est mis à chanter
\endverse

\beginverse
Je chante pour les filles
Les filles à marier
\endverse

\beginverse
Si je reviens à Nantes
Oui, je l'épouserai 
\endverse

\beginverse
Dans les prison de Nantes 
Y'avait un prisonnier
\endverse

\beginsong{Dans les yeux d'Emilie}[by={Joe Dassin (1977)}]

\beginverse
Dans son quartier du vieux Québec
Les rues ont l'air d'avoir l'accent
Et l'an deux mille voisine avec
Les maisons grises du vieux temps
Mais l'hiver vient d'éclater
Le Saint-Laurent est prisonnier
D'un décembre qui va bien durer six mois
Quand les jours ressemblent aux nuits
Sans éclaircie à espérer
Qui peut croire que l'été nous reviendra
\endverse

\beginverse
Refrain
Moi, j'avais le soleil
Jour et nuit dans les yeux d'Émilie
Je réchauffais ma vie à son sourire
Moi, j'avais le soleil
Nuit et jour dans les yeux de l'amour
Et la mélancolie au soleil d'Émilie
Devenait joie de vivre
\endverse

\beginverse
Dans son quartier du vieux Québec
Quand les toits redeviennent verts
Quand les enfants ont les pieds secs
On tourne le dos à l'hiver
C'est la fête du printemps
Le grand retour du Saint-Laurent
On dirait que les gens sortent de la terre
Mais Émilie n'est plus à moi
J'ai froid pour la première fois
Je n'ai plus ni sa chaleur, ni sa lumière
\endverse

\beginverse
Refrain
\endverse

\beginverse
En ce temps-là j'avais le soleil
Jour et nuit dans les yeux d'Émilie
Je réchauffais ma vie à son sourire
Moi, j'avais le soleil
Nuit et jour dans les yeux de l'amour
Et la mélancolie au soleil d'Émilie
Devenait joie de vivre
\endverse

\beginsong{Debout les gars}[by={Hugues Aufray (1964)}]

\beginverse
Refrain
Deb\\[Lam]out les gars, réveillez-vous
Il\\[Sol] va falloir en mettre un coup
D\\[Lam]ebout les gars, réveillez-vous
On v\\[Do]a au b\\[Sol]out du mo\\[Lam]nde.
\endverse

\beginverse
C\\[Lam]ette montagne que tu vois
O\\[Sol]n en viendra à bout mon gars
U\\[Lam]n bulldozer et deux cents bras
Et p\\[Do]asser\\[Sol]a la r\\[Lam]oute.
\endverse

\beginverse
Refrain
\endverse

\beginverse
Il ne faut pas se dégonfler
Devant les tonnes de rocher
On va faire un 14 juillet
A coups de dynamite.
\endverse

\beginverse
Refrain
\endverse

\beginverse
Encore un mètre et deux et trois
En dix-neuf cent quatre-vingt-trois
Tes enfants seront fiers de toi
La route sera belle.
\endverse

\beginverse
Refrain
\endverse

\beginverse
Il nous arrive parfois le soir
Comme un petit coup de cafard
Mais ce n'est qu'un peu de brouillard
Que le soleil déchire.
\endverse

\beginverse
Refrain
\endverse

\beginverse
Les gens nous prenaient pour des fous
Mais nous on passera partout
Et nous serons au rendez-vous
De ceux qui nous attendent.
\endverse

\beginverse
Refrain
\endverse

\beginverse
Et quand tout sera terminé
Il faudra bien se séparer
On n'oubliera jamais, jamais
Ce qu'on a fait ensemble.
\endverse

\beginverse
Refrain\\[bis]
\endverse

\beginsong{Dégénérations}[by={Mes Aïeux (2004)}]

\beginverse
Ton arrière-arrière-grand-père, il a défriché la terre
Ton arrière-grand-père, il a labouré la terre 
Et pis ton grand-père a rentabilisé la terre
Pis ton père, il l'a vendu pour devenir fonctionnaire
\endverse

\beginverse
Et pis toi mon p'tit gars, tu sais pu c'que tu vas faire
Dans ton p'tit trois et d'mi ben trop cher, frette en hiver
Il te vient des envies de dev'nir propriétaire
Et tu rêves la nuit d'avoir ton petit lopin d'terre
\endverse

\beginverse
Ton arrière-arrière-grand-mère, elle a eu quatorze enfants
Ton arrière-grand-mère en a eu quasiment autant
Et pis ta grand-mère en a eu trois ctait suffisant
Pis ta mère en voulait pas, toi t'étais un accident
\endverse

\beginverse
Et pis toi ma p'tite fille, tu changes de partenaires tout I'temps
Quand tu fais des conn'ries, tu t'en sauves en avortant
Mais y'a des matins, tu te réveilles en pleurant
Quand tu rêves la nuit d'une grande table entourée d'enfants
\endverse

\beginverse
Ton arrière-arrière-grand-père, a vécu la grosse misère
Ton arrière-grand-père, il ramassait les cennes noires
Et pis ton grand-père, miracle, est devenu millionnaire
Ton père en a hérité, il l'a tout mis dans ses réers
\endverse

\beginverse
Et pis toi p'tite jeunesse tu dois ton cul au ministère
Pas moyen d'avoir un prêt dans une institution bancaire
Pour calmer tes envies de hold-uper la caissière
Tu lis des livres qui parlent de simplicité volontaire
\endverse

\beginverse
Tes arrières-arrières grands-parents, ils savaient comment fêter
Tes arrières-grands-parents ça swinguait fort dans les veillées
Pis tes grands-parents ont connu l'époque yé-yé
Tes parents c'tait les discos c'est là qu'ils se sont rencontrés
\endverse

\beginverse
Et pis toi mon ami qu'est-ce que tu fais de la soirée
Éteins donc ta T.V. faut pas rester encabané
Heureusement que dans la vie certaines choses refusent de changer
Enfile tes plus beaux habits car nous allons ce soir danser
\endverse

\beginsong{Déjeuner en paix}[by={Stephane Eicher (1991)}]

\beginverse
J'abandonne sur une chaise le journal du matin
Les nouvelles sont mauvaises d'où qu'elles viennent
J'attends qu'elle s'réveille et qu'elle se lève enfin
Je souffle sur les braises, pour qu'elles prennent
\endverse

\beginverse
Cette fois je ne lui annoncerai pas
La dernière hécatombe
Je gard'rai pour moi ce que m'inspire le monde
Elle m'a dit qu'elle voulait, si je le permettais
Déjeuner en paix
\endverse

\beginverse
Déjeuner en paix
\endverse

\beginverse
Je vais à la f'nêtre et le ciel ce matin
N'est ni rose, ni honnête pour la peine
"Est-ce que tout va si mal? Est-ce que rien ne va bien?
L'homme est un animal" me dit-elle
\endverse

\beginverse
Elle prend son café en riant, elle me regarde à peine
Plus rien ne la surprend sur la nature humaine
C'est pourquoi elle voudrait, enfin si je le permets
Déjeuner en paix
\endverse

\beginverse
Oui, déjeuner en paix
Ah déjeuner en paix
\endverse

\beginverse
Je regarde sur la chaise le journal du matin
Les nouvelles sont mauvaises d'où qu'elles viennent
"Crois-tu qu'il va neiger?" me demande-t-elle soudain
"Me feras-tu un bébé pour Noël?"
\endverse

\beginverse
Elle prend son café en riant, elle me regarde à peine
Plus rien ne la surprend sur la nature humaine
C'est pourquoi elle voudrait, enfin si je le permets
Déjeuner en paix
\endverse

\beginverse
Oui, déjeuner en paix
Ah déjeuner en paix
Oui, déjeuner en paix
En paix, en paix
\endverse

\beginverse
Enfin déjeuner en paix
Oui, déjeuner en paix
Enfin déjeuner en paix
\endverse

\beginsong{Les démons de minuit}[by={Images (1987)}]

\beginverse
Rue déserte
Dernière cigarette, plus rien ne bouge
Juste un bar qui éclaire le trottoir
D'un néon rouge
J'ai besoin
De trouver quelqu'un, j'veux pas dormir
Je cherche un peu de chaleur à mettre dans mon cœur
\endverse

\beginverse
Refrain
Ils m'entraînent au bout de la nuit
Les démons de minuit
M'entraînent jusqu'à l'insomnie
Les fantômes de l'ennui
\endverse

\beginverse
Dans mon verre
Je regarde la mer qui se balance
J'veux un disque
De Funky Music, faut que ça danse
J'aime cette fille
Sur talons-aiguilles qui se déhanche
Ça met un peu de chaleur au fond de mon cœur
\endverse

\beginverse
Refrain \\[bis]
\endverse

\beginverse
J'aime cette fille
Sur talons-aiguilles qui se déhanche
Ça met un peu de chaleur au fond de mon cœur
Ils m'entraînent au bout de la nuit (jusqu'au bout de la nuit)
Les démons de minuit
M'entraînent jusqu'à l'insomnie (ils m'entraînent)
Les fantômes de l'ennui (Hou hou hou)
Ils m'entraînent au bout de la nuit
\endverse

# 

\beginsong{Dernière danse}[by={Indila (2014)}]

\beginverse
Oh ma douce souffrance
Pourquoi s’acharner tu r’commences
Je ne suis qu’un être sans importance
Sans lui je suis un peu « paro »
Je déambule seule dans le métro
Une dernière danse
Pour oublier ma peine immense
Je veux m’enfuir
Que tout r’commence
Oh ma douce souffrance
\endverse

\beginverse
Refrain
Je remue le ciel, le jour, la nuit
Je danse avec le vent, la pluie
Un peu d’amour, un brin de miel
Et je danse, danse, danse, danse
Danse, danse, danse
Et dans le bruit, je cours et j’ai peur
Est-ce mon tour ?
Revient la douleur…
Dans tout Paris, je m’abandonne
Et je m’envole, vole, vole, vole, vole, vole, vole
\endverse

\beginverse
Que d’espérance…
Sur ce chemin en ton absence
J’ai beau trimer, sans toi ma vie
N’est qu’un décor qui brille, vide de sens
\endverse

\beginverse
Refrain
\endverse

\beginverse
Dans cette douce souffrance
Dont j’ai payé toutes les offenses
Écoute comme mon coeur est immense
Je suis une enfant du monde
\endverse

\beginverse
Refrain
\endverse

\beginsong{Dernière danse}[by={Kyo (2003)}]

\beginverse
J'ai longtemps parcouru son corps
Effleuré cent fois son visage
J'ai trouvé de l'or et même quelques étoiles
En essuyant ses larmes
J'ai appris par cœur la pureté de ses formes
Parfois, je les dessine encore
Elle fait partie de moi
\endverse

\beginverse
Refrain
Je veux juste une dernière danse
Avant l'ombre et l'indifférence
Un vertige puis le silence
Je veux juste une dernière danse
\endverse

\beginverse
Je l'ai connue trop tôt, mais c'est pas de ma faute
La flèche a traversé ma peau
C'est une douleur qui se garde
Qui fait plus de bien que de mal
Mais je connais l'histoire, il est déjà trop tard
Dans son regard, on peut apercevoir qu'elle se prépare
Au long voyage
\endverse

\beginverse
Refrain
\endverse

\beginverse
Ouh-ouh-ouh-ouh-ouh-ouh-ouh-ouh-ouh (une dernière danse)
Ouh-ouh-ouh-ouh-ouh-ouh-ouh-ouh-ouh (une dernière danse)
\endverse

\beginverse
Je peux mourir demain, ça ne change rien
J'ai reçu de ses mains
Le bonheur ancré dans mon âme
C'est même trop pour un seul homme
Je l'ai vu partir, sans rien dire
Il fallait seulement qu'elle respire
Merci, d'avoir enchanté ma vie
\endverse

\beginverse
Avant l'ombre et l'indifférence
Un vertige puis le silence
Je veux juste une dernière danse
\endverse

\beginverse
J'ai longtemps parcouru son corps
Effleuré cent fois son visage
J'ai trouvé de l'or et même quelques étoiles
En essuyant ses larmes
Et j'ai appris par cœur la pureté de ses formes
Parfois, je les dessine encore
Elle fait partie de moi
Hmm-hmm-hmm-hmm-hmm-hmm-hmm-hmm
Hmm-hmm-hmm-hmm-hmm-hmm-hmm-hmm
\endverse

\beginsong{Dès que le vent soufflera}[by={Renaud (1983)}]

\beginverse
C'est pas l'h\\[Mim]omm' qui prend la mer,
C'est la m\\[Ré]er qui prend l'h\\[Mi]omme, ta-ta-tin
Moi, la m\\[Mim]er elle m'a p\\[Ré]ris, j'me souv\\[Mim]iens, un mardi
J'ai t\\[Mim]roqué mes santiag' et mon c\\[Ré]uir un peu z\\[Mim]one,
Contre un' paire de Docks\\[Ré]ides et un vieux ciré j\\[Mim]aune.
J'ai déserté les crasses qui m'd\\[Ré]isaient : « Sois prud\\[Mim]ent,
La mer c'est dégueul\\[Ré]asse, les poissons baisent ded\\[Mi]ans. »
\endverse

\beginverse
Refrain
D\\[Mi]ès que le vent souffler\\[Ré]a, je repartir\\[Mim]a,
Dès que l\\[Sol]es vents toumer\\[Ré]ont, nous nous e\\[Si7]n all'r\\[Mim]ons
\endverse

\beginverse
C'est pas l'homm' qui prend la mer, 
C'est la mer qui prend l’homme,
Moi, la mer elle m'a pris, au dépourvu, tant pis \!
J'ai eu si mal au cœur sur la mer en furie
Qu'j'ai vomi mon quatr'heures, et mon minuit aussi,
Je m'suis cogné partout, j'ai dormi dans des draps mouillés
Ça m'a pas coûté des sous, c'est d'la plaisance, c'est l'pied \!
\endverse

	Refrain

\beginverse
C'est pas l'homm' qui prend la mer, C'est la mer qui prend l'homme,
Mais elle prend pas la femme qui préfère la campagne
La mienne m'attend au port au bout de la jetée
L'horizon est bien mort dans ses yeux délavés.
Assise sur une bitte d'amarrage elle pleure
Son homme qui la quitte, la mer, c'est son malheur.
\endverse

	Refrain

​​C'est pas l'homm' qui prend la mer,
C'est la mer qui prend l'homme,
Moi, la mer, elle m'a pris, comme on prend un taxi.
Je f’rai le tour du monde pour voir à chaque étape
Si tous les gars au monde veulent bien m'lâcher la grappe.
J'irai aux quatre vents foutre un peu le boxon
Jamais les océans n'oublieront mon prénom.

\beginverse
Refrain 
\endverse

\beginverse
C'est pas l'homm' qui prend la mer,
C'est la mer qui prend l’homme.
Moi, la mer, elle m'a pris, et mon bateau aussi.
Il est fier mon navire, il est beau mon bateau,
C'est un fameux trois-mâts fin comme un oiseau, hisse-ho
Mais Tabarly, Pajot, Kersauzon et Riguidel,
Naviguent pas sur des cageots, ni des poubelles.
\endverse

	Refrain

\beginverse
C'est pas l'homm' qui prend la mer.
C'est la mer qui prend l’homme,
Moi, la mer, elle m'a pris, j’me souviens, un vendredi.
Ne pleure plus ma mère, ton fils est matelot,
Ne pleure plus mon père, je vis au fil de l'eau.
Regardez votre enfant, il est parti marin
Je sais c'est pas marrant, mais c'était mon destin.
\endverse

\beginsong{Le déserteur}[by={Boris Vian (1954)}]

\beginverse
Monsieur le président,
Je vous fais une lettre
Que vous lirez peut-être,
Si vous avez le temps
Je viens de recevoir
Mes papiers militaires
Pour partir à la guerre
Avant mercredi soir
Monsieur le président,
Je ne veux pas la faire
Je ne suis pas sur terre
Pour tuer des pauvres gens
C'est pas pour vous fâcher
Il faut que je vous dise
Ma décision est prise,
Je m'en vais déserter
\endverse

\beginverse
Depuis que je suis né,
J'ai vu mourir mon père
J'ai vu partir mes frères
Et pleurer mes enfants
Ma mère a tant souffert,
Qu'elle est dedans sa tombe
Et se moque des bombes,
Et se moque des vers
\endverse

\beginverse
Quand j'étais prisonnier
On m'a volé ma femme
On m'a volé mon âme
Et tout mon cher passé
Demain de bon matin
Je fermerai ma porte
Au nez des années mortes
J'irai sur les chemins
\endverse

\beginverse
Je mendierai ma vie
Sur les routes de France
De Bretagne en Provence
Et je crierai aux gens:
« Refusez d'obéir,
Refuser d e la faire
N'allez pas à la guerre
Refusez de partir \! »
S'il faut donner son sang
Allez donner le votre \!
Vous êtes bon apôtre
Monsieur le président
Si vous me poursuivez
Prévenez vos gendarmes
Que je n'aurai pas d'armes
Et qu'ils pourront tirer
\endverse

\beginsong{Destination ailleurs}[by={Yannick Noah (2006)}]

\beginverse
On laisse nos chaussures au placard
Et on prend la guitare
Un CD de Marley
On laisse les enfants aux parents
On prendra tout notre temps
Je te garde pour moi
\endverse

\beginverse
Refrain
De Paris ou d'ailleurs
Si tu me suis
On prendra le meilleur
Tout est permis
Et si ça nous va bien
On ira encore plus loin
Destination ailleurs
Destination ailleurs
\endverse

\beginverse
On peut fermer un peu la porte
Oublier un peu les autres
La voiture et la ville
On pourrait couper la télé
Nos deux portables et s'en aller
Toi et moi pour une fois
\endverse

\beginverse
Refrain
\endverse

\beginverse
On peut faire une pause dans nos vies
C'est peut-être pas mal aussi
\endverse

\beginverse
Refrain
\endverse

\beginverse
Si tu veux bien
Un mois, un jour, une heure
On en sait rien
On en sait rien
Alors, suis-moi
\endverse

\beginverse
De Paris ou d'ailleurs
Si tu me suis
On prendra le meilleur
Tout est permis
Tout est permis
Destination ailleurs
\endverse

\beginsong{Le dîner}[by={Bénabar (2005)}]

\beginverse
J\\[Lam]'veux pas y aller, à ce dîner
J'ai pas l'moral, j'suis fati\\[Fa]gué
Ils nous en voudront pa\\[Mi]s
Allez on n'y va pa\\[Lam]s
\endverse

\beginverse
En plus faut qu'j'fasse un régime
Ma chemise me boudi\\[Fa]ne
J'ai l'air d'une chipola\\[Mi]ta
Je peux pas sortir comme ç\\[Lam]a
\endverse

\beginverse
Ç\\[LA]a n'a rien à vo\\[Rém]ir
J'le\\[Si]s aime bi\\[Mim]en, tes amis
Ma\\[Rém]is je veux pas les vo\\[Do]ir
P\\[Sol]arce que j'ai pas envie
\endverse

\beginverse
Refrain
On s'en fo\\[Do]ut, on n'y va pas
On n'a qu'à s'cac\\[Mim]her sous les draps
On com\\[Lam]mandera des pizzas
Toi, la tél\\[Sol]é et moi
\endverse

\beginverse
On app\\[Do]elle, on s'excuse
On impro\\[Mim]vise, on trouve quelqu'chose
On n'a qu'à di\\[Do]re à tes am\\[Mi]is
Qu'on les aime pa\\[Lam]s et puis tant pis
\endverse

\beginverse
J'suis pas d'humeur, tout me déprime
Et il se trouve que par hasard
Y a un super bon film
À la télé ce soir
\endverse

\beginverse
Un chef-d'oeuvre du septième art
Que je voudrais revoir
Un drame très engagé
Sur la police de Saint-Tropez
\endverse

\beginverse
C'est une satire sociale
Dont le personnage central
Est joué par De Funès
En plus y a des extraterrestres
\endverse

\beginverse
Refrain
\endverse

\beginverse
On appelle, on s'excuse
On improvise, on trouve quelqu'chose
On n'a qu'à dire à tes amis
Qu'on les aime pas et puis tant pis
\endverse

\beginverse
Refrain
\endverse

\beginverse
J'ai des frissons, je me sens faible
Je crois qu'je suis souffrant
Ce serait pas raisonnable
De sortir maintenant
\endverse

\beginverse
Je préfère pas prend' de risque
C'est peut-être contagieux
Il vaut mieux que je reste
Ça m'ennuie mais c'est mieux
\endverse

\beginverse
Tu me traites d'égoïste
Comment oses-tu dire ça?
Moi qui suis malheureux et triste
Et j'ai même pas de home-cinéma
\endverse

\beginverse
Refrain
\endverse

\beginverse
On appelle, on s'excuse
On improvise, on trouve quelqu'chose
On n'a qu'à dire à tes amis
Qu'on les aime pas et puis tant pis
\endverse

\beginverse
Refrain
\endverse

\beginsong{Dis-moi}[by={BB Brunes (2007)}]

\beginverse
Une légère envie de violence quand elle relace ses bas
Je ne suis plus à vendre, Houna je ne suis plus comme ça
Des rumeurs adolescentes disent que je ne suis pas
À toi et je pense qu'une part de vrai se cache
\endverse

\beginverse
Refrain
\endverse

\beginverse
Dis-moi si je dois partir ou pas
Dis-moi hou hou
Dis-moi si tu aimes ça Houna
Car je suis fou de toi Houna
Quand tu ne m'appartiens pas
\endverse

\beginverse
Une violente envie de descente lorsque t'embrasse ces gars
Je ne ferai point l'enfant tout ça ne m'atteint pas
Des rumeurs adolescentes disent que je ne suis pas
Un homme à femmes et rien d'autre qu'un homme à toi
\endverse

\beginverse
Refrain
\endverse

\beginverse
Quand tu me mords où ça dérange
Et tu m'attaches les bras
Quand je fais sautiller sa frange
Ses cris se tirent dans les graves
\endverse

\beginverse
Les voyeurs en redemandent
Moi je ne veux que Houna
La plus belle des plus belles jambes
Et de la place pour trois
\endverse

\beginverse
Dis-moi si je dois partir ou pas
Dis-moi hou hou
Dis-moi si tu aimes ça Houna
Dis-moi hou hou
Dis-moi non je ne craquerai pas
Dis-moi hou hou
Dis-moi si tu aimes ça Houna
Car je suis fou de toi Houna
Quand tu ne m'appartiens pas
\endverse

\beginsong{Dommage}[by={Bigflo & Oli (2017)}]

\beginverse
Intro : Lam Rém Sol7 Lam
\endverse

\beginverse
Louis prend son b\\[Lam]us, comme tous les mat\\[Rém]ins
Il croisa cette même f\\[Sol7]ille, avec son doux parf\\[Lam]um
Qu'elle vienne lui parl\\[Lam]er, il espère tous les j\\[Rém]ours
Ce qu'il ressent au fond d\\[Sol7]'lui, c'est ce qu'on appelle l'am\\[Lam]our
Mais Louis, il est tim\\[Lam]ide et elle, elle est si b\\[Rém]elle
Il ne veut pas y all\\[Sol7]er, il est collé au fond d'son s\\[Lam]iège
Une fois elle lui a sour\\[Lam]i quand elle est descend\\[Rém]ue
Et depuis ce jour l\\[Sol7]à, il ne l'a jamais rev\\[Lam]ue
\endverse

\beginverse
Refrain
Ah il aurait dû y a\\[Lam]ller, il aurait dû le f\\[Rém]aire, crois-m\\[Lam]oi
On a tous dit : "Ah c'\\[Lam]est dommage, ah c'\\[Rém]est dommage, c'\\[Sol7]est p't'être la dernière f\\[Lam]ois"
\endverse

\beginverse
Yasmine a une belle voix, elle sait qu'elle est douée
Dans la tempête de sa vie, la musique est sa bouée
Face à ses mélodies, le monde est à ses pieds
Mais son père lui répétait: "Trouve-toi un vrai métier"
Parfois elle s'imagine sous la lumière des projecteurs
Sur la scène à recevoir les compliments et les jets de fleurs
Mais Yasmine est rouillée, coincée dans la routine
Ça lui arrive de chanter quand elle travaille à l'usine
\endverse

\beginverse
Refrain
\endverse

\beginverse
Ah elle aurait dû y aller, elle aurait dû le faire, crois-moi
On a tous dit : "Ah c'est dommage, ah c'est dommage, c'est p't'être la dernière fois"
\\[bis]
\endverse

\beginsong{Don’t worry, be happy}[by={Bobby McFerrin (1988), }]

\beginverse
Intro: Sol Lam Do Sol 
\endverse

\beginverse
H\\[Sol]ere's a little song I wrote 
You mig\\[Lam]ht want to sing it note for note 
Don't wor\\[Do]ry, be hap\\[Sol]py
I\\[Sol]n every life we have some trouble
But wh\\[lam]en you worry you make it double 
Don't wo\\[Do]rry, be hap\\[Sol]py
\endverse

\beginverse
Refrain
O\\[Sol]oh, ooh ooh ooh-ooooh ooh oo-ooh oo-ooh oo\\[Lam]-oooh
Ooh, ooh ooh ooh-ooooh ooh oo-ooh oo-ooh oo\\[Do]-oooh
Ooh, ooh ooh ooh-ooooh ooh oo-ooh oo-ooh oo\\[Sol]-oooh
\endverse

\beginverse
Ain’t got no place to lay your head
Somebody came and took your bed
Don’t worry, be happy
The landlord says your rent is late
He may have to litigate
Don’t worry, be happy
\endverse

\beginverse
Refrain
\endverse

\beginverse
Ain't got no cash, ain't got no style
Ain't got no one to make you smile 
Don't worry, be happy Cos when you worry, your face will frown
And that will bring everybody down 
So don't worry, be happy 
\endverse

\beginverse
Refrain
\endverse

\beginverse
Now there, is this song I wrote
I hope you learned note for note
Like good little children, don’t worry, be happy
Now listen to what I said, in your life expect some trouble
When you worry you make it double
\endverse

\beginverse
Don't worry, be happy 
Don't worry, be happy
\endverse

\beginsong{Donnez-moi}[by={Les Frangines (2019)}]

\beginverse
J'aurais beau parler les langues du monde
J'aurais beau être un gagnant
J'aurais beau n'être pas des gens de l'ombre
J'aurais beau être puissant
\endverse

\beginverse
Refrain
Donnez-moi l'automne
Donnez-moi moi du temps
Donnez-moi de l'été
Donnez-moi de l'art
Donnez du printemps
Donnez de la beauté
Donnez-moi de l'or
Donnez de l'argent
Donnez-moi un voilier
Oh oh hé
Si je m'aime pas
Si je t'aime pas
Ça sert à quoi
À quoi bon les honneurs et la gloire
Si je m'aime pas
Si je t'aime pas
Ça rime à quoi
Sans amour nos vies sont dérisoires
\endverse

\beginverse
J'aurais beau plaire et conquérir la terre
J'aurais beau être un Don Juan
J'aurais beau faire la plus belle carrière
J'aurais beau être important
\endverse

\beginverse
Refrain
\endverse

\beginverse
Aimer c'est recevoir
Et savoir tout donner
C'est s'oublier et voir
Ce qu'on a oublié
Donnez-moi l'automne
Donnez-moi moi du temps
Donnez-moi de l'été
Donnez-moi de l'art
Donnez du printemps
Donnez de la beauté
Donnez-moi de l'or
Donnez de l'argent
Donnez-moi un voilier
Oh oh hé
Si je m'aime pas
Si je t'aime pas
Ça sert à quoi
À quoi bon les honneurs et la gloire
Si je m'aime pas
Si je t'aime pas
Ça rime à quoi
Sans amour nos vies sont dérisoires
Ça sert à quoi
Si je m'aime pas
Si je t'aime pas
Ça rime à quoi
\endverse

\beginsong{Éducation sentimentale}[by={Maxime Le Forestier (1974)}]

\beginverse
C\\[Mi]e soir à la brume n\\[Lam]ous irons ma brune
C\\[Rém]ueillir des serm\\[Sol7]ents
C\\[Do]ette fleur sauvage qui f\\[Lam]ait des ravages
D\\[Rém]ans les c\\[Sol7]œurs d'enf\\[Do]ants
P\\[Fa]our toi ma princesse j\\[Sol]'en ferai des tresses
E\\[Sol7]t dans tes chev\\[Do]eux
C\\[Mi7]es serments, ma belle, t\\[Lam]e rendront cruelle
P\\[Rém]our tes am\\[Sol7]our\\[Do]eux.
\endverse

\beginverse
Demain à l'aurore nous irons encore
Glaner dans les champs
Cueillir des promesses, des fleurs de tendresse
Et de sentiments
Et sur la colline, dans les sauvagines
Tu te coucheras
Dans mes bras ma brune, éclairée de lune,
Tu te donneras.
\endverse

\beginverse
C'est au crépuscule, quand la libellule
S'endort au marais
Qu'il faudra, voisine, quitter la colline
Et vite rentrer
Ne dis rien ma brune, pas même à la lune
Et moi, dans mon coin
J'irai solitaire, je saurai me taire,
Je ne dirai rien.
\endverse

\beginverse
Ce soir à la brume nous irons ma brune
Cueillir des serments
Cette fleur sauvage qui fait des ravages
Dans les cœurs d'enfants
Pour toi ma princesse j'en ferai des tresses
Et dans tes cheveux
Ces serments, ma belle, te rendront cruelle
Pour tes amoureux.
\endverse

\beginsong{L’effet papillon}[by={Bénabar (2008)}]

\beginverse
Si le b\\[Do]attement d'ailes d'un papillon qu\\[Lam]elque part au Cambodge
Décl\\[Do]enche, sur un autre continent, le plus v\\[Lam]iolent des orages
Le c\\[Fa]hoix de quelques-uns dans un bur\\[Lam]eau occidental
Bouleverse des m\\[Fa]illions de destins, surtout si le b\\[Sol]ureau est ovale
\endverse

\beginverse
Il n'y a que l'ours blanc qui s'étonne que sa banquise fonde
Ça ne surprend plus personne, de notre côté du monde
Quand le financier s'enrhume, ce sont les ouvriers qui toussent
C'est très loin la couche d'ozone, mais c'est d'ici qu'on la perce
\endverse

\beginverse
Refrain
C\\[Do]'est l'effet papill\\[Lam]on : petite c\\[Fa]aus\\[Sol]e, grande conséquenc\\[Do]e
P\\[Do]ourtant jolie comme express\\[Lam]ion : petite c\\[Fa]hos\\[Sol]e, dégât immens\\[Do]e
\endverse

\beginverse
Qu'on l'appelle "retour de flamme" ou "théorie des dominos"
"Un murmure devient vacarme" comme dit le proverbe à propos :
"Si au soleil tu t'endors, de Biafine tu t'enduiras
Si tu mets une claque au videur, courir très vite tu devras"
\endverse

\beginverse
Si on se gave au resto, c'est un fait, nous grossirons
Mais ça c'est l'effet cachalot, revenons à nos moutons
... à nos papillons
Un hôtel un après-midi "aventure extra-conjugale"
Puis, le coup de boule de son mari, alors si ton nez te fait mal
\endverse

\beginverse
Refrain
\endverse

\beginverse
Avec les baleines on fabrique du rouge à lèvres, des crèmes pour filles
Quand on achète ces cosmétiques, c'est au harpon qu'on se maquille
Si tu fais la tournée des bars, demain, tu sais qu't'auras du mal
Pour récupérer, à huit heures, ton permis au tribunal
\endverse

\beginverse
Refrain
\endverse

\beginverse
Le papill\\[Do]on s'envole, le papill\\[Lam]on s'envole
T\\[Fa]out b\\[Sol]at de l'ail\\[Do]e
\\[bis]
\endverse

## 

## 

\beginsong{Elle m'a dit}[by={Cali (2003)}]

\beginverse
Je crois que je ne t'aime plus.
Elle m'a dit ça hier,
Ça a claqué dans l'air
Comme un coup de revolver.
Je crois que je ne t'aime plus.
Elle a jeté ça hier,
Entre le fromage et le dessert
Comme mon cadavre à la mer.
\endverse

\beginverse
Je crois que je ne t'aime plus.
Ta peau est du papier de verre
Sous mes doigts... sous mes doigts.
Je te regarde et je pleure
Juste pour rien... comme ça.
\endverse

\beginverse
Sans raison je pleure,
A gros bouillons je pleure,
Comme devant un oignon je pleure, arrêtons là...
\endverse

\beginverse
Refrain
Elle m'a dit
Elle m'a dit
\endverse

\beginverse
Je crois que je ne t'aime plus.
Relève-toi, relève-toi.
Ne te mouche pas dans ma robe,
Pas cette fois... relève-toi.
Tu n'as plus d'odeur,
Tes lèvres sont le marbre
De la tombe de notre amour,
Elle m'a dit ça son sang était froid.
Quand je fais l'amour avec toi
Je pense à lui.
Quand je fais l'amour avec lui
Je ne pense plus à toi
\endverse

\beginverse
Refrain
\endverse

\beginverse
Je crois que je ne t'aime plus.
Elle m'a dit ça hier,
Ça a pété dans l'air
Comme un vieux coup de tonnerre.
Je crois que je ne t'aime plus.
Je te regarde, je ne vois rien.
Tes pas ne laissent plus de traces
A côté des miens.
Je ne t'en veux pas,
Je ne t'en veux plus,
Je n'ai juste plus d'incendie
Au fond du ventre c'est comme ça
\endverse

\beginverse
Refrain \\[bis]
\endverse

\beginverse
Alors j'ai éteint la télé
Mais je n'ai pas trouvé le courage,
Par la fenêtre de me jeter :
Mourir d'amour n'est plus de mon âge...
\endverse

\beginverse
Refrain
\endverse

\beginsong{Emmène-moi}[by={Boulevard des airs (2015)}]

\beginverse
J\\[Lam]’suis comme u\\[Sol]n grain de sable
P\\[Lam]erdu dans l’\\[Sol]océan
J\\[Lam]’ai perdu m\\[Sol]on cartable
J\\[Do]’ai perdu mes p\\[Sol]arents
\endverse

\beginverse
J\\[Lam]’suis comme l’eau d\\[Sol]es courants
F\\[lam]atigué d\\[Sol]’ignorer
S\\[Lam]i je coule d\\[Sol]ans le vent
S\\[Do]i je fais que pas\\[Sol]ser
\endverse

\beginverse
Refrain
E\\[Rém]mmène-moi vo\\[Lam]ir la mer
F\\[Mi]ais-moi boire l’océ\\[Lam]an
E\\[Rém]mmène-moi d\\[Lam]ans les airs
A\\[Mi]ime-moi dans le ve\\[Lam]nt
\\[bis]
\endverse

\beginverse
J’suis comme une poussière
Si je m’envole un matin
Je retourne à la terre
Je m’en vais et je viens
\endverse

\beginverse
J’suis comme l’eau des fontaines
Impuissant et lassé
Poussé par ce système
Qui poursuit sans cesser
\endverse

\beginverse
Refrain
\endverse

\beginverse
J’suis comme les autres en fait
Je ne saurai jamais
Si je poursuis la quête
Si j’ai laissé tomber
J’suis comme rempli d’espoir
Ce matin je renais
Emmène-moi près du phare
Allons jusqu’aux rochers
\endverse

\beginverse
Refrain
\endverse

\beginsong{En chemin \- je m’en vais}[by={Phill collins \- Frère des ours (2003)}]

\beginverse
Dites à mes amis que je m'en vais
Je pars vers de nouveaux pays
Où le ciel est tout bleu, dites que je m'en vais
Et c'est tout ce qui compte dans ma vie
\endverse

\beginverse
Dites à mes amis que je m'en vais
Et j'aime chacun des pas que je fais
Le soleil est mon guide et moi je m'en vais
Je ne peux m'empêcher de sourire
\endverse

\beginverse
Car il n'y a rien de mieux que de se revoir
Peu importe ce qui nous sépare
Vous ne pouvez que sourire de nos histoires
Oh ça me fait chaud au cœur
\endverse

\beginverse
Alors dites leurs que je m'en vais
Je pars vers de nouveaux amis
Et dormir sous les étoiles c'est mon idéal
Sous la lune qui protège mes nuits
\endverse

\beginverse
Ni la neige ni la pluie changeront ma vie
Le soleil se lèvera c'est écrit
Le vent qui frotte mon visage réchauffe mon cœur
Je pars vers un avenir meilleur
\endverse

\beginverse
C'est vrai, je m'en vais
Et je souris
C'est vrai je m'en vais \\[x3]
\endverse

\beginsong{Ensemble on est mieux}[by={Les scouts ASBL (2021)}]

\beginverse
Intro: Mim Lam Do Si7
\endverse

\beginverse
J'm'en souviens, j'\\[Mim]avais six ans
J'comprenais p\\[Lam]as c'que voulaient mes parents
Quand ils m'ont d\\[Do]it “mon enfant
Va chez les s\\[Si7]couts, c'est important"
Moi, j'flippais g\\[Mim]rave au début
J'sortais d'chez m\\[Lam]oi, d'mes habitudes
Mais la rib\\[Do]ambelle m'ouvre la porte
Et au f\\[Si7]inal le rire m'emporte
\endverse

\beginverse
Refrain
E\\[Mim]nsemble on est mieux
On a du m\\[Do]al à s'dire adieu
Les scouts nous p\\[Sol]ortent, nous transportent
Nous font d\\[Si7]anser comme le feu
\\[bis]
\endverse

\beginverse
Puis j'me rappelle de mes huit ans
Chez les Louveteaux tout était différent
Je découvrais la vie au grand air
Les jeux dans l'bois, la vie sans manières
On court partout, on fait les fous
On rit, on chante, comme des loups
Avec la meute, j'deviens plus grande
La force du loup, c'est le clan
\endverse

\beginverse
Refrain
\\[bis]
\endverse

\beginverse
Arrivé chez les Éclaireurs
Plus personne ne me faisait peur
Quand j'suis en bande avec mes potes
J'me sens plus fort même pour la tot'
On vit ensemble et en patrouille
On devient les rois d'la débrouille
On construit même nos pilotis
À bout de bras et d'énergie
Ensemble on est mieux
On a du mal à s'dire adieu
Les scouts nous portent, nous transportent
Nous font danser comme le feu
\endverse

\beginverse
Refrain
\\[bis]
\endverse

\beginverse
Puis viennent les pi's et leurs envies
D'aventure et de fantaisie
Pour ça on rêve toute l'année
On veut qu'une chose, c'est tout changer
Durant le camp, on s'ouvre à tout
Une autre culture, un autre chez nous
On vient, on aide, si on peut
On échange même avec des vieux
Ensemble on est mieux
On a du mal à s'dire adieu
Les scouts nous portent, nous transportent
Nous font danser comme le feu
\endverse

\beginverse
Refrain
\\[bis]
\endverse

\beginverse
Enfin, tu es animateur
Du temps, du talent et du cœur
Et tu t'engages bénévolement
À transmettre tout c'que t'as dans l'sang
Parfois, c'est vrai, c'est la galère
Mais c'est pas grave, en staff tu gères
Baden-Powell est fier de toi
Chante avec moi et lève trois doigts
\endverse

\beginverse
Refrain
\\[bis]
\endverse

\beginsong{Eté 90}[by={Thérapie Taxi (2021)}]

\beginverse
On a dévalé la pente en moins de deux
On a fait comme si on savait pas
On a évité les regards ambigus
On a fait comme si on pouvait pas
On a dessiné la zone, évité les roses
Repoussé la faune, compliqué les choses
\endverse

\beginverse
Mais maudit ami je veux plus
Danser ce slow avec toi
Souviens-toi des années 90
Quand dans la cour, tous les jours, j’étais ton roi
Tu as bien grandi et tu me brusques
Et parfois même tu te loves dans mes bras
\endverse

\beginverse
Mais jamais jamais jamais plus
Car je le sais, je suis l’homme qu’on ne voit pas
\endverse

\beginverse
Et si le soleil se lèvе sur les autres
Je sais quе c’est moi qui ai chassé les roses
À mon amour, tu le sais, c’est ma faute
J’ai bien trop peur pour casser les choses
On va s’en tenir simplement à nos rôles
(Simplement à nos rôles, simplement à nos rôles)
Je sais que c’est triste mais je suis sous hypnose
(Je suis sous hypnose, je suis sous hypnose)
\endverse

\beginverse
Tu as décidé des règles en fin de jeu
J’étais teenager, amoureuse
Puis le temps s’est écoulé en moins de deux
Finies les années délicieuses
\endverse

\beginverse
Mais maudit ami je veux plus
Danser ce slow avec toi
Souviens-toi des années 90
Quand dans la cour, tous les jours, t’étais mon roi
\endverse

\beginverse
Et si le soleil se lève sur les autres
Je sais que c’est moi qui ai chassé les roses
À mon amour, tu le sais, c’est ma faute
J’ai bien trop peur pour casser les choses
On va s’en tenir simplement à nos rôles
(Simplement à nos rôles, simplement à nos rôles)
Je sais que c’est triste mais je suis sous hypnose
(Que tu t’approches et que j’appuie sur)
Pause \\[pause]
\endverse

\beginverse
Et seul tous les soirs (Et seul tous les soirs)
Et seul tous les soirs (Et seul tous les soirs)
Et seul tous les soirs, je reste dans le noir \\[4x]
\endverse

(Si le soleil se lève sur les autres)
(Je sais que c’est moi qui ai chassé les roses)
(À mon amour, tu le sais, c’est ma faute)
(J’ai bien trop peur pour casser les choses)
Et si le soleil se lève sur les autres
Je sais que c’est moi qui ai chassé les roses
À mon amour, tu le sais, c’est ma faute
J’ai bien trop peur pour casser les choses
On va s’en tenir simplement à nos rôles
(Je serais là pour toujours, ton épaule)
Je sais que c’est triste mais je suis sous hypnose
Que tu t’approches et que j’appuie sur pause

\beginsong{Etoiles des neiges}[by={Line Renaud (1949)}]

\beginverse
Dans un coin perdu des montagnes
Un tout petit savoyard
Chantait son amour dans le calme du soir
Près de sa bergère au doux regard…
\endverse

\beginverse
Refrain
Etoile des Neiges, mon cœur amoureux
S'est pris au piège, de tes grands yeux
Je te donne en gage cette croix d'argent
Et de l'aimer toute ma vie, je fais serment
\endverse

\beginverse
Hélas \! ... soupirait la bergère
Que répondront nos parents ?
Comment ferons-nous, nous n'avons plus d'argent
Pour nous marier dès le printemps ?
\endverse

\beginverse
Refrain
Etoile des Neiges, sèche tes beaux yeux
Le ciel protège les amoureux
Je pars en voyage pour qu'à mon retour
A tout jamais plus rien n'empêche notre amour.
\endverse

\beginverse
Alors il partit vers la ville
Et ramoneur il se fit :
Sur les chemins dans le vent et la pluie
Comme un petit diable noir de suie.
\endverse

\beginverse
Refrain
Etoile des neiges, sèche tes beaux yeux
Le ciel protège les amoureux
Ne perds pas courage, il te reviendra
Et tu seras bientôt encore entre ses bras.
\endverse

\beginverse
Et quand les beaux jours refleurirent
Il s'en revint au hameau
Et sa fiancée l'attendait tout là-haut
Parmi les clochettes des troupeaux
\endverse

\beginverse
Refrain
Etoile des Neiges, fus garçons d'honneur
Vont en cortège, portant des fleurs
Par un mariage, fit mon histoire
De la bergère et de son petit savoyard.
\endverse

\beginsong{Les étoiles filantes}[by={Les Cowboys fringants (2004)}]

\beginverse
Si je m\\[lam]'arrête un instant
Pour te par\\[Sol]ler de ma vie
Juste comme ça\\[Mim] tranquillement
Dans un b\\[Fa]ar, rue St-Denis
\endverse

\beginverse
J'te raconterai les souvenirs
Bien gravés dans ma mémoire
De cette époque où vieillir
Était encore bien illusoire
\endverse

\beginverse
Quand j'agaçais les p'tites filles
Pas loin des balançoires
Et que mon sac de billes
Devenait un vrai trésor
\endverse

\beginverse
Et ces hivers enneigés
À construire des igloos
Et rentrer les pieds g'lés
Juste à temps pour Passe-Partout
\endverse

\beginverse
Refrain
Mais au bo\\[Lam]ut du ch'min, dis-moi c'qui va rest\\[sol]er
De la p'\\[Mim]tite école et d'la cour de r\\[Fa]écré
Quand les avi\\[Lam]ons en papier ne partent plus au ve\\[Sol]nt
On se di\\[Mim]t que l'bon temps passe finalem\\[Fa]ent
Comme une étoile fila\\[lam]nte
\endverse

\beginverse
Pont: Lam Sol Mim Fa
\endverse

\beginverse
Si je m'arrête un instant
Pour te parler de la vie
Je constate que, bien souvent
On choisit pas, mais on subit
\endverse

\beginverse
Et que les rêves des ti-culs
S'évanouissent ou se refoulent
Dans cette réalité crue
Qui nous embarque dans le moule
\endverse

\beginverse
La trentaine, la bedaine
Les morveux, l'hypothèque
Les bonheurs et les peines
Les bons coups et les échecs
\endverse

\beginverse
Travailler, faire d'son mieux
N'arracher, s'en sortir
Et espérer être heureux
Un peu avant de mourir
\endverse

\beginverse
Refrain
\endverse

\beginverse
Si je m'arrête un instant
Pour te parler de la vie
Juste comme ça tranquillement
Pas loin du carré Saint-Louis
\endverse

\beginverse
C'est qu'avec toi je suis bien
Et que j'ai pu' l'goût de m'en faire
Parce que tsé, voir trop loin
C'pas mieux que r'garder en arrière
\endverse

\beginverse
Malgré les vieilles amertumes
Et les amours qui passent
Les chums qu'on perd dans' brume
Et les idéaux qui se cassent
\endverse

\beginverse
La vie s'accroche et renaît
Comme les printemps reviennent
Dans une bouffée d'air frais
Qui apaise les coeurs en peine
\endverse

\beginverse
Ça fait que si à' soir t'as envie de rester
Avec moi, la nuit est douce, on peut marcher
Et même si on sait ben que tout' dure rien qu'un temps
J'aimerais ça que tu sois pour un moment
Mon étoile filante
\endverse

\beginverse
Mais au bout du ch'min, dis-moi c'qui va rester \\[bis]
Que des étoiles filantes
\endverse

\beginsong{Être un homme comme vous}[by={José Bartel (1967), Ben l'oncle soul (2016) \- Le Livre de la jungle }]

\beginverse
Je suis le roi de la danse,oh
La jungle est à mes pieds,
De la puissance j'suis au plus haut
Et pourtant j'dois vous envier
\endverse

\beginverse
Je voudrais devenir un homme,
Ce serait merveilleux
Vivre pareil aux autres hommes
Loin des singes ennuyeux
\endverse

\beginverse
Oh, woupidou \!
J'voudrais voudrais marcher comme vous,
Et parler comme vous,
Faire comme vous, tout \!
Un singe comme moi
Pourrait je crois
Être parfois
Bien plus humain que vous \!
\endverse

\\[Scat]

\beginverse
Pourtant crois moi bien j'suis pas dupe.
Si je marchande avec vous
c'est que j'désire le moyen d'être un homme
Un point c'est tout.
Dis moi le secret pour être un homme,
Est-ce vraiment si mystérieux ?
Pour moi faire éclore la grande fleur rouge ce serait merveilleux.
\endverse

\\[Scat]

\beginverse
J'voudrais marcher comme vous,
Et parler comme vous,
Faire comme vous, tout \!
Car je l'avoue,
Quelqu'un comme moi,
C'est vrai je crois peut devenir comme vous \!
C'est vrai je crois peut devenir comme moi \! \\[bis]
\endverse

\beginsong{Faramaz}[by={Rosana Spadavecchia }]

\beginverse
Refrain
V\\[Lam]a, cette fête est termin\\[Rém]ée
Mais la vraie f\\[Sol7]ête comm\\[Do]ence
V\\[Lam]a, prends la route et va sem\\[Mi]er
Les fleurs de l'amit\\[Lam]ié
\endverse

\beginverse
I\\[Mi]l est arriv\\[Lam]é
Le t\\[Sol]emps de partag\\[Do]er
Faram\\[Rém]az, c'est toi, c'est m\\[Do]oi
Nos p\\[Lam]eines e\\[Rém]t nos j\\[Mi]oies
\endverse

\beginverse
Refrain
\endverse

\beginverse
Il faut repartir
Et bâtir l'avenir
Faramaz, c'est un départ
C'est un début d'espoir
\endverse

\beginverse
Refrain
\endverse

\beginverse
Marche d'un bon pas
Ne te retourne pas
Faramaz, c'est le chemin
Du scoutisme de demain
\endverse

\beginsong{Femme libérée}[by={Cookie Dinger (1984)}][by={Capo X}]

\beginverse
E\\[Lam]lle est abo\\[Fa]nnée à\\[Do] Marie-Cl\\[Sol]aire
D\\[LAm]ans le Nouvel O\\[Fa]bs, ell'ne lit q\\[Do]ue Brétéch\\[Sol]er
L\\[Lam]e Monde y a l\\[Fa]ongtemps qu'elle fa\\[Do]it plus sembl\\[Sol]ant
Elle ach\\[Lam]ète Match en ca\\[Fa]chette, c'est bi\\[Do]en plus marr\\[Sol7]ant.
\endverse

\beginverse
Refrain
Ne la laisse p\\[Lam]as tomb\\[Fa]er, elle\\[Do] est si fragile
Etre une femme libérée, tu sais, c'est pas si fac\\[Sol]ile
\\[bis]
\endverse

\beginverse
Au fond de son lit, un macho s'endort
Qui ne l'aimera pas plus loin que l'aurore
Mais, elle s'en fout, elle s'éclate quand même
Il lui ronronne des tonnes de «je t'aime".
\endverse

\beginverse
Refrain
\endverse

\beginverse
Sa première ride lui fait du souci
Le reflet du miroir pèse sur sa vie
Elle rentre son ventre à chaque fois qu'elle sort
Même dans «Elle» ils disent qu'il faut faire un effort.
\endverse

\beginverse
Refrain
\endverse

\beginverse
Elle fume beaucoup, elle a des avis sur tout
Elle aime raconter qu'elle sait changer une roue
Elle avoue son âge, celui d'ses enfants
Et goûte même un p'tit joint de temps en temps
\endverse

\beginsong{Le festin}[by={Camille (2007)}]

\beginverse
Les rêves des amoureux sont comme le bon vin
Ils donnent de la joie ou bien du chagrin
Affaibli par la faim je suis malheureux
Volant en chemin tout ce que je peux
Car rien n'est gratuit dans la vie
\endverse

\beginverse
Éspoire est un plât bien trop vitе consommé
À sauter les repas, je suis habitué
Un voleur, solitaire, est triste à nourrir
À nous je suis amer je veux réussir
Car rien n'est gratuit dans...
\endverse

\beginverse
La vie... Jamais on ne redira
Que la course aux étoiles, ça n'est pas pour moi
Laisser-moi vous émerveillez, prendre mon en vol
Nous allons enfin nous régaler
\endverse

\beginverse
La fête va enfin commencer
Et sortez les bouteilles, finis les ennuis
Je dresse la table, demain nouvelle vie
Je suis heureux à l'idée de ce nouveau destin
Une vie à me cacher, et puis libre enfin
Le festin est sur mon chemin
\endverse

\beginverse
Une vie à me cacher et puis libre enfin
Le festin est sur mon chemin
\endverse

\beginsong{La fleur au chapeau}[by={Traditionnel}]

\beginverse
Vous qui nous regardez passer
Sous le soleil ou sous l'orage
Peut-être bien que vous pensez
Que nous avons bien du courage
Pour ainsi nous harasser
A courir le long des routes
Vous ne savez ce que c'est
Et vous n'aurez jamais sans doute
\endverse

\beginverse
Refrain
Une fleur au chapeau
A la bouche une chanson
Un cœur joyeux et sincère
Et c'est tout ce qu'il faut
A nous autres gais lurons
Pour aller au bout de la terre
\endverse

\beginverse
Ah \! Comme nous serions heureux
Si nous pouvions la vie entière
Courir par les chemins ombreux
Ou sur les routes familières
Depuis les sommets neigeux
Jusqu'au bord des mers profondes
A travers nos cris joyeux
Nous dirons au vaste monde
\endverse

\beginverse
Refrain
\endverse

\beginverse
Hélas \! il n'en est pas ainsi
Et notre tâche est plus aride
Mais il nous faut du cœur aussi
Il faut aussi des bras solides
Pour combattre sans merci
La laideur et la paresse
A travers lutte et soucis
Il nous faut garder sans cesse
\endverse

\beginsong{Forever young}[by={Alphaville (1984)}]

\beginverse
Let's dance in style, let's dance for a while
Heaven can wait, we're only watching the skies
Hoping for the best, but expecting the worst
Are you gonna drop the bomb or not?
\endverse

\beginverse
Let us die young or let us live forever
We don't have the power, but we never say never
Sitting in a sandpit, life is a short trip
The music's for the sad men
\endverse

\beginverse
Can you imagine when this race is won?
Turn our golden faces into the sun
Praising our leaders, we're getting in tune
The music's played by the, the madmen
\endverse

\beginverse
Refrain
\endverse

\beginverse
Forever young
I want to be forever young
Do you really want to live forever?
Forever and ever
Forever young
I want to be forever young
Do you really want to live forever?
Forever young
\endverse

\beginverse
Some are like water, some are like the heat
Some are a melody and some are the beat
Sooner or later, they all will be gone
Why don't they stay young?
\endverse

\beginverse
It's so hard to get old without a cause
I don't want to perish like a fading horse
Youth's like diamonds in the sun
And diamonds are forever
\endverse

\beginverse
So many adventures couldn't happen today
So many songs we forgot to play
So many dreams swinging out of the blue
We'll let them come true
\endverse

\beginverse
Forever young
I want to be forever young
Do you really want to live forever?
Forever and ever
\endverse

\beginverse
Refrain
\endverse

\beginsong{Foule sentimentale}[by={Alain Souchon (1993), Yael Naim (2018)}]

\beginverse
Oh la la la vie en rose
Le rose qu'on nous propose
D'avoir les quantités d'choses
Qui donnent envie d'autre chose
Aïe, on nous fait croire
Que le bonheur c'est d'avoir
De l'avoir plein nos armoires
Dérisions de nous dérisoires car
\endverse

\beginverse
Refrain
Foule sentimentale
On a soif d'idéal
Attirée par les étoiles, les voiles
Que des choses pas commerciales
Foule sentimentale
Il faut voir comme on nous parle
Comme on nous parle
\endverse

\beginverse
Il se dégage
De ces cartons d'emballage
Des gens lavés, hors d'usage
Et tristes et sans aucun avantage
On nous inflige
Des désirs qui nous affligent
On nous prend faut pas déconner dès qu'on est né
Pour des cons alors qu'on est
\endverse

\beginverse
Refrain
\endverse

\beginverse
On nous Claudia Schieffer
On nous Paul-Loup Sulitzer
Oh le mal qu'on peut nous faire
Et qui ravagea la moukère
Du ciel dévale
Un désir qui nous emballe
Pour demain nos enfants pâles
Un mieux, un rêve, un cheval
\endverse

\beginverse
Refrain
\endverse

## 

\beginsong{Le galérien}[by={Yves Montand (1950)}][by={Capo III}]

\beginverse
J\\[Do]e m'souviens ma m\\[Sol7]ère m'aimait
Et je suis aux g\\[Do]alèr\\[Mi7]es.
J\\[Lam]e m'souviens ma m\\[Mi]ère disait
M\\[Mi7]ais je n'ai pas cru ma m\\[Lam]èr\\[Sol7]e.
N\\[Do]e traîne pas dans l\\[Sol7]es ruisseaux
T'bats pas comme un sauv\\[Do]ag\\[Mi7]e
T\\[Lam]'amuse pas comme l\\[Mi]es oiseaux
E\\[mi7]lle me disait d'être s\\[Lam]ag\\[Sol7]e.
\endverse

\beginverse
J'ai pas tué, j'ai pas volé
J'voulais courir ma chance.
J'ai pas tué, j'ai pas volé
J'voulais qu'chaque jour soit dimanche.
Je m'souviens comme elle pleurait
Dès que j'passais la porte
Je m'souviens comme elle pleurait
Elle voulait pas que je sorte.
\endverse

\beginverse
Toujours, toujours elle disait
T'en vas pas chez les filles
Fais donc pas toujours c'qui t'plais
Dans les prisons y a des grilles.
J'ai pas tué, j'ai pas volé
Mais j'ai cru Madeleine 
J'ai pas tué, j'ai pas volé
J'voulais pas lui faire de peine.
\endverse

\beginverse
Je m'souviens ma mère disait
Suis pas les bohémiennes
Je m'souviens comme elle disait
On ramasse les gens qui trainent.
Un jour les soldats du roi
T emmèneront aux galères
Tu t'en iras trois par trois
Comme ils ont emmené ton père.
\endverse

\beginverse
Tu auras la tête rasée
On te mettra des chaînes.
T'en auras les reins brisés
Et moi j'en mourrai de peine.
Toujours, toujours tu ram'ras
Quand tu s'ras aux galères
Toujours, toujours tu ram'ras
Tu pens'ras p't-être à ta mère.
\endverse

\beginverse
J'ai pas tué, j'ai pas volé
Mais j'ai pas cru ma mère
Et je m'souviens qu'elle m'aimait
Pendant que j'rame aux galères
\endverse

\beginsong{Le gâteau empoisonné}[by={Gérard Calvi \- Astérix et Cléopâtre (1968)}]

\beginverse
Dans un grand bol de strychnine
Délayez de la morphine
Faites tiédir à la casserole
Un bon verre de pétrole
Ho ho, je vais en mettre deux
\endverse

\beginverse
Quelques gouttes de ciguë
De la bave de sangsue
Un scorpion coupé très fin
\endverse

\beginverse
Et un peu de poivre en grains \!
\endverse

\beginverse
Non \!
\endverse

\beginverse
Ah ? Bon
\endverse

\beginverse
Émiettez votre arsenic
Dans un verre de narcotique
Deux cuillères de purgatif
Qu'on fait bouillir à feu vif
Ho ho, je vais en mettre trois
\endverse

\beginverse
Dans un petit plat à part
Tiédir du sang de lézard
La valeur d'un dé à coudre
\endverse

\beginverse
Et un peu de sucre en poudre \!
\endverse

\beginverse
Non \!
\endverse

\beginverse
Ah ? Bon
\endverse

\beginverse
Vous versez la mort-aux-rats
Dans du venin de cobra
Pour adoucir le mélange
Pressez trois quartiers d'orange
\endverse

\beginverse
Ho ho, je vais en mettre un seul
\endverse

\beginverse
Décorez de fruits confits
Moisis dans du vert-de-gris
Tant que votre pâte est molle
\endverse

\beginverse
Et un peu de vitriol \!
\endverse

\beginverse
Non... Oui \!
\endverse

\beginverse
Ah... Je savais bien que ça serait bon
\endverse

\beginverse
Le pudding à l'arsenic
Nous permet ce pronostic
Demain sur les bords du Nil
Que mangeront les crocodiles ?
Des gaulois \!
\endverse

\beginsong{Les gens qu'on aime}[by={Patrick Fiori (2017)}]

\beginverse
J'aurais pu traîner le long de mes rêves
J'aurais pu, l'air de rien
Attendre ici que la journée s'achève
Sortir le chien, si j'en avais un
J'aurais pu m'inventer des inventaires
Refaire et faire le point
Mais ce matin j'ai bien plus cher à faire
Bamdabadabam
\endverse

\beginverse
Refrain
Ce matin, j'irai dire aux gens que j'aime
Ou juste merci d'être ceux qu'ils sont
Qu'ils changent mes heures amères en poèmes
Et tous ces mots que nous taisons
Ce matin, j'irai dire aux gens que j'aime
Oh\! Comme ils comptent pour moi chaque instant
Des mots doux c'est mieux qu'un beau requiem
Et les dire c'est important
Et dire avant tant qu'il est temps
\endverse

\beginverse
On veut toujours attendre la prochaine
Remettre au lendemain
C'est bien plus simple d'émettre des haines
Bien anonymes tapis dans son coin
Et coulent nos vies et l'eau des fontaines
La vie du quotidien
Et passent les jours et puis les semaines
Bamdabadabam
\endverse

\beginverse
Refrain
\endverse

\beginverse
J'aurais pu traîner le long de mes rêves
J'aurais pu l'air de rien
Attendre ici que la journée s'achève
Bamdabadabam
\endverse

\beginverse
On devrait dire aux gens quand on les aime
Trouver les phrases, trouver le temps
Qu'ils changent nos heures amères en poèmes
On devrait tout se dire avant
Il faut le dire aux gens quand on les aime
Comme ils comptent pour nous chaque instant
Les mots doux c'est mieux qu'un beau requiem
Et tant qu'on est là bien vivant
Tout se dire tant qu'il est temps
\endverse

\beginsong{Golden Baby}[by={Cœur de pirate (2011)}]

\beginverse
Je t’ai vu d’un œil solitaire
Le pied dans l’arène pour te plaire
Et briller aux regards que j’ignorais
\endverse

\beginverse
Le tien comptait plus que les autres
Même si tu ne t’en rendais pas compte
Et j’aurais tout fait pour connaître tes fins
\endverse

\beginverse
Refrain
\endverse

\beginverse
Golden Baby, c’en est assez
De courir te faire désirer
Dans ces lumières qui donnent vie à nos nuits
Golden Baby, sans tout pour plaire
Dans ton silence, tu restes fier
De croire en ce qui n’existerait pas
Et si tu veux de moi
\endverse

\beginverse
On s’est finalement embrassés
Des mois sans silence, sans parler
Dans l’attente qui, de loin, m’a déchirée
\endverse

\beginverse
Et j’aurais aimé être ces filles
Qui, dans tes chansons, reprennent vie
Même si, de loin, je sais qu’on s’est menti
\endverse

\beginverse
Refrain
\endverse

\beginverse
J’ai voulu tout laisser tomber
Pour ne pas être ombre du passé
Et retrouver tes rires et tes secrets
\endverse

\beginverse
Mais quand je l’ai vue près de toi
Celle qui en chanson reprend vie
Je sais maintenant que tu m’avais menti
\endverse

\beginverse
Refrain
\endverse

\beginsong{Le gorille\*}[by={Georges Brassens (1952)}]

\beginverse
C'est à t\\[Ré]ravers de larges grilles,
Que les femelles du cant\\[La7]on,
Contemplaient un puissant gorille,
Sans souci du qu'en-dira-t-\\[Ré]on ;
Avec impudeur, ces commères
Lorgnaient même un endroit préc\\[La7]is
Que, rigoureusement, ma mère
M'a défendu d’ nommer ic\\[Ré]i
Gare au gori\\[Ré]ii-i\\[La7]ii-i\\[Ré]ii-i\\[La7]ii-lle \!
\endverse

\beginverse
Tout à coup la prison bien close
Où vivait le bel animal
S'ouvre, on n' sait pourquoi (je suppose
Qu'on avait dû la fermer mal) ;
Le singe, en sortant de sa cage,
Dit : "C'est aujourd'hui que j'le perds \!"
Il parlait de son pucelage,
Vous aviez deviné, j'espère \!
Gare au goriii-iii-iii-iii-lle \!
\endverse

\beginverse
L'patron de la ménagerie
Criait, éperdu : "Nom de nom \!
C'est assommant, car le gorille
N'a jamais connu de guenon \!"
Dès que la féminine engeance
Sut que le singe était puceau,
Au lieu de profiter de la chance,
Elle fit feu des deux fuseaux \!
Gare au goriii-iii-iii-iii-lle \!
\endverse

\beginverse
Celles là même qui, naguère,
Le couvaient d'un œil décidé,
Fuirent, prouvant qu'ell’s n'avaient guère
De la suite dans les idé’s ;
D'autant plus vaine était leur crainte,
Que le gorille est un luron
Supérieur à l'homm’ dans l'étreinte,
Bien des femmes vous le diront \!
Gare au goriii-iii-iii-iii-lle \!
\endverse

\beginverse
Tout le monde se précipite
Hors d'atteinte du singe en rut,
Sauf une vieille décrépite
Et un jeune juge en bois brut.
Voyant que toutes se dérobent,
Le quadrumane accéléra
Son dandinement vers les robes
De la vieille et du magistrat \!
Gare au goriii-iii-iii-iii-lle \!
\endverse

"Bah \! soupirait la centenaire,
Qu'on pût encor me désirer,
Ce serait extraordinaire,
Et, pour tout dire, inespéré \!" ;
Le juge pensait, impassible :
"Qu'on me prenn’ pour une guenon,
C'est complètement impossible..."
La suite lui prouva que non \!
Gare au goriii-iii-iii-iii-lle \!

\beginverse
Supposez que l'un de vous puisse être,
Comme le singe, obligé de
Violer un juge ou une ancêtre,
Lequel choisirait-il des deux ?
Qu'une alternative pareille,
Un de ces quatre jours, m'échoie,
C'est, j'en suis convaincu, la vieille
Qui sera l'objet de mon choix \!
Gare au goriii-iii-iii-iii-lle \!
\endverse

\beginverse
Mais, par malheur, si le gorille
Aux jeux de l'amour vaut son prix,
On sait qu'en revanche il ne brille
Ni par le goût ni par l'esprit.
Lors, au lieu d'opter pour la vieille,
Comme l'aurait fait n'importe qui,
Il saisit le juge à l'oreille
Et l'entraîna dans un maquis \!
Gare au goriii-iii-iii-iii-lle \!
\endverse

\beginverse
La suite serait délectable,
Malheureusement, je ne peux
Pas la dire, et c'est regrettable,
Ça nous aurait fait rire un peu ;
Car le juge, au moment suprême,
Criait : "Maman \!", pleurait beaucoup,
Comme l'homme auquel, le jour même,
Il avait fait trancher le cou.
Gare au goriii-iii-iii-iii-lle \!
\endverse

\beginsong{Guantanamera}[by={Joe Dassin (1966)}]

\beginverse
Refrain
G\\[Sol]uantanamera\\[La],
Ma ville G\\[Ré]uantanamera\\[La],
G\\[Ré]uantana\\[Sol]mera\\[La],
Ma ville G\\[Ré]uantanam\\[Sol]er\\[La]a.
\endverse

\beginverse
C'était un h\\[Ré]omme en dér\\[La]oute,
C'était un f\\[Ré]rère sans do\\[La]ute,
Il n'avait n\\[Ré]i lieu ni p\\[La]lace,
Et sur les r\\[Ré]outes de l'exil,
Sur les s\\[Ré]entiers, sur les p\\[Sol]lace\\[La]s,
Il me par\\[Ré]lait de sa v\\[Sol]ille\\[La].
\endverse

\beginverse
Refrain
\endverse

\beginverse
Là-bas sa maison de misère
Etait plus blanche que le coton
Les rues de sable et de terre
Sentaient le rhum et le melon
Sous leurs jupons de dentelle.
Dieu que les femmes étaient belles.
\endverse

\beginverse
Refrain 
\endverse

\beginverse
Il me reste toute la terre.
Mais je n'en demandais pas autant.
Quand j'ai passé la frontière.
Il n'y avait plus rien devant.
J'allais d'escale en escale,
Loin de ma terre natale.
\endverse

\beginsong{Hakuna Matata}[by={Elton John, Tim Rice \- Le Roi Lion (1994)}]

\beginverse
Refrain
Hakuna Matata,
Mais quelle phrase magnifique
Hakuna Matata,
Quel chant fantastique \!
\endverse

\beginverse
Ces mots signifient
Que tu vivras ta vie,
Sans aucun souci,
Philosophie
\endverse

\beginverse
Hakuna Matata
\endverse

\beginverse
Un jour, quelle horreur
Il comprit que son odeur
Au lieu de sentir la fleur
Soulevait les cœurs.
\endverse

\beginverse
Mais y'a dans tout cochon
Un poète qui sommeille.
Quel martyr
Quand personne
Peut plus vous sentir\!
\endverse

\beginverse
Disgrâce infâme (Parfum d'infâme)
Inonde mon âme (Oh \! Ça pue le drame)
Je déclenche une tempête (Pitié, arrête\!)
Chaque fois que je ...
Non Pumbaa, pas devant les enfants\!
Oh\! Pardon\!
\endverse

\beginverse
Refrain
Hakuna Matata\!
\endverse

\beginverse
Hakuna Matata,\\[3x]
\endverse

\beginverse
Refrain
\endverse

\beginverse
Hakuna Matata, \\[3x]
\endverse

\beginsong{Hallelujah}[by={Leonard Cohen (1984), M Pokora (2012)}]

\beginverse
Now, I’ve h\\[Sol]eard there was a s\\[Mim]ecret chord
\endverse

\beginverse
That D\\[Sol]avid played and it pl\\[Mim]eased the Lord
\endverse

\beginverse
But y\\[Do]ou don't really c\\[Ré]are for music, d\\[Sol]o you ?\\[Ré]
\endverse

\beginverse
Well, it g\\[Sol]oes like this, the f\\[Do]ourth, the f\\[Ré]ifth
\endverse

\beginverse
The m\\[Mim]inor fall and the m\\[Do]ajor lift
\endverse

\beginverse
The b\\[Ré]affled king comp\\[Si7]osing “Halleluj\\[Mim]ah”
\endverse

\beginverse
Refrain
\endverse

\beginverse
Hallel\\[Do]ujah, Hallel\\[Mim]ujah, Hallel\\[Do]ujah, Hallel\\[Sol]u-uhu-u\\[Ré]-uj\\[Do]ah
\endverse

\beginverse
Your faith was strong but you needed proof
\endverse

\beginverse
You saw her bathing on the roof
\endverse

\beginverse
Her beauty and the moonlight overthrew ya
\endverse

\beginverse
She tied you to her kitchen chair
\endverse

\beginverse
She broke your throne and she cut your hair
\endverse

\beginverse
And from your lips she drew the Hallelujah
\endverse

\beginverse
Refrain
\endverse

\beginverse
You say I took the name in vain
\endverse

\beginverse
I don’t even know the name
\endverse

\beginverse
But if I did, well really, what’s it to you?
\endverse

\beginverse
There’s a blaze of light in every word
\endverse

\beginverse
It doesn’t matter which you heard
\endverse

\beginverse
The holy or the broken Hallelujah
\endverse

\beginverse
Refrain
\endverse

\beginverse
I did my best, it wasn’t much
\endverse

\beginverse
I couldn’t feel, so I tried to touch
\endverse

\beginverse
I’ve told the truth, I didn’t come to fool you
\endverse

\beginverse
And even though it all went wrong
\endverse

\beginverse
I’ll stand before the Lord of song
\endverse

\beginverse
With nothing on my tongue but Hallelujah…
\endverse

\beginverse
Refrain
\endverse

\beginverse
Maybe there's a God above
\endverse

\beginverse
But all I've ever learned from love
\endverse

\beginverse
Was how to shoot somebody who outdrew ya
\endverse

\beginverse
It's not a cry that you hear at night
\endverse

\beginverse
It's not somebody who's seen the light
\endverse

\beginverse
It's a cold and it's a broken hallelujah
\endverse

\beginverse
Refrain
\endverse

\beginsong{Here’s to You}[by={Joan Baez (1971)}]

\beginverse
H\\[Do]ere's to y\\[Sol]ou, Nicol\\[Lam]a and Ba\\[Mim]rt
R\\[do]est foreve\\[Sol]r here i\\[lam]n our hea\\[sol]rts
T\\[Mim]he last and fi\\[Rém]nal m\\[Sol]oment is yo\\[Do]urs
T\\[Do]hat ago\\[Sol]ny is yo\\[Lam]ur tri\\[Mim]um\\[Lam]ph
\\[8x]
\endverse

\beginsong{Hexagone\*}[by={Renaud (1975)}]

\beginverse
Ils s'embrassent au mois de janvier,
car une nouvelle année commence,
mais depuis des éternités
l'a pas tell'ment changé la France.
Passent les jours et les semaines,
y'a qu'le décor qui évolue,
la mentalité est la même,
tous des tocards, tous des faux culs.
\endverse

\beginverse
Ils sont pas lourds en février,
à se souvenir de Charonne,
des matraqueurs assermentés
qui fignolèrent leur besogne.
La France est un pays' de flics,
à tous les coins d'rue y'en a cent,
pour faire régner l'ordre public
ils assassinent impunément.
\endverse

\beginverse
Quand on exécute au mois d'mars,
de l'autr'côté des Pyrénées,
un anarchiste du Pays Basque,
pour lui apprendre à s'révolter,
ils crient, ils pleurent et ils s'indignent
de cette immonde mise à mort,
mais ils oublient qu'la guillotine
chez nous aussi fonctionne encore.
\endverse

\beginverse
Être né sous l'signe de l'hexagone,
c'est pas c'qu'on fait de mieux en c'moment,
et le roi des cons, sur son trône,
j'parierais pas qu'il est allemand.
\endverse

\beginverse
On leur a dit, au mois d'avril,
à la télé, dans les journaux,
de pas se découvrir d'un fil,
que l'printemps c'était pour bientôt,
Les vieux principes du seizième siècle,
et les vieilles traditions débiles,
ils les appliquent tous à la lettre,
y m'font pitié ces imbéciles.
\endverse

\beginverse
Ils se souviennent, au mois de mai,
d'un sang qui coula rouge et noir,
d'une révolution manquée
qui faillit renverser l'histoire.
J'me souviens surtout d'ces moutons,
effrayés par la liberté, s'en allant voter par millions
pour l'ordre et la sécurité.
\endverse

\beginverse
Ils commémorent au mois de juin,
un débarquement d'Normandie,
ils pensent au brave soldat ricain
qu'est v'nu se faire tuer loin d'chez lui.
Ils oublient qu'à l'abri des bombes,
les Français craient : vive Pétain,
qu'ils étaient bien planqués à Londres,
qu'y'avait pas beaucoup d'Jean Moulin.
\endverse

\beginverse
Être né sous l'signe de l'hexagone,
c'est pas la gloire en vérité
et le roi des cons, sur son trône,
me dites pas qu'il est portugais.
\endverse

\beginverse
Ils font la fête au mois d'juillet,
en souv'nir d'une révolution
qui n'a jamais éliminé
la misère et l'exploitation.
Ils s'abreuvent de bals populaires,
d'feux d'artifice et de flonflons,
ils pensent oublier dans la bière
qu'ils sont gouvernés comme des pions.
\endverse

\beginverse
Au mois d'août c'est la libertén
après une longue année d'usine,
ils crient : vive les congés payés ;
ils oublient un peu la machine.
En Espagne, en Grèce ou en France,
ils vont polluer toutes les plages,
et, par leur unique présence,
abîmer tous les paysages.
\endverse

\beginverse
Lorsqu'en septembre on assassine
un peuple et une liberté
au coeur de l'Amérique latine,
ils sont pas nombreux à gueuler.
Un ambassadeur se ramène,
bras ouverts il est accueuilli,
le fascisme c'est la gangrène,
à Santiago comme à Paris.
\endverse

\beginverse
Être né sous l'signe de l'hexagone,
c'est vraiment pas une sinécure,
et le roi des cons, sur son trône,
il est français, ça j'en suis sûr.
\endverse

\beginverse
Finies les vendanges en octobre,
le raisin fermente en tonneaux,
ils sont très fiers de leurs vignobles,
leurs côtes-du-rhône et leurs bordeaux.
Ils exportent le sang de la terre
un peu partout à l'étranger,
leur pinard et leur camembert,
c'est leur seule gloire, à ces tarés.
\endverse

\beginverse
En novembre, au Salon d'l'auto,
ils vont admirer par milliers
l'dernier modèle de chez Peugeot,
qu'il pourront jamais se payer.
La bagnole, la télé, l'tiercé,
c'est l'opium du peuple de France,
lui supprimer c'est le tuer,
c'est une drogue à accoutumance.
\endverse

\beginverse
En décembre, c'est l'apothéose,
la grande bouffe et les les p'tits cadeaux,
ils sont toujours aussi moroses,
mais y'a d'la joie dans les ghettos.
La Terre peut s'arrêter d'tourner,
ils rat'ront pas leur réveillon,
moi j'voudrais tous les voir crever,
étouffés de dinde aux marrons.
\endverse

\beginverse
Etre né sous l'signe de l'Hexagone,
on peut pas dire qu'ça soit bandant.
Si l'roi des cons perdait son trône,
y'aurait cinquante millions de prétendants.
\endverse

\beginsong{Hey Jude}[by={The Beatles (1968)}]

\beginverse
Hey Jude, don't make it bad,
Take a sad song and make it better.
Remember to let it into your heart,
Then you can start to make it better.
\endverse

\beginverse
Hey Jude, don't be afraid,
You were made to go out and get her.
The minute you let under your skin
Then you begin to make it better.
\endverse

\beginverse
And anytime you feel the pain,
Hey Jude, refrain, don't carry the world upon your shoulder.
For well you know that it's a fool who plays it cool
By making his world a little colder.
Da da da da
\endverse

\beginverse
Jude, don't let me down,
You have found her, now go and get her.
Remember to let her into your heart,
Then you can start to make it better.
\endverse

\beginverse
So let it out and let it in,
Hey Jude, begin, you're waiting for someone to perform with.
And don't you know that it's just you, hey Jude, you'll do,
The movement you need is on your shoulder
\endverse

\beginverse
Hey Jude, don't make it bad,
Take a sad song and make it better.
Remember to her it into your skin,
Then you'll begin to make it better, better,
Better, better, better, better oh
Da, da, da, da, da, da, da, da, da, da,da, hey Jude.
\endverse

\beginsong{L’homme de Cro-Magnon}[by={Les 4 Barbus (1997)}]

\beginverse
C'était au temps d'la préhistoire
Voici deux ou trois cent mille ans
Vint au monde un être bizarre
Proche parent d'l'orang-outang
Perché sur ses pattes de derrière
Vêtu d'un slip en peau d'bison
Il partait conquérir la terre
C'était l'Homme de Cromagnon.
\endverse

\beginverse
Refrain
L'Homme de Cro, l'Homme de Ma, I'Homme de Gnon
L'Homme de Cromagnon
L'Homme de Cro de Magnon
Ce n'est pas du bidon
L'Homme de Cromagnon on on
(bis 3 der)
\endverse

\beginverse
Armé de sa hache de pierre
De son couteau de pierre itou
Il chassait Fours et la panthère
En serrant les fesses malgré tout
Devant I'diplodocus en rage
Il était tout d'même un peu p'tit
En se disant dans son langage
« Vivement qu'on invente le fusil. »
\endverse

\beginverse
Refrain
\endverse

\beginverse
Il était poète à ses heures
Disant à sa femme en émoi
« Tu es belle comme un dinosaure
Tu r'ssemble à Lolobrigida
Si tu veux voir des cartes postales
Viens dans ma caverne tout là-haut
J'te ferai voir des peintures murales
On dirait du vrai Picasso. »
\endverse

\beginverse
Refrain
\endverse

\beginverse
Trois cent mille ans après sur terre
Comme nos ancêtres nous admirons
Les monts, les bois et les rivières
Mais s'ils rev'naient, quelle déception
Nous voyant suer 6 jours sur 7
Ils diraient sans faire de détail
« Vraiment qu'nos héritiers sont bêtes
D'avoir inventé le travail. »
\endverse

\beginsong{Un homme debout}[by={Claudio Capéo (2016)}]

\beginverse
Refrain
\endverse

\beginverse
Si je m'endors, me réveillerez-vous ?
Il fait si froid dehors, le ressentez-vous ?
Il fut un temps où j'étais comme vous
Malgré toutes mes galères, je reste un homme debout
\endverse

\beginverse
Priez pour que je m'en sorte
Priez pour que mieux je me porte
Ne me jetez pas la faute
Ne me fermez pas la porte
\endverse

\beginverse
Oui je vis, de jour en jour
De squat en squat, un troubadour
Si je chante, c'est pour qu'on m'regarde,
Ne serait-ce qu'un p'tit bonjour
J'vous vois passer, quand j'suis assis
Vous êtes debout, pressés, j'apprécie
Un p'tit regard, un p'tit sourire
Peu prennent le temps, ils ne font que courir
\endverse

\beginverse
Refrain
\endverse

\beginverse
La la la la la la la
La la la la la la la la
\endverse

\beginverse
Merci bien pour la pièce
En c'moment c'est dur, je confesse
'Fin j'vais m'en sortir je l'atteste
Un jour avoir un toit, une adresse
Même si de toi à moi c'est dur, je stresse
\endverse

\beginverse
Le moral n'est pas toujours bon, le temps presse
Mais bon comment faire, à part l'ivresse comme futur
Et des promesses, en veux-tu ?
\endverse

\beginverse
Voilà ma vie, j'me suis pris des coups dans la tronche
Sois sûr que si j'tombe par terre tout l'monde passe mais personne ne bronche
Franchement à part les gosses qui m'regardent étrangement
Tout l'monde trouve ça normal que j'fasse la manche
M'en veuillez pas, mais parfois, j'ai qu'une envie abandonner
\endverse

\beginverse
Refrain
\endverse

\beginverse
Priez pour que je m'en sorte
Priez pour que mieux je me porte
Ne me jetez pas la faute
Ne me ferme pas la porte
\endverse

\beginverse
Refrain \\[bis]
\endverse

\beginsong{Les hommes que j'aime}[by={La Rue Kétanou (2002)}]

\beginverse
Refrain
\endverse

\beginverse
Je voudrais vous parler des hommes que j'aime
Ceux qui m'ont embrassé, au bord de la Seine
Ou j'allais me jeter, jeté par une reine
Que j'avais aimé, plus que les hommes que j'aime.
\endverse

\beginverse
Ils ont des gueules cassées, il faut les voir au petit jour
Se coucher tout étonnés, du monde qui les entoure.
Ils volent, ils viennent, ils trainent, ils parlent fort, ne parlent pas
Ils entendent des carmen, qui leur disaient: "viens par la\!"
Et chaque fois ils y vont, et chaque fois ils en reviennent
Entre un ange et un démon, ainsi j'aime les hommes que j'aime.
\endverse

\beginverse
Refrain
\endverse

\beginverse
Ce sont des géants, qui savent le chagrin d'amour
Des amitiés de survivants, qui fêtent votre retour
Et quand passe un drame et que l'un de nous il r'touche
On se donnent des prénoms de femmes, et on s'embrasse sur la bouche.
Mon dieu c'est mon tour, j'vais au bord de la seine
Je cris au secours, ainsi m'aiment les hommes que j'aime.
\endverse

\beginverse
Refrain
\endverse

\beginverse
Et je lève mon cœur, à la tendresse de ces voyous,
qu'elle me porte bonheur, ce soir j'ai rendez-vous
Et j'irai comme je suis, non je ne changerai rien,
Acte mes folies, à mon feu dans mes mains
A mon amour sans pudeur, a mon amour qui se déchaine
Et même si ça fait peur, ainsi aiment les hommes que j'aime.
\endverse

\beginverse
Refrain \\[bis]
\endverse

\beginsong{L’horloge tourne}[by={Mickael Miro (2010)}]

\beginverse
Un SMS vient d’arriver, j’ai 18 ans,
Envolée ma virginité, j'suis plus un enfant.
L’horloge tourne, les minutes sont torrides
Et moi je rêve d’accélérer le temps.
\endverse

\beginverse
Dam dam déo oh oh oh, dam dam déo oh oh oh oh.
\endverse

\beginverse
Un SMS vient d’arriver, j’ai 20 ans,
On l’a fait sans se protéger mais j'veux pas d’un enfant,
L’horloge tourne, les minutes infanticides
Et moi je rêve de remonter le temps.
\endverse

\beginverse
Dam dam déo oh oh oh, dam dam déo oh oh oh oh.
\endverse

\beginverse
Un SMS vient d’arriver, j’ai 21 ans,
9 mois se sont écoulés et toujours pas d’enfants. Ouh...
L’horloge tourne, les minutes se dérident
Et moi je rêve, tranquille je prends mon temps.
\endverse

\beginverse
Dam dam déo oh oh oh, dam dam déo oh oh oh oh.
\endverse

\beginverse
Un SMS vient d’arriver, j’ai 25 ans,
Un tsunami a tout emporté, même les jeux d’enfants,
L’horloge tourne, les minutes sont acides
Et moi je rêve que passe le mauvais temps
\endverse

\beginverse
Dam dam déo oh oh oh, dam dam déo oh oh oh oh.
\endverse

\beginverse
Un SMS vient d’arriver, j’ai 28 ans,
Mamie est bien fatiguée mais j’suis plus un enfant,
L’horloge tourne mais son coeur se suicide
Et moi je rêve, je rêve du bon vieux temps
\endverse

\beginverse
Dam dam déo oh oh oh, dam dam déo oh oh oh oh. \\[bis]
\endverse

\beginverse
Un SMS va arriver, j’aurai 30 ans,
30 ans de liberté et soudain le bilan,
L’horloge tourne, les minutes sont des rides
Et moi je rêve, je rêve d’arrêter le temps
\endverse

\beginverse
Dam dam déo oh oh oh, dam dam déo oh oh oh oh. \\[3x]
\endverse

\beginsong{L’hymne de nos campagnes\*}[by={Tryo (1988)}]

\beginverse
Si tu es né dans une cité HLM
Je te dédicace ce poème
En espérant qu'au fond de tes yeux ternes
Tu puisses y voir un petit brin d'herbe
Et les mans faut faire la part des choses
Il est grand temps de faire une pause
De troquer cette vie morose
Contre le parfum d'une rose
\endverse

\beginverse
Refrain
C'est l'hymne de nos campagnes
De nos rivières, de nos montagnes
De la vie man, du monde animal
Crie-le bien fort, use tes cordes vocales \!
\endverse

\beginverse
Pas de boulot, pas de diplômes
Partout la même odeur de zone
Plus rien n'agite tes neurones
Pas même le shit que tu mets dans tes cônes
Va voir ailleurs, rien ne te retient
Va vite faire quelque chose de tes mains
Ne te retourne pas ici tu n'as rien
Et sois le premier à chanter ce refrain
\endverse

\beginverse
Refrain
\endverse

\beginverse
Assieds-toi près d'une rivière
Écoute le coulis de l'eau sur la terre
Dis-toi qu'au bout, hé \! il y a la mer
Et que ça, ça n'a rien d'éphémère
Tu comprendras alors que tu n'es rien
Comme celui avant toi, comme celui qui vient
Que le liquide qui coule dans tes mains
Te servira à vivre jusqu'à demain matin \!
\endverse

\beginverse
Assieds-toi près d'un vieux chêne
Et compare le à la race humaine
L'oxygène et l'ombre qu'il t'amène
Mérite-t-il les coups de hache qui le saignent ?
Lève la tête, regarde ces feuilles
Tu verras peut-être un écureuil
Qui te regarde de tout son orgueil
Sa maison est là, tu es sur le seuil…
\endverse

\beginverse
Refrain
\endverse

\beginverse
Peut-être que je parle pour ne rien dire
Que quand tu m'écoutes tu as envie de rire
Mais si le béton est ton avenir
Dis-toi que c'est la forêt qui fait que tu respires
J'aimerais pour tous les animaux
Que tu captes le message de mes mots
Car un lopin de terre, une tige de roseau
Servira à la croissance de tes marmots \!
\endverse

\beginsong{Il changeait la vie}[by={Jean-Jacques Goldmann (1987)}]

\beginverse
C'était un cordonnier, sans rien d'particulier
Dans un village dont le nom m'a échappé
Qui faisait des souliers si jolis, si légers
Que nos vies semblaient un peu moins lourdes à porter
Il y mettait du temps, du talent et du cœur
Ainsi passait sa vie au milieu de nos heures
Et loin des beaux discours, des grandes théories
À sa tâche chaque jour, on pouvait dire de lui
Il changeait la vie
\endverse

\beginverse
C'était un professeur, un simple professeur
Qui pensait que savoir était un grand trésor
Que tous les moins que rien n'avaient pour s'en sortir
Que l'école et le droit qu'a chacun de s'instruire
Il y mettait du temps, du talent et du cœur
Ainsi passait sa vie au milieu de nos heures
Et loin des beaux discours, des grandes théories
À sa tâche chaque jour, on pouvait dire de lui
Il changeait la vie
\endverse

\beginverse
C'était un p'tit bonhomme, rien qu'un tout p'tit bonhomme
Malhabile et rêveur, un peu loupé en somme
Se croyait inutile, banni des autres hommes
Il pleurait sur son saxophone
Il y mit tant de temps, de larmes et de douleur
Les rêves de sa vie, les prisons de son cœur
Et loin des beaux discours, des grandes théories
Inspiré jour après jour de son souffle et de ses cris
Il changeait la vie
\endverse

\beginsong{Il en faut peu pour être heureux}[by={Jean Scout, Pascal Bressy \- Le Livre de la jungle (2002)}]

\beginverse
Il en faut peu pour être heureux
Vraiment très peu pour être heureux
Il faut se satisfaire du nécessaire
Un peu d'eau fraîche et de verdure
Que nous prodigue la nature
Quelques rayons de miel et de soleil.
\endverse

\beginverse
Je dors d'ordinaire sous les frondaisons
Et toute la jungle est ma maison
Toutes les abeilles de la forêt
Butinent pour moi dans les bosquets
Et quand je retourne un gros caillou
Je sais trouver des fourmis dessous.
Essaye c'est bon, c'est doux, oh\!
\endverse

\beginverse
Il en faut vraiment peu,
Très peu pour être heureux\!
Mais oui\!
Pour être heureux.
\endverse

\beginverse
Il en faut peu pour être heureux
Vraiment très peu pour être heureux
Chassez de votre esprit tous vos soucis
Prenez la vie du bon côté
Riez, sautez, dansez, chantez
Et vous serez un ours très bien léché\!
\endverse

\beginverse
Cueillir une banane, oui
Ça se fait sans astuce
\endverse

\beginverse
Aïe\!
Mais c'est tout un drame
Si c'est un cactus
Si vous chipez des fruits sans épines
Ce n'est pas la peine de faire attention
Mais si le fruit de vos rapines
Est tout plein d'épines
C'est beaucoup moins bon\!
Alors petit, as-tu compris?
Il en faut vraiment peu,
Très peu, pour être heureux\!
Pour être heureux?
Pour être heureux\!
\endverse

\beginverse
Et tu verras qu' tout est résolu
Lorsque l'on se passe
Des choses superflues
Alors tu ne t'en fais plus.
Il en faut vraiment peu, très peu, pour être heureux.
\endverse

\beginverse
Il en faut peu pour être heureux
Vraiment très peu pour être heureux
Chassez de votre esprit
Tous vos soucis ... Youpi
Prenez la vie du bon côté
Riez, sautez, dansez, chantez
Et vous serez un ours très bien léché
Waouh
Et vous serez un ours très bien léché.
\endverse

\beginverse
Youpi\!
\endverse

\beginsong{Il est libre Max}[by={Hervé Christiani (1981)}]

\beginverse
Refrain
Il est libre Max, il est libre Max
Y'en a même qui disent qu'ils l'ont vu voler.
\endverse

\beginverse
Il met de la magie mine de rien dans tout c'qu'il fait
Il a l'sourire facile même pour les imbéciles
Il s'amuse bien, il tombe jamais dans les pièges
Il s'laisse pas étourdir par les néons des manèges
Il vit sa vie sans s'occuper des grimaces
Que font autour de lui les poissons dans la nasse.
\endverse

\beginverse
Refrain
\endverse

\beginverse
Il travaille un p'tit peu quand son corps est d'accord
Pour lui faut pas s'en faire, il sait doser son effort
Dans l'panier d'crabes il joue pas les homards
Il cherche pas à tout prix à faire des bulles dans la mare.
\endverse

\beginverse
Refrain
\endverse

\beginverse
Il r'garde autour de lui avec les yeux de l'amour
Avant qu't'aies rien pu dire il t'aime déjà au départ
Il fait pas d'bruit il joue pas au tambour
Mais la statue de marbre lui sourit dans la cour.
\endverse

\beginverse
Refrain
\endverse

\beginverse
Et bien sûr toutes les filles lui font leurs yeux de velours
Lui pour leur faire plaisir il leur raconte des histoires
Il les emmène par-delà le labour
Chevaucher les licornes à la tombée du soir.
\endverse

\beginverse
Refrain
\endverse

\beginverse
Comme il n'a pas d'argent pour faire le grand voyageur
Il va parler souvent aux habitants de son cœur
Qu'est-ce qu'ils s'racontent ? C'est ça qu'il faudrait savoir
Pour avoir comme lui autant d'amour dans l'regard.
\endverse

\beginsong{Il faut que je m’en aille}[by={Graeme Allwright (1966)}]

\beginverse
Le temps est loin, de nos vingt ans
Des coups de poings, des coups de sang
Mais qu'à c'la ne tienne, c'est pas fini
On peut chanter quand le verre est bien rempli
\endverse

\beginverse
Refrain
Buvons encore une dernière fois
A l'amitié, l'amour, la joie
On a fêté nos retrouvailles
Ça m' fait d' la peine,
Mais il faut que je m'en aille
\endverse

\beginverse
Refrain
\endverse

\beginverse
Et souviens-toi de cet été
La première fois qu'on s'est saoulé
Tu m'as ram'né à la maison
En chantant, on marchait à reculons
\endverse

\beginverse
Refrain
\endverse

\beginverse
Je suis parti changer d'étoile
Sur un navire, l'ai mis la voile
Pour n'être plus qu'un étranger
Ne sachant plus très bien où il allait
\endverse

\beginverse
Refrain
\endverse

\beginverse
J't'ai raconté mon mariage
A la mairie d'un p'tit village
Je rigolais dans mon plastron
Quand le maire essayait d'prononcer mon nom
\endverse

\beginverse
Refrain
\endverse

\beginverse
J'n'ai pas écrit, toutes ces années
Et toi aussi, t'es marié
T'as trois enfants à faire manger
Moi j'en ai cinq, si ça peut te consoler
\endverse

\beginsong{Il jouait du piano debout}[by={France Gall (1980)}]

\beginverse
Ne dites pas que ce garçon était fou
Il ne vivait pas comme les autres, c'est tout
Et pour quelles raisons étranges
Les gens qui n'sont pas comme nous,
Ça nous dérange
\endverse

\beginverse
Ne dites pas que ce garçon n'valait rien
Il avait choisi un autre chemin
Et pour quelles raisons étranges
Les gens qui pensent autrement
Ça nous dérange
Ça nous dérange
\endverse

\beginverse
Refrain
\endverse

\beginverse
Il jouait du piano debout
C'est peut-être un détail pour vous
Mais pour moi, ça veut dire beaucoup
Ça veut dire qu'il était libre
Heureux d'être là malgré tout
\endverse

\beginverse
Il jouait du piano debout
Quand les trouillards sont à genoux
Et les soldats au garde à vous
Simplement sur ses deux pieds,
Il voulait être lui, vous comprenez
\endverse

\beginverse
Il n'y a que pour sa musique, qu'il était patriote
Il s'rait mort au champ d'honneur pour quelques notes
Et pour quelles raisons étranges,
Les gens qui tiennent à leurs rêves,
Ça nous dérange
\endverse

\beginverse
Lui et son piano, ils pleuraient quelques fois
Mais c'est quand les autres n'étaient pas là
Et pour quelles raisons bizarres,
Son image a marqué ma mémoire,
Ma mémoire..
\endverse

\beginverse
Refrain
\endverse

\beginverse
Il jouait du piano debout
Il chantait sur des rythmes fous
Et pour moi ça veut dire beaucoup
Ça veut dire essaie de vivre
Essaie d'être heureux,
Ça vaut le coup.
\endverse

\beginverse
Refrain
\endverse

\beginverse
Il jouait du piano debout
Quand les trouillards sont à genoux
Et les soldats au garde à vous
Simplement sur ses deux pieds,
Il voulait être lui, vous comprenez
\endverse

\beginsong{Imagine}[by={John Lennon (1988)}]

\beginverse
Imagine there's no heaven
It's easy if you try
No hell below us
Above us only sky
Imagine all the people
Living for today
\endverse

\beginverse
Refrain
Aha you may say I'm a dreamer.
But I'm not the only one
I hope some day you'll join us
And the world will be as one
\endverse

\beginverse
Imagine there's no countries
It isn't hard to do
Nothing to kill and die for
And no religion too
Imagine all the people
Living life in peace
\endverse

\beginverse
Refrain
\endverse

\beginverse
Imagine no possessions
I wonder if you can
No need for greed or hunger
A brotherhood of man
Imagine all the people
Sharing all the world
\endverse

\beginsong{L’italiano}[by={Toto Cutugno (1983)}]

\beginverse
Lasciatemi cantare
Con la chitarra in mano
Lasciatemi cantare
Sono un italiano
\endverse

\beginverse
Buongiorno Italia, gli spaghetti al dente
E un partigiano come presidente
Con l'autoradio sempre nella mano destra
Un canarino sopra la finestra
\endverse

\beginverse
Buongiorno Italia con i tuoi artisti
Con troppa America sui manifesti
Con le canzoni, con amore
Con il cuore
Con più donne e sempre meno suore
\endverse

\beginverse
Buongiorno Italia, buongiorno Maria
Con gli occhi pieni di malinconia
Buongiorno Dio
Lo sai che ci sono anch'io
\endverse

\beginverse
Refrain
\endverse

\beginverse
Lasciatemi cantare
Con la chitarra in mano
Lasciatemi cantare
Una canzone piano piano
Lasciatemi cantare
Perché ne sono fiero
Sono un italiano
Un italiano vero
\endverse

\beginverse
Buongiorno Italia che non si spaventa
Con la crema da barba alla menta
Con un vestito gessato sul blu
E la moviola la domenica in TV
\endverse

\beginverse
Buongiorno Italia col caffè ristretto
Le calze nuove nel primo cassetto
Con la bandiera in tintoria
E una Seicento giù di carrozzeria
\endverse

\beginverse
Buongiorno Italia, buongiorno Maria
Con gli occhi pieni di malinconia
Buongiorno Dio
Lo sai che ci sono anch'io
\endverse

\beginverse
Refrain \\[bis]
\endverse

\beginsong{J'ai demandé à la lune}[by={Indochine (2002)}]

\beginverse
J'ai demandé à la lune
Et le soleil ne le sait pas
Je lui ai montré mes brûlures
Et la lune s'est moquée de moi
Et comme le ciel n'avait pas fière allure
Et que je ne guérissais pas
Je me suis dit quelle infortune
Et la lune s'est moquée de moi
\endverse

\beginverse
J'ai demandé à la lune
Si tu voulais encore de moi
Elle m'a dit "j'ai pas l'habitude
De m'occuper des cas comme ça"
Et toi et moi
On était tellement sûr
Et on se disait quelquefois
Que c'était juste une aventure
Et que ça ne durerait pas
\endverse

\beginverse
Je n'ai pas grand chose à te dire
Et pas grand chose pour te faire rire
Car j'imagine toujours le pire
Et le meilleur me fait souffrir
\endverse

\beginverse
J'ai demandé à la lune
Si tu voulais encore de moi
Elle m'a dit "j'ai pas l'habitude
De m'occuper des cas comme ça"
Et toi et moi
On était tellement sûr
Et on se disait quelques fois
Que c'était juste une aventure
Et que ça ne durerait pas
\endverse

\beginsong{J'avais rendez-vous}[by={Carrousel (2012)}]

\beginverse
Il y a la foule partout autour, elle se déroule le souffle court
Presser le pas sur les pavés, les pas pressés sur le côté
Il y a la ville, des verticales qui s′empilent en capitales
Mais je me construis dans son élan, même les taxis n'ont plus le temps
\endverse

\beginverse
Reftrain
Mais j′avais rendez-vous mais, j'avais rendez-vous mais
J'avais rendez-vous, dis-moi après quoi on court
J′avais rendez-vous mais, j′avais rendez-vous mais
J'avais rendez-vous, dis-moi après quoi on court
\endverse

\beginverse
verse
Il y a des trains dans le brouillard, rien de certain dans les regards
Au quotidien, des trajectoires, est-ce que l′on revient si on s'égare
Il y a qu′on trace toujours plus vite que l'on efface les limites
Courir en vain mais dans le vent, se dire combien on est vivant
\endverse

\beginverse
Refrain
\endverse

\beginverse
verse
Il y a des jours où l′on s'ennuie, il y a des nuits où l′on s'en fout
Il y a des milliards de fourmis qui voudraient pas qu′on les oublie
Il y a des jours où l'on s'ennuie, il y a des nuits où l′on s′en fout
Il y a des milliards de fourmis qui voudraient pas qu'on les oublie
\endverse

\beginverse
Refrain \\[bis]
\endverse

\beginsong{J'entends siffler le train}[by={Richard Anthony (1962)}]

\beginverse
J'ai pensé qu'il valait mieux nous quitter sans un adieu
Je n'aurais pas eu le cœur de te revoir
Mais j'entends siffler le train \\[bis]
Que c'est triste un train qui siffle dans le soir.
\endverse

\beginverse
Je pouvais t'imaginer toute seule abandonnée
Sur le quai dans la cohue des «au revoir»
Et j'entends siffler le train \\[bis]
Que c'est triste un train qui siffle dans le soir.
\endverse

\beginverse
J'ai failli courir vers toi, j'ai failli crier vers toi
C'est à peine si j'ai pu me retenir
Que c'est loin où tu t'en vas \\[bis]
Auras-tu jamais le temps de revenir ?
\endverse

\beginverse
J'ai pensé qu'il valait mieux nous quitter sans un adieu
Mais je sens que maintenant tout est fini
Et j'entends siffler le train \\[bis]
J'entendrai siffler ce train toute ma vie \\[bis]
\endverse

\beginsong{J't'emmène au vent}[by={Louise Attaque (1997)}]

\beginverse
Allez viens, j't'emmène au vent
Je t'emmène au-dessus des gens
Et je voudrais que tu te rappelles
Notre amour est éternel
Et pas artificiel
\endverse

\beginverse
Refrain
Je voudrais que tu te ramènes devant
Que tu sois là de temps en temps
Et je voudrais que tu te rappelles
Notre amour est éternel
Et pas artificiel
\endverse

\beginverse
Je voudrais que tu m'appelles plus souvent
Que tu prennes parfois les devants
Et je voudrais que tu te rappelles
Notre amour est éternel
Et pas artificiel
\endverse

\beginverse
Je voudrais que tu sois celle que j'entends
Allez viens j't'emmène au-dessus des gens
Et je voudrais que tu te rappelles
Notre amour est éternelle
Artificielle
\endverse

\beginverse
Refrain\\[5x]
\endverse

\beginsong{J’traîne des pieds}[by={Olivia Ruiz (2005)}]

\beginverse
J'traînais les pieds et des casseroles
J'n'aimais pas beaucoup l'école
J'traînais les pieds et mes guiboles abîmées
J'explorais mon quartier
\endverse

\beginverse
J'traînais des pieds dans mon café
Les vieux à la belote braillaient
Papi, mamie, tonton André et toutes ces pépées
A mes p'tits soins, à m'pouponner
\endverse

\beginverse
Refrain
Ecorché mon visage, écorchés mes genoux
Écorché mon p'tit coeur tout mou
Bousillées mes godasses, bousillé sur ma joue
Bousillées les miettes de nous
\endverse

\beginverse
La fumée du bœuf bourguignon
Toute la famille tête dans l'guidon
Du temps où on pouvait faire les cons
Les pensionnaires, les habitués, les gens d'passage surtout l'été
Joyeux bordel dans mon café
\endverse

\beginverse
Refrain
\endverse

\beginverse
Je traîne les pieds, j'traîne mes casseroles
J'n'aime toujours pas l'école
\endverse

\beginverse
Refrain
\endverse

\beginsong{Je l’aime à mourir}[by={Francis Cabrel (1979)}]

\beginverse
Moi je n'étais rien, mais voilà qu'aujourd'hui
Je suis le gardien du sommeil de ses nuits. Je l'aime à mourir
Vous pouvez détruire tout ce qu'il vous plaira
Elle n'a qu'à ouvrir l'espace de ses bras
Pour tout reconstruire, pour tout reconstruire. Je l'aime à mourir
\endverse

\beginverse
Elle a gommé les chiffres des horloges du quartier
Elle a fait de ma vie des cocottes en papiers, des éclats de rire
Elle a bâti des ponts entre nous et le ciel
Et nous les traversons chaque fois qu'elle
Ne peut pas dormir, ne peut pas dormir. Je l'aime à mourir
\endverse

\beginverse
Refrain
Elle a dû faire toutes les guerres
Pour être aussi forte aujourd'hui
Elle a dû faire toutes les guerres de la vie et l'amour aussi
\endverse

\beginverse
Elle vit de son mieux son rêve d'opaline
Elle danse au milieu des forêts qu'elle dessine. Je l'aime à mourir
Elle porte des rubans qu'elle laisse s'envoler
Elle me chante souvent que j'ai tort d'essayer
De les retenir, de les retenir. Je l'aime à mourir
\endverse

\beginverse
Pour monter dans sa grotte cachée sous les toits
Je dois clouer des notes à mes sabots de bois. Je l'aime à mourir
Je dois juste m'asseoir, je ne dois pas parler
Je ne dois rien vouloir, je dois juste essayer
De lui appartenir, de lui appartenir. Je l'aime à
\endverse

\beginverse
Refrain puis 1er couplet
\endverse

\beginsong{Je marche seul}[by={Jean-Jacques Goldmann (1985)}]

\beginverse
Comme un bateau dérive / Sans but et sans mobile
Je marche dans la ville / Tout seul et anonyme
La ville et ses pièges, ce sont mes privilèges
Je suis riche de ça, mais ça ne s'achète pas.
\endverse

\beginverse
Refrain
J'm'en fous, j'm'en fous, de tout
De ces chaînes qui pendent à nos cous
J'm'en enfuis, j'oublie
J'm'offre une parenthèse, un sursis.
\endverse

\beginverse
Refrain
Je marche seul
Dans les rues qui se donnent
Et la nuit me pardonne, je marche seul
En oubliant les heures.
Je marche seul
Sans témoin sans personne
Que mes pas qui résonnent, je marche seul
Acteur et voyageur.
\endverse

\beginverse
Se rencontrer, séduire, / Quand la nuit fait des siennes
Promettre sans le dire / Juste des yeux qui trainent
Oh quand la vie s'obstine en ces heures assassines
Je suis riche de ça, mais ça ne s'achète pas.
\endverse

\beginverse
Refrain
Et j'm'en fous, j'm'en fous, de tout
De ces chaines qui pendent à nos cous
J'm'en enfuis, j'oublie
J'm'offre une parenthèse, un sursis.
Je marche seul
Quand ma vie déraisonne
Quand l'envie m'abandonne, je marche seul
Pour me noyer d'ailleurs.
\endverse

\beginsong{Je n’aurai pas le temps}[by={Michel Fugain (1969)}]

\beginverse
Refrain
Je n'aurai pas le temps, 
pas le temps.
\endverse

\beginverse
Même en courant,
Plus vite que le vent,
Plus vite que le temps.
Même en volant,
Je n'aurai pas le temps,
Pas le temps.
De visiter toute l'immensité
D'un si grand univers.
Même en cent ans,
Je n'aurai pas le temps
De tout faire.
\endverse

\beginverse
Refrain
\endverse

\beginverse
J'ouvre tout grand mon coeur.
J'aime de tous mes yeux.
C'est trop peu
Pour tant de coeurs
Et tant de fleurs.
Des milliers de jours,
C'est bien trop court,
C'est bien trop court.
\endverse

\beginverse
Refrain
\endverse

\beginverse
Et pour aimer
Comme l'on doit aimer
Quand on aime vraiment.
Même en cent ans,
Je n'aurai pas le temps,
Pas le temps.
\endverse

\beginsong{Je veux}[by={Zaz (2010)}]

\beginverse
Donnez-moi une suite au Ritz, je n'en veux pas\!
Des bijoux de chez Chanel, je n'en veux pas\!
Donnez-moi une limousine, j'en ferais quoi?
\endverse

\beginverse
Offrez-moi du personnel, j'en ferais quoi?
Un manoir à Neuchâtel, ce n'est pas pour moi.
Offrez-moi la Tour Eiffel, j'en ferais quoi?
\endverse

\beginverse
Refrain
Je veux d'l'amour, d'la joie, de la bonne humeur,
C'n'est pas votre argent qui f'ra mon bonheur,
Moi j'veux crever la main sur le cœur.
Allons ensemble, découvrir ma liberté,
Oubliez donc tous vos clichés,
Bienvenue dans ma réalité.
\endverse

\beginverse
J'en ai marre d'vos bonnes manières, c'est trop pour moi\!
Moi je mange avec les mains et j'suis comme ça\!
J'parle fort et je suis franche, excusez-moi\!
\endverse

\beginverse
Finie l'hypocrisie. Moi, j'me casse de là\!
J'en ai marre des langues de bois\!
Regardez-moi, d'toute manière j'vous en veux pas
Et j'suis comme ça (j'suis comme ça)
\endverse

\beginverse
Refrain \\[3x]
\endverse

\beginsong{Je voudrais déjà être roi}[by={Elton John, Time rice \- Le Roi Lion (1994)}]

\beginverse
C’est moi Simba, c’est moi le roi
Du royaume animal
\endverse

\beginverse
C’est la première fois qu’on voit un roi
Avec si peu de poils
\endverse

\beginverse
Je vais faire dans la cour des grands
Une entrée triomphale
En poussant, très royalement
Un rugissement bestial
\endverse

\beginverse
Majesté, tu ne te mouches pas du coude \!
\endverse

\beginverse
Je voudrais déjà être roi \!
\endverse

\beginverse
Tu as encore un long chemin à faire
Votre altesse, tu peux me croire
\endverse

\beginverse
Au roi, on ne dit pas...
\endverse

\beginverse
D’ailleurs quand je dis ça...
\endverse

\beginverse
Tiens ta langue et tais-toi
\endverse

\beginverse
Ce que j’essaie de dire c’est...
\endverse

\beginverse
Surtout ne fais pas ça \!
\endverse

\beginverse
Il faut que tu comprennes que...
\endverse

\beginverse
Reste ici, assieds-toi
\endverse

\beginverse
Restez ici
\endverse

\beginverse
Sans jamais dire où je vais
\endverse

\beginverse
Ce lion a une tête de mule...
\endverse

\beginverse
Je veux faire ce qu’il me plait
\endverse

\beginverse
Il est grand temps votre grandeur
Qu’on parle de coeur à coeur
\endverse

\beginverse
Le roi n’a que faire
Des conseils d’une vieille corneille
\endverse

\beginverse
Si tu confonds la monarchie avec la tyrannie
Vive la république
Adieu l’Afrique \!
Je ferme la boutique
Oh prend garde, lion, ne te trompe pas de voie
\endverse

\beginverse
Je voudrais déjà être roi \!
\endverse

\beginverse
Regardez bien à l’ouest (Ah pitié, au secours \!)
Regardez bien à l’est (Non \! Non \!)
Mon pouvoir, sans conteste
Est sans frontière
\endverse

\beginverse
Pas encore \!
\endverse

\beginverse
C’est une rumeur qui monte jusqu’au ciel
Les animaux répandent la nouvelle
Simba sera le nouveau roi soleil
\endverse

\beginverse
Je voudrais déjà être roi \! \\[3x]
\endverse

\beginsong{Jeune et con}[by={Saez (1999)}]

\beginverse
Encore un jour se lève sur la planète France
Et je sors doucement de mes rêves je rentre dans la danse
Comme toujours il est huit heures du soir j'ai dormi tout le jour
Je me suis encore couché trop tard je me suis rendu sourd encore
\endverse

\beginverse
Encore une soirée où la jeunesse France
Encore elle va bien s'amuser puisqu'ici rien a de sens
Alors on va danser faire semblant d'être heureux
Pour aller gentiment se coucher mais demain rien n'ira mieux
\endverse

\beginverse
Refrain
Puisqu'on est jeune et con
Puisqu'ils sont vieux et fous
Puisque des hommes crèvent sous les ponts
Mais ce monde s'en fout
Puisqu'on est que des pions
Content d'être à genoux
Puisque je sais qu'un jour nous gagnerons a devenir fous
\endverse

\beginverse
Encore un jour se lève sur la planète France
Mais j'ai depuis longtemps perdu mes rêves je connais trop la danse
Comme toujours il est huit heure du soir j'ai dormi tout le jour
Mais je sais qu'on est quelques milliards a chercher l'amour encore
\endverse

\beginverse
Encore une soirée ou la jeunesse France
Encore elle va bien s'amuser dans cet état d'urgence
Alors elle va danser faire semblant d'exister
Qui sait si l'on ferme les yeux on vivra vieux
\endverse

\beginverse
Refrain
\endverse

\beginverse
Encore un jour se lève sur la planète France
Et j'ai depuis longtemps perdu mes rêves je connais trop la danse
Comme toujours il est huit heures du soir j'ai dormi tout le jour
Mais je sais qu'on est quelques milliards à chercher l'amour
\endverse

\beginsong{Jeunesse, lève-toi\*}[by={Saez (2008)}]

\beginverse
Comme un éclat de rire
Vient consoler tristesse
Comme un souffle avenir
Viens raviver les braises
Comme un parfum de souffre
Qui fait naître la flamme
Jeunesse lève toi
\endverse

\beginverse
Contre la vie qui va qui vient
Puis qui s'éteint
Contre l'amour qu'on prend, qu'on tient
Mais qui tient pas
Contre la trace qui s'efface
Au derrière de soi
Jeunesse lève toi
\endverse

\beginverse
Moi contre ton épaule
Je repars à la lutte
Contre les gravités qui nous mènent à la chute
Pour faire du bruit encore
A réveiller les morts
Pour redonner éclat
A l'émeraude en toi
\endverse

\beginverse
Pour rendre au crépuscule
La beauté des aurores
Dis moi qu'on brûle encore
Dis-moi que brûle encore cet espoir que tu tiens
Parce que tu n'en sais rien de la fougue et du feu
Que je vois dans tes yeux ?
Jeunesse lève toi \!
\endverse

\beginverse
Quand tu vois comme on pleure
A chaque rue sa peine
Comment on nous écoeure
Perfusion dans la veine
A l'ombre du faisceau
Mon vieux tu m'aura plus \!
Ami dis quand viendra la crue
\endverse

\beginverse
Contre courant toujours sont les contre-cultures,
Au gré des émissions leurs gueules de vide-ordures ?
Puisque c'en est sonné la mort du politique,
L'heure est aux rêves
Aux Utopiques \!
\endverse

\beginverse
Pour faire nos ADN
Un peu plus équitables,
Pour faire de la poussière
Un peu plus que du sable
Dans ce triste pays
Tu sais un jour ou l'autre
Faudra tuer le père
Faire entendre ta voix
Jeunesse lève toi \!
\endverse

\beginverse
Au clair de lune indien
Toujours surfer la vague
A l'âme
Au creux des reins
Faut aiguiser la lame
Puisqu'ici il n'y a qu'au combat qu'on est libre
De ton triste sommeil, je t'en prie libère-toi \!
\endverse

\beginverse
Puisqu'ici il faut faire des bilans et du chiffre
Sont nos amours toujours au bord du précipice,
N'entends-tu pas ce soir chanter le chant des morts
Ne vois tu pas le ciel à la portée des doigts ?
Jeunesse lève toi \!
\endverse

\beginverse
Comme un éclat de rire
Vient consoler tristesse,
Comme un souffle avenir
Vient raviver les braises
Comme un parfum de souffre
Qui fait naître la flamme
Quand plongé dans le gouffre on sait plus où est l'âme
Jeunesse lève toi \!
\endverse

\beginverse
Contre la vie qui va qui vient
Puis qui nous perd,
Contre l'amour qu'on prend qu'on tient
Puis qu'on enterre
Contre la trace qui s'efface
Au derrière de soi ?
JEUNESSE LÈVE-TOI \!
\endverse

\beginverse
Au clair de lune indien
Toujours surfer la vague
A l'âme
Au creux des reins
Faut aiguiser la lame
Puisqu'ici il n'y a qu'au combat qu'on est libre
De ton triste coma, je t'en prie libère-toi \!
Puisqu'ici il faut faire des bilans et du chiffre
Sont nos amours toujours au bord du précipice,
N'entends-tu pas ce soir chanter le chant des morts
A la mémoire de ceux qui sont tombés pour toi
Jeunesse lève toi
\endverse

\beginsong{Jolene}[by={Dolly Parton (1973)}]

\beginverse
Refrain
Jolene \\[Jolene], Jolene \\[Jolene], Jolene, Jolene
I'm begging of you, please don't take my man
Jolene \\[Jolene], Jolene \\[Jolene], Jolene, Jolene
Please don't take him just because you can
\endverse

\beginverse
Your beauty is beyond compare
With flaming locks of auburn hair
Ivory skin and eyes of emerald green
\endverse

\beginverse
Your smile is like a breath of spring
Your voice is soft like summer rain
I cannot compete with you, Jolene
\endverse

\beginverse
He talks about you in his sleep
There's nothing I can do to keep
From crying when he calls your name, Jolene
\endverse

\beginverse
Refrain
\endverse

\beginverse
Now you could have your choice of men
I could never love again
'Cause he's the only one for me, Jolene
\endverse

\beginverse
I had to have this talk with you
My happiness depends on you
And whatever you decide to do, Jolene
\endverse

\beginverse
Refrain
\endverse

\beginverse
I can easily understand
How you could easily take my man
But you don't know what he means to me, Jolene
\endverse

\beginverse
Jolene, Jolene\\[bis]
\endverse

\beginsong{La jument de Michao}[by={Tri Yann (1976), Nolwenn Leroy (2010)}]

\beginverse
C'est dans dix ans je m'en irai,
J'entends le loup et le renard chanter, 
J'entends le loup, le renard et la belette, 
J'entends le loup et le renard chanter.
(bis 2 par 2\)
\endverse

\beginverse
C'est dans neuf ans je m'en irai.
La jument de Michao a passé dans le pré.
La jument de Michao et son petit poulain,
A passé dans le pré et mangé tout le foin.
L'hiver viendra, les gars, l’hiver viendra,
La jument de Michao, elle s'en repentira.
(bis 2 par 2\)
\endverse

\beginverse
C'est dans huit ans…
\endverse

\beginverse
etc. 
\endverse

\beginsong{Junge}[by={Die Ärtze (2007)}]

\beginverse
Junge, warum hast du nichts gelernt?
Guck dir den Dieter an, der hat sogar ein Auto
Warum gehst du nicht zu Onkel Werner in die Werkstatt?
Der gibt dir 'ne Festanstellung, wenn du ihn darum bittest
Junge
\endverse

\beginverse
Und wie du wieder aussiehst, Löcher in der Hose und ständig dieser Lärm
(Was sollen die Nachbarn sagen?)
Und dann noch deine Haare, da fehlen mir die Worte
Musst du die denn färben?
(Was sollen die Nachbarn sagen?)
Nie kommst du nach Hause, wir wissen nicht mehr weiter
\endverse

\beginverse
Junge, brich deiner Mutter nicht das Herz
Es ist noch nicht zu spät, dich an der Uni einzuschreiben
Du hast dich doch früher so für Tiere interessiert,
Wäre das nichts für dich, eine eigene Praxis?
Junge
\endverse

\beginverse
Und wie du wieder aussiehst, Löcher in der Nase und ständig dieser Lärm
(Was sollen die Nachbarn sagen?)
Elektrische Gitarren und immer diese Texte
Das will doch keiner hören
(Was sollen die Nachbarn sagen?)
Nie kommst du nach Hause, soviel schlechter Umgang
Wir werden dich enterben
(Was soll das Finanzamt sagen?)
Wo soll das alles enden, wir machen uns doch Sorgen
\endverse

\beginverse
Und immer deine Freunde, ihr nehmt doch alle Drogen
Und ständig dieser Lärm
(Was solln die Nachbarn sagen?)
Denk an deine Zukunft, denk an deine Eltern
Willst du dass wir sterben?
\endverse

\beginsong{Kumbaya}[by={Traditionnel}]

\beginverse
Refrain
Kumbaya, my Lord, Kumbaya \\[3x]
Oh lord, Kumbaya
\endverse

\beginverse
Someone's singing, Lord, Kumbaya…
Someone's sleeping, Lord, Kumbaya…
Someone's praying, Lord, Kumbaya…
Someone's crying, Lord, Kumbaya…
Someone's thinking, Lord, Kumbaya…
Kumbaya, my Lord, Kumbaya…
\endverse

\beginsong{La même}[by={GIMS, Vianney (2018)}]

\beginverse
Mes amis entendez la vie que j'ai eue
Où les gens m'attendaient, je n'suis pas venu
Si je les emmêle, si je dérange
C'est qu'je suis un pêle-mêle, un mélange
J'suis trop compliqué, je n'choisirai jamais
Que les deux côtés, ne me demandez
Pas où je veux aller, même les singes singent les sages
Et tous ces sages ont fait des cases où tous nous ranger
\endverse

\beginverse
Refrain
Eh, eh, aye aye aye
Aye, aye, aye
Si je vous gêne, bah c'est la même
Si je vous gêne, bah c'est la même
Eh, eh, aye aye aye
Aye, aye, aye
Si je vous gêne, bah c'est la même
Si je vous gêne, bah c'est la même
\endverse

\beginverse
On prend des boîtes, on y range les gens qu'au fond jamais, jamais l'on ne comprend
Comme l'Homme est fait de mille boîtes, ces boîtes que l'on prend ne sont jamais assez grandes
J'ai suivi mille chemins et serré dix mille mains
On peut aimer Brel et Megui, aimer même nos ennemis
\endverse

\beginverse
J'suis trop compliqué, je ne rentrerai jamais
Dans vos petites cases, je vis au jour le jour
Alors je zigzague toujours avec ces lunettes noires
J'entends les gens se demander : "Quand est-ce que tombe le masque ?"
\endverse

\beginverse
Refrain
\endverse

\beginverse
T'es entré dans ma vie, ô ma liberté chérie
La vie, c'est des envies, l'envie avant les avis
T'es entré dans ma vie, ô ma liberté chérie
La vie, c'est des envies, l'envie avant les avis
Eh, eh, aye aye aye
Aye, aye, aye
Si je vous gêne, bah c'est la même
Si je vous gêne, bah c'est la même
\endverse

\beginverse
Refrain
\endverse

\beginsong{Lady melody}[by={Tom Frager (2009)}]

\beginverse
Je me laisse aller souvent
C'est vrai j'attends
Que passe le mauvais temps
Et qu'on fasse comme avant
Je suis pas certain d'avoir trouvé ma place
Je suis pas certain mais pour éviter la casse
J'ai trouvé ma petite Lady Melody
\endverse

\beginverse
Refrain
Elle est dans ma tête
Elle ne m'abandonne jamais
Je la trouve encore plus belle
Quand elle s'habille en reggae
Elle me suit...
A chaque voyage loin d'ici
Elle est ma Lady Melody
\endverse

\beginverse
Ma petite Lady
Elle est ce qui me reste
Quand j'ai déjà tout essayé
Elle chante quand la vie me blesse
\endverse

\beginverse
Et je chante à ses côtés
Dans les orages, les tempêtes
Jamais elle ne m'a quitté quand je m'arrête
D'avancer... j'ai trouvé...
Elle est le soleil que j'attendais
\endverse

\beginverse
Refrain
\endverse

\beginverse
Avec elle je fly ouais
Tu vois comme un oiseau là-haut
Je fly away
Quand j'entend sa mélodie
Je fly ouais
Il n'y a qu'elle qui me comprenne
Je fly away
Elle me donne le faya
Et je fly ouais
J'évite les failles de la vie
Et je fly away
Tu vois comme un oiseau là-haut
Je fly ouais
Comme un oiseau qui plane
Ouais je fly tout là haut, tout là haut...
\endverse

\beginverse
Refrain\\[2x]
\endverse

\beginverse
Ohoho
Elle est dans ma tête
\endverse

\beginsong{Lambé an dro\*}[by={Matmatah (1998)}]

\beginverse
Mélaouache Fanch \!
\endverse

\beginverse
Si tu cherches un peu de gaieté
Viens donc faire un tour à  Lambé
\\[bis]
Si aux exams tu t'es planté
Viens donc faire un tour à  Lambé
\\[bis]
Si t'as quelque chose à  fêter
Viens donc faire un tour à  Lambé
\\[bis]
Y a du chouchen à  volonté
Viens donc faire un tour à  Lambé
\\[bis]
Si t'as rien trouvé pour squatter
Viens donc faire un tour à  Lambé$
\\[bis]
Si ton mec vient de te plaquer
Viens donc faire un tour à  Lambé
\\[bis]
Si du Bouguen tu veux te jeter
Viens donc faire un tour à  Lambé
\\[bis]
Si pour le mélo y a plus d'entrées
Viens donc faire un tour à  Lambé
\\[bis]
Si t'en a marre de galérer
Viens donc faire un tour à  Lambé
\\[bis]
Si dans le bus tu t'es fait choper
Viens donc faire un tour à  Lambé
\\[bis]
Si dans le bus tu t'es fait pécho
Viens donc faire un tour à  Lambé
\\[bis]
Si t'as de la beuh à  partager
Viens donc faire un tour à  Lambé
\\[bis]
Et si t'aimes bien la marche à  pied
Viens donc faire un tour à  Lambé
\\[bis]
Viens donc faire un tour à  Lambé\\[8x]
\endverse

\beginsong{Le légionnaire\*}[by={Traditionnel}]

\beginverse
Il est sur la terre africaine,
Un bataillon dont les soldats, dont les soldats
Sont tous des gars qu'ont pas eu d'veine,
C'est le Bat. d'Af. que nous voilà, que nous voilà
Pour y entrer chose spéciale,
Faut avoir fait de la prison, de la prison.
Prison militaire ou centrale,
C'est tous de là que nous sortons, que nous sortons.
\endverse

\beginverse
Refrain
Mais après tout qu'est-ce que ça fout et l'on s'en fout, ou, ou, ou.
En marchant sur la grand route
Souviens-toi, oui souviens-toi, ah, ah, ah
Qu’les anciens font fait sans doute
Avant toi, oui avant foi ah, ah, an.
De Gabes à Gadouine, de Meknès à Bédouine,
Sac au dos dans la poussière
Marchons les légionnaires
(bis 2 der)
\endverse

\beginverse
J'ai vu mourir un pauvre gosse.
Un pauvre gosse de 18 ans, de 18 ans;
Atteint d'une balle féroce.
Il est mort en criant : « Maman », criant : « Maman ».
C'est moi qu'ai fermé ses paupières
Recueilli son dernier soupir, dernier soupir
J'ai écrit à sa pauvre mère.
Qu'un Légionnaire savait mourir, savait mourir.
\endverse

\beginverse
Refrain
\endverse

\beginverse
Comme nous n'avons jamais eu d'veine
Pour sûr qu'un jour on y crèvera, on y crèvera
Sur cette putain d'terre africaine,
Dans le sable on nous enterrera, nous enterrera
Avec pour croix une baïonnette,
A l'endroit où nous sommes tombés, sommes tombés.
Qui voulez-vous qui nous regrette,
Car nous sommes tous des réprouvés, des réprouvés.
\endverse

\beginsong{Lemon tree}[by={Fool's garden (1995)}]

\beginverse
I'm sittin' here in the boring room
It's just another rainy Sunday afternoon
I'm wastin' my time, I got nothin' to do
I'm hangin' around, I'm waitin' for you
But nothing ever happens
\endverse

\beginverse
I'm drivin' around in my car
I'm drivin' too fast, I'm drivin' too far
I'd like to change my point of view
I feel so lonely, I'm waitin' for you
But nothing ever happens and I wonder
\endverse

\beginverse
Refrain
I wonder how, I wonder why
Yesterday you told me 'bout the blue, blue sky
And all that I can see is just another lemon tree
I'm turnin' my head up and down
I'm turnin', turnin', turnin', turnin', turnin' around
And all that I can see is just another lemon tree
\endverse

\beginverse
dah, dah-dah-dah-dah-dah, di-dah-dah
Dah-dah-dah-dah-dah, di-dah-dah
Dah-di-di-dah (dah-di-di-dah)
\\[Bis]
\endverse

\beginverse
I'm sittin' here, I missed the power
I'd like to go out, takin' a shower
But there's a heavy cloud inside my head
I feel so tired, put myself into bed
Well, nothing ever happens and I wonder
\endverse

\beginverse
Refrain
And all that I can see is just another lemon tree
And I wonder (wonder, wonder, wonder)
\endverse

\beginverse
dah, dah-dah-dah-dah-dah, di-dah-dah
Dah-dah-dah-dah-dah, di-dah-dah
Dah-di-di-dah
\\[Bis]
\endverse

\beginsong{Let it be}[by={The Beatles (1970)}]

\beginverse
When I find myself in times of trouble
Mother Mary comes to me speaking words of wisdom,
Let it be. And in my hour of darkness she is
Standing right in front of me speaking words of wisdom,
\endverse

\beginverse
Refrain
Let it be. Let it be, let it be, let it be, let it be,
Whisper words of wisdom, let it be.
\endverse

\beginverse
And when the broken hearted people living in the world agree.
There will be an answer, let it be.
For though they may be parted
There is still a chance that they will see, there will be an answer,
\endverse

\beginverse
Refrain
Let it be. Let it be, let it be, let it be, let it be,
There will be an answer, let it be.
\endverse

\beginverse
And when the night is cloudy, there is still a light that shines on me.
Shines until tomorrow, let it be.
I wake up to the sound of music, Mother Mary comes to me,
Speaking words of wisdom, let it be.
\endverse

\beginverse
Refrain
Let it be, let it be, let it be, let it be,
Whisper words of wisdom, let it be.
\endverse

\beginsong{Let my people go}[by={Louis Armstrong (1958)}]

\beginverse
Un grand navire est arrivé
Let my people go
Des soldats blancs ont débarqué
Let my people go
\endverse

\beginverse
Refrain
Descends, Seigneur,
Reviens sur cette ferre
De la peur, de la faim, Seigneur,
Délivre nos frères \!
\endverse

\beginverse
Ils ont pillé, ils ont brité
Et massacré nos derniers nés
\endverse

\beginverse
Refrain
\endverse

\beginverse
Les soldats nous ont enchaînés
Les planteurs nous ont achetés
\endverse

\beginverse
Refrain
\endverse

\beginverse
Ils ont frappé ceux qui tombaient
Ils ont tué ceux qui fuyaient
\endverse

\beginverse
Refrain
\endverse

\beginverse
Depuis trois siècles ont passé
Quand viendras-tu nous délivrer
\endverse

\beginverse
Refrain
\endverse

\beginverse
Les noirs sont las de pardonner
Les noirs voudraient pouvoir aimer
\endverse

\beginsong{La lettre\*}[by={Renan Luce (2006)}]

\beginverse
J'ai reçu une lettre
Il y a un mois peut-être
Arrivée par erreur
Maladresse de facteur
Aspergée de parfum
Rouge à lèvres carmin
J'aurais dû cette lettre
Ne pas l'ouvrir peut-être
Mais moi je suis un homme
Qui aime bien ce genre de jeu
\\[Je] veux bien qu'elle me nomme
Alphonse ou Fred, c'est comme elle veut
Des jolies marguerites
Sur le haut de ses « i »
Des courbes manuscrites
Comme dans les abbayes
Quelques fautes d'orthographe
Une légère dyslexie
Et en guise de paraphe
« Ta petite blonde sexy »
Et moi je suis un homme
Qui aime bien ce genre de jeu
\\[Je] n'aime pas les nonnes
Et j'en suis tombé amoureux
\endverse

\beginverse
Elle écrit que dimanche
Elle s'ra sur la falaise
Où j'l'ai prise par les hanches
Et que dans l'hypothèse
Où j'n'aurais pas le tact
D'assumer mes ébats
Elle choisira l'impact
30 mètres plus bas
Et moi je suis un homme
Qui aime bien ce genre d'enjeu
\\[Je] n'veux pas qu'elle s'assomme
Car j'en suis tombé amoureux
Grâce au cachet d'la poste
D'une ville sur la Manche
J'étais à l'avant-poste
Au matin du dimanche
L'endroit était désert
Il faudra être patient
Des blondes suicidaires
Il n'y en a pas cent
Et moi je suis un homme
Qui aime bien ce genre d'enjeu
\\[Je] veux battre Newton
Car je suis tombé amoureux
\endverse

\beginverse
Elle surplombait la Manche
Quand je l'ai reconnue
J'ai saisi par la manche
Ma petite ingénue
Qui ne l'était pas tant
Au regard du profil
Qu'un petit habitant
Lui f'sait sous le nombril
Et moi je suis un homme
Qui aime bien ce genre d'enjeu
\\[Je] veux bien qu'il me nomme
Papa \- s'il le veut
\endverse

\beginsong{Libérée délivrée}[by={Anaïs Delva \- La Reine des Neiges (2013)}]

\beginverse
L'hiver s'installe doucement dans la nuit
La neige est reine à son tour
Un royaume de solitude
Ma place est là pour toujours
\endverse

\beginverse
Le vent qui hurle en moi ne pense plus à demain
Il est bien trop fort
J'ai lutté, en vain
\endverse

\beginverse
Cache tes pouvoirs, n'en parle pas
Fais attention, le secret survivra
Pas d'états d'âme, pas de tourments
De sentiments
\endverse

\beginverse
Libérée, Délivrée
Je ne mentirai plus jamais
Libérée, Délivrée
C'est décidé, je m'en vais
J'ai laissé mon enfance en été
Perdue dans l'hiver
Le froid est pour moi le prix de la liberté.
\endverse

\beginverse
Quand on prend de la hauteur
Tout semble insignifiant
La tristesse, l'angoisse et la peur
M'ont quittée depuis longtemps
\endverse

\beginverse
Je veux voir ce que je peux faire
De cette magie pleine de mystères
Le bien, le mal, je dis tant pis, tant pis
\endverse

\beginverse
Libérée, Délivrée
Les étoiles me tendent les bras
Libérée, Délivrée
Non, je ne pleure pas
Me voilà \!
Oui, je suis là \!
Perdue dans l'hiver
\endverse

\beginverse
Mon pouvoir vient du ciel et envahit l'espace
Mon âme s'exprime en dessinant et sculptant dans la glace
Et mes pensées sont des fleurs de cristal gelées.
\endverse

\beginverse
Non je ne reviendrai pas
Le passé est passé \!
\endverse

\beginverse
Libérée, Délivrée
Désormais plus rien ne m'arrête
Libérée, Délivrée
Plus de princesse parfaite
Je suis là \!
Comme je l'ai rêvé \!
Perdue dans l'hiver
\endverse

\beginverse
Le froid est pour moi le prix de la liberté.
\endverse

\beginsong{Liberta}[by={Peps (2003)}]

\beginverse
Tu sais qu’il y a un bateau qui mène au pays des rêves
Là-bas où il fait chaud, où le ciel n’a pas son pareil
Tu sais qu’au bout de cette terre
Oui, les gens sèment
Des milliers de graines de joie
Où pousse ici la haine
On m’avait dit p’tit gars
Là-bas on t’enlève tes chaînes
On te donne une vie
Sans te jeter dans l’arène
Comme ici tout petit
Après neuf mois à peine
On te plonge dans une vie
Où tu perds vite haleine
Alors sans hésiter, j’ai sauté dans la mer
Pour rejoindre ce vaisseau
Et voir enfin cette terre
Là-bas trop de lumière
J’ai dû fermer les yeux
Mais rien que les odeurs
Remplissaient tous mes vœux
\endverse

\beginverse
Refrain
I just wanna be free in this way
Just wanna be free in my world
Vivere per libertà
Vivere nella libertà
\endverse

\beginverse
Alors une petite fille aussi belle que nature
Me prit par la main et me dit : suis cette aventure
On disait même, oh oui, que la mer l’enviait
Que la montagne se courbait pour la laisser passer
Elle m’emmena au loin avec une douceur sans fin
Et ses bouclettes dorées dégageaient ce parfum
Qui depuis des années guidait ce chemin
Ton chemin, mon chemin, le chemin
\endverse

\beginverse
Refrain
\endverse

\beginverse
Pour arriver enfin à ces rêves d’enfants
Qui n’ont pas de limites comme on a maintenant
J’ai vu des dauphins nager dans un ciel de coton
Où des fleurs volaient, caressant l’horizon
J’ai vu des arbres pousser, remplaçant les gratte-ciels
J’ai vu au fond de l’eau une nuée d’hirondelles
\endverse

\beginverse
Refrain
\endverse

\beginsong{La ligne Holworth}[by={Graeme Allwright (1968)}]

\beginverse
Ted Holworth était un notable dont l'argent venait de la mer
Tous les paroissiens respectables (admiraient sa piété de fer) \\[bis]
Sans doute il ne confondit guère les affaires et les sentiments
Mais sa parole était sincère, (c'est du moins ce que disaient les gens) \\[bis]
\endverse

\beginverse
Il avait tout d'un homme honnête mais il faut vous dire la vérité
Il était noir sous l'étiquette (et ses bateaux étaient damnés) \\[bis]
Il transportait aux Antipodes des hommes attachés par les pieds
Bagnards de sang ou de maraude, (criminels de majestés) \\[bis]
\endverse

\beginverse
Ils avaient offensé la reine ou bien massacré pour voler
Mais ils tiraient à la même chaîne (des innocents humiliés) \\[bis]
Ceux-là s'en allaient vers l'enfer pour un crime abominé
Ils n'avaient pas voulu se taire par (amour de la vérité) \\[bis]
\endverse

\beginverse
La coque était puante et noire, les gars bien comme des loups
Tant de misères, de désespoir (avait de quoi vous rendre fou) \\[bis]
\endverse

\beginverse
Depuis le monde a bien changé, la ligne Holworth a fait peau neuve
Et est très bien considérée, (sa réussite est un chef-d'œuvre) \\[bis]
Il n'y a plus de bagnards dans les cales, mais les marins crient
Sous son pavillon triomphal, (elle transporte des émigrants) \\[bis]
\endverse

\beginsong{Le lion est mort ce soir}[by={Henri Salvador (1962)}]

\beginverse
Dans la jungle,
terrible jungle,
le lion est mort ce soir.
\endverse

\beginverse
Et les hommes,
tranquilles s'endorment
le lion est mort ce soir.
\endverse

\beginverse
Le message
dans le village,
le lion est mort ce soir.
\endverse

\beginverse
Plus de rage,
plus de camage,
le lion est mort ce soir.
\endverse

\beginverse
L'indomptable,
le redoutable,
le lion est mort ce soir.
\endverse

\beginverse
Viens ma belle,
viens ma gazelle,
le lion est mort ce soir.
\endverse

\beginsong{Loin du froid de décembre}[by={Hélène Ségara \- Anastasia (1996)}]

\beginverse
Des images me reviennent
Comme le souvenir tendre
D'une ancienne ritournelle
Autrefois en décembre
\endverse

\beginverse
Refrain
Je me souviens, il me semble
Des jeux qu'on inventait ensemble.
Je retrouve dans un sourire
La flamme des souvenirs
\endverse

\beginverse
Doucement un écho
Comme une braise sous la cendre
Un murmure à mi-mots
Que mon cœur veut comprendre
\endverse

\beginverse
Refrain
\endverse

\beginverse
De très loin un écho
Comme une braise sous la cendre,
Un murmure à mi mots
Que mon coeur veut comprendre
\endverse

\beginverse
Une ancienne ritournelle
Loin du froid de décembre.
\endverse

\beginsong{Love}[by={Jean-Claude Gianadda (2008)}]

\beginverse
Refrain
Love, c'était son nom
La la la la la la la Love
Un vagabond qui vivait de
Soleil, d'espace et de chanson
\endverse

\beginverse
Il est venu chez nous
Guitare en bandoulière
Venant d'on ne sait où
Il parcourait la terre
Et dans ses longs cheveux
Le vent semblait chanter
Tout au fond de ses yeux
Dansait la liberté. Aïe Aïe Aïe
\endverse

\beginverse
Refrain
\endverse

\beginverse
Il écoutait le vent
Les fleurs et les rivières
Jouait comme un enfant
Pariall de la lumière
Il partageait ses rires
Ses rêves et ses projets
Et sur chaque sourire
Dansait la liberté. Aïe Aïe Aïe
\endverse

\beginverse
Refrain
\endverse

\beginverse
Il est parti un jour
Nui ne sait où il est
Au pays de l'amour
Tu peux le rencontrer
Et dans notre maison
Il nous aura laissé
Avec quelques chansons
Un peu de liberté. Aïe Aïe Aïe
\endverse

\beginsong{Love lioubov amour}[by={Les Poppys (1971)}]

\beginverse
Refrain
Love, love, love, dit-on en Amérique, Lioubov, en Russie soviétique
Amour aux quatre coins de France. Moi, je crois, crois, crois, 
Qu'avec tous ces mots-la, la paix enfin aura, un jour sa chance.
\endverse

\beginverse
Léonid qui habite Moscou, est monté dans sa troïka
Acheter du caviar et de la vodka, de chez Poutchkine.
Tandis que Richard de Washington, a pris des hamburgers par tonne
Et Georges à Paris a cueilli des fleurs, 
Et tous trois sont allés en Chine.
\endverse

\beginverse
Refrain
\endverse

\beginverse
C'était une surprise partie, on a dansé toute la nuit
De temps en temps, on peut bien rire entre voisins,
Voisins de terre.
C'était une soirée à Pékin, où se retrouvaient de vieux copains
Qui aiment rire, boire et chanter
Mais qui n'aiment pas faire la guerre.
\endverse

\beginsong{Ma liberté}[by={Georges Moustaki (1970)}]

\beginverse
Ma liberté, longtemps je t'ai gardée, comme une perle rare,
Ma liberté, c'est toi qui m'as aidé à larguer les amarres.
On allait n'importe où, on allait jusqu'au bout
Des chemins de fortune; pour cueillir en rêvant
Une rose des vents sur un rayon de lune.
\endverse

\beginverse
Ma liberté, devant tes volontés, ma vie était soumise,
Ma liberté, je t'avais tout donné, ma dernière chemise.
Et combien j'ai souffert pour pouvoir satisfaire
Tes moindres exigences; j'ai changé de pays.
J'ai perdu mes amis, pour gagner ta confiance.
\endverse

\beginverse
Ma liberté, tu as su désarmer mes moindres habitudes,
Ma liberté, toi qui m'as fait aimer même la solitude.
Toi qui m'as fait sourire quand je voyais finir
Une belle aventure; toi qui m'as protégé
Quand j'allais me cacher pour soigner mes blessures.
\endverse

\beginverse
Ma liberté, pourtant je t'ai quittée, une nuit de décembre,
J'ai déserté les chemins écartés que nous suivions ensemble.
Lorsque sans me méfier, les pieds et poings liés,
Je me suis laissé faire; et je t'ai trahie
Pour une prison d'amour et sa belle geôlière.
Et je t'ai trahie pour une prison d'amour et sa belle geôlière.
\endverse

\beginsong{Ma philosophie}[by={Amel Bent (2004)}]

\beginverse
Je n'ai qu'une philosophie
Être acceptée comme je suis
Malgré tout ce qu'on me dit
Je reste le poing levé
Pour le meilleur comme le pire
Je suis métisse mais pas martyre
J'avance le cœur léger
Mais toujours le poing levé
\endverse

\beginverse
Lever la tête, bomber le torse
Sans cesse redoubler d'efforts
La vie ne m'en laisse pas le choix
Je suis l'as qui bat le roi
Malgré nos peines, nos différences
Et toutes ces injures incessantes
Moi je lèverai le poing
Encore plus haut, encore plus loin
\endverse

\beginverse
Refrain
Viser la Lune
Ça me fait pas peur
Même à l'usure
J'y crois encore et en cœur
Des sacrifices
S'il le faut j'en ferai
J'en ai déjà fait
Mais toujours le poing levé
\endverse

\beginverse
Je ne suis pas comme toutes ces filles
Qu'on dévisage, qu'on déshabille
Moi j'ai des formes et des rondeurs
Ça sert à réchauffer les coeurs
Fille d'un quartier populaire
J'y ai appris à être fier
Bien plus d'amour que de misère
Bien plus de coeur que de pierre
\endverse

\beginverse
Je n'ai qu'une philosophie
Être acceptée comme je suis
Avec la force et le sourire
Le poing levé vers l'avenir
Lever la tête, bomber le torse
Sans cesse redoubler d'efforts
La vie ne m'en laisse pas le choix
Je suis l'as qui bat le roi
\endverse

\beginverse
Refrain\\[3x]
\endverse

\beginsong{Ma révolution}[by={Jenifer (2004)}]

\beginverse
Refrain
Ma révolution porte ton nom
Ma révolution n'a qu'une seule façon
De tourner le monde
De le changer
Pour toi je ne cesserai jamais de marcher
Ma révolution porte ton nom
\endverse

\beginverse
J'étais à peu près
Je suis exactement
La femme que j'espérais
Le coeur enfin vivant
Pour toi j'ai soulevé
Un amour de géant
J'ai fait ma guerre
Marqué la Terre
\endverse

\beginverse
Je me suis battue
En ton nom j'ai crié
Sur les toits ma venue
Ma raison d'exister
Je t'aurais voulu
Depuis tellement d'années
Que le temps vienne
Je serais sienne
\endverse

\beginverse
Refrain
\endverse

\beginverse
La nuit mon amour
Je te rêve à côté
A côté de moi
Pour au moins l'éternité
Que ta vie me parcourt
Mais je rêve éveillée
Je suis aux anges
A toi mon ange
\endverse

\beginverse
Refrain
\endverse

\beginverse
Avec toi
Ma vie rêvée
Toi pour m'accompagner
J'ai tout changé
Tout renversé
Et me fait tout un monde
Que j'imagine parfait
T'es ma révolution
\endverse

\beginverse
Refrain\\[bis]
\endverse

\beginsong{Manhattan-Kaboul}[by={Renaud, Axelle Red (2002)}]

\beginverse
Petit Portoricain
Bien intégré, quasiment New-Yorkais
Dans mon building tout de verre et d’acier
Je prends mon job, un rail de coke, un café
\endverse

\beginverse
Petite fille Afghane
De l’autre côté de la terre
Jamais entendu parler de Manhattan
Mon quotidien c’est la misère et la guerre
\endverse

\beginverse
Refrain
Deux étrangers au bout du monde, si différents
Deux inconnus, deux anonymes, mais pourtant
Pulvérisés sur l’autel
De la violence éternelle
\endverse

\beginverse
Un 747
S’est explosé dans mes fenêtres
Mon ciel si bleu est devenu orage
Lorsque les bombes ont rasé mon village
\endverse

\beginverse
Refrain
\endverse

\beginverse
So long \! Adieu mon rêve américain
Moi plus jamais esclave des chiens
Ils t’imposaient l’Islam des tyrans
Ceux-là ont-ils jamais lu le Coran ?
\endverse

\beginverse
Suis redev’nu poussière
Je s’rai pas maître de l’univers
Ce pays que j’aimais tell’ment serait-il
Finalement colosse aux pieds d’argile ?
\endverse

\beginverse
Les dieux, les religions
Les guerres de civilisation
Les armes, les drapeaux, les patries, les nations
F’ront toujours de nous de la chair à canon
\endverse

\beginverse
Refrain\\[bis]
\endverse

\beginsong{Manu Chao}[by={Les Wampas (2006)}]

\beginverse
Je chante dans les Glaviots un groupe punk de Normandie
on répète dans la grange tous les mardis et les jeudis
quand au bout d'un quart d'heure on a assez fait de bruit
on s'assoie dans le foin et on chante ce refrain
\endverse

\beginverse
si j'avais le portefeuille de Manu Chao
je partirais en vacances au moins jusqu'au Congo
si j'avais le compte en banque de Louise Attaque
je partirais en vacances au moins jusqu'à pâques
c'est beau la Normandie comme le dit ma grand tante Marie
mais si j'avais du blé je partirais bien loin d'ici
souvent les soirs d'été je m'assoie dans les champs de blé
je ferme doucement les yeux et j'écoute les pommiers chanter
\endverse

\beginverse
si j'avais le portefeuille de Manu Chao
je partirais en vacances avec tous mes poteaux
si j'avais le compte en banque de Louise Attaque
je partirais en vacances au moins jusqu'à pâquessi j'avais le portefeuille de Manu Chao
je partirais en vacances dans une superbe auto
\endverse

\beginverse
si j'avais le compte en banque de Louise Attaque
je partirais en vacances au moins jusqu'à pâques
moi aussi si je pouvais j'irais bien jusqu'au Mexique
boire de la téquila avec le commandant Marcos
mais j'ai encore au moins cinq hectares à labourer
je remonte sur mon tracteur et je chante pour me donner du coeur
\endverse

\beginverse
si j'avais le portefeuille de Manu Chao
je partirais en vacances au moins jusqu'au Congo
si j'avais le compte en banque de Louise Attaque
je partirais en vacances au moins jusqu'à pâquesmais j'ai pas un beau chapeau comme Manu Chao
et j'irai en vacances seulement à Saint Lô
et j'ai pas de la classe comme Didier Wampas
je resterai pour les vacances
tout seul avec mes vaches si j'avais le portefeuille de Manu Chao
je partirais en vacances avec tous mes poteaux
si j'avais le compte en banque de la Louise Attaque
je partirais en vacances au moins jusqu'à pâques
\endverse

\beginsong{Matador}[by={Mickey 3D (2005)}]

\beginverse
Je n'ai pas peur des Américains,
Ni des cons, ni des politiciens,
Mais j'ai peur de t'attraper la main
Et que tu m'esquives encore
Je ne sais pas si cet amour est fort
Ou s'il ressemble à la chasse au trésor
Si t'en veux pas, saches que je le déplore
Et que je m'excuse encore
\endverse

\beginverse
Refrain
Je n'ai pas peur de la mort
Mais que tu m'évites encore
Je te préviens matador
Qu'un jour je t'aurai alors
\endverse

\beginverse
On a vu des taureaux, aimer des toreros
On a vu des taureaux, aimer les toreros
\\[bis]
\endverse

\beginverse
Je n'ai pas peur des ordinateurs
Ni des virus exterminateurs
J'ai défoncé tellement de gladiateurs
Qu'ils ont disparu alors
\endverse

\beginverse
J'aimerais bien t'emmener sur le port
Te refaire le coup du con qui t'adore
J'ai peur que tu joues les toréadors
Et que tu m'esquives encore
\endverse

\beginverse
Refrain
\endverse

\beginverse
On a vu des taureaux, aimer des toreros
On a vu des taureaux, aimer les toreros
\endverse

\beginsong{La mauvaise réputation}[by={Georges Brassens (1952)}]

\beginverse
Au village, sans prétention,
J'ai mauvaise réputation ;
Que je me démène ou je reste coi,
Je pass’ pour un je-ne-sais-quoi.
Je ne fais pourtant de tort à personne,
En suivant mon ch’min de petit bonhomme ;
Mais les brav’s gens n'aiment pas que
L'on suive une autre route qu'eux…
Non, les brav’s gens n'aiment pas que
L'on suive une autre route qu'eux…
Tout le monde médit de moi,
Sauf les muets, ça va de soi.
\endverse

\beginverse
Le jour du quatorze-Juillet,
Je reste dans mon lit douillet ;
La musique qui marche au pas,
Cela ne me regarde pas.
Je ne fais pourtant de tort à personne,
En n'écoutant pas le clairon qui sonne ;
Mais les braves gens n'aiment pas que
L'on suive une autre route qu'eux…
Non les braves gens n'aiment pas que
L'on suive une autre route qu'eux…
Tout le monde me montre du doigt,
Sauf les manchots, ça va de soi.
\endverse

\beginverse
Quand je croise un voleur malchanceux,
Poursuivi par un cul-terreux;
Je lance la patte et pourquoi le taire,
Le cul-terreux se r’trouv’ par terre.
Je ne fait pourtant de tort à personne,
En laissant courir les voleurs de pommes ;
Mais les brav’s gens n'aiment pas que
L'on suive une autre route qu'eux…
Non les braves gens n'aiment pas que
L'on suive une autre route qu'eux…
Tout le monde se ru’ sur moi,
Sauf les culs-d’-jatt’, ça va de soi.
\endverse

\beginverse
Pas besoin d'être Jérémi’,
Pour d’viner l’ sort qui m'est promis :
S'ils trouv’nt une corde à leur goût,
Ils me la passeront au cou.
Je ne fais pourtant de tort à personne,
En suivant les ch’mins qui ne mèn’nt pas à Rome ;
Mais les brav’s gens n'aiment pas que
L'on suive une autre route qu'eux…
Non les brav’s gens n'aiment pas que
L'on suive une autre route qu'eux…
Tout le monde viendra me voir pendu,
Sauf les aveugl’s, bien entendu.
\endverse

\beginsong{Me gustas tu\*}[by={Manu Chao (2001)}]

¿Que horas son mi corazon ?
Doce de la noche en la Habana de Cuba.
Once de la noche en San Salvador El Salvador.
Once de la noche en Managua Nicaragua.

\beginverse
Me gustan los aviones, me gustas tu.
Me gusta viajar, me gustas tu.
Me gusta la mañana, me gustas tu.
Me gusta el viento, me gustas tu.
Me gusta soñar, me gustas tu.
Me gusta la mar, me gustas tu
\endverse

\beginverse
Refrain
Que voy a hacer, je ne sais pas
Que voy a hacer, je ne sais plus
Que voy a hacer, je suis perdu
¿Que hora son mi corazon ?
\endverse

\beginverse
Me gusta la moto, me gustas tu.
Me gusta correr, me gustas tu.
Me gusta la lluvia, me gustas tu.
Me gusta volver, me gustas tu
Me gusta marijuana, me gustas tu.
Me gusta colombiana, me gustas tu.
Me gusta la montaña, me gustas tu.
Me gusta la noche, me gustas tu.
\endverse

\beginverse
Refrain
\endverse

\beginverse
Me gusta la cena, me gustas tu.
Me gusta la vecina, me gustas tu.
Me gusta su cocina, me gustas tu.
Me gusta camelar, me gustas tu.
Me gusta la guitarra, me gustas tu.
Me gusta el reggae, me gustas tu.
\endverse

\beginverse
Refrain
\endverse

\beginverse
Me gusta la canela, me gustas tu.
Me gusta el fuego, me gustas tu.
Me gusta menear, me gustas tu.
Me gusta la Coruña, me gustas tu.
Me gusta Malasana, me gustas tu.
Me gusta la castana, me gustas tu.
Me gusta Guatemala, me gustas tu.
\endverse

\beginverse
Refrain
\endverse

\beginsong{Mistral gagnant}[by={Renaud (1985)}]

\beginverse
Ah... m'asseoir sur un banc cinq minutes avec toi
Et regarder les gens tant qu'en y a
\endverse

\beginverse
Te parler du bon temps qu'est mort ou qui r'viendra
En serrant dans ma main tes petits doigts
\endverse

\beginverse
Pi donner à bouffer à des pigeons idiots
Leur filer des coups d'pied pour de faux
\endverse

\beginverse
Et entendre ton rire qui lézarde les murs
Qui sait surtout guérir mes blessures
\endverse

\beginverse
Te raconter un peu comment j'étais,
Mino les bombecs fabuleux
Qu'on piquait chez le marchand car-en-sac
Et Minhto caramels à un franc 
Et les Mistrals gagnants.
\endverse

\beginverse
Ah... marcher sous la pluie cinq minutes avec toi
Et regarder la vie tant qu'y en a
\endverse

\beginverse
Te raconter la Terre en te bouffant des yeux
Te parler de ta mère un p'tit peu
\endverse

\beginverse
Et sauter dans les flaques pour la faire râler
Bousiller nos godasses et s'marrer
\endverse

\beginverse
Et entendre ton rire comme on entend la mer
S'arrêter, repartir en arrière
\endverse

\beginverse
Te raconter surtout les Carambars d'antan
Et les coco-boers et les vrais roudoudou
Qui nous coupaient les lèvres et nous niquaient les dents
Et les Mistral gagnants.
\endverse

\beginverse
Ah... m'asseoir sur un banc cinq minutes avec toi
Regarder le soleil qui s'en va
\endverse

\beginverse
Te parier du bon temps qu'est mort et je m'en fous
Te dire qu'les méchants c'est pas nous
\endverse

\beginverse
Que si moi je suis barge ce n'est pas que de tes yeux
Car ils ont l'avantage d'être deux
Et entendre ton rire s'envoler aussi haut
Que s'envolent des oiseaux
\endverse

\beginverse
Te raconter enfin qu'il faut aimer la vie
L'aimer même si le temps est assassin
Et emporte avec lui le rire des enfants
Et les Mistrals gagnants
\endverse

\beginsong{Mon ancêtre Gurdil\*}[by={Pen of chaos et le Naheulband \- Donjon de Naheulbeuk (2001)}]

\beginverse
Refrain
Nous sommes les nains sous la montagne
On creuse le jour, on boit la nuit
Et on n'aime pas ceux d'la surface \!
\endverse

\beginverse
Voici l'histoire d'un nain capable
De courir vite et de voyager loin
Dans son épopée formidable
Nous le suivrons, une bière à la main \!
\endverse

\beginverse
Refrain
\endverse

\beginverse
Un jour mon ancêtre Gurdil
Fut envoyé creuser dans la forêt
Y'avait soit disant du mithril
Si y'en avait, on sait pas où il s'trouvait \!
Il fit sa cabane en bordure
D'un bois touffu peuplé d'elfes sylvains
Des gens qui bouffent de la verdure
Evidemment, ça n'fait pas des bons voisins \!
\endverse

\beginverse
Refrain
\endverse

"Arrière tu n'es pas bienvenue \!"
Lui dirent les elfes, en lui jetant des pierres
Voyant que tout était foutu
Il prit la fuite, en suivant la rivière
Il fut recueilli par les fées
Ondines bleues, bullant sur le rivage
De l'eau de pluie lui fut donnée
Il recracha WAA \! Tout dans leur visage \!

\beginverse
Refrain
\endverse

\beginverse
Courant à travers les fougères
Il arriva près d'un village humain
Bien sûr qu'on y vendait d'la bière
Mais aucun homme ne voulait servir un nain \!
Gurdil massacra la patron
D'une taverne, à coups de tabourets \! Aïe \!
Puis il rentra à la maison
Et de la mine, il ne repartit jamais \!
\endverse

\beginverse
Refrain
\endverse

\beginverse
Amis restons bien à l'abri
Mangeons, buvons dans nos maisons de pierres
Là haut, c'est peuplé d'abrutis
ALLEZ PATRON \! Ressers donc une bière \!
\endverse

\beginverse
Refrain
\endverse

\beginsong{Mon étoile }[by={Tafta (2004)}]

\beginverse
Tu me dis que c'est facile d'oublier
Tous ces moments que l'on a passés à s'aimer
Qu'il te suffit de partir pour tirer
Un trait sur tous nos souvenirs, ce passé
Et si tu crois que c'est facile pour moi
Sans même une main pour me guider, me protéger
Que l'amour est inutile, j'y crois pas
J'ai besoin d'aimer pour exister
\endverse

\beginverse
Je n'en peux plus de croire en cet amour
Sans jamais pouvoir le crier, le prouver
Et que c'est tellement difficile chaque jour
J'ai besoin d'aimer pour exister
Besoin d'aimer pour exister
\endverse

\beginverse
Refrain
Et je suis comme un homme à la mer
J'ai perdu mon étoile
Et je me sens seul dans l'univers
Et si le vent souffle dans mes voiles
J'irai au bout de la Terre pour y décrocher mon étoile \\[bis]
\endverse

\beginverse
Tu me dis que c'est facile d'oublier
Tous ces moments que l'on a passés à s'aimer
Qu'il te suffit de partir pour tirer
Un trait sur tous nos souvenirs, ce passé
Et que l'amour est inutile, j'y crois pas
J'ai besoin d'aimer pour exister
Besoin d'aimer pour exister
\endverse

\beginverse
Refrain
\endverse

\beginverse
J'irai au bout de la Terre pour y décrocher mon étoile\\[bis]
Mon étoile
Eh, eh \\[4x]
\endverse

\beginsong{Mon Européenne\*}[by={Saez (2017)}]

\beginverse
Elle est Gauloise au p’tit vin blanc
Elle est contre gouvernements
Elle est pas fille des religions
Elle est pas putain du pognon
Elle est vent du Nord ou d’Ouest
Elle est vent du Sud ou de l’Est
Elle est sans-abri à la rue
Elle est toujours peine perdue
Elle est gitane elle est profane
Elle est quand la gauloise plane
Elle toujours fumeuse de joints
Elle dort dans les gares en chemin
Elle est solidaire au combat
Elle est Varsovie Messina
Elle est pas banquière pour un sou
Elle est pas bottes au garde à vous
Elle est sans-abri sans frontière
Elle est contre totalitaire
Elle est j’t’emmerde avec ta thune
Allez vas-y ressers une brune
Elle est ma gueule de Picasso
Elle est tous mes potes au pinceau
Kusturica Sarajevo
Elle est pas loin la Gestapo
\endverse

\beginverse
Mon Européenne c’est pas la Bruxelles
Mon Européenne c’est pas Genève
C'est pas la thune tu marches ou crèves
Tu sais moi mon Européenne
Elle a pas vraiment de frontières
Son corps c’est la planète entière
N’en déplaise au peuple bourgeois
Tu sais mon Européenne à moi
\endverse

\beginverse
Elle est keupon rat sur l’épaule
Elle est tatouage de la taule
Elle est accordéon sanglot
Elle est accorde-moi un tango
Elle est destin des origines
Elle est racine gréco-latine
Elle est contre l’union bancaire
Elle est mes révolutionnaires
\endverse

\beginverse
Elle est pote à Mimi Pinson
Elle est Roumanie sans pognon
Elle est guillotine pour les rois
Elle est plutôt comme toi et moi
Elle pas médiatique je crois
Elle est pas politique bourgeois
Elle est paysanne au combat
Elle est partisane quand elle boit
\endverse

\beginverse
Elle est ouvrière licenciée
Non c’est pas la fille du progrès
Elle est bandonéon métro
Elle est plutôt Manu Crado
Elle est nordique nord-africaine
Elle est un peu baltique aussi
Elle a des airs de statue grecque
Elle a des airs des Italies
Qu’on dirait Paris à Venise
Qu’on dirait Namur aux Marquises
C’est Gauguin qui peint la terre
Comme un pinceau vous dit mon frère
Mon Européenne c’est pas Bruxelles
Mon Européenne c’est pas Genève
C'est pas la thune tu marches ou crèves
Tu sais moi mon Européenne
\endverse

\beginverse
Elle est pas Merkel ou Hollande
C’est pas la valse des propagandes
Des discours de haine au bistrot
Elle est roumaine dans les métros
\endverse

\beginverse
Elle a pas un rond fin du mois
N’en déplaise au peuple bourgeois
Elle est pas Mercedes je crois
Elle est plutôt Grec au combat
\endverse

\beginverse
Elle est Suédoise plans à trois
Elle est mon ardoise quand je bois
Elle est gréco-latine Germaine
Elle est Britannique quand elle traîne
Elle aime les bars elle aime la bière
Elle aime l’odeur du populaire
Elle est moitié louve moitié chienne
Elle est d’où qu’on aille d’où qu’on vienne
\endverse

\beginverse
Elle est Barcelona corazon
Elle est Venise elle est Vérone
C’est pas la boursière de London
C’est l’enfer des Babylones
Elle est Cherbourg Saint-Pétersbourg
Elle est toutes les putains d’Hambourg
Elle est Russie américaine
Tu sais moi ma République haine
\endverse

\beginverse
Elle est polka dans les métros
Elle est Gypsi elle est Django
Elle est pas ghetto à Calais
Elle est pas règne du billet
Elle est Flamenco sous Franco
Elle a le sourire du prolo
Elle est p’tit matin au bistrot
Elle a la gueule Greta Garbo
Elle est accordéon sanglot
Elle est accorde-moi un tango
Elle a la beauté Ukrainienne
Tu sais moi mon Européenne
C’est pas Bruxelles c’est pas Genève
C’est pas la thune tu marches ou crèves
C’est pas c’qui passe dans les radios
C’est pas c’qu’on lit sur tes réseaux
\endverse

\beginverse
Elle est Allemande elle est Anglaise
Elle est Flamande elle est Française
Elle est Bulgare elle est Slovaque
Poing levé contre la matraque
Mon Espagnole mon Italienne
En farandole mon Européenne
Elle est Léttonne elle est Hongroise
Elle est Wallonne elle est Liégeoise
Elle est baltique elle est bohème
Ma Bolchévique ma Norvégienne
Elle est d’Athènes elle est Danoise
Elle m’fout la trique ma Suédoise
Elle est latine anglo-saxone
Puis souvent c’est vrai qu’elle est conne
Elle est continent vieille histoire
Elle est souvent sur des comptoirs
\endverse

\beginverse
Elle est Galloise elle est Gauloise
Elle sait surtout m’laisser l’ardoise
Elle est révoltée polonaise
Elle a le sang nord-irlandaise
Elle est statue gréco-romaine
Tu la verrais mon Européenne
Ma Vénus à moi quand j’la traîne
Plus que tout mon Européenne
\endverse

\beginverse
Qu’elle soit Chinoise ou Japonaise
Elle peut même être Américaine
De Saïgon de Tian’anmen
Tu sais moi mon Européenne
Elle peut venir de toutes les terres
Tant qu’elle me chante des missionnaires
Ouais c’est sûr elle a pas d’frontières
\endverse

\beginverse
Elle a le corps d’la Terre entière
\endverse

\beginsong{Mon gros loup, mon p'tit loup}[by={Henri Dès (2002)}]

\beginverse
Refrain
Je t'aime mon loup
Mon gros loup mon p'tit loup
Je t'aime mon loup
Mon gros loup p'tit loup
On dit que t'es mauvais
C'est pas vrai c'est pas vrai
On dit que t'es mauvais
C'est pas vrai pas vrai
Paraît que t'es méchant
C'est navrant c'est navrant
Paraît que t'es méchant
C'est navrant navrant
\endverse

\beginverse
Y a des hommes faut voir comme
Ils ont la dent dure
Dure dure dure dure dure
Ils ont la dent dure garanti sur facture
\endverse

\beginverse
Refrain
\endverse

\beginverse
Y a des dames c'est un drame
Qui portent un manteau
Teau teau teau teau teau
Qui portent un manteau de ta peau sur le dos
\endverse

\beginverse
Refrain
\endverse

\beginverse
Des chasseurs enfants de chœur
J'en n'ai pas connu nus nus nus nus nus
J'en n'ai pas connu
Et j'en n'ai jamais vus
\endverse

\beginverse
Refrain
\endverse

\beginsong{Mon héroïne}[by={Joyce Jonathan (2019)}]

\beginverse
Être une femme, être mille
Se créer toute ses nuances
Être une fille, pas facile
De vivre sa vie sous influence
Être viril, c’est oser,
Passer pour un garçon manqué
Être aimé, l’accepter
Et ne plus jamais regretter
\endverse

\beginverse
Et si je ne suis pas la meilleure
Que pour certains j’en fais trop
Je ne suis pas une erreur
Je n’y laisserai jamais ma peau
\endverse

\beginverse
Refrain
Je serais mon héroïne
Battante jusqu’à la fin du film
Une force qui surmonte
Les imbéciles de tous les contes
Je porterai mes idées sur le monde
Je serais mon héroïne
Vivante jusqu’à la fin du film,
Une femme qui affronte
Et qui n’aura plus jamais honte
Je porterai mes idées sur le monde
\endverse

\beginverse
Être aphone, à parler
Des heures durant des soirées
Être sûr, être soi
Jamais se laissé insulter
Être une fille, être une meuf,
Être plutôt trop que pas assez
Être vivante, attachante
Et accepter son passé
\endverse

\beginverse
Je ne serai toujours pas la meilleure
Pour certains j’en ferais toujours top
Je ne suis pas une erreur,
Et je n’y laisserais jamais la peau
\endverse

\beginverse
Refrain
\endverse

\beginverse
Une femme qui affronte
Qui n’aura plus jamais honte
Une force qui surmonte
\endverse

\beginverse
Tu seras mon héroïne \\[héroïne]
Battante jusqu’à la fin du film
Une force qui surmonte
Les imbéciles de tous les contes
Je porterai mes idées sur le monde
Tu seras mon héroïne
Vivante jusqu’à la fin du film,
Une femme qui affronte
Et qui n’aura plus jamais honte
Je porterai mes idées sur le monde
\endverse

\beginsong{Mon précieux}[by={Soprano (2016)}]

\beginverse
Ta douce mélodie me réveille chaque matin
Avant même d'embrasser ma femme je te prends par la main
Puis je te caresse le visage pour voir si tout va bien
Tellement inséparable qu'on part ensemble au petit coin
Mon café, mon jus d'orange on le partage aux amis
En voiture mes yeux sont dans les tiens donc quelques feux je grille
Au boulot on parle tellement ensemble que des dossiers j'oublie
Au diner vu le silence tout le monde t'aime dans ma famille, baby
\endverse

\beginverse
Refrain
Je te partage ma vie, au lieu de la vivre
Tu me partages la vie des autres pour me divertir
Je ne regarde plus le ciel depuis que tu m'as pris mes yeux dans tes applis, baby
Je ne sais plus vivre sans toi à mes cotés
Ton regard pixélisé m'a envoûté, toi mon précieux, mon précieux Mon précieux, mon précieux, mon précieux, mon précieux
Quand tu sonnes ou quand tu commences à vibrer
Je perds la tête , comment pourrais-je te quitter
Toi mon précieux, mon précieux, mon précieux
Mon précieux, mon précieux, mon précieux
\endverse

\beginverse
Tu es ma secrétaire , tu gères mon organisation
Tu allèges mes neurones grâce à tes notifications
Plus besoin d'aller voir la famille vu que tu me les follow
Pour leur prouver que je les aime, je n'ai qu'à liker leur photos
Pourquoi aller en concert, tu m'as tout mis sur Youtube
Tu m'aides à consommer car tu ne me parles qu'avec des pubs
J'fais plus gaffes à l'orthographe depuis que je te parle avec mes doigts
Mes gosses font plus de toboggan, ils préfèrent jouer avec toi, baby
\endverse

\beginverse
Refrain
\endverse

\beginverse
Mais là je deviens fou , l'impression que mon pouls ralenti
J'ai plus de repères , je suis perdu
Depuis que tu n'as plus de batterie
\endverse

\beginverse
Mon précieux , mon précieux , mon précieux \\[4x]
\endverse

\beginverse
Vous avez 39 nouvelles demandes d'amis
Vous avez 120 nouveaux likes
Vous n'avez pas vu vos amis depuis deux mois
Votre vie est digitale, LOL
\endverse

\beginsong{La montagne}[by={Jean Ferrat (1965)}]

\beginverse
Ils quittent un à un le pays
Pour s'en aller gagner leur vie
Loin de la terre où ils sont nés
\endverse

\beginverse
Depuis longtemps, ils en rêvaient
De la ville et de ses secrets
Du formica et du ciné
\endverse

\beginverse
Les vieux, ça n'était pas original
Quand ils s'essuyaient machinal
D'un revers de manche, les lèvres
\endverse

\beginverse
Mais ils savaient tous à propos
Tuer la caille ou le perdreau
Et manger la tomme de chèvre
\endverse

\beginverse
Refrain
Pourtant
Que la montagne est belle
Comment peut-on s'imaginer
En voyant un vol d'hirondelles
Que l'automne vient d'arriver?
\endverse

\beginverse
Avec leurs mains dessus leurs têtes
Ils avaient monté des murettes
Jusqu'au sommet de la colline
\endverse

\beginverse
Qu'importent les jours, les années
Ils avaient tous l'âme bien née
Noueuse comme un pied de vigne
\endverse

\beginverse
Les vignes, elles courent dans la forêt
Le vin ne sera plus tiré
C'était une horrible piquette
\endverse

\beginverse
Mais il faisait des centenaires
À ne plus que savoir en faire
S'il ne vous tournait pas la tête
\endverse

\beginverse
Refrain
\endverse

\beginverse
Deux chèvres et puis quelques moutons
Une année bonne et l'autre non
Et sans vacances, et sans sorties
\endverse

\beginverse
Les filles veulent aller au bal
Il n'y a rien de plus normal
Que de vouloir vivre sa vie
\endverse

\beginverse
Leur vie, ils seront flics ou fonctionnaires
De quoi attendre sans s'en faire
Que l'heure de la retraite sonne
\endverse

\beginverse
Il faut savoir ce que l'on aime
Et rentrer dans son HLM
Manger du poulet aux hormones
\endverse

\beginverse
Refrain
\endverse

\beginsong{Morgane de toi}[by={Renaud (1983)}]

\beginverse
Y'a un mariole qu'a au moins quatre ans
Y veut t'piquer ta pelle et ton seau
Ta couche-culotte avec les bombecs dedans
Lolita défends-toi, fous-y un coup d'râteau dans l'dos
Attends un peu avant d'te faire emmerder
Par ces p'tits machos qui pensent qu'à une chose
Jouer au docteur non-conventionné
J'y ai joué aussi je sais de quoi j'cause
J'les connais bien les play-boys des bacs à sable
J'draguais leur mère avant d'connaître la tienne
Si tu les écoutes y t'feront porter leur cartable
Heureusement que j'suis là, que j'te regarde et que j't'aime
\endverse

\beginverse
Refrain
Lola, j'suis qu'un fantôme
Quand tu vas où j'suis pas
Tu sais ma môme que j'suis morgane de toi.
\endverse

\beginverse
Comme j'en ai marre de m'faire tatouer des machins
Qui m' font comme une bande dessinée sur la peau
J'ai écrit ton nom avec des clous dorés plantés un à un
Dans l'cuir de mon blouson dans l'dos
T'es la seule gonzesse que j'peux t'nir dans mes bras
Sans m'démettre une épaule, sans plier sous ton poids
Tu pèses moins lourd qu'un moineau qui mange pas
Déploie jamais tes ailes Lolita, t'envole pas
Avec tes miches de rat qu'on dirait des noisettes
Et ta peau plus sucrée qu'un pain de chocolat
Tu risques de donner faim à un tas de p'tits mecs
Quand t'iras à l'école si jamais t'y vas.
\endverse

\beginverse
Refrain
\endverse

\beginverse
Qu'est-ce qu'tu m' racontes, tu veux un p'tit frangin
Tu veux que j't'achète un ami Pierrot
Eh les bébés ça s'trouve pas dans les magasins
Et je crois pas qu'ta mère voudrait qu'j'lui fasse un petit dans l'dos.
Ben quoi Lola, on n'est pas bien ensemble ?
Tu crois pas qu'on est déjà bien assez nombreux
T'entends pas ce bruit, c'est le monde qui tremble
Sous les cris des enfants qui sont malheureux
Allez viens avec moi j T'embarque dans ma galère
Dans mon arche y a d'la place pour tous les marmots
Avant que ce monde devienne un grand cimetière
Faut profiter du vent qu'on a dans l'dos.
\endverse

\beginsong{My Bonnie }[by={Tony Sheridan & The Beatles (1964)}]

\beginverse
My Bon\\[Sol]nie lies ov\\[Do]er the oc\\[Sol]ean
My Bonnie lies ov\\[la7]er the se\\[Ré7]a
Well, my Bo\\[Sol]nnie lies ov\\[Do]er the oc\\[Sol]ean
Yeah, bri\\[La7]ng back my Bon\\[Ré7]nie to m\\[Sol]e
Yeah, bri\\[Sol]ng ba\\[Sol7]ck, ah, br\\[Do]ing ba\\[La7]ck
Oh, br\\[Ré7]ing back my Bonnie to m\\[Sol]e, to me
Ah, bri\\[Sol]ng, oh, bring ba\\[Sol7]ck, ah, b\\[Do]ring ba\\[La7]ck
Oh, br\\[Ré7]ing back my Bonnie to m\\[Sol]e
\endverse

\beginverse
Well, my Bonnie lies over the ocean
My Bonnie lies over the sea
Yeah, my Bonnie lies over the ocean
Oh, I said bring back my Bonnie to me
\endverse

\beginverse
Yeah, bring back, ah, bring back
Oh, bring back my Bonnie to me, to me
Oh, bring back, ah, bring back
Oh, bring back my Bonnie to me
\endverse

\beginsong{Les mystérieuses Cités d'or}[by={Jacques Cardona (1982)}]

\beginverse
Enfant du Soleil,
Tu parcours la Terre, le Ciel,
Cherches ton chemin,
C'est ta vie, c'est ton destin.
\endverse

\beginverse
Et le jour, la nuit,
Avec tes deux meilleurs amis,
A bord du Grand Condor,
Tu recherches les Cités d'Or.
\endverse

\beginverse
Ah, ah, ah, ah, ah,
Esteban, Zia, Tao, les Cités d'Or
Ah, ah, ah, ah, ah,
Esteban, Zia, Tao, les Cités d'Or
\endverse

\beginverse
Les Cités d'Or, les Cités d'Or
\endverse

\beginverse
Enfant du soleil,
Ton destin est sans pareil,
L'aventure t'appelles,
N'attends pas et cours vers elle.
\endverse

\beginverse
Ah, ah, ah, ah, ah,
Esteban, Zia, Tao, les Cités d'Or.
\endverse

\beginsong{Ne pleure pas Jeanette\*}[by={Henri Dès (1990)}]

\beginverse
Ne pleure pas Jeannette
Tra la la la la la la la
Ne pleure pas Jeannette
Nous te marierons
Nous te marierons
\endverse

\beginverse
Avec le fils d'un prince
Ou celui d'un baron
\endverse

\beginverse
Je ne veux pas de prince
Encor’ moins d'un baron
\endverse

\beginverse
Je veux mon ami Pierre
Qui est dans la prison
\endverse

\beginverse
Tu n'auras pas ton Pierre
Nous le pendouillerons
\endverse

\beginverse
Si vous pendouillez Pierre
Pendouillez-moi avec aussi
\endverse

\beginverse
Et l'on pendouilla Pierre
Et sa Jeannette avec aussi
\endverse

\beginverse
Sur la plus haute branche
Un rossignol chantait
\endverse

\beginsong{Ne sens-tu pas claquer tes doigts ?}[by={Traditionnel}]

\beginverse
Ne sens-tu pas claquer tes doigts ?
La musique monter en toi ?
Jusqu'à ce que le feu soit mort
Ne veux-tu pas chanter encore ?
\endverse

\beginverse
Ne sens-tu pas battre ton cœur
Qui s'éparpille en mille fleurs ?
Et prends la main de ton ami
Regarde-le et puis souris
\endverse

\beginverse
Ne vois-tu pas ? Le soleil brille
Dans le cœur des garçons et filles
Ne veux-tu pas chanter encore
Jusqu'à ce que le feu soit mort ?
\endverse

\beginverse
Ne sens-tu pas cligner tes yeux
Et le sommeil monter en toi ?
Jusqu'à ce que le jour se lève
Ne veux-tu pas chanter encore ?
\endverse

\beginverse
La route s'ouvre devant toi
Tu marcheras toujours tout droit
Et l'amitié te mènera
Dans un pays rempli de joie
\endverse

\beginverse
Ne sens-tu pas cet air de joie
Monter en toi, se fondre en toi ?
Oublie tes peines et tes chagrins
Pour ne penser qu'à ce refrain
\endverse

\beginverse
Ne sens-tu pas chauffer tes pieds
La fatigue monter en toi ?
Jusqu'à ce que le soir soit là
Ne veux-tu pas marcher encore ?
\endverse

\beginsong{Non, non, rien n’a changé}[by={Les Poppys (1971)}]

\beginverse
C'est l'histoire d'une trêve que j'avais demandée,
C'est l'histoire d'un soleil que j'avais espéré,
C'est l'histoire d'un amour que je croyais vivant,
C'est l'histoire d'un beau jour que moi petit enfant,
Je voulais très heureux, pour toute la planète,
Je voulais, j'espérais que la paix règne en maître
En ce soir de Noël, (mais tout a continué) \\[3x]
\endverse

\beginverse
Refrain
Non, non, rien a changé, tout, tout a continué,
Non, non, rien a changé, tout, tout a continué, eh, eh.
\endverse

\beginverse
Et pourtant bien des gens ont chanté avec nous,
Et pourtant bien des gens se sont mis à genoux,
Pour prier, oui pour prier \\[bis]
Et j'ai vu tous les jours à la télévision
Même le soir de Noël, des fusils, des canons
J'ai pleuré, oui, j'ai pleuré \\[bis]
Qui pourra m'expliquer ? Que…
\endverse

\beginverse
Refrain
\endverse

\beginverse
Moi je pense à l'enfant entouré de soldats,
Moi je pense à l'enfant qui demande pourquoi,
Tout le temps, oui, tout le temps \\[bis]
Moi je pense à tout ça, mais je ne devrais pas,
Et pourtant, oui et pourtant \\[bis]
Je chante, je chante...
\endverse

\beginsong{Nous}[by={Julien Doré (2020)}]

\beginverse
Nous, on ira voir la mer
Voir si les gens sont fiers
Imaginer monter l'eau
Bien qu'on n'ait rien su faire
On n'a plus rien à perdre
Un peu de ventre et d'égo
Et quelques langues à défaire
Pour les revoir se plaire
\endverse

\beginverse
Refrain
Nous, nous, nous
Nous on s'en fout de vous
Vous pouvez prendre tout
Tant qu'on est tendre nous
Nous, nous, nous
Nous on s'en fout de vous
Vous pouvez prendre tout
Tant qu'on est tendre nous
\endverse

\beginverse
Nous on ira voir la mer
Voir si la lune éclaire
De quelques têtes hors de l'eau
Un monde où tout se perd
Demain c'est juste hier
Un peu laissé sur le dos
Un peu blessé par les pierres
Qu'on n'a pas voulu perdre
\endverse

\beginverse
Refrain
\endverse

\beginverse
Nous, nous, nous
Nous
\\[bis]
\endverse

\beginsong{Nuit et brouillard}[by={Jean Ferrat (1963)}]

\beginverse
Ils étaient vingt et cent, ils étaient des milliers
Nus et maigres, tremblant dans des wagons plombés
Qui déchiraient la nuit de leurs ongles battants
Ils étaient des milliers, ils étaient vingt et cent.
Ils se croyaient des hommes, n'étaient plus que des nombres
Depuis Longtemps leurs dés avaient été gelés
Dès que la main retombe, il ne reste qu'une ombre
Ils ne devaient jamais plus revoir un été.
\endverse

\beginverse
La fuite monotone et sans hâte du temps
Survivre encore un jour, une heure obstinément
Combien de tours de roues, d'arrêts et de départs
Qui n'en finissent pas de distiller l'espoir
IIs s'appelaient Jean-Pierre, Natacha ou Samuel
Certains priaient Jésus, Jéhovah ou Vichnou,
D'autres ne priaient pas, mais qu'importe le ciel
Ils voulaient simplement ne plus vivre à genoux.
\endverse

\beginverse
Ils n'arrivaient pas tous à la fin du voyage
Ceux qui sont revenus, peuvent-ils être heureux ?
Ils essaient d'oublier, étonnės qu'à leur age
Les veines de leurs bras soient devenues si bleues.
Les Allemands guettaient du haut des miradors
La lune se taisait comme vous vous taisiez
En regardant au loin, en regardant dehors
Votre chair était tendre à leurs chiens policiers
\endverse

\beginverse
On me dit à présent que ces mots n'ont plus cours
Qu'il vaut mieux ne chanter que des chansons d’amour
Que le sang sèche vite en entrant dans l'histoire
Et qu'il ne sert à rien de prendre une guitare
Mais qui donc est de taille à pouvoir m'arrêter
L'ombre s'est faite humaine, aujourd'hui c'est l'été
Je twisterais les mots s'il fallait les twister
Pour qu'un jour les enfants sachent qui vous étiez.
\endverse

\beginverse
Vous étiez vingt et cent, vous étiez des milliers
Nus et maigres, tremblant dans des wagons plombés
Qui déchiriez la nuit de vos ongles battants
Vous étiez des milliers, vous étiez vingt et cent.
\endverse

\beginsong{L’oiseau}[by={Cécile Aubry (1968)}]

\beginverse
Je connais les brumes claires
La neige rose des matins d'hiver
Je pourrais te retrouver
Le lièvre blanc qu'on ne voit jamais
Mais l'oiseau, l'oiseau s'est envolé
Et moi jamais je ne le retrouverai
Car j'ai vu l'oiseau voler
J'ai vu l'oiseau je sais qu'il partait
Je l'ai entendu pleurer
Le bel oiseau que le vent chassait.
\endverse

\beginverse
Je voudrais tout te donner
Mais toi pourquoi ne me dis-tu rien ?
Quel est-il ton grand secret
Un secret d'homme je le comprends bien
Moi tu sais je peux le raconter
Combien l'oiseau est parti à regret
Si un jour tu m'écoutais
Tu apprendrais tout ce que je sais
L'oiseau part et puis revient
Tu le verras peut-être demain.
\endverse

\beginverse
Si jamais je rencontrais
Le bel oiseau qui s'est envolé
S'il revient de son voyage
Tout près de toi le long du rivage
Moi vois-tu je lui raconterai
Combien pour toi je sais qu'il a compté
C'est l'oiseau que tu aimais
L'oiseau jaloux je l'ai deviné
Si jamais il revenait
Je lui dirai que tu l'attendais.
\endverse

\beginsong{L’oiseau et l’enfant}[by={Marie Myriam (1977)}]

\beginverse
Comme un enfant aux yeux de lumière
Qui voit passer au loin les oiseaux
Comme l'oiseau bleu survolant la terre
Vois comme le monde, le monde est beau.
\endverse

\beginverse
Beau le bateau dansant sur les vagues
Ivre de vie, d'amour et de vent
Belle, la chanson naissante des vagues
Abandonnée au sable blanc.
\endverse

\beginverse
Blanc l'innocent, le sang du poète
Qui en chantant invente l'amour
Pour que la vie s'habille de fête
Et que la nuit se change en jour.
\endverse

\beginverse
Jour d'une vie où l'aube se lève
Pour réveiller la ville aux yeux lourds
Où les matins effeuillent les rêves
Pour nous donner un monde d'amour.
\endverse

\beginverse
Refrain
L'amour c'est toi, L'amour c'est moi
L'amour c'est toi, L'enfant c'est moi.
\endverse

\beginverse
Moi, je ne suis qu'une fille (qu'un homme) de l'ombre
Qui voit briller l'étoile du soir
Toi, mon étoile, qui tisses ma ronde
Viens allumer mon soleil noir.
\endverse

\beginverse
Noire la misère, les hommes et la guerre
Qui croient tenir les rênes du temps
Pays d'amour n'a pas de frontières
Pour ceux qui ont un coeur d’enfant.
\endverse

\beginverse
Comme un enfant aux yeux de lumière
Qui voit passer au loin les oiseaux
Comme l’oiseau bleu survolant la terre
Nous trouverons ce monde amour.
\endverse

\beginverse
Refrain
\endverse

\beginsong{On a qu'une terre}[by={Stress (2006)}]

\beginverse
Quand il sera grand, il m'demandera "pourquoi y a plus d'poissons dans la mer?"
Je vais dire quoi? Qu'j'savais pas ou qu'j'en avais rien à faire
Et quand il demandera "Papa, est-ce juste pour le bois"
"Qu'vous avez rasé l'poumon d'la planète? J'vais respirer avec quoi?"
J'aurais l'air d'un irresponsable, incapable, un coupable au comportement inexcusable
\endverse

\beginverse
Une nature bousillée, un monde de CO2
Est-ce vraiment le futur qu'on voulait construire pour eux?
Ça commence par l'respect et l'une des choses à faire
C'est un commerce équitable pour eux, nous et notre Terre
Les grands discours c'est bien mais les petits gestes c'est mieux
La différence, on doit la faire aujourd'hui car on le peut
\endverse

\beginverse
Refrain
Vas-y consomme, consomme, consume, consume
Tronçonne, tronçonne, allume, allume
Mais que fais-tu si notre futur s'retrouve entre l'marteau et l'enclume?
Si ça brûle et qu'ça s'consume et qu'notre Terre ressemble à la lune
Que fais-tu si notre futur s'retrouve entre l'marteau et l'enclume?
\endverse

\beginverse
Dites-moi pas qu'vous l'voyez pas, qu'vous l'sentez pas?
Ce changement, ne me mentez pas
Le climat part en vrille, vous attendez quoi?
Combien d'Katrina nous faudra-t-il pour accepter ça?
J'veux pas marcher sur le sol d'une mer asséchée
En m'disant "j'aurais peut-être dû trier mes déchets"
\endverse

\beginverse
À nos yeux, c'est une erreur, aux yeux d'nos enfants, un péché
Tout l'monde crie au drame mais personne n'a l'air pressé
J'veux pas voir le jour où l'eau aura la valeur du pétrole
Où l'pétrole ne sera plus mais on paiera encore pour ces bémols
J'suis pas devenu "monsieur écolo", c'est clair
Mais avec c'que je sais aujourd'hui, je peux faire mieux qu'hier
\endverse

\beginverse
Refrain \\[bis]
\endverse

\beginsong{On écrit sur les murs}[by={Kids United (2015)}]

\beginverse
On écrit sur les murs le nom de ceux qu’on aime
Des messages pour les jours à venir
On écrit sur les murs à l ‘encre de nos veines
On dessine tout ce que l’on voudrait dire
\endverse

\beginverse
Partout autour de nous,
Y’a des signes d’espoir dans les regards
Donnons-leur un cri, car dans la nuit
Tout s’efface même leur trace
\endverse

\beginverse
Refrain 
On écrit sur les murs le nom de ceux qu’on aime
Des messages pour les jours à venir
On écrit sur les murs à l ‘encre de nos veines
On dessine tout ce que l’on voudrait dire
On écrit sur les murs la force de nos rêves
Nos espoirs en forme de graffitis
On écrit sur les murs pour que l’amour se lève
Un beau jour sur le monde endormi
\endverse

\beginverse
Des mots seulement gravés
Pour ne pas oublier pour tout changer
Mélangeons demain dans un refrain
Nos visages, métissages
\endverse

\beginverse
Refrain 
\endverse

\beginverse
On écrit sur les murs le nom de ceux qu’on aime
Des messages pour les jours à venir
On écrit sur les murs à l ‘encre de nos veines
On dessine tout ce que l’on voudrait dire
\endverse

\beginverse
Refrain 
Un beau jour sur le monde endormi
\endverse

\beginsong{On ira}[by={Zaz (2013)}]

\beginverse
On ira écouter Harlem au coin de Manhattan
On ira rougir les thés dans les souks à Amman
On ira nager dans le lit du fleuve Sénégal
Et on verra brûler Bombay sous un feu de Bengale
\endverse

\beginverse
On ira gratter le ciel en-dessous de Kyoto
On ira sentir Rio battre au coeur de Janeiro
On lèvera nos yeux sur le plafond de la chapelle Sixtine
Et on lèvera nos verres dans le café Pouchkine
Ahahah
\endverse

\beginverse
Refrain
Oh qu'elle est belle notre chance
Aux milles couleurs de l'être humain
Mélangées de nos différences
A la croisée des destins
\endverse

\beginverse
Vous êtes les étoiles, nous sommes l'univers
Vous êtes un grain de sable, nous sommes le désert
Vous êtes mille pages et moi je suis la plume
Oohohohohohohoh
\endverse

\beginverse
Vous êtes l'horizon et nous sommes la mer
Vous êtes les saisons et nous sommes la Terre
Vous êtes le rivage et moi je suis l'écume
Oohohohohohohoh
\endverse

\beginverse
On dira que les poètes n'ont pas de drapeaux
On fera des jours fêtes autant qu'on a de héros
On saura que les enfants, sont les gardiens de l'âme
Et qu'il y a des Reines autant qu'il y a de femmes
\endverse

\beginverse
On dira que les rencontres font les plus beaux voyages
On verra qu'on ne mérite que ce qui se partage
On entendra chanter des musiques d'ailleurs
Et l'on saura donner, ce qu'on a de meilleur
\endverse

\beginverse
Refrain
\endverse

\beginverse
Vous êtes les étoiles, nous sommes l'univers
Vous êtes un grain de sable, nous sommes le désert
Vous êtes mille pages et moi je suis la plume
Oohohohohohohoh
\endverse

\beginverse
Vous êtes l'horizon et nous sommes la mer
Vous êtes les saisons et nous sommes la Terre
Vous êtes le rivage et moi je suis l'écume
Oohohohohohohoh
(bis 2 dernier couplet)
\endverse

\beginsong{On s'attache }[by={Christophe Maé (2007)}]

\beginverse
J'ai pas le style
Pourtant pas hostile
Mais c'est pas pour moi le costard uniforme
\endverse

\beginverse
J'ai pas l'intégral
Du gendre idéal
J'aurais toujours l'impression qu'on m'espionne
\endverse

\beginverse
Refrain
Pourtant pas contre l'amour
Je s'rais même plutôt pour
Mais c'est pas pour autant qu'il faut
Qu'on s'attache
Et qu'on s'empoisonne
Avec une flèche
Qui nous illusionnes
Faut pas qu'on s'attache
Et qu'on s'emprisonnes
Mais rien n'empêche
Que l'on s'abandonne, non \!
\endverse

\beginverse
D'un chef de file
J'en ai pas l'profil
Mais sur l'oreiller j'aime pas qu'on me questionne
\endverse

\beginverse
Je suis pas James Bond
Entouré de belles blondes
J'envie même pas les hommes
Qui papillonnent
\endverse

\beginverse
Refrain
\endverse

\beginverse
Qu'on s'attache
Et qu'on s'emprisonne
Mais rien n'empêche
Que l'on s'abandonne, non
\endverse

\beginverse
On le sait bien
Le quotidien ça nous tue, ça nous tient
Ca nous fait mal
Rien de plus normal
Mais tu t'enfiles dans la file
Mais faut pas qu'tu dépasses
A chaque fois, qu'tu resquilles, mais t'es qui ?
T'es pas normal
\endverse

\beginverse
On s'attache
Et on s'empoisonne
Avec une flèche
Qui nous illusionne, faut pas, non
\endverse

\beginverse
Qu'on s'attache
Et qu'on s'empoisonne
Mais rien n'empêche
Que l'on s'abandonne, non \!
\endverse

\beginverse
On s'attache
Et qu'on s'empoisonne
Avec une flèche
Qui nous illusionne, faut pas, non
\endverse

\beginverse
Qu'on s'attache
Et qu'on s'empoisonne
Mais rien n'empêche
Que l'on s'abandonne, non, non \!
\endverse

\beginverse
Non, je sais pas faire comme
Après tout je suis qu'un homme
Non je veux pas vivre comme
Dans la case de l'oncle Tom
En papa Dalton
Me retrouver dans un album
\endverse

\beginverse
Après tout je ne suis qu'un homme
\endverse

\beginsong{Papa ô papa}[by={Jean-Claude Darnal (1960)}]

\beginverse
Y'avait dans l'désert qui conduit tout là-bas
Un homme rude et fort qui marchait à grands pas
Derrière un p'tit gars lui emboitait le pas
Mais n'y arrivait pas.
Papa, ô papa, attends-moi, je n'peux pas
Papa, si tu vas à grands pas, faire ton pas
Un pas c'est un pas, mais ton pas je n'l'ai pas
Papa, ô papa, je n'peux pas.
\endverse

\beginverse
Parfois le bonhomme s'arrêtait pour laisser
Laisser au gamin le temps de l'attraper
Sitôt qu'ils étaient à nouveau rassemblés
Alors ils repartaient
Papa, ô papa, attends-moi, je n'peux pas
Viens-là mon p'tit gars, t'en fais pas prends mon pas
Un pas, c'est un pas, mais ton pas, je n'l'ai pas
Viens-là, mon gars, ne t'en fais pas.
\endverse

\beginverse
La marche avançait, mais le temps défilait Notre homme vieillissait, le gamin grandissait Son pas s'allongea, et maint'nant pas à pas
Il s'en allaient là-bas.
Papa, ô papa, regard'moi faire mon pas
Papa, si tu vas à grands pas, 'fais comme toi
Un pas, c'est un pas, et ton pas, c'est mon pas
Papa, ô papa, je fais ton pas.
\endverse

\beginverse
Y'avait dans l'désert qui conduit tout là-bas
Un homme rude et fort qui marchait à grands pas
Laissant derrièr lui un vieillard qui a dit :
« Adieu, j'arrête ici.
Va, va, mon p'tit gars, d'un bon pas, n'attends pas
Mon pas est trop las, va tout seul vers là-bas
Un jour tu verras un gamin qui suivra
Ton pas et le rattrapera
Un jour tu verras un gamin qui suivra
Ton pas et le dépassera. »
\endverse

\beginsong{Papaoutai}[by={Stromaé (2013)}]

\beginverse
dites-moi d’où il vient
enfin je saurai où je vais
maman dit que lorsqu’on cherche bien
on finit toujours par trouver
elle dit qu’il n’est jamais très loin
qu’il part très souvent travailler
maman dit: “travailler c’est bien”
bien mieux qu’être mal accompagné, pas vrai ?
\endverse

\beginverse
Refrain
où est ton papa
dis-moi où est ton papa
sans même devoir lui parler il sait ce qui ne va pas
ah sacré papa
dis-moi où es-tu caché
ça doit faire au moins mille fois
que j’ai compté mes doigts et
\endverse

\beginverse
où t’es papa où t’es ?
où t’es papa où t’es ?
où t’es papa où t’es ?
où t’es où t’es où papa où t’es ?
\\[bis]
\endverse

\beginverse
quoi, qu’on y croie ou pas
il y aura bien un jour où on n’y croira plus
un jour ou l’autre on sera tous papa
et d’un jour à l’autre on aura disparu
serons-nous détestables ?
serons-nous admirables ?
des géniteurs ou des génies
dites-nous qui donne naissance aux irresponsables
\endverse

\beginverse
hein, dites-nous qui, tiens
tout le monde sait comment on fait des bébés
mais personne ne sait comment on fait des papas
monsieur “je sais tout” en aurait hérité ?
c’est ça, faut l’sucer de son pouce ou quoi ?
dites-nous où c’est caché
ça doit faire au moins mille fois
qu’on a bouffé nos doigts et
\endverse

\beginverse
où t’es papa où t’es ?
où t’es papa où t’es ?
où t’es papa où t’es ?
où t’es où t’es où papa où t’es ?
\\[bis]
\endverse

\beginverse
Refrain\\[bis]
\endverse

\beginverse
où t’es papa où t’es ?
où t’es papa où t’es ?
où t’es papa où t’es ?
où t’es où t’es où papa où t’es ?
\\[Bis]
\endverse

\beginsong{Pelot d'Hennebont\*}[by={Tri Yann (1974)}]

\beginverse
Ma chère Maman je vous écris
Que nous sommes entrés dans Paris,
Que j'étions déjà caporal,
Et j'serions bientôt général.
\endverse

\beginverse
A la bataille je combattions
Les ennemis de la nation,
Et tous ceux qui se présentions,
A grands coups d'sabre, j'les émondions.
\endverse

\beginverse
Le Roi Louis m'a z'appelé,
C'est «sans quartier» qu'il m'a nommé,
« Sans quartier », c'est point mon nom,
J'lui dis \['m'appelle Pelot d'Hennebont.
\endverse

\beginverse
Il a quéri z'un biau ruban
Et je n'sais quoi z'au bout d'argent,
M'a dit boute ça d'ssus ton habit,
Et combats toujours l'ennemi.
\endverse

\beginverse
Faut qu'ce soye que que chose de précieux
Pour les aut' m'appellent Monsieur,
Et boutent lou main à lou chapiau,
Quand ils veulent conter au Pelot.
\endverse

\beginverse
Ma Mère, si j'meurs en combattant,
J'vous enverrai ce biau ruban,
Et vous l'courrez à vot' fusiau,
En souvenir du gars Pelot.
\endverse

\beginsong{Le pénitencier}[by={Johnny Hallyday (1964)}]

\beginverse
Les portes du pénitencier
Bientôt vont se refermer
Et c'est là que je finirai ma vie
Comm'd'autres gars l'ont finie
\endverse

\beginverse
Pour moi ma mère m'a tout donné
Sa robe de mariée
Peux-tu jamais me pardonner
Je t'ai trop fait pleurer
\endverse

\beginverse
Le soleil n'est pas fait pour nous
C'est la nuit qu'on peut tricher
Toi qui ce soir as tout perdu
Demain tu peux gagner
\endverse

\beginverse
O mères écoutez-moi
Ne laissez jamais vos garçons
Seuls la nuit traîner dans les rues
Ils iront tout droit en prison
\endverse

\beginverse
Toi la fille qui m'as aimé
Je t'ai trop fait pleurer
Les larmes de honte que tu as versées
Il faut les oublier
\endverse

\beginverse
Les portes du pénitencier
Bientôt vont se refermer
Et c'est là que je finirai ma vie
Comm' d'autres gars l'ont finie
\endverse

# 

\beginsong{Le petit âne gris}[by={Hugues Aufray (1968)}]

\beginverse
Ecoutez cette histoire
Que l'on m'a raconté
Du fond de ma mémoire
Je vais vous la chanter
Elle se passe en Provence
Au milieu des moutons
Dans le sud de la France bis
Au pays des santons
(bis 2 der)
\endverse

\beginverse
Quand il vint au domaine
Y avait un beau troupeau
Les étables étaient pleines
De brebis et d'agneaux
Marchant toujours en tête
Aux premières lueurs
Pour tirer sa charrette
Il mettait tout son cabur
(bis 2 der)
\endverse

\beginverse
Au temps des transhumances
Il s'en allait heureux
Remontant la Durance
Honnête et courageux
Mais un jour de Marseille
Des messieurs sont venus
La ferme était bien vieille
Alors on l'a vendue
(bis 2 der)
\endverse

\beginverse
Il resta au village
Tout le monde l'aimait bien
Vaillant malgré son âge
Et malgré son chagrin
Image d'évangile
Vivant d'humilité
Il se rendait utile
Auprès du cantonnier.
(bis 2 der)
\endverse

\beginverse
Cette vie honorable
Un soir s'est terminée
Dans le fond d'une étable
Tout seul il s'est couché
Pauvre bête de somme
Il a fermé les yeux
Abandonné des hommes
Il est mort sans adieux.
(bis 2 der)
\endverse

\beginverse
Cette chanson sans gloire
Vous racontait la vie
Vous racontait l'histoire
D'un petit âne gris.
\endverse

\beginsong{Le petit bonheur}[by={Félix Leclerc (1948)}]

\beginverse
C’est un petit bonheur
Que j’avais ramassé
Il était tout en pleurs
Sur le bord d’un fossé
Quand il m’a vu passer
Il s’est mis à crier:
"Monsieur, ramassez-moi
Chez vous emmenez-moi".
Mes frères m’ont oublié, je suis tombé, je suis malade
Si vous n’me cueillez point, je vais mourir, quelle ballade \!
Je me ferai petit, tendre et soumis, je vous le jure
Monsieur, je vous en prie, délivrez-moi de ma torture".
\endverse

\beginverse
J’ai pris le p’tit bonheur
L’ai mis sous mes haillons
J’ai dit: " Faut pas qu’il meure,
Viens-t’en dans ma maison".
Alors le p’tit bonheur
A fait sa guérison
Sur le bord de mon coeur
Y avait une chanson.
Mes jours, mes nuits, mes peines, mes deuils, mon mal, tout fut oublié;
Ma vie de désoeuvré, j’avais dégoûts d'la recommencer
Quand il pleuvait dehors ou qu’mes amis m’faisaient des peines,
J’prenais mon p’tit bonheur et j’lui disais: "C’est toi ma reine".
\endverse

\beginverse
Mon bonheur a fleuri,
Il a fait des bourgeons.
C’était le paradis,
Ça s’voyait sur mon front.
Or un matin joli
Que j’sifflais ce refrain,
Mon bonheur est parti
Sans me donner la main.
J’eus beau le supplier, le cajoler, lui faire des scènes,
Lui montrer le grand trou qu’il me faisait au fond du coeur,
Il s’en allait toujours, la tête haute, sans joie, sans haine,
Comme s’il ne pouvait plus voir le soleil dans ma demeure.
\endverse

\beginverse
J’ai bien pensé mourir
De chagrin et d’ennui,
J’avais cessé de rire
C’était toujours la nuit.
Il me restait l’oubli,
Il me restait l’mépris,
Enfin que j’me suis dit:
"Il me reste la vie".
J’ai repris mon bâton, mes deuils, mes peines et mes guenilles,
Et je bats la semelle dans des pays de malheureux.
Aujourd’hui quand je vois une fontaine ou une fille,
Je fais un grand détour ou bien je me ferme les yeux...\\[Bis].
\endverse

\beginsong{La petite fugue}[by={Catherine, Maxime Le Forestier (1969)}]

\beginverse
Refrain
C'était toujours la même
Mais on l'aimait quand même
La fugue d'autrefois
Qu'on jouait tous les trois
On était malhabile
Elle était difficile
La fugue d'autrefois
Qu'on jouait tous les trois
\endverse

\beginverse
Éléonore attaquait le thème au piano
On trouvait ça tellement beau
Qu'on en oubliait de jouer pour l'écouter
Elle s'arrêtait brusquement et nous regardait
Du haut de son tabouret
Elle disait : "Reprenez mi fa mi fa mi ré"
\endverse

\beginverse
Refrain
\endverse

\beginverse
Souviens-toi qu'un violon fut jeté sur le sol
Car c'était toujours le sol
Qui gênait Nicolas quand il était bémol
Quand les voisins commençaient à manifester
C'était l'heure du goûter
Salut Jean-Sébastien et à jeudi prochain
\endverse

\beginverse
Refrain
\endverse

\beginverse
Un jour Éléonore a quitté la maison
Emportant le diapason
Depuis ce jour nous n'accordons plus nos violons
L'un après l'autre nous nous sommes dispersés
La fugue seule est restée
Et chaque fois que je l'entends c'est le printemps
\endverse

\beginverse
Refrain
\endverse

\beginsong{Petite Marie}[by={Francis Cabrel (1977)}]

\beginverse
Petite Marie, je parle de toi
Parce qu'avec ta petite voix
Tes petites manies, tu as versé sur ma vie
Des milliers de roses
\endverse

\beginverse
Petite furie, je me bats pour toi
Pour que dans dix mille ans de ça
On se retrouve à l'abri, sous un ciel aussi joli
Que des milliers de roses
\endverse

\beginverse
Refrain
Je viens du ciel et les étoiles entr' elles
Ne parlent que de toi
D'un musicien qui fait jouer ses mains
Sur un morceau de bois
De leur amour plus bleu que le ciel autour
\endverse

\beginverse
Petite Marie, je t'attends transi
Sous une tuile de ton toit
Le vent de la nuit froide me renvoie la ballade
Que j'avais écrite pour toi
\endverse

\beginverse
Petite furie, tu dis que la vie
C'est une bague à chaque doigt
Au soleil de Floride, moi mes poches sont vides
Et mes yeux pleurent de froid
\endverse

\beginverse
Refrain
\endverse

\beginverse
Dans la pénombre de ta rue
Petite Marie, m'entends-tu ?
Je n'attends plus que toi pour partir...
Dans la pénombre de ta rue
Petite Marie, m'entends-tu ?
Je n'attends plus que toi pour partir...
\endverse

\beginverse
Refrain
\endverse

\beginsong{La Piémontaise}[by={Traditionnel (1705)}]

\beginverse
Mon Dieu que je suis à mon aise
Quand j'ai ma mie auprès de moi, auprès de moi
De temps en temps je la regarde
En lui disant: « Embrasse-moi, embrasse-moi »
(bis 2 der)
\endverse

\beginverse
Comment veux-tu que je t'embrasse
Quand on me dit du mal de toi, du mal de toi
On dit que tu pars pour la guerre
Dans le Piémont, servir le Roi, servir le Roi
(bis 2 der)
\endverse

\beginverse
Ceux qui t'ont dit cela ma belle
Ils t'ont bien dit la vérité, la vérité
Mon cheval est à l'écurie
Sellé, bridé, prêt à partir, prêt à partir
(bis 2 der)
\endverse

\beginverse
Quand tu seras dans ces campagnes
Tu ne penseras plus à moi, non plus à moi
Tu trouveras des Piémontaises
Qui sont cent fois plus belles que moi, plus belles que moi
(bis 2 der)
\endverse

\beginverse
Si fait, si fait, si fait ma belle
Je penserai toujours à toi, toujours à toi
Je ferai faire une peinture
Toute à la semblance de toi, semblance de toi
(bis 2 der)
\endverse

\beginverse
Quand je serai z'à table à boire
A mes compagnons je dirai, je dirai :
« Chers compagnons venez voir
Celle que mon coeur a tant aimée, a tant aimé »
(bis 2 der)
\endverse

# 

\beginsong{Le pingouin}[by={Marie Henchoz (1991)}]

\beginverse
Un pingouin du Pôle Nord
Un beau jour dit : J’en ai assez \!
Donnez-moi un passeport
Et je m’en vais sous les cocotiers.
\endverse

\beginverse
Refrain 
Chez moi, il fait froid, et j’ai les pieds gelés.
Y’a pas de soleil et je ne peux pas bronzer.
\endverse

\beginverse
Je dois mettre un cache-nez,
La chaleur est frigorifique,
Je m’abîme la santé
Et j’ai besoin de fruits exotiques.
\endverse

\beginverse
Refrain
\endverse

\beginverse
Le pingouin a pris le train,
Direction les îles Hawaï.
Et enfin, un matin,
A débarqué chez les ouistitis.
\endverse

\beginverse
Refrain
\endverse

\beginverse
C’est bien chaud les tropiques :
Je ne peux ôter mon habit,
Les fruits donnent la colique.
Me voilà cloué au fond du lit.
Mon Dieu, je transpire, je suis raplapla,
Je n’ai plus de forces pour la lambada.
\endverse

\beginverse
Le pingouin a repris le train
Et laissé les noix de coco
En rêvant d’aiglefin.
Et tant pis pour le curaçao.Lalala la lala…
\endverse

\beginsong{Plus rien}[by={Les Cowboys fringants (2004)}]

\beginverse
Refrain
Il ne reste que quelques minutes à ma vie
Tout au plus quelques heures, je sens que je faiblis
Mon frère est mort hier au milieu du désert
Je suis maint'nant le dernier humain de la Terre
\endverse

\beginverse
On m'a décrit jadis, quand j'étais un enfant
Ce qu'avait l'air le monde il y a très très longtemps
Quand vivaient les parents de mon arrière grand-père
Et qu'il tombait encore de la neige en hiver
\endverse

\beginverse
En ces temps on vivait au rythme des saisons
Et la fin des étés apportait la moisson
Une eau pure et limpide coulait dans les ruisseaux
Où venaient s'abreuver chevreuils et orignaux
\endverse

\beginverse
Mais moi je n'ai vu qu'une planète désolante
Paysages lunaires et chaleur suffocante
Et tous mes amis mourir par la soif ou la faim
Comme tombent les mouches...
Jusqu'à c'qu'il n'y ait plus rien...
Plus rien...
Plus rien...
\endverse

\beginverse
Refrain
\endverse

\beginverse
Tout ça a commencé il y a plusieurs années
Alors que mes ancêtres étaient obnubilés
Par des bouts de papier que l'on appelait argent
Qui rendaient certains hommes vraiment riches et puissants
\endverse

\beginverse
Et ces nouveaux dieux ne reculant devant rien
Étaient prêts à tout pour arriver à leur fins
Pour s'enrichir encore ils ont rasé la Terre
Pollué l'air ambiant et tari les rivières
\endverse

\beginverse
Mais au bout de cent ans des gens se sont levés
Et les ont averti qu'il fallait tout stopper
Mais ils n'ont pas compris cette sage prophétie
Ces hommes là ne parlaient qu'en termes de profits
\endverse

\beginverse
C'est des années plus tard qu'ils ont vu le non-sens
Dans la panique ont déclaré l'état d'urgence
Quand tous les océans ont englouti les îles
Et que les innondations ont frappé les grandes villes
\endverse

\beginverse
Et par la suite pendant toute une décennie
Ce fut les ouragans et puis les incendies
Les tremblements de terre et la grande séch'resse
Partout sur les visages on lisait la détresse
\endverse

\beginverse
Les gens ont dû se battre contre les pandémies
Décimés par millions par d'atroces maladies
Puis les autres sont morts par la soif ou la faim
Comme tombent les mouches...
Jusqu'à c'qu'il n'y ai plus rien...
Plus rien...
Plus rien...
\endverse

\beginverse
Mon frère est mort hier au milieu du désert
Je suis maintenant le dernier humain de la terre
Au fond l'intelligence qu'on nous avait donnée
N'aura été qu'un beau cadeau empoisonné
\endverse

\beginverse
Car il ne reste que quelques minutes à la vie
Tout au plus quelques heures, je sens que je faiblis
Je ne peux plus marcher, j'ai peine à respirer
Adieu l'humanité... Adieu l'humanité...
\endverse

\beginsong{Polyglotte}[by={Henri Dès (2002)}]

\beginverse
Refrain
Moi j'sais parler
Toutes les langues, toutes les langues
Moi j'sais parler
Les langues du monde entier
J'en savais rien
Mais maintenant que tu le dis
C'est enfantin
Ça va changer ma vie
\endverse

\beginverse
J'ai des baskets
Ça c'est un mot en anglais
J'ai des baskets
Pour faire mes p'tits trajets
Un anorak
Mot qui vient des esquimaux
Un anorak
Pour quand il fait pas beau
\endverse

\beginverse
Refrain
\endverse

\beginverse
Les spaghetti
Mot qui nous vient d'Italie
Les spaghetti
Me mettent en appétit
C'est le yaourt
Mot qui vient de Bulgarie
C'est le yaourt
Mon dessert de midi
\endverse

\beginverse
Refrain
\endverse

\beginverse
J'achète au kiosque
Mot qui nous vient de Turquie
J'achète au kiosque
Mes journaux favoris
Sans un kopeck
Mot qui nous vient de Russie
Sans un kopeck
J'peux pas faire des folies
\endverse

\beginverse
Refrain
\endverse

\beginverse
C'est sur un yacht
Mot qui vient du hollandais
C'est sur un yacht
Que j'passe le mois d'juillet
Grâce au judo
Mot qui nous vient du Japon
Grâce au judo
J'ne suis plus un poltron
\endverse

\beginverse
Refrain
\endverse

\beginverse
Par cette chanson
Mot qui nous vient du français
Par cette chanson
J'peux dire désormais
J'suis Polyglotte
Mot qui vient du grec ancien
J'suis Polyglotte
Et j'épate les copains
\endverse

\beginverse
Refrain
\endverse

\beginsong{Potemkine}[by={Jean Ferrat (1965)}]

\beginverse
M'en voudrez-vous beaucoup si je vous dis un monde
Qui chante au fond de moi au bruit de l'océan ?
M'en voudrez-vous beaucoup si la révolte gronde
Dans ce nom que je dis au vent des quatre vents ?
Ma mémoire chante en sourdine Potemkine
\endverse

\beginverse
Ils étaient des marins durs à la discipline
Ils étaient des marins, ils étaient des guerriers
Et le cœur d'un marin au grand vent se burine
Ils étaient des marins sur un grand cuirassé
Sur les flots je l'imagine, Potemkine
\endverse

\beginverse
M'en voudrez-vous beaucoup si je vous dis un monde
Où celui qui a faim va être fusillé ?
Le crime se prépare et la mer est profonde
Que face aux révoltés montent les fusiliers
C'est mon frère qu'on assassine, Potemkine
\endverse

\beginverse
Mon frère, mon ami, mon fils, mon camarade
Tu ne tireras pas sur qui souffre et se plaint
Mon frère, mon ami, je te fais notre alcade
Marin, ne tire pas sur un autre marin \!
Ils tournèrent leurs carabines, Potemkine
\endverse

\beginverse
M'en voudrez-vous beaucoup si je vous dis un monde
Où l'on punit ainsi qui veut donner la mort ?
M'en voudrez-vous beaucoup si je vous dis un monde
Où l'on n'est pas toujours du côté du plus fort ?
Ce soir, l'aime la marine, Potemkine
\endverse

\beginsong{Le pouvoir des fleurs}[by={Laurent Voulzy (1992)}]

\beginverse
Je m'souviens on avait des projets pour la terre
pour les hommes comme la nature
faire tomber les barrières, les murs,
les vieux parapets d'Arthur
fallait voir
imagine notre espoir
on laissait nos cœurs
au pouvoir des fleurs
jasmin, lilas,
c'étaient nos divisions nos soldats
pour changer tout ça
\endverse

\beginverse
Refrain
changer le monde
changer les choses avec des bouquets de roses
changer les femmes
changer les hommes
avec des géraniums
\endverse

\beginverse
je m'souviens, on avait des chansons, des paroles
comme des pétales et des corolles
qu'écoutait en rêvant
la petite fille au tourne-discophone
le parfum
imagine le parfum
l'Eden, le jardin,
c'était pour demain,
mais demain c'est pareil,
le même désir veille
là tout au fond des cœurs
tout changer en douceur
\endverse

\beginverse
Refrain
\endverse

\beginverse
ah\! sur la terre
il y a des choses à faire
pour les enfants, les gens, les éléphants
ah\! tant de choses à faire
et moi pour
te donner du cœur
je t'envoie des fleurs
\endverse

\beginverse
tu verras qu'on aura des foulards, des chemises
et que voici les couleurs vives
et que même si l'amour est parti
ce n'est que partie remise
pour les couleurs, les accords, les parfums
changer le vieux monde
pour faire un jardin
tu verras
tu verras
le pouvoir des fleurs
y a une idée pop dans mon air
\endverse

\beginverse
Refrain \\[bis]
\endverse

\beginverse
changer les...
Changer les cœurs...
\endverse

# 

\beginsong{La prairie}[by={Yves Montand (1949)}]

\beginverse
J'étouffe dans la ville
Et je m'y meurs d'ennui
Car tout me semble gris
Les rues me sont hostiles
Les toits cachent le soleil
Ah rendez-moi mon ciel \!
Laissez-moi retrouver ma prairie
Retrouver mes vastes horizons
Le galop effréné des noirs troupeaux en furie
Poursuivis par de fougueux garçons
Rendez-moi les copains qui m'attendent
L'oncle Joe et Jimmy l'Tatoué
Laissez-moi les revoir avant qu'un jour on les pende
Avant que le shérif les ait tués.
\endverse

\beginverse
La vieille diligence me conduira là-bas
Là-bas dans mon Texas
Au galop on s'élance
Allez mon gars, vas-y
Là-bas c'est le pays \!
Au Saloon, point de groom a la porte
C'est du pied que l'on pousse le battant
Le revolver au poing, rien ne vaut la manière forte
Pour se faire respecter en entrant
Car là-bas sans qu'on y prenne garde Dans un coin quelques mauvais garçons
Pour fêter ce retour joyeusement vous canardent
Les cow-boys ont de drôles de façons.
\endverse

\beginverse
Refrain
Laissez-moi chanter sur ma guitare
Les exploits de tous les gars de l'Ouest
Laissez-moi retrouver mes rodéos, mes bagarres
Laissez-moi retrouver mon pays
Laissez-moi retrouver mon Far West
\endverse

# 

\beginsong{La quête}[by={Orelsan (2021)}]

\beginverse
Rien peut m' ramener plus en arrière
Que l'odeur d' la pâte à modeler
Maman est prof' de maternelle
C'est même la maîtresse d'à côté
J'ai cinq ans et j' passe par la fenêtre
Pour aller m' planquer dans sa classe
Elle m' dit : T'es pas censé être là
J'ai dit : Près d' toi, c'est là ma place
J'aime que les livres, j' préfère être seul
Donc j' suis plus content quand il pleut
J' fais quelques cours de catéchisme
Mais j' suis pas sûr de croire en Dieu
J'ai sept ans, la vie est facile
Quand j' sais pas, j' demande à ma mère
Un jour elle m'a dit : J' sais pas tout
J'ai perdu foi en l'univers
\endverse

\beginverse
Refrain 1
À cinq ans, j' voulais juste en avoir sept
À sept ans, j'étais pressé d' voir le reste
Aujourd'hui, j'aimerais mieux qu' le temps s'arrête
Ah, c' qui compte c'est pas l'arrivée, c'est la quête
\endverse

\beginverse
J' balaye les feuilles mortes sur le terrain
Le froid m' fait des cloques sur les mains
J'ai dix ans, j' suis fan de basket
J' m'habille en p'tit américain
Mon père, mon héros, m'a offert
Les Jordan huit avec les scratch
Donc j' fais tout pour le rendre fier
Quand il vient m' voir à tous les matchs
J' rentre au collège, on m' traite de bourge
Normal, mes chaussures coûtent une blinde
J' veux plus les mettre, mon père s'énerve
Toi t'as tout, nous on n'avait rien
J'ai douze ans, j' fous l' bordel en cours
Pour essayer d' me faire des potes
Le prof de musique s' fout en l'air
Il est au paradis des profs
\endverse

\beginverse
Refrain 2
À onze ans, j' voulais juste en avoir treize
À treize ans, j'étais pressé d' voir le reste
Aujourd'hui, j'aimerais mieux qu' le temps s'arrête
Ah, c' qui compte c'est pas l'arrivée, c'est la quête
\endverse

\beginverse
Souvent j' suis tombé amoureux
Mais pour une fois, c'est réciproque
J'abandonne lâchement tous mes potes
J' vois plus qu' ma meuf, on fume des clopes
Quatorze ans, j' suis juste un fantôme
Du moins c'est c' que disent mes parents
Chérie veut qu' j' traîne plus qu'avec elle
Pourtant elle m' fait la gueule tout l' temps
Vu qu' j' déménage, ça nous sépare
J' me dis qu' l'amour c'est surcoté
Mon frangin m'éclate au basket
Alors j' préfère abandonner
J'ai quinze ans, j' regarde Kids en boucle
J' traîne avec des gars comme Casper
Mon père est sévère avec moi
Donc j' le répercute sur mon frère
\endverse

\beginverse
Refrain 3
À quinze ans, j' voulais juste en avoir seize
À seize ans, j'étais pressé d' voir le reste
Aujourd'hui, j'aimerais mieux qu' le temps s'arrête
Ah, c' qui compte c'est pas l'arrivée, c'est la quête
\endverse

\beginverse
J' descends les marches, la peur au ventre
Pour intercepter mon bulletin
À la maison, c'est la guerre froide
On s' comprend plus donc j' dis plus rien
J'ai seize ans et j' passe par la fenêtre
Pour rejoindre les autres au skatepark
On boit des bières, on fume des joints
Et j' raconte tout ça dans mes raps
Les années passent, même un peu trop
Au point qu' j'ose plus chanter mon âge
Mon frangin filme quand j' mets la bague
Ma frangine anime le mariage
Les choses que j'ose dire à personne
Sont les mêmes qui remplissent des salles
Maman est là, mon père est fier
Et l'univers est pas si mal
\endverse

\beginverse
Refrain 4
À seize ans, j' voulais juste avoir dix-sept
Dix-sept ans, j'étais pressé d' voir le reste
Aujourd'hui, j'aimerais mieux qu' le temps s'arrête
Ah, c' qui compte c'est pas l'arrivée, c'est la quête
\endverse

\beginverse
Refrain 1
À cinq ans, j' voulais juste en avoir sept
À sept ans, j'étais pressé d' voir le reste
Aujourd'hui, j'aimerais mieux qu' le temps s'arrête
Ah, c' qui compte c'est pas l'arrivée, c'est la quête
\endverse

# 

\beginsong{Red river Valley}[by={Traditionnel (1927)}]

\beginverse
Les pionniers sont passés avant le jour,
Dans les rues du village accablé.
Et mon cœur a frémi à leurs pas lourds,
Sur les bords de la Red River Valley.
\endverse

\beginverse
Refrain
Oh, Seigneur la roue tourne entre tes mains,
Où je vais aujourd'hui je ne sais.
Oh, Seigneur la roue tourne entre tes mains,
Mais je veux retrouver les pionniers.
\endverse

\beginverse
Les pionniers ont peiné pour le village,
A leurs mains, la vallée s'est pliée.
Et mes yeux ont vu naitre un barrage.
Sur les bords de la Red River Valley.
\endverse

\beginverse
Refrain
\endverse

\beginverse
Les pionniers ont marqué dans la clairière,
Que le pain se partage entre tous.
Et ma main s'est ouverte à mes frères,
Sur les bords de la Red River Valley.
\endverse

\beginverse
Refrain
\endverse

\beginverse
Les pionniers ont chanté dans la nuit claire,
Que la terre est à qui la voulait.
Et ma voix s'est unie à leur chant fier,
Sur les bords de la Red River Valley.
\endverse

\beginverse
Refrain
\endverse

\beginverse
Les pionniers ont promis de revenir
L'herbe pousse aujourd'hui à mes pieds.
Et mon cœur s'est trouvé fait pour servir
Sur les bords de la Red River Valley.
\endverse

\beginsong{Résiste}[by={France Gall (1981)}]

\beginverse
Si on t'organise une vie bien dirigée
Où tu t'oublieras vite
Si on te fait danser sur une musique sans âme
Comme un amour qu'on quitte
Si tu réalises que la vie n'est pas là
Que le matin tu te lèves
Sans savoir où tu vas
\endverse

\beginverse
Refrain
Résiste
Prouve que tu existes
Cherche ton bonheur partout, va,
Refuse ce monde égoïste
Résiste
Suis ton cœur qui insiste
Ce monde n'est pas le tien, viens,
Bats-toi, signe et persiste
Résiste
\endverse

\beginverse
Tant de libertés pour si peu de bonheur
Est-ce que ça vaut la peine
Si on veut t'amener à renier tes erreurs
C'est pas pour ça qu'on t'aime
Si tu réalises que l'amour n'est pas là
Que le soir tu te couches
Sans aucun rêve en toi
\endverse

\beginverse
Refrain
\endverse

\beginverse
Danse pour le début du monde
Danse pour tous ceux qui ont peur
Danse pour les milliers de cœurs
Qui ont droit au bonheur...
Résiste \\[3x]
\endverse

\beginverse
Refrain
\endverse

\beginsong{Rimini}[by={Les Wampas (2006)}]

\beginverse
Le soir quand l'Italie est triste
Elle ressemble à Rimini
Non mais vraiment
Qu'est ce qu'il t'a pris
D'aller mourir à Rimini
\endverse

\beginverse
Barbe Noire t'attendait là-haut
Les pirates étaient fiers de toi
Non mais vraiment
Qu'est ce qu'il t'a pris
D'aller mourir à Rimini
\endverse

\beginverse
Tu allais plus haut,
Plus vite que les autres
J'espère que tu n'as pas raté
Le Paradis
\endverse

\beginverse
Oui à côté de Rimini
Même Palavas a l'air sexy
Car à côté de Rimini
La Grande Motte ressemble à Venise
\endverse

\beginverse
Ouhouhouh
Ouhouhouh
Ouhouhouh
\endverse

\beginverse
Tu allais plus haut,
Plus vite que les autres
Oui pour toi Rimini c'est bien fini
Tu allais plus haut,
Plus vite que les autres
J'espère que tu n'as pas raté
Le Paradis
\endverse

\beginverse
Un jour avec tous les pirates
Tu reviendras crier vengeance
Le bandana sous les étoiles
Pour réduire Rimini en cendres
\endverse

\beginsong{Riptide}[by={Vance Joy (2013)}]

\beginverse
I was scared of dentists and the dark
I was scared of pretty girls and starting conversations
Oh, all my friends are turnin' green
You're the magician's assistant in their dream
\endverse

\beginverse
Refrain 1
Ah-ooh, ah-oh
And they come unstuck
\endverse

\beginverse
Refrain 2
Lady, runnin' down to the riptide
Taken away to the dark side
I wanna be your left-hand man
I love you when you're singin' that song
And I got a lump in my throat 'cause
You're gonna sing the words wrong
\endverse

\beginverse
There's this movie that I think you'll like
This guy decides to quit his job and heads to New York City
This cowboy's runnin' from himself
And she's been livin' on the highest shelf
\endverse

\beginverse
Refrain 1
\endverse

\beginverse
Refrain 2
\endverse

\beginverse
I just wanna, I just wanna know
If you're gonna, if you're gonna stay
I just gotta, I just gotta know
I can't have it, I can't have it any other way
I swear she's destined for the screen
Closest thing to Michelle Pfeiffer that you've ever seen, oh
\endverse

\beginverse
Refrain 2 \\[3x]
\endverse

\beginverse
I got a lump in my throat 'cause
You're gonna sing the words wrong
\endverse

\beginsong{Les rois du monde}[by={Damien Sargue, Cyril Niccolaï, Joy Eyzen (2000)}]

\beginverse
Les rois du monde vivent au sommet
Ils ont la plus belle vue mais y a un mais
Ils ne savent pas ce qu'on pense d'eux en bas 
Ils ne savent pas qu'ici c'est nous les rois
\endverse

\beginverse
Les rois du monde font tout ce qu'ils veulent
Ils ont du monde autour d'eux mais ils sont seuls
Dans leurs châteaux là-haut, ils s'ennuient
Pendant qu'en bas nous on danse toute la nuit
\endverse

\beginverse
Refrain
Nous on fait l'amour on vit la vie
Jour après jour, nuit après nuit
À quoi ça sert d'être sur la terre
Si c'est pour faire nos vies à genoux
On sait que le temps c'est comme le vent
De vivre y a que ça d'important
On se fout pas mal de la morale
On sait bien qu'on fait pas de mal
\endverse

\beginverse
Les rois du monde ont peur de tout
C'est qu'ils confondent les chiens et les loups
Ils font des pièges où ils tomberont un jour
Ils se protègent de tout même de l'amour
\endverse

\beginverse
Les rois du monde se battent entre eux
C'est qu'y a de la place, mais pour un pas pour deux
Et nous en bas leur guerre on la fera pas
On sait même pas pourquoi tout ça c'est jeux de rois
\endverse

\beginverse
Refrain
\endverse

\beginsong{Salade de fruit\*}[by={Bourvil (1959)}]

\beginverse
Ta mère t'a donné comme prénom
Salade de fruits, ah\! quel joli nom
Au nom de tes ancêtres hawaïens
Il faut reconnaître que tu le portes bien
\endverse

\beginverse
Refrain
Salade de fruits, jolie, jolie, jolie
Tu plais à mon père, tu plais à ma mère
Salade de fruits, jolie, jolie, jolie
Un jour ou l'autre il faudra bien
Qu'on nous marie
\endverse

\beginverse
Pendus dans la paillote au bord de l'eau
Y a des ananas, y a des noix de cocos
J'en ai déjà goûté je n'en veux plus
Le fruit de ta bouche serait le bienvenu
\endverse

\beginverse
Refrain
\endverse

\beginverse
Je plongerai tout nu dans l'océan
Pour te ramener des poissons d'argent
Avec des coquillages lumineux
Oui mais en échange tu sais ce que je veux
\endverse

\beginverse
Refrain
\endverse

\beginverse
On a donné chacun de tout son cœur
Ce qu'il y avait en nous de meilleur
Au fond de ma paillote au bord de l'eau
Ce panier qui bouge c'est un petit berceau
\endverse

\beginverse
Salade de fruits, jolie, jolie, jolie
Tu plais à ton père, tu plais à ta mère
Salade fruits, jolie, jolie, jolie
C'est toi le fruit de nos amours \!
Bonjour petit \!
\endverse

\beginsong{Salut les amoureux}[by={Joe Dassin (1973), Les Frangines (2019), Garou (2022)}]

\beginverse
Les matins se suivent et se ressemblent
Quand l'amour fait place au quotidien
On n'était pas fait pour vivre ensemble
Ça n'suffit pas toujours de s'aimer bien
\endverse

\beginverse
C'est drôle, hier, on s'ennuyait
Et c'est à peine si l'on trouvait
Des mots pour se parler du mauvais temps
Et maintenant qu'il faut partir
On a cent mille choses à dire
Qui tiennent trop à cœur pour si peu de temps
\endverse

\beginverse
Refrain
On s'est aimé comme on se quitte
Tout simplement, sans penser à demain
A demain qui vient toujours un peu trop vite
Aux adieux qui quelquefois se passent un peu trop bien
\endverse

\beginverse
On fait c'qu'il faut, on tient nos rôles
On se regarde, on rit, on crâne un peu
On a toujours oublié quelque chose
C'est pas facile de se dire adieu
\endverse

\beginverse
Et l'on sait trop bien que tôt ou tard
Demain peut-être, ou même ce soir
On va se dire que tout n'est pas perdu
De ce roman inachevé, on va se faire un conte de fées
Mais on a passé l'âge, on n'y croirait plus
\endverse

\beginverse
Refrain
\endverse

\beginverse
Roméo, Juliette, et tous les autres
Au fond de vos bouquins, dormez en paix
Une simple histoire comme la nôtre
Est de celles qu'on écrira jamais
\endverse

\beginverse
Allons petite il faut partir
Laisser ici nos souvenirs
On va descendre ensemble si tu veux
Et quand elle va nous voir passer
La patronne du café
Va encore nous dire " Salut les amoureux "
\endverse

\beginverse
Refrain
\endverse

\beginsong{San Francisco}[by={Maxime le Forestier (1972)}]

\beginverse
C'est une maison bleue adossée à la colline
On y vient à pieds, on ne frappe pas
Ceux qui vivent là ont jeté la clé
On se retrouve ensemble après des années de route
Et on vient s'asseoir autour du repas
Tout le monde est là à cinq heures du soir
\endverse

\beginverse
Refrain
Quand San Francisco s'embrume, San Francisco s'allume,
San Francisco...
Où êtes-vous Lizzard et Luc, Psylvia ? Attendez-moi \!
\endverse

\beginverse
Nageant dans le brouillard, enlacés, roulant dans l'herbe
On écoutera Tom à la guitare
Phil à la Kéna jusqu'à la nuit noire
Un autre arrivera pour nous dire des nouvelles
D'un qui reviendra dans un an ou deux,
Puisqu'il est heureux, on s'endormira
\endverse

\beginverse
Refrain
Quand San Francisco se lève, San Francisco se lève,
San Francisco..
Où êtes-vous Lizzard et Luc, Psylvia ? Attendez-moi \!
\endverse

\beginverse
C'est une maison bleue accrochée à ma mémoire
On y vient à pieds, on ne frappe pas
Ceux qui vivent là ont jeté la clé
Peuplée de cheveux longs, de grands lits et de musique
Peuplée de lumière et peuplée de fous
Elle sera dernière à rester debout
\endverse

\beginverse
Refrain
Si San Francisco s'effondre, si San Francisco s'effondre
San Francisco...
Où êtes-vous Lizzard et Luc, Psylvia ? Attendez-moi \!
\endverse

\beginsong{Santiano}[by={Hugues Aufray (1961)}]

\beginverse
C'est un fameux trois-mâts fin comme un oiseau
Hisse et ho, \\[bis] Santiano,
Dix-huit noeuds, quatre cents tonneaux
Je suis fier d'y être matelot.
\endverse

\beginverse
Refrain
Tiens bon la vague et tiens bon le vent
Hisse et ho, \\[bis] Santiano
Si Dieu veut, toujours droit devant
Nous irons jusqu'à San Francisco
\endverse

\beginverse
Je pars pour de longs mois en laissant Margot
Hisse et ho, \\[bis] Santiano,
D'y penser j'avais le cœur gros,
En doublant les feux de Saint-Malo.
\endverse

\beginverse
Refrain
\endverse

\beginverse
On prétend que là-bas l'argent coule à flot
Hisse et ho, \\[bis] Santiano,
On trouve l'or au fond des ruisseaux
J'en ramènerai plusieurs lingots.
\endverse

\beginverse
Refrain
\endverse

\beginverse
Un jour je reviendrai chargé de cadeaux,
Hisse et ho, \\[bis] Santiano,
Au pays j'irai voir Margot
A son doigt je passerai l'anneau.
\endverse

\beginverse
Refrain
Tiens bon le cap et tiens bon le flot,
Hisse et ho, \\[bis] Santiano.
Sur la mer qui fait le gros dos
Nous irons jusqu'à San Francisco.
\endverse

# 

\beginsong{Les sardines}[by={Patrick Sébastien (2006)}]

\beginverse
Pour faire une chanson facile, facile,
Faut d'abord des paroles débiles, débiles,
Une petite mélodie qui te prend bien la tête,
Et une chorégraphie pour bien faire la fête,
Dans celle là, on se rassemble, à 5, ou 6, ou 7
Et on se colle tous ensemble, en chantant à tue tête.
\endverse

\beginverse
Refrain
Ha \! Qu'est-ce qu'on est serré, au fond de cette boite,
Chantent les sardines, chantent les sardines,
Ha \! Qu'est-ce qu'on est serré, au fond de cette boite,
Chantent les sardines entre l'huile et les aromates. \\[bis]
\endverse

\beginverse
Bien sûr, que c'est vraiment facile, facile,
C'est même complètement débile, débile,
C'est pas fait pour penser, c'est fait pour faire la fête,
C'est fait pour se toucher, se frotter les arêtes ,
Alors on se rassemble, à 5, ou 6, ou 7,
Et puis on saute ensemble en chantant à tue tête,
\endverse

\beginverse
Refrain
\endverse

\beginverse
Et puis,… pour respirer un p'tit peu, on s'écarte en se tenant la main,
Et puis, … pour être encore plus heureux,
On fait là, là, là, en chantant mon refrain \!
Là, là, et les mains en l'air, là, là \!
Là, là, là, là, là, là, là, là, là, là, là, là, là, là, là, 
\endverse

\beginverse
Et maintenant, on se resserre tous \!
On se resserre, et maintenant qu'on l'a connaît,
On va chanter la chanson des sardines \! Attention \! Allez \!
\endverse

\beginverse
Refrain\\[bis]
\endverse

\beginverse
là, là, là, là, là, là, là, là, là, là, là, là, là, là, là, 
là, là, là, là, là, là, là, là, là, là, là, là, là, là, là, 
\endverse

# 

\beginsong{Saurus}[by={LCone (2020)}]

\beginverse
De chlini dino Joshua hed ganz en churze hals,
Und ihn macht das so truurig, will dass allne gar ned gfallt
Am morge ufem pauseplatz isch er immer versteckt
Kaputze über stirne so das ihn niemer entdeckt
Sie lachet und sie rüefed "lueg mal de, lueg mal wäh"
Und er seid denn "hey höred uf, mier isch das unagnehm"
Er stiigt ganz schnell ufs trottinett
Wett eifach hei is dino bett
Sii wie alli andere, isch alles was er wett, ooh
\endverse

\beginverse
Refrain
Saurus, chliine saurus
Lah sie rede dino, du bisch ned elei
Saurus, chliine saurus
D'welt isch grösser als du denksch
Chum zrugg uf d'bei
Saurus
Chum zrug uf dini bei
Saurus
\endverse

\beginverse
Am nöchste morge wacht er uf
Und wott nümm zrugg i d'klass
Mit dene fiese hänsli dinos macht das halt kei spass
Er hirnet und er überleid, es raubt ihm fast de schnuuf
Wie söll er das da mache, da gaht ihm es liechtli uf
Er seid, "mami ich bin chrank",
Sie seid, "isch doch gar ned wohr"
Er seid die sind so fies und plaget mich scho s'ganze jahr
Nie meh seid er gäng er zrugg, zu diesne zrugg i d'schuel
Da seid d'mama dino joshua jetzt los mer zue
\endverse

\beginverse
Refrain
\endverse

\beginverse
De chliini dino Joshua
Au de wird einisch gross
Au er wird einisch gmerke s'lebe isch ned nur famos
Die andere wend ihn abezieh
Das isch ned immer lieb
De dino wird sin weg no gah und singt dezue das lied
Saurus, chliine saurus
Lah sie rede dino, du bisch ned elei
Saurus, chliine saurus
D'welt isch grösser als du denksch
Chum zrugg uf d'bei
Saurus
Saurus
\endverse

# 

\beginsong{Saute le mur}[by={Traditionnel }]

\beginverse
Refrain
Saute le mur, saute la haie
Et viens me rejoindre.
Saute le mur, saute la haie
Si c'qu'on dit est vrai.
\endverse

\beginverse
On dit dans tout le village
Que de toutes les filles
C'est toi qui es la plus sage
Et la plus jolie.
Mais quand on veut l'approcher
C'est une vraie misère
Ton père le fusil chargé
Guette par derrière.
\endverse

\beginverse
Refrain
\endverse

\beginverse
Ton père veut te marier
Au fils de l'épicière
Sans même t'avoir consultée
Lui n'pense qu'aux affaires.
Je n'aime pas te voir pleurer
Aussi la nuit prochaine
Si tu veux je l'enlèverai
A celui qui l'enchaîne.
\endverse

\beginverse
Refrain
\endverse

\beginverse
Dans ma vieille cabane en bois
Au bord de l'étang
Je frai une p'tite place pour toi
Et pour nos enfants.
J'étais seul pour admirer Les beaux clairs de lune
Viens tu pourras partager
Toute ma fortune.
\endverse

\beginverse
Refrain
Saute le mur, saute la haie…
…c'est toi que j'épouserai.
\endverse

\beginsong{La scoutitude}[by={Matthieu Koffer \- parodie d' Oldelaf (2016)}]

\beginverse
La scoutitude.
C'est quand on t'dis qu'tu pars en camp à Douarnenez
C'est quand tu pars en camp oubliant ton Kway
C'est quand tu montes ta tente et qu't'as pas les piquets
Et t'es trempé
\endverse

\beginverse
La scoutitude,
C'est quand tu fais un' nuit à la belle en Bretagne
C'est quand ton pilot' veut camper sur la montagne
C'est quand tu fais la vaisselle un soir de lasagnes
Et c’est ballot
\endverse

\beginverse
Refrain
La scoutitude
C'est moi, c'est toi
C'est nous, c'est quoi ?
C'est un camp pourri comm' tu n'le souhaitais pas
La scoutitude
C'est mmmh, c'est wou \!
C'est eux, c'est vous
C'est les chefs qui te disent que ça va pas du tout.
\endverse

\beginverse
La scoutitude.
C'est quand tu fais tomber ta lamp' dans les feuillets
C'est quand tu trouves une limac' dans ton duvet.
C'est quand c'est toi qu'as la gamell' qu'est mal lavée.
Et ça fait crade.
\endverse

\beginverse
La scoutitude.
C'est quand dans la gourde des pionniers y a pas que d'l'eau
C'est faire une tent' surélevée à quatr' mètr' de haut
Et pour monter il ne te rest' qu'un escabeau
Et ça fait haut \!
\endverse

\beginverse
Refrain
\endverse

\beginverse
La scoutitude.
C'est quand ton chef de group' veut voir ton dossier d'camp
C'est quand ton coach te réclam' ton dossier d'camp
C'est quand tu te demand' : c'est quoi un dossier d'camp
Et ça fait tard
\endverse

\beginverse
La scoutitude.
C'est quand tu pars au jamboree avec ton père
C'est quand tu pars au jamboree avec ta mère
C'est quand tu pars au jamboree avec ton frère
Et ça fait chier…
\endverse

\beginverse
Refrain
\endverse

\beginsong{La Seine}[by={Vanessa Paradis (2011)}]

\beginverse
Elle sort de son lit, tellement sûre d'elle
La Seine, la Seine, la Seine.
Tellement jolie elle m'ensorcelle
La Seine, la Seine, la Seine.
\endverse

\beginverse
Extralucide, la lune est sur
La Seine, la Seine, la Seine
Tu n'es pas saoul, Paris est sous
La Seine, la Seine, la Seine
\endverse

\beginverse
Refrain
Je ne sais, ne sais, ne sais pas pourquoi
On s'aime comme ça, la Seine et moi
\\[bis]
\endverse

\beginverse
Extra Lucille, quand tu es sur
La Seine, la Seine, la Seine
Extravagante, quand l'ange est sur
La Seine, la Seine, la Seine
\endverse

\beginverse
Refrain
\endverse

\beginverse
Sur le Pont des Arts, mon coeur vacille.
Entre les eaux, l'air est si bon
Cet air si pur, je le respire
Nos reflets perchés sur ce pont.
\endverse

\beginverse
On s'aime comme ça, la Seine et moi.
On s'aime comme ça, la scène et moi.
\\[bis]
\endverse

\beginsong{Sensualité}[by={Axelle Red (1993)}]

\beginverse
Jamais je n'aurais pensé...
"Tant besoin de lui"
Je me sens si envoûtée
Que ma maman me dit: ralentis
Désir ou amour
Tu le sauras un jour
\endverse

\beginverse
Refrain
J'aime j'aime Tes yeux,
 j'aime ton odeur
Tous tes gestes en douceur
Lentement dirigés
Sensualité
Oh stop un instant
J'aimerais que ce moment
Fixe pour des tas d'années
Ta sensualité
\endverse

\beginverse
Il parait qu'après quelques temps
La passion s'affaiblit
Pas toujours apparemment
Et maman m'avait dit: ralentis
Désir et amour tu le sauras un jour
\endverse

\beginverse
Refrain
\endverse

\beginverse
Je te demande si simplement
Ne fais pas semblant
Je t'aimerai encore
Et encore
\endverse

\beginverse
Désir ou amour
Tu le sauras un jour
Refrain
\endverse

\beginverse
Tes yeux, j'aime ton odeur
Tous tes gestes en douceur
Lentement dirigés
Sensualité
Ouh, stop, un instant 
J'aimerais que ce moment
Fixe pour des tas d'années
Ta sensualité
\\[Bis]
\endverse

\beginsong{Si t'étais là}[by={Louane (2017)}]

\beginverse
Parfois je pense à toi dans les voitures
Le pire, c'est les voyages, c'est d'aventure
Une chanson fait revivre un souvenir
Les questions sans réponse ça c'est le pire
\endverse

\beginverse
Refrain
Est-ce que tu m'entends? Est-ce que tu me vois?
Qu'est-ce que tu dirais, toi, si t'étais là ?
Est-ce que ce sont des signes que tu m'envoies ?
Qu'est-ce que tu ferais, toi, si t'étais là ?
\endverse

\beginverse
Je me raconte des histoires pour m'endormir
Pour endormir ma peine et pour sourire
J'ai des conversations imaginaires
Avec des gens qui ne sont pas sur la Terre
\endverse

\beginverse
Refrain
\endverse

\beginverse
Je m'en fous si on a peur que je tienne pas le coup
Je sais que t'es là pas loin, même si c'est fou
Les fous c'est fait pour faire fondre les armures
Pour faire pleurer les gens dans les voitures
\endverse

\beginverse
Refrain
\endverse

\beginsong{Siffler sur la colline}[by={Joe Dassin (1968)}]

\beginverse
Refrain
Elle m'a dit d'aller siffler là-haut sur la colline
De l'attendre avec un petit bouquet d'églantines
J'ai cueilli les fleurs, et j'ai sifflé tant que j'ai pu
J'ai attendu, attendu, elle n'est jamais venue
Zaï zaï zaï zaï, zaï zaï zaï zaï, \\[bis] oh oh, oh oh \\[bis]
\endverse

\beginverse
Je l'ai vue près d'un laurier, elle gardait ses blanches brebis
Quand j'ai demandé d'où venait sa peau fraîche, elle m'a dit:
«C'est d'rouler dans la rosée qui rend les bergères jolies»
Mais quand j'ai dit qu'avec elle, je voudrais y rouler aussi
Elle m'a dit…
\endverse

\beginverse
Refrain
\endverse

\beginverse
A la foire du village, un jour je lui ai soupiré
Que je voudrais être une pomme, suspendue à un pommier
Et qu'à chaque fois qu'elle passe, elle vienne me mordre dedans
Mais elle est passée, et tout en me montrant ses jolies dents
Elle m'a dit… 
\endverse

\beginsong{Le sirop typhon}[by={Richard Anthony (1969)}]

\beginverse
Refrain
Buvons, buvons, buvons le sirop typhon (2),
L'universelle panacé-éée 
A la cuillère, ou bien dans un verre, 
Rien ne pourra nous résister 
\endverse

\beginverse
Monsieur Carouge avait le nez rouge, et cela le désolait
Une cuillère lui fut salutaire, il a maintenant le nez violet
Mme Leprince se trouvait trop mince, elle ressemblait à un bâton
Elle fit un'cure sans demi-mesure, elle est plus ronde qu'un ballon.
\endverse

\beginverse
Refrain
\endverse

\beginverse
Monsieur le Maire avait des misères, dans ses discours, il bégayait
Un p'tit verre lui fut salutaire, il n'bégaie plus car il est muet
Monsieur Léon ponpon (2) si gentil et si rond patapon (2)
Ne gagnait jamais au tiercé-é
Un'cur' sévère lui fut salutaire, il est gagnant mais comme jockey
\endverse

\beginverse
Refrain
\endverse

\beginverse
Le petit Pierre était célibataire, et voulait le rester longtemps
Il but un verre, puis un autre verre, il a 10 femmes et 30 enfants
Dans le village, tous les enfants sages écoutent les cloches sonner
Rêvent qu'un mage venu des nuages, du bon sirop va leur donner.
\endverse

\beginsong{Le sorbier de l’Oural}[by={Evgueni Rodyguine (1950)}]

\beginverse
Sous le vent des plaines
Un arbre m'est donné
Un ami de peine,
Mon arbre de liberté.
Toi, tu es la jeunesse
Quand je n'y croyais pas
Tu m'apportes l'ivresse
Et tu me tends les bras.
\endverse

\beginverse
Refrain
Sous la neige, la terre
N'avait pas oublié
Au pays solitaire
A fleuri le sorbier.
\endverse

\beginverse
Sous le vent des plaines
Il m'a donné ses fruits
Et depuis je sème
Ses graines pour mes amis.
Il nous montre la route
Quand nous perdons nos pas
A la saison du doute
En nous tendant les bras.
\endverse

\beginsong{Sous l’océan}[by={Henri Salvador \- La petite sirène (1989)}]

\beginverse
Ariel, écoute-moi
Chez les humains, c'est la pagaille
La vie sous la mer, c'est bien mieux qu'la vie qu'ils ont là-haut sur terre
\endverse

\beginverse
Le roseau est toujours plus vert dans le marais d'à côté
Toi, t'aimerais bien vivre sur terre, bonjour la calamité
Regarde bien le monde qui t'entoure dans l'océan parfumé
On fait Carnaval tous les jours, mieux tu pourras pas trouver
\endverse

\beginverse
Sous l'océan, sous l'océan
Doudou, c'est bien mieux, tout l'monde est heureux sous l'océan
Là-haut, ils bossent toute la journée
Esclavagés et prisonniers
Pendant qu'on plonge comme des éponges sous l'océan
\endverse

\beginverse
Chez nous, les poissons, s'fendent la pipe
Les vagues sont un vrai régal
Là-haut, ils s'écaillent et ils flippent
À tourner dans leur bocal
Le bocal, faut dire, c'est l'extase
Chez leurs copains cannibales
Si monsieur poisson n'est pas sage
Il finira dans la poêle (oh, non)
\endverse

\beginverse
Sous l'océan (sous l'océan) sous l'océan (sous l'océan)
Y a pas d'court-bouillon, pas d'soupe de poisson, pas d'marmiton
Pour la tambouille, on leur dit "non"
Sous l'océan, y a pas d'hameçon
On déambule, on fait des bulles sous l'océan
\endverse

\beginverse
La vie est super, mieux que sur la terre, je te le dis (je te le dis)
Tu vois l'esturgeon et la raie
Se sont lancés dans le Reggae
On a le rythme, c'est d'la dynamite sous l'océan
\endverse

\beginverse
Triton au flûtiau, la carpe joue d'la harpe
La rascasse, de la basse, c'est les rois du rap
Maquereau au saxo, turbo aux bongos
Le lieu est le Dieu d'la soul \\[ouais]
La raie au djembé, l'gardon au violon
Les soles rock'n'roll, le thon garde le ton
Le bar et le sprat se marrent et s'éclatent
Vas-y, souffle, mon doudou
\endverse

\beginverse
Ouais, ha-ha
Ouais, sous l'océan (sous l'océan)
Sous l'océan (sous l'océan)
Quand la sardine "begin the beguine" ça balance, ça swingue (ça balance, ça swingue)
Ils ont du sable, ça c'est certain
Nous, le jazz-band et les copains
On a les clim-clams pour faire une jim-jam sous l'océan (ha-ha)
Les limaces des mers au rythme d'enfer sous l'océan
Et les bigorneaux pour donner l'tempo
C'est frénétique, c'est fantastique
On est en transe, faut qu'ça balance sous l'océan
\endverse

\beginsong{Sous le vent}[by={Céline Dion, Garou (2005)}]

\beginverse
Et si tu crois que j'ai eu peur
C'est faux
Je donne des vacances à  mon cœur
Un peu de repos
\endverse

\beginverse
Et si tu crois que j'ai eu tort
Attends
Respire un peu le souffle d'or
Qui me pousse en avant
Et...
\endverse

\beginverse
Refrain
Fais comme si j'avais pris la mer
J'ai sorti la grand'voile
Et j'ai glissé sous le vent
Fais comme si je quittais la terre
J'ai trouvé mon étoile
Je l'ai suivie un instant
Sous le vent
\endverse

\beginverse
Et si tu crois que c'est fini
Jamais
C'est juste une pause, un répit
Après les dangers
Et si tu crois que je t'oublie
Écoute
Ouvre ton corps aux vents de la nuit
Ferme les yeux
Et...
\endverse

\beginverse
Refrain
\endverse

\beginverse
Et si tu crois que c'est fini
Jamais
C'est juste une pause, un répit
Après les dangers
\endverse

\beginverse
Refrain \\[bis]
\endverse

\beginverse
Sous le vent... sous le vent.....
\endverse

\beginsong{Stewball}[by={Hugues Aufray (1984)}]

\beginverse
Il s'appelait Stewball
C'était un cheval blanc
Il était mon idole
Et moi j'avais dix ans
\endverse

\beginverse
Notre pauvre père
Pour acheter ce pur-sang
Avait mis dans l'affaire
Jusqu'à son dernier franc
\endverse

\beginverse
Il avait dans la tête
D'en faire un grand champion
Pour liquider nos dettes
Et payer la maison
\endverse

\beginverse
Il croyait à sa chance
Il engagea Stewball
Par un beau dimanche
Au grand Prix de Saint-Paul
\endverse

« Je sais dit mon père.
Que Stewball va gagner »
Mais après la rivière,
Stewball est tombé

\beginverse
Quand le vétérinaire
D'un seul coup l'acheva,
J'ai vu pleurer mon père
Pour la première fois...
\endverse

\beginverse
Il s'appelait Stewball
C'était un cheval blanc,
Il était mon idole
Et moi j'avais dix ans
\endverse

\beginsong{Le Sud}[by={Nino Ferrer (1975)}]

\beginverse
C'est un endroit qui ressemble à la Louisiane, à l'Italie
Il y a du linge étendu sur les terrasses, et c'est joli.
\endverse

\beginverse
Refrain
On dirait le Sud, le temps dure longtemps
Et la vie sûrement, plus d'un million d'années
Et toujours en été.
\endverse

\beginverse
Y a plein d'enfants qui se roulent sur la terrasse
Y a plein de chiens
Y a même un chat, une tortue, des poissons rouges
Il ne manque rien.
\endverse

\beginverse
Refrain
\endverse

\beginverse
Un jour ou l'autre, il faudra qu'il y ait la guerre
On le sait bien
On n'aime pas ça, mais on ne sait pas quoi faire
On dit: «C'est le destin.»
\endverse

\beginverse
Refrain
Tant pis pour le Sud, c'était pourtant bien
On aurait pu vivre plus d'un million d'années
Et toujours en été.
\endverse

\beginsong{Sur la lune}[by={Bigflo & Oli (2021)}]

\beginverse
Refrain
Un jour, j'irai sur la lune, un jour, j’irai
Et si j'disais que j'en étais sûr, j’te mentirais
Et je sais qu'elle me voit
Parce que je la vois aussi
Alors je la montre du doigt
Et ça devient possible
\endverse

\beginverse
Un jour, je serai vieux
J'aurai, enfin trouvé ma place
Parce que j'ai beau courir
Je ne rattrape pas le temps qui passe
Un jour, je serai père
J'aurai un fils à élever
Et je lui apprendrai que chaque erreur est un essai
Un jour je serai fort
J'aurai plus de fourmis dans les jambes
Quand le monde est immobile
Pourquoi c'est moi qui tremble ?
Un jour, je serai mieux
Je sais, je le serai un jour
Tu peux pas quitter la Terre
Tu peux juste en faire le tour
\endverse

\beginverse
Refrain
\endverse

\beginverse
Un jour, je serai fou
J'aurai fait le tour de la Terre
J'aurai rayé chaque ligne de la grande liste de mes rêves
Un jour, je serai moi
J’aurai assumé toutes mes fautes
Je sais j'suis différent
Donc au final j'suis comme les autres
Un jour, je serai sage
J'aurai fini de faire le con
J'irai voir mes ennemis
Pour tous leur demander pardon
Un jour, je serai mort
J'aurai fait le tour de mon âge
Une plaque avec mon nom
Une place dans les nuages
\endverse

\beginverse
Refrain
\endverse

\beginverse
Un jour, je serai moi-même
J'aurai trouvé le sourire
J'aurai réglé mes problèmes
J'en ai marre de courir, marre de courir
Un jour, je serai moi-même
J'aurai trouvé le sourire
J'aurai réglé mes problèmes
J'en ai marre de courir, marre de courir
\endverse

\beginverse
Refrain \\[bis]
\endverse

\beginsong{Take me home, country roads}[by={John Denver (1971)}]

\beginverse
Almost Heaven, West Virginia
Blue Ridge Mountains, Shenandoah River
Life is old there, older than the trees
Younger than the mountains, growin' like a breeze
\endverse

\beginverse
Refrain
Country roads, take me home
To the place I belong
West Virginia, mountain mama
Take me home, country roads
\endverse

\beginverse
All my memories gather 'round her
Miner's lady, stranger to blue water
Dark and dusty, painted on the sky
Misty taste of moonshine, teardrop in my eye
\endverse

\beginverse
Refrain
\endverse

\beginverse
I hear her voice in the mornin' hour, she calls me
The radio reminds me of my home far away
Drivin' down the road, I get a feelin'
That I should've been home yesterday, yesterday
\endverse

\beginverse
Refrain \\[Bis]
\endverse

\beginverse
Take me home, \\[down] country roads \\[bis]
\endverse

\beginsong{Tant qu’on aura de l’amour}[by={Les Cowboys Fringants (2008)}]

\beginverse
Tant qu'on aura d' l'amour", chanson naïve
Trois, quatre \!
\endverse

\beginverse
Depuis qu'on a lâché prise
On voit de la couleur dans les zones grises
Il y a du bon dans la froidure de novembre
Elle nous permet de nou coller plus ensemble
Sous une couette
Tout nus pas d'bobettes
\endverse

\beginverse
Refrain
Tant qu'on aura de l'amour
De l'eau fraîche et de l'air pur
Un toit et puis 4 murs
Ce sera la joie dans not'cour
\endverse

\beginverse
On apprécie les p'tites choses
Trop d'attentes vaines rendent la vie morose
À c't'heure si on a l'vent dans'face en partant
B'en on s'dit qu'on l'aura dans l'dos en rev'nant
Ou vice et versa
On s'badre p'us avec ça
\endverse

\beginverse
Refrain
\endverse

\beginverse
On se plaint pas quand y mouille
C'est ça qui fait pousser nos plants de citrouilles
L'été on est heureux quand il fait très chaud
Car le soleil réchauffe nos coeurs d'artichauts
Comme je t'aime
Veux-tu que l'on sème
\endverse

\beginverse
Refrain \\[bis]
\endverse

\beginsong{Le temps des fleurs}[by={Dalida (1968)}]

\beginverse
Dans une taverne du vieux Londres
Où se retrouvaient des étrangers
Nos voix criblées de joie montaient de l'ombre
Et nous écoutions nos cœurs chanter
\endverse

\beginverse
Refrain1
C'était le temps des fleurs
On ignorait la peur
Les lendemains avaient un goût de miel
Ton bras prenait mon bras
Ta voix suivait ma voix
On était jeunes et l'on croyait au ciel
La, la, la...
On était jeunes et l'on croyait au ciel
\endverse

\beginverse
Et puis sont venus les jours de brume
Avec des bruits étranges et des pleurs
Combien j'ai passé de nuits sans lune
A chercher la taverne dans mon cœur
\endverse

\beginverse
Refrain 2
Tout comme au temps des fleurs
Où l'on vivait sans peur
Où chaque jour avait un goût de miel
Ton bras prenait mon bras
Ta voix suivait ma voix
On était jeunes et l'on croyait au ciel
La, la, la ......
On était jeunes et l'on croyait au ciel
\endverse

\beginverse
Je m'imaginais chassant la brume
Je croyais pouvoir remonter le temps
Et je m'inventais des clairs de lune
Où tous deux nous chantions comme avant
C'était le temps des fleurs
On ignorait la peur
Les lendemains avaient un goût de miel
Ton bras prenait mon bras
Ta voix suivait ma voix
On était jeunes et l'on croyait au ciel
La, la, la ......
On était jeunes et l'on croyait au ciel
\endverse

\beginverse
Et ce soir je suis devant la porte
De la taverne où tu ne viendras plus
Et la chanson que la nuit m'apporte
Mon cœur déjà ne la connaît plus
\endverse

\beginverse
Refrain 1
\endverse

# 

\beginsong{The sound of silence}[by={Simon & Garfunkel (1964)}]

\beginverse
Hello darkness, my old friend
I've come to talk with you again
Because a vision softly creeping
Left its seeds while I was sleeping
And the vision that was planted in my brain
Still remains
Within the sound of silence
\endverse

\beginverse
In restless dreams I walked alone
Narrow streets of cobblestone
'Neath the halo of a street lamp
I turned my collar to the cold and damp
When my eyes were stabbed by the flash of a neon light
That split the night
And touched the sound of silence
\endverse

\beginverse
And in the naked light I saw
Ten thousand people, maybe more
People talking without speaking
People hearing without listening
People writing songs that voices never share
No one dared
Disturb the sound of silence
\endverse

"Fools" said I, "You do not know
Silence like a cancer grows
Hear my words that I might teach you
Take my arms that I might reach you"
But my words like silent raindrops fell
And echoed in the wells of silence

\beginverse
And the people bowed and prayed
To the neon god they made
And the sign flashed out its warning
In the words that it was forming
\endverse

\beginverse
And the sign said, "The words of the prophets
Are written on the subway walls
And tenement halls
And whispered in the sounds of silence"
\endverse

# 

\beginsong{The wellerman}[by={Nathan Evans (2021)}]

\beginverse
Paroles…
\endverse

# 

\beginsong{Toi \+ moi}[by={Grégoire (2008)}]

\beginverse
Paroles…
\endverse

\beginsong{Tous les cris les S.O.S}[by={Daniel Balavoine (1985)}]

\beginverse
Paroles…
\endverse

\beginsong{Tout le bonheur du monde}[by={Sinsémilia (2004)}]

\beginverse
Paroles…
\endverse

\beginsong{La tribu de Dana}[by={Manau (1998)}]

\beginverse
Paroles…
\endverse

\beginsong{Ulysse}[by={Joachim Du Bellay (1558), Ridan (2007)}]

\beginverse
Paroles…
\endverse

\beginsong{Un autre monde}[by={Téléphone (1984)}]

\beginverse
Je rêvais d'un autre monde
Où la terre serait ronde
Où la lune serait blonde
Et la vie serait féconde
\endverse

\beginverse
Refrain
Je dormais à poings fermés
Je ne voyais plus mes pieds
Je rêvais réalité
Ma réalité..
\endverse

\beginverse
Je rêvais d'une autre terre
Qui resterait un mystère
Une terre moins terre à terre
Oui je voulais tout foutre en l'air
\endverse

\beginverse
Refrain
Je marchais les yeux fermés
Je ne voyais plus mes pieds
Je rêvais réalité
Ma réalité…
\endverse

\beginverse
Oui je rêvais de notre monde
Et la terre est bien ronde Et la lune est si blonde
Que ce soir dansent les ombres du monde
A la rêver immobile
Elle m'a trouvé bien futile
Mais quand bouger l'a faite tourné
Ma réalité m'a pardonné
Et dansent les ombres du monde bis
Eh \! Eh \!
Danse, danse, danse…
\endverse

\beginsong{Un monde parfait}[by={Illona Mitrecey (2005)}]

\beginverse
Paroles…
\endverse

\beginsong{Une seule vie}[by={De Palmas (2000)}]

\beginverse
Paroles…
\endverse

\beginsong{Ursule}[by={Traditionnel}]

\beginverse
Refrain
Ohhue \! ohhue \! oh Ursule
D’amour pour toi mon cœur brûle
Il faudrait, il faudrait une pompe à vapeur
Pour éteindre le feu qui consume mon cœur. 
\endverse

\beginverse
J'aime tes beaux yeux
Derrière tes lunettes
On dirait les feux
De ma camionnette.
\endverse

\beginverse
Refrain
\endverse

\beginverse
J'aime ton gros nez
Ton nez plein de moque
Et quand tu te mouches
C'est un peu loufoque.
\endverse

\beginverse
Refrain
\endverse

\beginverse
J'aime tes cheveux
Tes cheveux filasses
Tombant sur tes yeux
Ça fait dégueulasse.
\endverse

\beginverse
Refrain
\endverse

\beginverse
J'aime tes oreilles
Tes oreilles de vache
Qui penchent en avant
Dans le sens de marche.
\endverse

\beginverse
Refrain
\endverse

\beginverse
J'aime tes gros pieds
Qui puent le fromage
Le fromage râpé
Qu'on met dans l'potage.
\endverse

# 

\beginsong{La vache à l’école \- j’rentre dans mon étable}[by={Carmen Campagne (2012)}]

\beginverse
Je rentre dans l'étable pour tirer mon vache
Pas capable tirer mon vache
Je prends un petit banc pour tirer mon vache
Pas capable tirer mon vache
Je prends un peu de bouse
Je le mets sur mon face
J'ai les yeux tout barbouillés
Et un grand \\[surp] chocolat chaud \\[bis]
\endverse

# 

\beginsong{La vache sentimentale}[by={Traditionnel}]

\beginverse
J’ai acheté une vache, une vache à cinq cents francs
Et je la menais paître et paître dans un champ
\endverse

\beginverse
Refrain
Elle a du sentiment ma vache
Elle a du sentiment
\endverse

\beginverse
Et je la menais paître et paitre dans un champ
Quand elle fut arrêtée par cinq ou six agents
\endverse

\beginverse
Quand elle fut arrêtée par cinq ou six agents
On la fit amener devant le parlement
\endverse

\beginverse
On la fit amener devant le parlement
Elle releva la queue et s'assit sur un banc
\endverse

\beginverse
Elle releva la queue et s'assit sur un banc
Elle envoya sa queue dans l'œil du président
\endverse

\beginverse
Elle envoya sa queue dans l'œil du président
Le président furieux l'envoya chez l'huissier
\endverse

\beginverse
Le président furieux l'envoya chez l'huissier
L'huissier tout étonné la mit en liberté
\endverse

\beginsong{Le vent se lève}[by={Patrick Chamblas (2017)}]

\beginverse
La nuit semblait profonde, l’hiver interminable
Les Hommes asservis, les rois indétrônables
La guerre sur le monde, l’argent impitoyable
Détruisaient nos espoirs et nos châteaux de sable
\endverse

\beginverse
Refrain
Mais le vent se lève, emportant les géants de papier
Mais le vent se lève, et plus rien ne peut nous arrêter
Et plus rien ne peut nous arrêter \\[bis]
\endverse

\beginverse
Nos coeurs semblaient fanés, nos larmes inutiles
La mer charriait les corps des enfants de l’exil
La terre empoisonnée et sa forêt à vendre
Les cieux indifférents et nos étoiles en cendre 
\endverse

\beginverse
Refrain
\endverse

\beginverse
Les géants de papier, l’empire aux pieds d’argile
Vos jours sont comptés, l’espoir a pris racine
Car la nuit va finir, le printemps va venir
Les Hommes insoumis partagent l’avenir
\endverse

\beginverse
Refrain
\endverse

\beginsong{Les vents contraires}[by={Carrousel (2010)}]

\beginverse
Paroles…
\endverse

\beginsong{La vie en rose}[by={Edith Piaf (1946)}]

\beginverse
Des yeux qui font baisser les miens
\endverse

\beginverse
Un rire qui se perd sur sa bouche
\endverse

\beginverse
Voilà le portrait sans retouches
\endverse

\beginverse
De l'homme auquel j'appartiens
\endverse

\beginverse
Quand il me prend dans ses bras
\endverse

\beginverse
Qu'il me parle tout bas
\endverse

\beginverse
Je vois la vie en rose
\endverse

\beginverse
Il me dit des mots d'amour
\endverse

\beginverse
Des mots de tous les jours
\endverse

\beginverse
Et ça m'fait quelque chose
\endverse

\beginverse
Il est entré dans mon cœur
\endverse

\beginverse
Une part de bonheur
\endverse

\beginverse
Dont je connais la cause
\endverse

\beginverse
C'est lui pour moi, moi pour lui dans la vie
\endverse

\beginverse
Il me l'a dit, l'a juré pour la vie
\endverse

\beginverse
Et dès que je l'aperçois
\endverse

\beginverse
Alors je sens dans moi
\endverse

\beginverse
Mon cœur qui bat
\endverse

\beginverse
Des nuits d'amour à plus finir
\endverse

\beginverse
Un grand bonheur qui prend sa place
\endverse

\beginverse
Des ennuis, des chagrins s'effacent
\endverse

\beginverse
Heureux, heureux à en mourir
\endverse

\beginverse
Quand il me prend dans ses bras
\endverse

\beginverse
Qu'il me parle tout bas
\endverse

\beginverse
Je vois la vie en rose
\endverse

\beginverse
Il me dit des mots d'amour
\endverse

\beginverse
Des mots de tous les jours
\endverse

\beginverse
Et ça me fait quelque chose
\endverse

\beginverse
Il est entré dans mon cœur
\endverse

\beginverse
Une part de bonheur
\endverse

\beginverse
Dont je connais la cause
\endverse

\beginverse
C'est toi pour moi, moi pour toi dans la vie
\endverse

\beginverse
Il me l'a dit, l'a juré pour la vie
\endverse

\beginverse
Et dès que je t'aperçois
Alors je sens dans moi
Mon cœur qui bat
Lala-lala-lala
\endverse

\beginsong{Viens un peu voir à côté}[by={Jean Pradelles, Pierre Pradelles (1989)}]

\beginverse
Viens un peu voir à côté
Viens avec nous si tu veux
Et tu verras dans mes yeux
Qu'il est des larmes de pluie
Qui font fleurir des mercis
Qu'il est des nouveaux matins
Pour inventer des chemins
Où nous pourrons bâtir un avenir.
\endverse

\beginverse
Un avenir où la guerre
N'appartiendra qu'au passé
Où violence et misère
Ne seront plus qu'une idée
Où l'espérance chaque jour
Cisélera des mots d'amour
Allez 1 Viens partager nos chemins.
\endverse

\beginverse
Viens un peu voir à côté
S'il n'y aurait pas des étés
Où l'on vivrait sans façon
Au rythme des quatre saisons
Où l'on gagnerait du temps
En le perdant chaque instant
Comme on trouve sa vie en la donnant.
\endverse

\beginverse
Donner sa vie pour la vie
C'est la vie à l'infini
Etre premier ou dernier
Du moment qu'on est aimé
Et si des larmes au paupières
N'existaient plus sur la terre
Que pour être des perles de lumière.
\endverse

\beginverse
Une lumière, un chemin
La vérité dans nos mains
Une vie de chaque instant
Où la porte à deux battants
Pourrait renaître à la vie
Qu'on appelle «paradis»
Je crois bien que j'ai rêvé tu sais.
\endverse

\beginverse
Tu sais, c'est vrai que la vie
Souvent ne fait pas envie
C'est vrai qu'elle va de travers
Et que ce pauvre univers
Si gonfié de son chagrin
Peut éclater au matin
Un beau jour d'un grand sanglot trop lourd.
\endverse

\beginverse
Viens avec nous si tu veux
On va faire de notre mieux
Pour qu'aux enfants de demain
On puisse tendre la main
Pour qu'ils fabriquent à leur tour
Un autre monde où l'amour
Signerait les pages des années.
\endverse

\beginverse
Viens retrouver le silence
Avant d'entrer dans la danse
Au tourbillon de la vie
Au cœur du monde en folie
Viens, ne me laisse pas seul
Je vais me casser la gueule
Dans les bas-fonds des douces illusions.
\endverse

\beginverse
Allez \! Dis, viens, n'aie pas peur
C'est pas quelques coups au cœur
Qui vont barrer le chemin
Vers de nouveaux lendemains
Allez \! C'est pour aujourd'hui
Qu'il faut faire la belle vie
\endverse

\beginsong{Vogulisi}[by={Matty Valentino (2009)}]

\beginverse
Paroles…
\endverse

\beginsong{Vois sur ton chemin}[by={Bruno Coulais, Les petits Chanteurs de Saint-Marc \- Les choristes (2004)}]

\beginverse
Paroles…
\endverse

\beginsong{Voyage dans le temps}[by={Lynn Ahrens \- Anastasia (1997)}]

\beginverse
Paroles…
\endverse

\beginsong{Voyage voyage}[by={Desireless (1989)}]

\beginverse
Paroles…
\endverse

\beginsong{Wild World}[by={Cat Stevens (1970)}]

\beginverse
Paroles…
\endverse

\beginsong{Wonderwall}[by={Oasis (1995)}]

\beginverse
Paroles…
\endverse

\beginsong{Ye jacobites}[by={Tri Yann (1985)}]

\beginverse
Paroles…
\endverse

\beginsong{Yellow submarines}[by={The Beatles (1968)}]

\beginverse
In the town where I was born
Lived a man who sailed to sea
And he told us from his life
In the land of submarines
So we sailed toward the sun
Till we found the sea of green
And we lived beneath the waves
In our yellow submarine
\endverse

\beginverse
Refrain
We all live in a yellow submarine
Yellow submarine, yellow submarine
\\[bis]
\endverse

\beginverse
And our friends are all aboard
Many more of them live next door
And the band begins to play
As we live a life of ease
Everyone of use has what we need
Sky of blue and sea of green
In our yellow submarine
\endverse

\beginsong{Yesterday}

\beginverse
Yesterday
All my troubles seemed so far away
Now it looks as if they're here to stay
O I believe in Yesterday
\endverse

\beginverse
Suddenly
I'm not half the man I used to be
There's a shadow hanging over me
Oh yesterday came suddenly
\endverse

\beginverse
Why she had to go I don't kno
She wouldn't say
I said something wrong now I long
For yesterday
\endverse

\beginverse
Yesterday
Love was such an easy game to play
No I need a place to hide away
O I believe in yesterday
\endverse

# 

\beginsong{Yvan, Boris et moi}[by={Marie Laforêt (1967)}]

\beginverse
Paroles…
\endverse
\endsong
\beginsong{Afferrare la banana}[by={Scoutismo Ticino}]

\beginverse
Afferrare la banana \\[rép]
\endverse

*Refrain*
La bana-nana-nanana
Me la mangio me la mangio
La bana-nana-nanana
Me la mangio per dessert

\beginverse
Afferrare… \\[rép]
E sbucciare la banana \\[rép]
\endverse

	*Refrain*

\beginverse
Afferare… \\[rép]
E sbucciare… \\[rép]
E tagliare… \\[rép]
E cucinare… \\[rép]
E mangiare… \\[rép]
E digerire… \\[rép]
E vomitare… \\[rép]
E rimangiare… \\[rép]
E ridigerire… \\[rép]
E cagare la banana \\[rép]
\endverse

# 

\beginsong{Boogie Woogie}

\beginverse
On met la main en avant, on met la main en arrière 
On met la main en avant et on secoue un petit peu 
On danse le boogie woogie, on fait un petit tour 
Et c'est toute ma chanson
\endverse

	Refrain
Oh \! boogie woogie \! oh boogie woogie \!
(et ainsi de suite avec presque toutes les parties du corps)

# 

\beginsong{Brousse, brousse}

\beginverse
Brousse, brousse, j'aime la brousse
J'aime la brousse et la jolie savane
Y'a des tigres, y'a des lions, y'a des léopards
J'aime la brousse et dans ma jolie savane
\endverse

# 

\beginsong{Canon du feu}

\beginverse
A. Entendez-vous dans le feu
B. Tous ces bruits mystérieux
C. Ce sont les tisons qui chantent
D. Eclaireur \\[Louveteau], soit joyeux
\endverse
\endsong
\beginsong{Canon du vent}

\beginverse
A. Vent frais, vent du matin
B. Vent qui chante au milieu des grands pins
C. Joie du vent qui chante
D. Allons dans le grand
\endverse

# 

\beginsong{Cavaliers}

	Refrain
Nous sommes tous, tous, tous
Des cavaliers, liers, liers
Celui qui n'fait pas attention n'aura pas droit à sa gamelle
Celui qui n'fait pas attention n'aura pas droit à son bidon

\beginverse
Attention, cavaliers, la main droite va commencer
Attention, cavaliers, la main gauche va commencer
Le pied droite..... le pied gauche.... la tête
Attention, cavalier, la tête va s'arrêter
Le pied gauche.... le pied droite.... la main gauche.... la main droite ....
Attention, cavaliers, la chanson va s'arrêter
\endverse

# 

\beginsong{Dans la troupe}

\beginverse
Roum roum \- roum roum roum
\endverse

\beginverse
Papa, maman, votre enfant n'a qu'un oeil
Papa, maman, votre enfant n'a qu'une dent.
\endverse

\beginverse
Ah \! que c'est embêtant d'avoir un enfant qui n'a qu'un oeil
Ah \! que c'est embêtant d'avoir un enfant qui n'a qu'une dent.
\endverse

\beginverse
Dans la troupe y a pas de jambe de bois
Y a des nouilles, mais ça ne se voit pas
La meilleure façon de marcher
C'est sûrement is notre
C'est de mettre un pied devant l'autre
Et de recommencer.
\endverse

\beginverse
Et son papa lui achètera une jolie trompette
Et son papa lui achètera une trompette en bois.
\endverse
\endsong
\beginsong{Dans mon pays d’Espagne}

\beginverse
Y a le soleil comme ça \\[bis]
\endverse

	Refrain
Dans mon pays d'Espagne... Olé \!
Dans mon pays d'Espagne... Olé \!

\beginverse
Y a la guitare comme ça \\[bis]
Y a le soleil comme ça \\[bis]
\endverse

\beginverse
Des castagnettes comme ça \\[bis]
Et la guitare comme ça \\[bis]
Et le soleil comme ça \\[bis]
\endverse

\beginverse
Y a des danseuses comme ça \\[bis]
Des castagnettes…
Et la guitare…
Et le soleil…
\endverse

\beginverse
Y a le taureau comme ça \\[bis]
Et les danseuses…
Des castagnettes…
Et la guitare…
Et le soleil…
\endverse
\endsong
\beginsong{Derrière chez moi}

\beginverse
Derrière chez moi, devinez quoi qu'il y a
Derrière chez moi, devinez quoi qu'il y a
\endverse

\beginverse
Y a un bois
Le plus beau des bois
Bois, petit bois derrière chez moi
Et la lonla lonlère et la lonla lonla \\[bis]
 
Et dans le bois, devinez quoi qu'il y a
Et dans le bois, devinez quoi qu'il y a
\endverse

\beginverse
Y a un arbre
Le plus beau des n'arbres
N'arbre, petit n’arbre derrière chez moi
\endverse

\beginverse
Et dans le n’arbre, devinez quoi qu’il y a
…
Y a une branche, la plus belle des branches
Branche sur le n’arbre, n’arbre dans le bois, bois, petit…
\endverse

\beginverse
Y a une feuille…
Y a un nid…
Y a un œuf…
Y a un p'tit zozio...
Y a un boyeau…
Y a une crotte…
Y a une graine…
Y a un arbre…
etc.
\endverse
\endsong
\beginsong{Et on pagaie}

\beginverse
Et on pagaie, on pagaie \\[rép]
Mais où t'as mis les pagaies \\[rép]
Sous les grands bananiers \\[rép]
Mais les crocos les ont bouffées \\[rép]
Et on peut plus pagayer \\[rép]
\endverse
\endsong
\beginsong{Rassemblement des sizaines}

\beginverse
Paroles…
\endverse
\endsong
\beginsong{Fli-Fly }

\beginverse
Fly \\[rép]
Fly flay \\[rép]
Fly flay flo \\[rép]
Westa \\[rép]
Oh na na westa \\[rép]
Ini mini tessamini ouh ouh ouh ah \\[rép]
Ini mini salamini ouh ouh ouh ah \\[rép]
Ix bi dedi dela de bobo dedi dela de gs gs \\[rép]
\endverse
\endsong
\beginsong{Lai lai leo}

\beginverse
Paroles…
\endverse
\endsong
\beginsong{Si tu vas au ciel}

\beginverse
Si tu vas au ciel \\[bis]
Bien avant moi \\[bis]
Fais un pt'it trou \\[bis]
Tire-moi par là \\[bis]
Si tu vas en enfer
Bien avant moi
Bouche tous les trous
Que j'n'y aille pas
\endverse

\beginverse
On n'va pas au ciel
En limousine
Car il n'y a pas
De gazoline
\endverse

\beginverse
On n'va pas au ciel
En patinant
Car y'a pas d'glace
Au firmament
\endverse

\beginverse
Si tu y vois Molière
Demande-lui voir
S'il pouvait faire
Des vers si beaux
\endverse

\beginverse
On n'va pas au ciel
En p'tit bateau
Car on s'casse la gueule
Dans les chutes d'eau
\endverse

\beginverse
Les beaux livres disent
Qu'Cain tua Abel
En frappant sur sa tête
Avec l'manche d'une pelle
\endverse

\beginverse
Et maintenant c'est fini
Saint-Pierre l'a dit
Ferma la porte
Et s'mit au lit
\endverse
\endsong
\beginsong{J'ai acheté une vache à lait}

\beginverse
J’ai acheté une vache à lait \\[rép]
En action à la Coop \\[rép]
J’l’ai monté à mon chalet \\[rép]
Chaque matin, elle me donnait \\[rép]
Une monstre boille à lait \\[rép]
\endverse
\endsong
\beginsong{Madelon}

\beginverse
Madeleine, elle a un pied mariton \\[rép]
Un pied mariton \\[rép]
\endverse

\beginverse
Un pied mariton, Madeleine
Un pied mariton, Madelon.
\endverse

\beginverse
Madeleine, elle a une jambe de bois \\[rép]
Une jambe de bois \\[rép]
Un pied mariton \\[rép]
\endverse

\beginverse
Un pied mariton, Madeleine
Un pied mariton, Madelon
\endverse

\beginverse
Madeleine, elle a …
… une cuisse de velours
… des g'noux cagneux
… un ventre d'acier
… un poumon gainé
… une dent d'ciment
… un poil dans la main
… un œil de verre
… les cheveux en botte de foin
… un nez en trompette
… etc.
\endverse

\beginverse
Mais la Madelon elle a du cœur … 
\endverse
\endsong
\beginsong{Ouaélé }

\beginverse
Paroles…
\endverse
\endsong
\beginsong{Les p’tits potes}

\beginverse
Paroles…
\endverse
\endsong
\beginsong{Petrouchka}

\beginverse
C'est le marchand Petrouchka qui revient
D'or, il a rempli son sac et il est content
Quand ses chevaux fatigués auront bu
Toute la nuit il pourra rire et chanter
\endverse

\beginsong{Mova on y va \!}[by={Verein BuLa 2021 (2021)}]

\beginverse
Paroles…
\endverse

# 

\beginsong{CAJU 2010 \- Scouts Airlines}[by={Scouts Perceval (2010)}]

\beginverse
Il me faudra bien plus qu'une nuit
Pour arriver à ma destination 
Mais cette fois j'ai pris ma décision 
Ciao Maman, je pars en voyage 
Ce matin j'ai préparé mes bagages 
Une brosse à dents, des tongues et un caleçon 
J'ai même entrepris d'écrire une chanson Pour chanter au-dessus des nuages 
\endverse

\beginverse
Refrain
J'ai recherché la meilleure compagnie.   Mais après tout celle que j'ai choisie.  
Une compagnie pour les petits et grands 
C'est Scout Airlines, celle qui va de l'avant.
 
Ça y est c'est fait, cette fois j'ai mon billet J'pars à Paris, New-York et à Sydney 
J'fais un détour par la Patagonie 
C'est l'aventure et ça ça m'plait 
Là-bas, j'ai rencontré tout plein de gens Avec qui j'ai partagé des traditions 
Ils m'ont fait découvrir leur beau pays 
Et maintenant, nous sommes d'excellents amis. 
\endverse

\beginverse
Refrain
\endverse

\beginverse
À toute bonne chose, il y a une fin 
C'est le moment de reprendre le chemin Vers ma maison, mon chat et ma famille Ciao les gars, je retourne au pays. 
Ne vous inquiétez pas je reviendrai 
Peut-être même plus vite que vous l'pensez 
D'ailleurs j'ai déjà trouvé mon billet 
Ciao Maman, je repars en voyage. 
\endverse

\beginverse
J'ai recherché la meilleure compagnie
Celle qui permet de se faire des amis...
\endverse

# 

\beginsong{CAJU 2016 \- A la recherche du temps perdu}[by={Scouts St-Germain \- parodie de Tim Toupet (2016)}]

\beginverse
Paroles…
\endverse

\beginsong{CAJU 2018 \- ???}[by={Scouts St-Louis \- parodie de Renaud (2018)}]

\beginverse
Paroles…
\endverse

# 

\beginsong{CAJU 2019 \- La cité des éléments}[by={Scouts Delémont (2019)}]

\beginverse
Capo I
\endverse

\\[Am]Au début du monde \\[F]apparurent deux \\[C]mages
\\[Am]L’un nommé Hou\\[F]ro, l’autre Ma\\[C]kanka
En\\[Am]semble ils créèrent \\[F]l’air, l’eau, le feu, la \\[C]terre
Ain\\[Am]si qu’une cité comme ja\\[F]mais n’a exis\\[C]té
A la \\[Am]voir, on \\[F]aurait pu pa\\[C]rier
Que \\[Am]rien ne pouvait \\[F]la troubler \\[C]

\beginverse
La \\[Am]cité des \\[F]éléments \\[C] est \\[Am]apparue \\[F]du néant \\[C]
For\\[Am]mée de terre, de \\[F]feu brûlant \\[C]
De \\[Am]l’air, de \\[F]l’eau des océans \\[C]
\endverse

\beginverse
Mais les deux mages se disputèrent
Jour après jour, parcourant ciel et terre
Un jour de grande dispute, Makanka en colère,
Avec son grand pouvoir, changea Houro en pierre
Dès lors, la cité fut prospère
Et ce, durant deux millénaires
\endverse

\beginverse
La cité des éléments est apparue du néant
Formée de terre, de feu brûlant
De l’air, de l’eau des océans
\endverse

\beginverse
Lorsqu’il se réveilla, Houro, hors de lui
Forma une armée de pierre, dans le secret de la nuit
Il terrorise depuis la paisible cité
C’est pourquoi Makanka nous a demandé de l’aider
\endverse

\beginverse
La cité des éléments est apparue du néant
Formée de terre, de feu brûlant
De l’air, de l’eau des océans
\endverse

\beginsong{CAJU 2021 \- Jura express?}[by={Scouts Pierre-Pertuis (2021)}]

\beginverse
Paroles…
\endverse

\beginsong{JUBACA 2025 \- }[by={Jubéquipe \- parodie d’Indochine (2025)}]

\beginverse
Paroles…
