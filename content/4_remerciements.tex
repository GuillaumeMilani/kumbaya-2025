\section*{Remerciements}
Cette nouvelle édition du Kumbaya n'aurait jamais vu le jour sans la passion, l'enthousiasme et l'engagement d'une foule de personnes talentueuses. Chaque accord, chaque mot et chaque page portent la trace de leur contribution. Nous leur adressons ici notre profonde reconnaissance :

\begin{flushleft}
\textbf{Couverture} \linebreak
Nadège Fleury / Loutre

\textbf{Accords et méthodologie musicale} \linebreak
Elori Baume / Alpaga

\textbf{Rédaction} \linebreak
Charline Unternährer / Écureuil \linebreak
Matteo Seragioli / Grizzly \linebreak
Quentin Gyseler / Pinson \linebreak
Raphaël Seuret / Wombat

\textbf{Mise en page} \linebreak
Guillaume Milani / Pécari

\textbf{Relectures} \linebreak
Charlie Valley / Luciole
Mathilde Heyer / Chouette
\end{flushleft}

Nous souhaitons également remercier chaleureusement l'équipe de rédaction du nouveau P'tit Romand, ainsi que les nombreuses et nombreux internautes des forums La Boîte à chansons et Ultimate Guitar, dont les ressources partagées ont grandement enrichi notre travail. Un grand merci aussi aux auteur·ice·s des chansons de camps jurassiens, de mova et de JUBACA, qui insufflent l'esprit scout à chaque couplet.

Ce chansonnier est le fruit d'un engagement entièrement bénévole visant pour une distribution non commerciale et non lucrative.

Achevé d'imprimer sur les presses de l'imprimerie Pressor SA à Delémont, en juillet 2025.
