\beginsong{Afferrare la banana}[by={Scoutismo Ticino}]

\beginverse
\bis{Afferrare la banana}{2}
\endverse

\beginchorus
La bana-nana-nanana
Me la mangio me la mangio
La bana-nana-nanana
Me la mangio per dessert
\endchorus

\beginverse
\bis{Afferrare…}{2}
\bis{E sbucciare la banana}{2}
\endverse

\beginchorus
Refrain
\endchorus

\beginverse
\bis{Afferare…}{2}
\bis{E sbucciare…}{2}
\bis{E tagliare…}{2}
\bis{E cucinare…}{2}
\bis{E mangiare…}{2}
\bis{E digerire…}{2}
\bis{E vomitare…}{2}
\bis{E rimangiare…}{2}
\bis{E ridigerire…}{2}
\bis{E cagare la banana}{2}
\endverse

\endsong
\beginsong{Boogie Woogie}

\beginverse
On met la \[Sol]main en avant, on met la \[Sol]main en arrière
On met la \[Sol]main en avant et on se\[Ré]coue un petit peu
On \[Ré]danse le boogie woogie, \[Ré]on fait un petit tour
Et \[Ré]c'est toute ma chanson
\endverse

\beginchorus
Oh ! boogie woogie ! oh boogie woogie !
(Et ainsi de suite avec les parties du corps)
\endchorus

\endsong
\beginsong{Brousse, brousse}

\beginverse
Brousse, brousse, j'aime la brousse
J'aime la brousse et la jolie savane
Y'a des tigres, y'a des lions, y'a des léopards
J'aime la brousse et dans ma jolie savane
\endverse

\endsong
\beginsong{Canon du feu}

\beginverse
A. Entendez-vous dans le feu
B. Tous ces bruits mystérieux
C. Ce sont les tisons qui chantent
D. Eclaireur (Louveteau), soit joyeux
\endverse

\endsong
\beginsong{Canon du vent}

\beginverse
A. Vent frais, vent du matin
B. Vent qui chante au milieu des grands pins
C. Joie du vent qui chante
D. Allons dans le grand
\endverse

\endsong
\beginsong{Cavaliers}

\beginchorus
Nous sommes \[Ré]tous, tous, tous
Des cavaliers, liers, liers
Celui qui n'\[La] fait pas attention n'aura pas \[Mi]droit à sa ga\[La]melle
Ce\[Ré]lui qui n'\[Mi] fait pas attention n'aura pas droit à son bi\[La]don
\endchorus

\beginverse
Attention, cavaliers, la main droite va commencer
Attention, cavaliers, la main gauche va commencer
Le pied droite..... le pied gauche.... la tête
Attention, cavalier, la tête va s'arrêter
Le pied gauche.... le pied droite.... la main gauche.... la main droite
Attention, cavaliers, la chanson va s'arrêter
\endverse

\endsong
\beginsong{Dans la troupe}

\beginverse
Roum roum \- roum roum roum
\endverse

\beginverse
Papa, maman, votre enfant n'a qu'un oeil
Papa, maman, votre enfant n'a qu'une dent
\endverse

\beginverse
Ah ! que c'est embêtant d'avoir un enfant qui n'a qu'un oeil
Ah ! que c'est embêtant d'avoir un enfant qui n'a qu'une dent
\endverse

\beginverse
Dans la troupe y a pas de jambe de bois
Y a des nouilles, mais ça ne se voit pas
La meilleure façon de marcher
C'est sûrement is notre
C'est de mettre un pied devant l'autre
Et de recommencer
\endverse

\beginverse
Et son papa lui achètera une jolie trompette
Et son papa lui achètera une trompette en bois
\endverse

\endsong
\beginsong{Dans mon pays d'Espagne}


\beginchorus
Dans \[Rém]mon pays d'Espagne... O\[La]lé !
Dans \[Rém]mon pays d'Espagne... O\[La]lé !
\endchorus

\beginverse
Y a le so\[La]leil comme ça (bis)
\endverse

\beginverse
Y a la gui\[Rém]tare comme ça
Y a le soleil comme ça (bis)
\endverse

\beginverse
Des castagnettes comme ça (bis)
Et la guitare comme ça (bis)
Et le soleil comme ça (bis)
\endverse

\beginverse
Y a des danseuses comme ça (bis)
Des castagnettes…
Et la guitare…
Et le soleil…
\endverse

\beginverse
Y a le taureau comme ça (bis)
Et les danseuses…
Des castagnettes…
Et la guitare…
Et le soleil…
\endverse

\endsong
\beginsong{Derrière chez moi}

\beginverse
Derrière chez \[Do]moi, de\[Sol]vinez quoi qu'il y \[Do]a
Derrière chez \[Do]moi, devinez q\[Sol]uoi qu'il y \[Do]a
\endverse

\beginverse
Y a un \[Sol]bois
\[Do]Le plus beau des \[Sol]bois
\[Ré7]Bois, petit bois, petite bois derrière chez \[Sol]moi
Et la lon\[Fa]la lon\[Sol]lère et la lon\[Do]la lon\[Do7]la (bis)
\endverse

\beginverse
Et dans le bois, devinez quoi qu'il y a
Et dans le bois, devinez quoi qu'il y a
\endverse

\beginverse
Y a un arbre
Le plus beau des n'arbres
N'arbre, petit n'arbre, petit n'arbre derrière chez moi
\endverse

\beginverse
Et dans le n'arbre, devinez quoi qu'il y a
Y a une branche, la plus belle des branches
Branche sur le n'arbre, n'arbre dans le bois, bois, petit
\endverse

\beginverse
Y a une feuille
Y a un nid
Y a un œuf
Y a un p'tit zozio
Y a un boyeau
Y a une crotte
Y a une graine
Y a un arbre
\endverse

\endsong
\beginsong{Et on pagaie}

\beginverse
\bis{Et on pagaie, on pagaie}{2}
\bis{Mais où t'as mis les pagaies}{2}
\bis{Sous les grands bananiers}{2}
\bis{Mais les crocos les ont bouffées}{2}
\bis{Et on peut plus pagayer}{2}
\endverse

\endsong
\beginsong{Rassemblement des sizaines}


\beginchorus
Qui \[Do]donc rassemblera sa sizaine
\[Sol]Au Rocher la prem\[Do]ière ?
Qui \[Do]donc rassemblera sa sizaine
\[Sol]Au Rocher d'Ake\[Do]la ?
\endchorus

\beginverse
\[Sol]Louvettes et lou\[Do]varts
Craig\[Sol]nez d'être en re\[Do]tard
\[Fa]La saute\[Rém]relle sau\[Do]te, bondit et \[Sol]court
\[Fa]Léger comme elle, cou\[Do]rons cou\[Sol]rons tou\[Do]jours
\endverse

\beginchorus
Refrain
\endchorus

\endsong
\beginsong{Fli-Fly}

\beginverse
Fly (bis)
Fly flay (bis)
Fly flay flo (bis)
Westa (bis)
Oh na na westa (bis)
Ini mini tessamini ouh ouh ouh ah (bis)
Ini mini salamini ouh ouh ouh ah (bis)
Ix bi dedi dela de bobo dedi dela de gs gs (bis)
\endverse

\endsong
\beginsong{Lai lai leo}

\beginverse
Lai lai leo (bis)
Lai lai leo leo leo lai
La lai lai leo (bis)
La lai lai leo leo lai
\endverse

\endsong
\beginsong{Si tu vas au ciel}

\beginverse
Si tu vas au \[Mi]ciel (bis)
Bien avant \[Mi]moi (bis)
Fais un pt'it t\[Si7]rou (bis)
Tire-moi par \[Mi]là (bis)
Si tu vas au c\[La]iel
Bien avant \[La]moi
Fais un pt'it t\[Mi]rou
Tire-moi par \[La]là
\endverse

\beginverse
Ayi \[Si7]ayoh-oh-oh !
Ayi ayoh-oh-\[Mi]oh !
(bis)
\endverse

\beginverse
Si tu vas en enfer
Bien avant moi
Bouche tous les trous
Que j'n'y aille pas
\endverse

\beginverse
On n'va pas au ciel
En limousine
Car il n'y a pas
De gazoline
\endverse

\beginverse
On n'va pas au ciel
En patinant
Car y'a pas d'glace
Au firmament
\endverse

\beginverse
Si tu y vois Molière
Demande-lui voir
S'il pouvait faire
Des vers si beaux
\endverse

\beginverse
On n'va pas au ciel
En p'tit bateau
Car on s'casse la gueule
Dans les chutes d'eau
\endverse

\beginverse
Les beaux livres disent
Qu'Cain tua Abel
En frappant sur sa tête
Avec l'manche d'une pelle
\endverse

\beginverse
Et maintenant c'est fini
Saint-Pierre l'a dit
Ferma la porte
Et s'mit au lit
\endverse

\endsong
\beginsong{J'ai acheté une vache à lait}

\beginverse
\bis{J'ai acheté une vache à lait}{2}
\bis{En action à la Coop}{2}
\bis{J'l'ai monté à mon chalet}{2}
\bis{Chaque matin, elle me donnait}{2}
\bis{Une monstre boille à lait}{2}
\endverse

\endsong
\beginsong{Madelon}

\beginverse
\bis{\[La]La Madelon, elle a un \[La7]pied mari\[Ré]ton}{2}
\bis{Un pied mariton}{2}
\endverse

\beginverse
Un pied mariton, Madeleine
Un pied mariton, Madelon
\endverse

\beginverse
\bis{La Madelon, elle a un mollet bien fait}{2}
\bis{Un mollet bien fait}{2}
\bis{Un pied mariton}{2}
\endverse

\beginverse
Un pied mariton, Madeleine
Un pied mariton, Madelon
\endverse

\beginverse
La Madelon, elle a
\ldots un g'noux d'quenouilles
\ldots une cuisse de velours
\ldots une jambe de bois
\ldots une hanche plastique
\ldots un ventre d'acier
\ldots un poumon gainé
\ldots un cou d'girafe
\ldots un cou d'girafe\ldots une dent d'ciment
\ldots un nez en trompette
\ldots un œil de vitre
\ldots les oreilles en bele-beleu
\ldots les cheveux en botte de paille
\ldots etc.
\endverse

\beginverse
Mais la Madelon elle a du cœur
\endverse

\endsong
\beginsong{Oh a élé}

\beginverse
Oh a élé (bis)
A vidi vidi gamba (bis)
A vélo sato (bis)
A vélo pompi (bis)
A caspe tao (bis)
A tchoutchou tchoutchou tchou véneli (bis)
\endverse

\beginverse
Ça chante pas encore assez (bis)
Il faut chanter beaucoup plus fort (bis)
Il faut gueuler encore plus fort (bis)
Il faut chanter beaucoup moins fort (bis)
\endverse

\endsong
\beginsong{Les p'tits potes}

\beginchorus
\[Do]Ah les p'tits potes (4x)
Ah les p'tits \[Sol]potes, potes \[Sol]potes
\endchorus

\beginverse
\bis{Chez les ptits potes}{2}
\bis{Y a un vieillard}{2}
\bis{Ils l'appellent tous}{2}
\bis{Le pote âgé, Le potager !}{2}
\endverse

\beginverse
\bis{Chez les p'tits potes}{2}
\bis{Y a un pompier}{2}
\bis{Ils l'appellent tous}{2}
\bis{Le pote au feu, Le pot au feu !}{2}
\endverse

\beginverse
\bis{Chez les p'tits potes{2}
\bis{Y a un aviateur{2}
\bis{Ils l'appellent tous{2}
Le pote en ciel, le potentiel !
\endverse

\beginverse
\bis{Chez les p'tits potes}{2}
\bis{Y a un député}{2}
\bis{Ils l'appellent tous}{2}
Le pote de chambre, Le pot' de chambre !
\endverse

\beginverse
\bis{Chez les p'tits potes}{2}
\bis{Y a un géant}{2}
\bis{Ils l'appellent tous}{2}
\bis{Le grand pote haut, Le grand poteau !}{2}
\endverse

\beginverse
\bis{Chez les p'tits potes}{2}
\bis{Y a un fleuriste}{2}
\bis{Ils l'appellent tous}{2}
\bis{Le pote aux roses, Le pot aux roses !}{2}
\endverse

\beginverse
\bis{Chez les p'tits potes}{2}
\bis{Y a un pharmacien}{2}
\bis{Ils l'appellent tous}{2}
\bis{L'apothicaire, l'apothicaire !}{2}
\endverse

\beginverse
\bis{Chez les p'tits potes}{2}
\bis{Y des triplés}{2}
\bis{Ils l'appellent tous}{2}
\bis{La tripotée, la tripotée !}{2}
\endverse

\beginverse
\bis{Chez les p'tits potes{2}
\bis{Y a un prof de maths{2}
\bis{Ils l'appellent tous{2}
L'hypoténuse, l'hypoténuse!
\endverse

\beginverse
Quand les p'tits potes (x2)
Partent en vacances (x2)
On n'sait jamais (x2)
Où les potes iront ! Les potirons !
\endverse

\endsong
\beginsong{Petrouchka}

\beginverse
\[Lam]C'est le marchand Petrouch\[Rém]ka qui revient
\[Mi]D'or, il a rempli son sac et \[Lam]il est content
\[Lam]Quand ses chevaux fati\[Rém]gués auront bu
\[Lam]Toute la \[Mi]nuit il pourra \[Lam]rire \[Mi]et \[Lam]chanter
\endverse

\endsong