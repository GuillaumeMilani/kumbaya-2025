\beginsong{Le festin}[by={Camille (2007)},sr={Capo III}]

\beginverse
Les \[Do]rêves des a\[Lam]moureux sont \[Rém]comme le bon \[Sol]vin
Ils \[Rém]donnent de \[Sol]la joie ou \[Mim]bien du cha\[Lam]grin
\[Sol]Affai\[Do]bli par la faim je suis malheu\[La7]reux
\[Rém]Volant en \[Sol]chemin tout \[Mim]ce que je \[La7]peux
Car \[Rém]rien n'est gra\[Sol]tuit dans la \[Do]vie
\endverse

\beginverse
Espoir est un plat bien trop vite consommé
À sauter les repas, je suis habitué
Un voleur, solitaire, est triste à nourrir
À nous je suis amer je veux réussir
Car rien n'est gratuit dans la vie
\endverse

\beginverse
\[Do]Jamais on ne me dir\[Fa]a
Que \[Rém]la course aux é\[Mim]toiles, \[Rém]ça n'est pas pour \[Sol]moi
\[Do]Laissez-moi vous émerveiller, prendre mon en\[Fa]vol
Nous \[Rém]allons enfin nous rég\[Sol]aler \[Do]
\endverse

\beginverse
La fête va enfin commencer
Et sortez les bouteilles, finis les ennuis
Je dresse la table, demain nouvelle vie
Je suis heureux à l'idée de ce nouveau destin
Une vie à me cacher, et puis libre enfin
Le festin est sur mon chemin
\endverse

\beginverse
Une \[Rém]vie à me \[Sol]cacher et \[Mi]puis libre en\[La7]fin
Le \[Fam]festin est \[Do]sur \[Sol7]mon che\[Do]min
\endverse

\endsong
\beginsong{La fleur au chapeau}[by={Traditionnel}]


\beginchorus
Une f\[Do]leur au chapeau
A la bouche une chan\[Sol7]son
Un cœur joyeux et sin\[Do]cère
Et c'est tout ce qu'il faut
A nous autres bons éc\[Sol7]lais
Pour aller au bout de la Te\[Do]rre
\endchorus

\beginverse
\[Do]Vous qui nous regar\[Fa]dez pas\[Do]ser
Sous le soleil ou \[Sol7]sous l'o\[Do]rage
Peut-être bien que \[Fa]vous pen\[Do]sez
Que nous avons bien \[Sol7]du cou\[Do]rage
Pour ain\[Lam]si nous haras\[Mi]ser
A cou\[Lam]rir le long des \[Mi]routes
Vous ne \[Lam]savez ce que \[Mi]c'est
Et vous n'au\[Lam]rez jamais sans \[Mi]doute
\endverse

\beginchorus
Refrain
\endchorus

\beginverse
Ah ! Comme nous serions heureux
Si nous pouvions la vie entière
Courir par les chemins ombreux
Ou sur les routes familières
Depuis les sommets neigeux
Jusqu'au bord des mers profondes
A travers nos cris joyeux
Nous dirons au vaste monde
\endverse

\beginchorus
Refrain
\endchorus

\beginverse
Hélas ! Il n'en est pas ainsi
Et notre tâche est plus aride
Mais il nous faut du cœur aussi
Il faut aussi des bras solides
Pour combattre sans merci
La laideur et la paresse
A travers luttes et soucis
Il nous faut garder sans cesse
\endverse

\endsong
\beginsong{Forever young}[by={Alphaville (1984)}]

\beginverse
\[Do]Let's dance in \[Sol]style, let's dance for a \[Lam]while
Heaven can \[Fa]wait, we're only watching the \[Sol]skies
Hoping for the \[Rém]best, but expecting the \[Fa]worst
Are you gonna drop the \[Lam]bomb or not ?\[Sol]
\endverse

\beginverse
Let us die young or let us live forever
We don't have the power, but we never say never
Sitting in a sandpit, life is a short trip
The music's for the sad men
\endverse

\beginverse
Can you imagine when this race is won ?
Turn our golden faces into the sun
Praising our leaders, we're getting in tune
The music's played by the, the mad men
\endverse

\beginchorus
Refrain
\endchorus

\beginchorus
\[Do] Forever youn\[Sol]g
I want to \[Lam]be forever \[Fa]young
\[Sol] Do you really want to \[Rém]live forever ?
\[Fa] Forever \[Sol] and ever ?
\[Do] Forever youn\[Sol]g
I want to \[Lam]be forever youn\[Fa]g
\[Sol] Do you really want to \[Rém]live forever ? \[Fa]
\[Sol] Forever \[Do]young \[Sol]
\endchorus

\beginverse
Some are like water, some are like the heat
Some are a melody and some are the beat
Sooner or later, they all will be gone
Why don't they stay young ?
\endverse

\beginverse
It's so hard to get old without a cause
I don't want to perish like a fading horse
Youth's like diamonds in the sun
And diamonds are forever
\endverse

\beginverse
So many adventures couldn't happen today
So many songs we forgot to play
So many dreams swinging out of the blue
We'll let them come true
\endverse

\beginverse
Forever young
I want to be forever young
Do you really want to live forever ?
Forever and ever ?
\endverse

\beginchorus
Refrain
\endchorus

\endsong
\beginsong{Foule sentimentale}[by={Alain Souchon (1993), Yael Naim (2018)}]

\beginchorus
\[Mim] Oh la la \[Do]la vie en ros\[Lam]e \[Si7]
\[Mim] Le rose \[Lam]qu'on nous propose \[Ré] \[Si7]
\[Mim] D'avoir les \[Do]quantités d'choses \[Lam]  \[Si7]
\[Mim] Qui donnent en\[Lam]vie d'autre chose \[Ré] \[Si7]
\endchorus

\beginverse
Aïe, on nous fait croire
Que le bonheur c'est d'avoir
De l'avoir plein nos armoires
Dérisions de nous dérisoires car
\endverse


\beginchorus
\[Mim]Foule senti\[Lam]mentale \[Ré]  \[Si7]
On a \[Mim]soif d'idé\[Do]al \[Lam] \[Si7]
\[Mim] Attirée \[Lam]par les étoiles, les voiles \[Ré] \[Si7]
\[Mim] Que des choses \[Do]pas commerciales \[Lam] \[Si7]
Foule sentimentale
Il faut voir comme on nous parle
Comme on nous parle
\endchorus

\beginverse
Il se dégage
De ces cartons d'emballage
Des gens lavés, hors d'usage
Et tristes et sans aucun avantage
On nous inflige
Des désirs qui nous affligent
On nous prend faut pas déconner dès qu'on est né
Pour des cons alors qu'on est
\endverse

\beginchorus
Refrain
\endchorus

\beginverse
On nous Claudia Schieffer
On nous Paul-Loup Sulitzer
Oh le mal qu'on peut nous faire
Et qui ravagea la moukère
Du ciel dévale
Un désir qui nous emballe
Pour demain nos enfants pâles
Un mieux, un rêve, un cheval
\endverse

\endsong
\beginsong{Le galérien}[by={Yves Montand (1950)},sr={Capo III}]

\beginverse
\[Do]Je m'souviens ma \[Sol7]mère m'aimait
Et je suis aux \[Do]galè\[Mi7]res
\[Lam]Je m'souviens ma \[Mi]mère disait
\[Mi7]Mais je n'ai pas cru ma \[Lam]mè\[Sol7]re
\[Do]Ne traîne pas dans \[Sol7]les ruisseaux
T'bats pas comme un sau\[Do]va\[Mi7]ge
\[Lam]T'amuse pas comme \[Mi]les oiseaux
\[Mi7]Elle me disait d'être \[Lam]sa\[Sol7]ge
\endverse

\beginverse
J'ai pas tué, j'ai pas volé
J'voulais courir ma chance
J'ai pas tué, j'ai pas volé
J'voulais qu'chaque jour soit dimanche
Je m'souviens comme elle pleurait
Dès que j'passais la porte
Je m'souviens comme elle pleurait
Elle voulait pas que je sorte
\endverse

\beginverse
Toujours, toujours elle disait
T'en vas pas chez les filles
Fais donc pas toujours c'qui t'plaît
Dans les prisons y a des grilles
J'ai pas tué, j'ai pas volé
Mais j'ai cru Madeleine
J'ai pas tué, j'ai pas volé
J'voulais pas lui faire de peine
\endverse

\beginverse
Je m'souviens ma mère disait
Suis pas les Bohémiennes
Je m'souviens comme elle disait
On ramasse les gens qui trainent
Un jour les soldats du roi
T emmèneront aux galères
Tu t'en iras trois par trois
Comme ils ont emmené ton père
\endverse

\beginverse
Tu auras la tête rasée
On te mettra des chaînes
T'en auras les reins brisés
Et moi j'en mourrai de peine
Toujours, toujours tu ram'ras
Quand tu s'ras aux galères
Toujours, toujours tu ram'ras
Tu pens'ras p't-être à ta mère
\endverse

\beginverse
J'ai pas tué, j'ai pas volé
Mais j'ai pas cru ma mère
Et je m'souviens qu'elle m'aimait
Pendant que j'rame aux galères
\endverse

\endsong
\beginsong{Le gâteau empoisonné}[by={Gérard Calvi \- Astérix et Cléopâtre (1968)}]

\beginverse
\[Lam]Dans un grand bol \[Mi]de strychnine
\[Lam]Délayez de \[Mi]la morphine
\[Sol]Faites tiédir à \[Do]la casserole
\[Sol]Un bon verre de pétrole
Ho ho, je vais en mettre deux
\endverse

\beginverse
Quelques gouttes de ciguë
De la bave de sangsue
Un scorpion coupé très fin
Et un peu de poivre en grains
Non
Ah ? Bon
\bis{\[Lam]  \[Lam]  \[Lam]  \[Mi]}{2}
\endverse

\beginverse
Émiettez votre arsenic
Dans un verre de narcotique
Deux cuillères de purgatif
Qu'on fait bouillir à feu vif
Ho ho, je vais en mettre trois
\endverse

\beginverse
Dans un petit plat à part
Tiédir du sang de lézard
La valeur d'un dé à coudre
Et un peu de sucre en poudre
Non
Ah ? Bon
\bis{\[Lam]  \[Lam]  \[Lam]  \[Mi]}{2}
\endverse

\beginverse
Vous versez la mort-aux-rats
Dans du venin de cobra
Pour adoucir le mélange
Pressez trois quartiers d'orange
\endverse

\beginverse
Ho ho, je vais en mettre un seul
\endverse

\beginverse
Décorez de fruits confits
Moisis dans du vert-de-gris
Tant que votre pâte est molle
Et un peu de vitriol
Non\dots Oui
Ah, je savais bien que ça serait bon
\endverse

\beginverse
Le pudding à l'arsenic
Nous permet ce pronostic
Demain sur les bords du Nil
Que mangeront les crocodiles ?
Des Gaulois
\endverse

\endsong
\beginsong{Les gens qu'on aime}[by={Patrick Fiori (2017)}]

\beginverse
Capo IV
\endverse

\beginchorus
\[Ré] J'aurais pu \[La]traîner le \[Sim]long de mes \[Sol]rêves
\[Ré] J'aurais pu, \[La]l'air de \[Sim]rien \[Ré]
\[Sol]  Attendre \[Ré]ici que la \[La]journée s'a\[Ré]chève
\[Sol] Sortir le \[Ré]chien, \[La]si j'en avais un
J'aurais pu m'inventer des inventaires
Refaire et faire le point
Mais ce matin j'ai bien plus cher à faire
\[Fa]  Bam\[Ré]dabada\[La]bam
\endchorus

\beginchorus
Ce matin-\[Sim]ci, j'irai \[Sol]dire aux \[La]gens que \[Ré]j'aime
Ou juste \[Sim]merci d'ê\[Sol]tre ce qu'ils \[La]sont
Qu'ils changent \[Ré]mes heures am\[Fa#]ères en po\MultiwordChords\[Sim]èmes
Et tous ces \[Sol]mots que \[Ré]nous tai\[La]sons
Ce matin-\[Sim]ci, j'irai \[Sol]dire aux \[La]gens que \[Ré]j'aime
Oh! Comme \[Sim]ils comptent pour \[Sol]moi chaque ins\[La]tant
Des mots doux \[Ré]c'est mieux qu\[Fa#]'un beau requi\[Sim]em
Et les di\[Ré]re c'est \[La]impor\[Sim]tant
\[La]  Et dire a\[Ré]vant tant \[La]qu'il est \[Do]temps
\endchorus

\beginverse
On veut toujours attendre la prochaine
Remettre au lendemain
C'est bien plus simple d'émettre des haines
Bien anonymes tapis dans son coin
Et coulent nos vies et l'eau des fontaines
La vie du quotidien
Et passent les jours et puis les semaines
Bamdabadabam
\endverse

\beginchorus
Refrain
\endchorus

\beginverse
J'aurais pu traîner le long de mes rêves
J'aurais pu l'air de rien
Attendre ici que la journée s'achève
Bamdabadabam
\endverse

\beginverse
On devrait dire aux gens quand on les aime
Trouver les phrases, trouver le temps
Qu'ils changent nos heures amères en poèmes
On devrait tout se dire avant
Il faut le dire aux gens quand on les aime
Comme ils comptent pour nous chaque instant
Les mots doux c'est mieux qu'un beau requiem
Et tant qu'on est là bien vivant
Tout se dire tant qu'il est temps
\endverse

\endsong
\beginsong{Golden Baby}[by={Cœur de pirate (2011)},sr={Capo V}]

\beginverse
Je \[Do]t'ai vu d'un œil \[Do7]solitaire
Le \[Do7]pied dans l'arène \[Fa]pour te plaire
Et \[Fam]briller aux re\[Do]gards que j'igno\[Sol7]rais
\endverse

\beginverse
Le \[Do]tien comptait plus \[Do7]que les autres
Même \[Do7]si tu ne t'en re\[Fa]ndais pas compte
Et \[Fam]j'aurais tout fait \[Do]pour connaître tes \[Sol]fins
\endverse

\beginchorus
Refrain
\endchorus

\beginverse
\[Fa]Golden Baby, c'en \[Do]est assez
\[Mi7]De courir te faire \[Lam]désirer
Dans \[Fa]ces lumières qui \[Do]donnent vie à nos \[Mi]nuits
\[Fa]Golden Baby, sans \[Do]tout pour plaire
Dans \[Mi7]ton silence, tu \[Lam]restes fier
De \[Fa]croire en ce qui \[Do]n'existerait \[Sol7]pas
Et si tu veux de \[Do]moi
\endverse

\beginverse
On s'est finalement embrassés
Des mois sans silence, sans parler
Dans l'attente qui, de loin, m'a déchirée
\endverse

\beginverse
Et j'aurais aimé être ces filles
Qui, dans tes chansons, reprennent vie
Même si, de loin, je sais qu'on s'est menti
\endverse

\beginchorus
Refrain
\endchorus

\beginverse
J'ai voulu tout laisser tomber
Pour ne pas être ombre du passé
Et retrouver tes rires et tes secrets
\endverse

\beginverse
Mais quand je l'ai vue près de toi
Celle qui en chanson reprend vie
Je sais maintenant que tu m'avais menti
\endverse

\beginchorus
Refrain
\endchorus

\endsong
\beginsong{Le gorille*}[by={Georges Brassens (1952)}]

\beginverse
C'est à \[Ré]travers de larges grilles
Que les femelles du Can\[La7]ton
Contemplaient un puissant gorille
Sans souci du qu'en-dira-t-\[Ré] on
Avec impudeur, ces commères
Lorgnaient même un endroit pré\[La7]cis
Que, rigoureusement, ma mère
M'a défendu d'nommer i\[Ré]ci
Gare au gor\[Ré]iii-\[La7]iii-\[Ré]iii-\[La7]iii-lle
\endverse

\beginverse
Tout à coup la prison bien close
Où vivait le bel animal
S'ouvre, on n'sait pourquoi, je suppose
Qu'on avait dû la fermer mal
Le singe, en sortant de sa cage
Dit : C'est aujourd'hui que j'le perds
Il parlait de son pucelage
Vous aviez deviné, j'espère
Gare au goriii-iii-iii-iii-lle
\endverse

\beginverse
L'patron de la ménagerie
Criait, éperdu : Nom de nom
C'est assommant, car le gorille
N'a jamais connu de guenon
Dès que la féminine engeance
Sut que le singe était puceau
Au lieu de profiter de la chance
Elle fit feu des deux fuseaux
Gare au goriii-iii-iii-iii-lle
\endverse

\beginverse
Celles là même qui, naguère
Le couvaient d'un œil décidé
Fuirent, prouvant qu'elles n'avaient guère
De la suite dans les idées
D'autant plus vaine était leur crainte
Que le gorille est un luron
Supérieur à l'homme dans l'étreinte
Bien des femmes vous le diront
Gare au goriii-iii-iii-iii-lle
\endverse

\beginverse
Tout le monde se précipite
Hors d'atteinte du singe en rut
Sauf une vieille décrépite
Et un jeune juge en bois brut
Voyant que toutes se dérobent
Le quadrumane accéléra
Son dandinement vers les robes
De la vieille et du magistrat
Gare au goriii-iii-iii-iii-lle
\endverse

\beginverse
Bah ! soupirait la centenaire
Qu'on put encor me désirer
Ce serait extraordinaire
Et, pour tout dire, inespéré
Le juge pensait, impassible
Qu'on me prenne pour une guenon
C'est complètement impossible
La suite lui prouva que non
Gare au goriii-iii-iii-iii-lle
\endverse

\beginverse
Supposez que l'un de vous puisse être
Comme le singe, obligé de
Violer un juge ou une ancêtre
Lequel choisirait-il des deux ?
Qu'une alternative pareille
Un de ces quatre jours, m'échoie
C'est, j'en suis convaincu, la vieille
Qui sera l'objet de mon choix
Gare au goriii-iii-iii-iii-lle
\endverse

\beginverse
Mais, par malheur, si le gorille
Aux jeux de l'amour vaut son prix
On sait qu'en revanche il ne brille
Ni par le goût ni par l'esprit
Lors, au lieu d'opter pour la vieille
Comme l'aurait fait n'importe qui
Il saisit le juge à l'oreille
Et l'entraîna dans un maquis
Gare au goriii-iii-iii-iii-lle
\endverse

\beginverse
La suite serait délectable
Malheureusement, je ne peux
Pas la dire, et c'est regrettable
Ça nous aurait fait rire un peu
Car le juge, au moment suprême
Criait : Maman !, pleurait beaucoup
Comme l'homme auquel, le jour même
Il avait fait trancher le cou
Gare au goriii-iii-iii-iii-lle
\endverse

\endsong
\beginsong{Guantanamera}[by={Joe Dassin (1966)}]


\beginchorus
\[Sol]Guantanamer\[La]a
Ma ville \[Ré]Guantanamer\[La]a
\[Ré]Guantan\[Sol]amer\[La]a
Ma ville \[Ré]Guantana\[Sol]me\[La]ra
\endchorus

\beginverse
C'était un \[Ré]homme en dé\[La]route
C'était un \[Ré]frère sans d\[La]oute
Il n'avait \[Ré]ni lieu ni \[La]place
Et sur les \[Ré]routes de l'exil
Sur les \[Ré]sentiers, sur les \[Sol]plac\[La]es
Il me pa\[Ré]rlait de sa \[Sol]ville\[La]
\endverse

\beginchorus
Refrain
\endchorus

\beginverse
Là-bas sa maison de misère
Etait plus blanche que le coton
Les rues de sable et de terre
Sentaient le rhum et le melon
Sous leurs jupons de dentelle
Dieu que les femmes étaient belles
\endverse

\beginchorus
Refrain
\endchorus

\beginverse
Il me reste toute la Terre
Mais je n'en demandais pas autant
Quand j'ai passé la frontière
Il n'y avait plus rien devant
J'allais d'escale en escale
Loin de ma terre natale
\endverse

\endsong
\beginsong{Hakuna Matata}[by={Elton John, Tim Rice \- Le Roi Lion (1994)}]


\beginchorus
\[Sol]Hakuna Matata
Mais quelle phrase magni\[Ré]fique
Hakuna Ma\[Sol]tata
\[Mi]Quel chant fanta\[La7]stique
\endchorus

\beginverse
Ces mots signi\[Sim]fient
Que tu vivras ta \[Mi]vie
Sans au\[Ré]cun souci
Phi\[La]losophie
Hakuna Ma\[La]tata
\endverse

\beginverse
Ce \[Do]très jeune \[Sol]phaco\[Ré]chère
J'é\[Do]tais jeune et \[Sol]phaco\[Ré]chère
Bel organe
Merci
\endverse

\beginverse
Un \[Fa]jour, quelle horreur
Il comprit \[Sol]que son odeur
Au lieu \[Fa]de sentir la fleur
Soule\[Sol]vait les cœurs
\endverse

\beginverse
Mais y'a \[Fa]dans tout cochon
Un \[Sol]poète qui som\[Ré]meille
Quel \[Fa]martyr
Quand personne
Peut plus vous sen\[La]tir
\endverse

\beginverse
Disgrâce in\[Ré]fâme
Inonde mon \[La]âme
Je déclenche une tem\[Do]pête
Chaque fois que je\dots
Non Pumbaa, pas devant les enfants
Oh! Pardon
\endverse


\beginchorus
Hakuna Matata
\endchorus

\beginverse
\bis{Hakuna Matata}{3}
\endverse

\beginchorus
Refrain
\endchorus

\beginverse
\bis{Hakuna Matata}{3}
\endverse

\endsong
\beginsong{Hallelujah}[by={Leonard Cohen (1984), M Pokora (2012)}]

\beginverse
Now, I've \[Sol]heard there was a \[Mim]secret chord
\endverse

\beginverse
That \[Sol]David played and it p\[Mim]leased the Lord
\endverse

\beginverse
But \[Do]you don't really \[Ré]care for music, \[Sol]do you ?\[Ré]
\endverse

\beginverse
Well, it \[Sol]goes like this, the \[Do]fourth, the \[Ré]fifth
\endverse

\beginverse
The \[Mim]minor fall and the \[Do]major lift
\endverse

\beginverse
The \[Ré]baffled king com\[Si7]posing Hallelu\[Mim]jah
\endverse

\beginchorus
Refrain
\endchorus

\beginverse
Halle\[Do]lujah, Halle\[Mim]lujah, Halle\[Do]lujah, Halle\[Sol]lu-uhu-\[Ré]u-u\[Do]jah
\endverse

\beginverse
Your faith was strong but you needed proof
\endverse

\beginverse
You saw her bathing on the roof
\endverse

\beginverse
Her beauty and the moonlight overthrew ya
\endverse

\beginverse
She tied you to a kitchen chair
\endverse

\beginverse
She broke your throne and she cut your hair
\endverse

\beginverse
And from your lips she drew the Hallelujah
\endverse

\beginchorus
Refrain
\endchorus

\beginverse
You say I took the name in vain
\endverse

\beginverse
I don't even know the name
\endverse

\beginverse
But if I did, well really, what's it to you ?
\endverse

\beginverse
There's a blaze of light in every word
\endverse

\beginverse
It doesn't matter which you heard
\endverse

\beginverse
The holy or the broken Hallelujah
\endverse

\beginchorus
Refrain
\endchorus

\beginverse
I did my best, it wasn't much
\endverse

\beginverse
I couldn't feel, so I tried to touch
\endverse

\beginverse
I've told the truth, I didn't come to fool you
\endverse

\beginverse
And even though it all went wrong
\endverse

\beginverse
I'll stand before the Lord of song
\endverse

\beginverse
With nothing on my tongue but Hallelujah
\endverse

\beginchorus
Refrain
\endchorus

\beginverse
Maybe there's a God above
\endverse

\beginverse
But all I've ever learned from love
\endverse

\beginverse
Was how to shoot somebody who outdrew ya
\endverse

\beginverse
It's not a cry that you hear at night
\endverse

\beginverse
It's not somebody who's seen the light
\endverse

\beginverse
It's a cold and it's a broken hallelujah
\endverse

\beginchorus
Refrain
\endchorus

\endsong
\beginsong{Here's to You}[by={Joan Baez (1971)}]

\beginverse
\bis{
    \MultiwordChords\[Do]Here's to \MultiwordChords\[Sol]you, Nico\MultiwordChords\[Lam]la and B\MultiwordChords\[Mim]art
    \MultiwordChords\[Do]Rest forev\MultiwordChords\[Sol]er here \MultiwordChords\[Lam]in our he\MultiwordChords\[Sol]arts
    \MultiwordChords\[Mim]The last and f\MultiwordChords\[Rém]inal \MultiwordChords\[Sol]moment is y\MultiwordChords\[Do]ours
    \MultiwordChords\[Do]That ag\MultiwordChords\[Sol]ony is y\MultiwordChords\[Lam]our tr\MultiwordChords\[Mim]iu\MultiwordChords\[Lam]mph
}{8}
\endverse

\endsong
\beginsong{Hexagone*}[by={Renaud (1975)}]

\beginverse
\[Mim]Ils s'embrassent au mois de janvier
Car une nouvelle année commence
Cais depuis des éternités
L'a pas tell'ment changé la \[Ré]France
Passent les jours et les semaines
Y'a qu'le décor qui évolue
La mentalité est la même
Tous des tocards, tous des faux-\[Mim]culs
\endverse

\beginverse
Ils sont pas lourds en février
À se souvenir de Charonne
Des matraqueurs assermentés
Qui fignolèrent leur besogne
La France est un pays de flics
À tous les coins d'rue y'en a cent
Pour faire régner l'ordre public
Ils assassinent impunément
\endverse

\beginverse
Quand on exécute au mois d'mars
De l'autr'côté des Pyrénées
Un anarchiste du Pays Basque
Pour lui apprendre à s'révolter
Ils crient, ils pleurent et ils s'indignent
De cette immonde mise à mort
Mais ils oublient qu'la guillotine
Chez nous aussi fonctionne encore
\endverse

\beginverse
Être né sous l'signe de l'Hexa\[Ré]gone
C'est pas c'qu'on fait de mieux en c'mo\[Mim]ment
Et le roi des cons, sur son t\[Ré]rône
J'parierais pas qu'il est Alle\[Mim]mand
\endverse

\beginverse
On leur a dit, au mois d'avril
À la télé, dans les journaux
De pas se découvrir d'un fil
Que l'printemps c'était pour bientôt
Les vieux principes du seizième siècle
Et les vieilles traditions débiles
Ils les appliquent tous à la lettre
Y m'font pitié ces imbéciles
\endverse

\beginverse
Ils se souviennent, au mois de mai
D'un sang qui coula rouge et noir
D'une révolution manquée
Qui faillit renverser l'histoire
J'me souviens surtout d'ces moutons
Effrayés par la liberté, s'en allant voter par millions
Pour l'ordre et la sécurité
\endverse

\beginverse
Ils commémorent au mois de juin
Un débarquement d'Normandie
Ils pensent au brave soldat ricain
Qu'est v'nu se faire tuer loin d'chez lui
Ils oublient qu'à l'abri des bombes
Les Français craient : vive Pétain
Qu'ils étaient bien planqués à Londres
Qu'y'avait pas beaucoup d'Jean Moulin
\endverse

\beginverse
Être né sous l'signe de l'Hexagone
C'est pas la gloire en vérité
Et le roi des cons, sur son trône
Me dites pas qu'il est Portugais
\endverse

\beginverse
Ils font la fête au mois d'juillet
En souv'nir d'une révolution
Qui n'a jamais éliminé
La misère et l'exploitation
Ils s'abreuvent de bals populaires
D'feux d'artifice et de flonflons
Ils pensent oublier dans la bière
Qu'ils sont gouvernés comme des pions
\endverse

\beginverse
Au mois d'août c'est la liberté
Après une longue année d'usine
Ils crient : vive les congés payés
Ils oublient un peu la machine
En Espagne, en Grèce ou en France
Ils vont polluer toutes les plages
Et, par leur unique présence
Abîmer tous les paysages
\endverse

\beginverse
Lorsqu'en septembre on assassine
Un peuple et une liberté
Au coeur de l'Amérique latine
Ils sont pas nombreux à gueuler
Un ambassadeur se ramène
Bras ouverts il est accueilli
Le fascisme c'est la gangrène
À Santiago comme à Paris
\endverse

\beginverse
Être né sous l'signe de l'Hexagone
C'est vraiment pas une sinécure
Et le roi des cons, sur son trône
Il est Français, ça j'en suis sûr
\endverse

\beginverse
Finies les vendanges en octobre
Le raisin fermente en tonneaux
Ils sont très fiers de leurs vignobles
Leurs Côtes-du-Rhône et leurs Bordeaux
Ils exportent le sang de la terre
Un peu partout à l'étranger
Leur pinard et leur camembert
C'est leur seule gloire, à ces tarés
\endverse

\beginverse
En novembre, au Salon d'l'auto
Ils vont admirer par milliers
L'dernier modèle de chez Peugeot
Qu'il pourront jamais se payer
La bagnole, la télé, l'tiercé
C'est l'opium du peuple de France
Lui supprimer c'est le tuer
C'est une drogue à accoutumance
\endverse

\beginverse
En décembre, c'est l'apothéose
La grande bouffe et les les p'tits cadeaux
Ils sont toujours aussi moroses
Mais y'a d'la joie dans les ghettos
La Terre peut s'arrêter d'tourner
Ils rat'ront pas leur réveillon
Moi j'voudrais tous les voir crever
Étouffés de dinde aux marrons
\endverse

\beginverse
Etre né sous l'signe de l'Hexagone
On peut pas dire qu'ça soit bandant
Si l'roi des cons perdait son trône
Y'aurait cinquante millions de prétendants
\endverse

\endsong
\beginsong{Hey Jude}[by={The Beatles (1968)},sr={Capo III}]

\beginverse
Hey \[Do]Jude, don't make it b\[Sol]ad
Take a \MultiwordChords\[Sol7]sad song and make it be\[Do]tter
Reme\[Fa]mber to let it into your h\[Do]eart
Then you can st\[Sol7]art to make it be\[Do]tter
\endverse

\beginverse
Hey Jude, don't be afraid
You were made to go out and get her
The minute you let under your skin
Then you begin to make it better
\endverse

\beginverse
\[Do7]And anytime you feel the \[Fa]pain
Hey Jude, ref\[Rém]rain
Don't carry the \[Sol7]world upon your \[Do]shoulder
\[Do7]For well you know that it's a \[Fa]fool who \[Do]plays
It \[Rém]cool by making his w\[Sol7]orld a little co\[Do]lder
Dadadada \[Do7]dadada \[Sol7]dadada \[Do]da
\endverse

\beginverse
Hey Jude, don't let me down
You have found her, now go and get her
Remember to let her into your heart
Then you can start to make it better
\endverse

\beginverse
So let it out and let it in
Hey Jude, begin
you're waiting for someone to perform with
And don't you know that it's just you
Hey Jude, you'll do
The movement you need is on your shoulder
\endverse

\beginverse
Hey Jude, don't make it bad
Take a sad song and make it better
Remember to her it into your skin
Then you'll begin to make it better, better
Better, better, better, better oh
Dadada dadada dadada da
\endverse

\endsong
\beginsong{L'homme de Cromagnon}[by={Les 4 Barbus (1997)}]

\beginverse
C'était au \[Do]temps d'la préhistoire
Voici deux ou trois cent mille \[Sol7]ans
Vint au monde un être bizarre
Proche parent d'l'orang-out\[Do]ang
Perché sur ses pattes de derrière
Vêtu d'un slip en peau d'bis\[Sol7]on
Il partait conquérir la t\[Do]erre
C'était l'\[Sol7]Homme de Croma\[Do]gnon
\endverse


\beginchorus
L'Homme de \[Do]Cro, l'Homme de Ma, I'Homme de Gnon
\[Sol7]L'Homme de Croma\[Do]gnon
\bis{\[Fa]L'Homme de Cro de Mag\[Do]non
    Ce n'est pas du b\[Sol]idon
    L'Homme de Croma\[Do]gnon on on}{3}
\endchorus

\beginverse
Armé de sa hache de pierre
De son couteau de pierre itou
Il chassait l'ours et la panthère
En serrant les fesses malgré tout
Devant I'diplodocus en rage
Il était tout d'même un peu p'tit
En se disant dans son langage
Vivement qu'on invente le fusil
\endverse

\beginchorus
Refrain
\endchorus

\beginverse
Il était poète à ses heures
Disant à sa femme en émoi
Tu es belle comme un dinosaure
Tu r'ssemble à Lolobrigida
Si tu veux voir des cartes postales
Viens dans ma caverne tout là-haut
J'te ferai voir des peintures murales
On dirait du vrai Picasso
\endverse

\beginchorus
Refrain
\endchorus

\beginverse
Trois cent mille ans après sur Terre
Comme nos ancêtres nous admirons
Les monts, les bois et les rivières
Mais s'ils rev'naient, quelle déception
Nous voyant suer 6 jours sur 7
Ils diraient sans faire de détail
Vraiment qu'nos héritiers sont bêtes
D'avoir inventé le travail
\endverse

\endsong
\beginsong{Un homme debout}[by={Claudio Capéo (2016)}]

\beginchorus
Refrain 1
\[Rém]Si je m'endors, me ré\[Lam]veillerez-vous ?
Il fait \[Do]si froid dehors, le re\[Sol]ssentez-vous ?
\[Rém]Il fut un temps où j'é\[Lam]tais comme vous
Malgré \[Do]toutes mes galères, je res\[Sol]te un homme debout
\endchorus

\beginchorus
Refrain 2
Priez pour que je m'en sorte
Priez pour que mieux je me porte
Ne me jetez pas la faute
Ne me fermez pas la porte
\endchorus

\beginverse
Oui je vis, de jour en jour
De squat en squat, un troubadour
Si je chante, c'est pour qu'on m'regarde
Ne serait-ce qu'un p'tit bonjour
J'vous vois passer, quand j'suis assis
Vous êtes debout, pressés, j'apprécie
Un p'tit regard, un p'tit sourire
Peu prennent le temps, ils ne font que courir
\endverse

\beginchorus
Refrain 1
\endchorus

\beginverse
La la la la la la la
La la la la la la la la
\endverse

\beginverse
Merci bien pour la pièce
En c'moment c'est dur, je confesse
'Fin j'vais m'en sortir je l'atteste
Un jour avoir un toit, une adresse
Même si de toi à moi c'est dur, je stresse
\endverse

\beginverse
Le moral n'est pas toujours bon, le temps presse
Mais bon comment ? À part l'ivresse comme futur
Et des promesses, en veux-tu ?
\endverse

\beginverse
Voilà ma vie, j'me suis pris des coups dans la tronche
Sois sûr que si j'tombe par terre tout l'monde passe mais personne ne bronche
Franchement à part les gosses qui m'regardent étrangement
Tout l'monde trouve ça normal que j'fasse la manche
M'en veuillez pas, mais parfois, j'ai qu'une envie : abandonner
\endverse

\beginchorus
Refrain 1
\endchorus

\beginchorus
Refrain 2
\endchorus

\beginchorus
\bis{Refrain 1}{2}
\endchorus

\endsong
\beginsong{Les hommes que j'aime}[by={La Rue Kétanou (2002)}]

\beginchorus
Refrain
\endchorus

\beginverse
Je vou\[Lam]drais vous par\[Sol]ler \[Do]des hom\[Fa]mes que \[Lam]j'aime
Ceux qui m'ont embra\[Sol]ssé, au \[Do]bord \[Fa]de la \[Lam]Seine
Ou j'allais me je\[Sol]ter, je\[Do]té \[Fa]par une \[Lam]reine
Que j'avais aim\[Sol]ée, plus \[Do]que les h\[Fa]ommes que \[Lam]j'aime
\endverse

\beginverse
Ils ont des gueules cassées, il faut les voir au petit jour
Se coucher tout étonnés, du monde qui les entoure
Ils volent, ils viennent, ils traînent, ils parlent fort, ne parlent pas
Ils entendent des Carmen, qui leur disaient : viens par là
Et chaque fois ils y vont, et chaque fois ils en reviennent
Entre un ange et un démon, ainsi j'aime les hommes que j'aime
\endverse

\beginchorus
Refrain
\endchorus

\beginverse
Ce sont des géants, qui savent le chagrin d'amour
Des amitiés de survivants, qui fêtent votre retour
Et quand passe un drame et que l'un de nous il r'touche
On se donne des prénoms de femmes, et on s'embrasse sur la bouche
Mon dieu c'est mon tour, j'vais au bord de la Seine
Je crie au secours, ainsi m'aiment les hommes que j'aime
\endverse

\beginchorus
Refrain
\endchorus

\beginverse
Et je lève mon cœur, à la tendresse de ces voyous
Qu'elle me porte bonheur, ce soir j'ai rendez-vous
Et j'irai comme je suis, non je ne changerai rien
Acte mes folies, à mon feu dans mes mains
A mon amour sans pudeur, à mon amour qui se déchaîne
Et même si ça fait peur, ainsi aiment les hommes que j'aime
\endverse

\beginchorus
\bis{Refrain}{2}
\endchorus

\endsong
\beginsong{L'horloge tourne*}[by={Mickael Miro (2010)}]

\beginverse
Un \[Lam]SMS vient \[Fa]d'arriver, \[Sol]j'ai 18 \[Lam]ans
En\[Lam]volée ma vir\[Fa]ginité, j'\[Sol]suis plus un en\[Lam]fant
\[Lam]L'horloge tourn\[Fa]e, les \[Sol]minutes sont to\[Lam]rrides
Et \[Lam]moi je \[Fa]rêve d'ac\[Sol]célérer le \[Lam]temps
\endverse

\beginverse
\[Lam]Dam dam \[Fa]déo \[Sol]oh oh \[Lam]oh, \[Lam]dam dam \[Fa]déo \[Sol]oh oh oh \[Lam]oh
\endverse

\beginverse
Un SMS vient d'arriver, j'ai 20 ans
On l'a fait sans se protéger mais j'veux pas d'un enfant
L'horloge tourne, les minutes infanticides
Et moi je rêve de remonter le temps
\endverse

\beginverse
Dam dam déo oh oh oh, dam dam déo oh oh oh oh
\endverse

\beginverse
Un SMS vient d'arriver, j'ai 21 ans
9 mois se sont écoulés et toujours pas d'enfants et
L'horloge tourne, les minutes se dérident
Et moi je rêve, tranquille je prends mon temps
\endverse

\beginverse
Dam dam déo oh oh oh, dam dam déo oh oh oh oh
\endverse

\beginverse
Un SMS vient d'arriver, j'ai 25 ans
Un tsunami a tout emporté, même les jeux d'enfants
L'horloge tourne, les minutes sont acides
Et moi je rêve que passe le mauvais temps
\endverse

\beginverse
Dam dam déo oh oh oh, dam dam déo oh oh oh oh
\endverse

\beginverse
Un SMS vient d'arriver, j'ai 28 ans
Mamie est bien fatiguée mais j'suis plus un enfant
L'horloge tourne mais son coeur se suicide
Et moi je rêve, je rêve du bon vieux temps
\endverse

\beginverse
\bis{Dam dam déo oh oh oh, dam dam déo oh oh oh oh}{2}
\endverse

\beginverse
Un SMS va arriver, j'aurai 30 ans
30 ans de liberté et soudain le bilan
L'horloge tourne, les minutes sont des rides
Et moi je rêve, je rêve d'arrêter le temps
\endverse

\beginverse
\bis{Dam dam déo oh oh oh, dam dam déo oh oh oh oh}{3}
\endverse

\endsong
\beginsong{L'hymne de nos campagnes*}[by={Tryo (1988)},sr={Capo V}]

\beginverse
Si \MultiwordChords\[Lam]tu es né dans une \[Fa]cité HL\[Mi]M
Je \MultiwordChords\[Lam]te dédicace \[Fa]ce poèm\[Mi]e
En esp\[Lam]érant qu'au fond \[Fa]de tes yeux \[Mi]ternes
\[Lam]Tu puisses y voir un \[Fa]petit brin d'\[Mi]herbe
\endverse

\beginverse
\MultiwordChords\[Lam]Eh les mans faut faire \MultiwordChords\[Fa]la part des \[Mi]choses
\[Lam]Il est grand temps de \[Fa]faire une \[Mi]pause
De tr\[Lam]oquer cette \[Fa]vie mo\[Mi]rose
\[Lam]Contre le \[Fa]parfum d'une r\[Mi]ose
\endverse


\beginchorus
C'est l'hymne de nos campagnes
De nos rivières, de nos montagnes
De la vie man, du monde animal
Crie-le bien fort, use tes cordes vocales
\endchorus

\beginverse
Pas de boulot, pas de diplômes
Partout la même odeur de zone
Plus rien n'agite tes neurones
Pas même le shit que tu mets dans tes cônes
Va voir ailleurs, rien ne te retient
Va vite faire quelque chose de tes mains
Ne te retourne pas ici tu n'as rien
Et sois le premier à chanter ce refrain
\endverse

\beginchorus
Refrain
\endchorus

\beginverse
Assieds-toi près d'une rivière
Écoute le coulis de l'eau sur la terre
Dis-toi qu'au bout, hé ! Il y a la mer
Et que ça, ça n'a rien d'éphémère
Tu comprendras alors que tu n'es rien
Comme celui avant toi, comme celui qui vient
Que le liquide qui coule dans tes mains
Te servira à vivre jusqu'à demain matin
\endverse

\beginverse
Assieds-toi près d'un vieux chêne
Et compare-le à la race humaine
L'oxygène et l'ombre qu'il t'amène
Mérite-t-il les coups de hache qui le saignent ?
Lève la tête, regarde ces feuilles
Tu verras peut-être un écureuil
Qui te regarde de tout son orgueil
Sa maison est là, tu es sur le seuil
\endverse

\beginchorus
Refrain
\endchorus

\beginverse
Peut-être que je parle pour ne rien dire
Que quand tu m'écoutes tu as envie de rire
Mais si le béton est ton avenir
Dis-toi que c'est la forêt qui fait que tu respires
J'aimerais pour tous les animaux
Que tu captes le message de mes mots
Car un lopin de terre, une tige de roseau
Servira à la croissance de tes marmots
\endverse

\endsong
\beginsong{Il changeait la vie}[by={Jean-Jacques Goldmann (1987)}]

\beginverse
C'était un cordonnier, sans \[La]rien d'particulier
Dans \[Sol]un village dont le \[La]nom m'a échappé
Qui \[Ré]faisait des souliers si \[La]jolis, si légers
Que \[Sol]nos vies semblaient un peu moins \[La]lourdes à porter
Il \[Sol]y mettait du temps, du ta\[La]lent et du cœur
Ain\[Ré]si passait sa vie au mi\[Sol]lieu de nos heur\[Ré]es
Et \[Mim]loin des beaux discours, des \[La]grandes théories
À sa \[Sib]tâche chaque jour, on pou\[Do]vait dire de lui
Il changeait la vie
\[Rém] \[Sib]\[Solm] \[Rém]\[La]  2x
\endverse

\beginverse
C'était un professeur, un simple professeur
Qui pensait que savoir était un grand trésor
Que tous les moins que rien n'avaient pour s'en sortir
Que l'école et le droit qu'a chacun de s'instruire
Il y mettait du temps, du talent et du cœur
Ainsi passait sa vie au milieu de nos heures
Et loin des beaux discours, des grandes théories
À sa tâche chaque jour, on pouvait dire de lui
Il changeait la vie
\endverse

\beginverse
C'était un p'tit bonhomme, rien qu'un tout p'tit bonhomme
Malhabile et rêveur, un peu loupé en somme
Se croyait inutile, banni des autres hommes
Il pleurait sur son saxophone
Il y mit tant de temps, de larmes et de douleur
Les rêves de sa vie, les prisons de son cœur
Et loin des beaux discours, des grandes théories
Inspiré jour après jour de son souffle et de ses cris
Il changeait la vie
\endverse

\endsong
\beginsong{Il en faut peu pour être heureux}[by={Jean Scout, Pascal Bressy \- Le Livre de la jungle (2002)}]

\beginverse
Il en faut \[Do]peu pour être heureux
Vrai\[Fa]ment très peu pour être heureux
Il \[Do]faut se satis\[La7]faire du néces\[Ré7]saire \[Sol7]
Un peu d'eau \[Do]fraîche et de verdure
Que \[Fa]nous prodigue la nature
Quelq\[Do]ues ra\[La7]yons de m\[Ré7]iel et \[Sol7]de so\[Do]leil
\endverse

\beginverse
Je dors d'ordi\[Sol7]naire sous les frondai\[Do]sons
Et toute la \[Sol7]jungle est ma mai\[Do]son
Toutes les a\[Fa]beilles de la forêt
Butinent pour \[Do]moi dans les bosq\[Ré7]uets
Et quand je retourne un gros caillou
Je \[Sol7]sais trouver des fourmis dessous
Es\[Do]saie c'est bon, c'est \[La7]doux, oh
\endverse

\beginverse
Il en faut vrai\[Ré7]ment peu
Très \[Sol7]peu pour être heu\[Do]reux
Mais oui
Pour être heureux
\endverse

\beginverse
Il en faut peu pour être heureux
Vraiment très peu pour être heureux
Chassez de votre esprit tous vos soucis
Prenez la vie du bon côté
Riez, sautez, dansez, chantez
Et vous serez un ours très bien léché
\endverse

\beginverse
Cueillir une ba\[Sol7]nane, oui
Ça se fait sans as\[Do]tuce
\endverse

\beginverse
Aïe
Mais c'est tout un \[Sol7]drame
Si c'est un cac\[Do]tus
Si vous chi\[Fa]pez des fruits sans épines
Ce n'est pas la \[Do]peine de faire atten\[Ré7]tion
Mais \[Ré7]si le fruit de vos rapines
Est tout plein d'épines
C'est \[Sol7]beaucoup moins bon
\[Do]Alors petit, as-tu comp\[La7]ris ?
Il en faut vrai\[Ré7]ment peu
Très \[Sol7]peu pour être heu\[Do]reux
Pour être heureux ?
Pour être heureux
\endverse

\beginverse
\[La7]Et tu verras qu'tout est résolu
Lorsque l'\[Ré7] on se passe
Des choses super\[Sol7]flues
\[Do]Alors tu ne t'en \[La7]fais plus
Il en faut \[Ré7]vraiment peu
Très \[Sol7]peu, pour être heu\[Do]reux
\endverse

\beginverse
Il en faut peu pour être heureux
Vraiment très peu pour être heureux
Chassez de votre esprit
Tous vos soucis ! Youpi
Prenez la vie du bon côté
Riez, sautez, dansez, chantez
Et vous serez un ours très bien léché
Waouh
Et vous serez un ours très bien léché
Youpi
\endverse

\endsong
\beginsong{Il est libre Max}[by={Hervé Christiani (1981)}]


\beginchorus
Il est \[Mim]libre Max, il est \[Do]libre Max
Y'en \[Ré]a même qui disent qu'ils l'ont \[Mim]vu voler
\endchorus

\beginverse
Il \[Mim]met de la magie mine de rien d\[Do]ans tout c'qu'il fait
Il \[Ré]a l'sourire facile même pour les i\[Mim]mbéciles
Il s'amuse bien, il tombe ja\[Do]mais dans les pièges
Il \[Ré]s'laisse pas étourdir par les n\[Mim]éons des manèges
Il vit sa vie sans s'occ\[Do]uper des grimaces
Que \[Ré]font autour de lui les poiss\[Mim]ons dans la nasse
\endverse

\beginchorus
Refrain
\endchorus

\beginverse
Il travaille un p'tit peu quand son corps est d'accord
Pour lui faut pas s'en faire, il sait doser son effort
Dans l'panier d'crabes il joue pas les homards
Il cherche pas à tout prix à faire des bulles dans la mare
\endverse

\beginchorus
Refrain
\endchorus

\beginverse
Il r'garde autour de lui avec les yeux de l'amour
Avant qu't'aies rien pu dire il t'aime déjà au départ
Il fait pas d'bruit il joue pas au tambour
Mais la statue de marbre lui sourit dans la cour
\endverse

\beginchorus
Refrain
\endchorus

\beginverse
Et bien sûr toutes les filles lui font leurs yeux de velours
Lui pour leur faire plaisir il leur raconte des histoires
Il les emmène par-delà le labour
Chevaucher les licornes à la tombée du soir
\endverse

\beginchorus
Refrain
\endchorus

\beginverse
Comme il n'a pas d'argent pour faire le grand voyageur
Il va parler souvent aux habitants de son cœur
Qu'est-ce qu'ils s'racontent ? C'est ça qu'il faudrait savoir
Pour avoir comme lui autant d'amour dans l'regard
\endverse

\endsong
\beginsong{Il faut que je m'en aille}[by={Graeme Allwright (1966)}]

\beginverse
Le temps est l\[Do]oin, de nos vingt ans
Des coups de p\[Fa]oings, des coups de \[Do]sang
Mais qu'à c'la ne \[Fa]tienne, c'est pas fi\[Do]ni
On peut chan\[Lam]ter quand le ve\[Fa]rre est b\[Sol7]ien rem\[Do]pli
\endverse


\beginchorus
Buvons e\[Fa]ncore une dernière \[Sol]fois
A l'am\[Fa]itié, l'\[Sol7]amour, la \[Sol]joie
On a f\[Fa]êté nos retrou\[Do]vailles
Ça m'fait d'la pe\[Lam]ine
Mais il \[Fa]faut que \[Sol7]je m'en \[Do]aille
\endchorus

\beginchorus
Refrain
\endchorus

\beginverse
Et souviens-toi de cet été
La première fois qu'on s'est saoulé
Tu m'as ram'né à la maison
En chantant, on marchait à reculons
\endverse

\beginchorus
Refrain
\endchorus

\beginverse
Je suis parti changer d'étoile
Sur un navire, l'ai mis la voile
Pour n'être plus qu'un étranger
Ne sachant plus très bien où il allait
\endverse

\beginchorus
Refrain
\endchorus

\beginverse
J't'ai raconté mon mariage
A la mairie d'un p'tit village
Je rigolais dans mon plastron
Quand le maire essayait d'prononcer mon nom
\endverse

\beginchorus
Refrain
\endchorus

\beginverse
J'n'ai pas écrit, toutes ces années
Et toi aussi, t'es marié
T'as trois enfants à faire manger
Moi j'en ai cinq, si ça peut te consoler
\endverse

\endsong
\beginsong{Il jouait du piano debout}[by={France Gall (1980)},sr={Capo III}]

\beginverse
\[Do]Ne dites pas que \[Sol]ce garçon ét\[Lam]ait fou
Il \[Do]ne vivait pas \[Sol]comme les autres, \[Lam]c'est tout
\[Fa]Et pour quelles raisons étranges
\[Sol]Les gens qui n'sont pas comme nous
\[Lam]Ça nous dérange \[Sol]
\endverse

\beginverse
Ne dites pas que ce garçon n'valait rien
Il avait choisi un autre chemin
Et pour quelles raisons étranges
Les gens qui pensent autrement
Ça nous dérange, ça nous dérange
\endverse


\beginchorus
Il jo\[Do]uait du piano debout
C'est peut-\[Mi]être un détail pour vous
Mais pour \[Lam]moi, ça veut dire beaucoup
\[Fa]Ça veut dire qu'il était libre
\[Sol]Heureux d'être là malgré tout
Il jouait du piano debout
Quand les trouillards sont à genoux
Et les soldats au garde à vous
Simplement sur ses deux pieds
Il voulait être lui, vous comprenez
\endchorus

\beginverse
Il n'y a que pour sa musique, qu'il était patriote
Il s'rait mort au champ d'honneur pour quelques notes
Et pour quelles raisons étranges
Les gens qui tiennent à leurs rêves
Ça nous dérange
\endverse

\beginverse
Lui et son piano, ils pleuraient quelques fois
Mais c'est quand les autres n'étaient pas là
Et pour quelles raisons bizarres
Son image a marqué ma mémoire
Ma mémoire
\endverse


\beginchorus
Il jouait du piano debout
Il chantait sur des rythmes fous
Et pour moi ça veut dire beaucoup
Ça veut dire essaie de vivre
Essaie d'être heureux
Ça vaut le coup
Il jouait du piano debout
Quand les trouillards sont à genoux
Et les soldats au garde à vous
Simplement sur ses deux pieds
Il voulait être lui, vous comprenez
\endchorus

\endsong
\beginsong{Imagine}[by={John Lennon (1988)}]

\beginverse
\[Do]Imagine there's \[Do7]no hea\[Fa]ven
\[Do]It's easy \[Do7]if you \[Fa]try
\[Do]No hell b\[Do7]elow \[Fa]us
\[Do]Above us o\[Do7]nly s\[Fa]ky
\[Do]Imagine \[Lam]all the \[Rém]peopl\[Do]e
\[Sol]Living for \[Do]toda\[Sol]y, h\[Sol7]a, ah, ahaha
\endverse

\beginverse
Imagine there's no countries
It isn't hard to do
Nothing to kill and die for
And no religion too
Imagine all the people
Living life in peace, youhou houhou
\endverse


\beginchorus
\[Fa]Y may \[Sol]say I'm a \[Do]dreame\[Mi7]r
\[FA]But I'm \[Sol]not the only \[Do]one u\[Mi7]s
\[Fa]I hope some\[Sol]day aou'll \[Do]join u\[Mi7]s
\[Fa]And the \[Sol]world will be as \[Do]one
\endchorus

\beginverse
Imagine no possessions
I wonder if you can
No need for greed or hunger
A brotherhood of man
Imagine all the people
Sharing all the world, youhou houhou
\endverse

\beginchorus
Refrain
\endchorus

\endsong
\beginsong{L'italiano}[by={Toto Cutugno (1983)}]

\beginverse
Lasciatemi cantare
Con la chitarra in mano
Lasciatemi cantare
Sono un italiano \[Lam]
\endverse

\beginverse
Buongiorno \[Lam]Italia, gli spaghetti al dente
E un par\[Lam]tigiano come presidente
Con l'au\[Lam]toradio sempre nella mano destra
Un canarino sopra \[Mi7]la finestra
\endverse

\beginverse
Buongiorno \[Mi7]Italia con i tuoi artisti
Con troppa \[Mi7]America sui manifesti
Con le \[Mi7]canzoni, con amore, con il cuore
Con più donne e sempre \[Lam]meno suore
\endverse

\beginverse
Buongiorno \[Do]Italia, buongiorno Maria
Con gli occhi \[Lam]pieni di malinconia
Buongiorno \[Mi7]Dio
Lo sai che ci sono anc\[Lam]h'io
\endverse


\beginchorus
Lasciatemi can\[Rém]tare
Con la chitarra in \[Lam]mano
Lasciatemi can\[Mi7]tare
Una canzone piano \[Lam]piano
Lasciatemi can\[Rém]tare
Perché ne sono \[Lam]fiero
Sono un ita\[Mi7]liano
Un italiano ve\[Lam]ro
\endchorus

\beginverse
Buongiorno Italia che non si spaventa
Con la crema da barba alla menta
Con un vestito gessato sul blu
E la moviola la domenica in TV
\endverse

\beginverse
Buongiorno Italia col caffè ristretto
Le calze nuove nel primo cassetto
Con la bandiera in tintoria
E una Seicento giù di carrozzeria
\endverse

\beginverse
Buongiorno Italia, buongiorno Maria
Con gli occhi pieni di malinconia
Buongiorno Dio
Lo sai che ci sono anch'io
\endverse

\beginchorus
\bis{Refrain}{2}
\endchorus

\endsong
\beginsong{J'ai demandé à la Lune}[by={Indochine (2002)}]

\beginverse
\MultiwordChords\[Do]J'ai de\MultiwordChords\[Sol]mandé à la \MultiwordChords\[Lam]Lune
\MultiwordChords\[Do]Et le So\MultiwordChords\[Sol]leil ne le sait \MultiwordChords\[Lam]pas
\MultiwordChords\[Rém]Je lui ai montré mes brû\MultiwordChords\[Lam]lures
\MultiwordChords\[Mim]Et la Lune s'est moquée de \MultiwordChords\[Sol]moi
\endverse

\beginverse
Et comme le ciel n'avait pas fière allure
Et que je ne guérissais pas
Je me suis dit quelle infortune
Et la Lune s'est moquée de moi
\endverse

\beginverse
J'ai demandé à la Lune
Si tu voulais encore de moi
Elle m'a dit j'ai pas l'habitude
De m'occuper des cas comme ça
\endverse

\beginverse
Et toi et moi
On était tellement sûr
Et on se disait quelquefois
Que c'était juste une aventure
Et que ça ne durerait pas
\endverse

\beginverse
\MultiwordChords\[Fam]Je n'ai pas grand chose à te \MultiwordChords\[Dom]dire
\MultiwordChords\[Fam]Et pas grand chose pour te faire \MultiwordChords\[Dom]rire
\MultiwordChords\[Mib]Car j'imagine toujours le \MultiwordChords\[Sib]pire
\MultiwordChords\[Mib]Et le meilleur me fait souf\MultiwordChords\[Sib]frir
\endverse

\beginverse
J'ai demandé à la Lune
Si tu voulais encore de moi
Elle m'a dit j'ai pas l'habitude
De m'occuper des cas comme ça
\endverse

\beginverse
Et toi et moi
On était tellement sûr
Et on se disait quelques fois
Que c'était juste une aventure
Et que ça ne durerait pas
\bis{\[Do] \[Sol]\[Lam] \[Do]\[Sol] \[Lam]}{2}
\endverse

\endsong
\beginsong{J'avais rendez-vous}[by={Carrousel (2012)},sr={Capo III}]

\beginverse
Il y a la \[Lam]foule partout au\[Mim]tour, elle se dé\[Fa]roule le souffle \[Sol]court
Presser le \[Lam]pas sur les pa\[Mim]vés, les pas pres\[Fa]sés sur le cô\[Sol]té
Il y a la \[Lam]ville, des verti\[Mim]cales qui s'em\[Fa]pilent en capi\[Sol]tales
Mais je me cons\[Lam]truis dans son é\[Mim]lan, même les ta\[Fa]xis n'ont plus le \[Sol]temps
\endverse


\beginchorus
Mais j'avais rendez-\[Lam]vous mais, j'avais rendez-\[Rém]vous mais
J'avais rendez-\[Do]vous, dis-moi après \[Sol]quoi  ?
J'avais rendez-\[Lam]vous mais, j'avais rendez-\[Rém]vous mais
J'avais rendez-\[Do]vous, dis-moi après \[Sol]quoi on court ?
\endchorus

\beginverse
Il y a des trains dans le brouillard, rien de certain dans les regards
Au quotidien, des trajectoires, est-ce que l'on revient si on s'égare ?
Il y a qu'on trace toujours plus vite que l'on efface les limites
Courir en vain mais dans le vent, se dire combien on est vivant
\endverse

\beginchorus
Refrain
\endchorus

\beginverse
Il y a des jours où l'on s'ennuie, il y a des nuits où l'on s'en fout
Il y a des milliards de fourmis qui voudraient pas qu'on les oublie
Il y a des jours où l'on s'ennuie, il y a des nuits où l'on s'en fout
Il y a des milliards de fourmis qui voudraient pas qu'on les oublie
\endverse

\beginchorus
\bis{Refrain}{2}
\endchorus

\endsong
\beginsong{J'entends siffler le train}[by={Richard Anthony (1962)}]

\beginverse
\[Do]J'ai pensé qu'il valait \[Lam]mieux nous qui\[Rém]tter sans un a\[Fa]dieu
Je n'au\[Rém]rais pas eu le \[Sol]cœur de te re\[Do]voir
\endverse


\beginchorus
\[Do]Mais j'entends siffler le \[Lam]train
Mais j'en\[Rém]tends siffler le \[Fa]train
Que c'est \[Rém]triste un train qui \[Sol]siffle dans le \[Do]soir
\endchorus

\beginverse
Je pouvais t'imaginer toute seule abandonnée
Sur le quai dans la cohue des au revoir
\endverse

\beginchorus
Refrain
\endchorus

\beginverse
J'ai failli courir vers toi, j'ai failli crier vers toi
C'est à peine si j'ai pu me retenir
\endverse

\beginverse
\bis{Que c'est loin où tu t'en vas}{2}
Auras-tu jamais le temps de revenir ?
\endverse

\beginverse
J'ai pensé qu'il valait mieux nous quitter sans un adieu
Mais je sens que maintenant tout est fini
\bis{Et j'entends siffler le train}{2}
\bis{J'entendrai siffler ce train toute ma vie}{2}
\endverse

\endsong
\beginsong{J't'emmène au vent}[by={Louise Attaque (1997)},sr={Capo IV}]

\beginverse
Allez \[Mim]viens, j't'emmène au \[Sol]vent
Je t'e\[Mim]mmène au-dessus des \[Sol]gens
Et je vou\[Lam]drais que tu te rap\[Mim]pelles
Notre amour est éter\[Ré]nel et pas artifi\[Fa]ciel
\endverse


\beginchorus
Je voudrais que tu te ra\[Mim]mènes de\[Sol]vant
Que tu \[Mim]sois là de temps en \[Sol]temps
Et je vou\[Lam]drais que tu te rap\[Mim]pelles
Notre amour est éter\[Ré]nel et pas artifi\[Fa]ciel
\endchorus

\beginverse
Je voudrais que tu m'appelles plus souvent
Que tu prennes parfois les devants
Et je voudrais que tu te rappelles
Notre amour est éternel et pas artifi\[Fa]ciel
\endverse

\beginverse
Je voudrais que tu sois celle que j'entends
Allez viens j't'emmène au-dessus des gens
Et je voudrais que tu te rappelles
Notre amour est éternel artifi\[Fa]ciel \[Mim]
\endverse

\beginchorus
\bis{Refrain}{5}
\endchorus

\endsong
\beginsong{J'traîne des pieds}[by={Olivia Ruiz (2005)}]

\beginchorus
\[Mim] J'traînais les pieds et des cass\[Si7]eroles
J'n'aimais pas beaucoup l'é\[Mim]cole \[Si7]
\[Mim] J'traînais les pieds et mes gui\[Si7]boles abîm\[Mim]ées
J'explorais mon quar\[Si7]tier

\[Mim] J'traînais des pieds dans mon ca\[Si7]fé
Les vieux à la belote brail\[Mim]laient \[Si]
\[Mim] Papi, mamie, tonton And\[Si7]ré et toutes ces pé\[Mim]pées
A mes p'tits soins, à m'poupon\[Si7]ner
\endchorus


\beginchorus
E\[Mim]corché mon visage, é\[Sim]corchés mes genoux
É\[Ré]corché mon p'tit coeur tout \[Do]mou
Bou\[Mim]sillées mes godasses, bou\[Sim]sillé sur ma joue
Bou\[Ré]sillées les miettes de \[Do]nous
\endchorus

\beginverse
La fumée du bœuf bourguignon
Toute la famille tête dans l'guidon
Du temps où on pouvait faire les cons
Les pensionnaires, les habitués, les gens d'passage surtout l'été
Joyeux bordel dans mon café
\endverse

\beginchorus
Refrain
\endchorus

\beginverse
Je traîne les pieds, j'traîne mes casseroles
J'n'aime toujours pas l'école
\endverse

\beginchorus
Refrain
\endchorus

\endsong
\beginsong{Je l'aime à mourir}[by={Francis Cabrel (1979)},sr={Capo V}]

\beginverse
\[Do]Moi je n'étais rien, mais voilà qu'aujourd'hui
Je suis le gardien du sommeil de ses nuits. Je l'aime à mo\[Lam]urir
Vous \[Rém]pouvez détruire tout ce qu'il vous plaira
Elle \[Fa]n'a qu'à ouvrir l'e\[Sol]space de ses bras
Pour tout r\[Do]econstruire, pour tout r\[Mim]econstruire
Je l'aime à mo\[Lam]urir
\endverse

\beginverse
Elle a gommé les chiffres des horloges du quartier
Elle a fait de ma vie des cocottes en papiers, des éclats de rire
Elle a bâti des ponts entre nous et le ciel
Et nous les traversons chaque fois qu'elle
Ne peut pas dormir, ne peut pas dormir
Je l'aime à mourir
\endverse


\beginchorus
Elle a dû fai\[Mi]re toutes les gue\[Lam]rres
Pour être \[Sol]si forte aujour\[Do]d'hui
Elle a dû fai\[Mi]re toutes les gue\[Lam]rres de la \[Sib]vie et l'a\[Do]mour aussi
\endchorus

\beginverse
Elle vit de son mieux son rêve d'opaline
Elle danse au milieu des forêts qu'elle dessine
Je l'aime à mourir
Elle porte des rubans qu'elle laisse s'envoler
Elle me chante souvent que j'ai tort d'essayer
De les retenir, de les retenir
Je l'aime à mourir
\endverse

\beginverse
Pour monter dans sa grotte cachée sous les toits
Je dois clouer des notes à mes sabots de bois
Je l'aime à mourir
Je dois juste m'asseoir, je ne dois pas parler
Je ne dois rien vouloir, je dois juste essayer
De lui appartenir, de lui appartenir
Je l'aime à mourir
\endverse

\beginchorus
Refrain puis 1er couplet
\endchorus

\endsong
\beginsong{Je marche seul}[by={Jean-Jacques Goldmann (1985)}]

\beginverse
Comme \[Dom]un bateau dérive \[Lab] \[Sib]
Sans \[Do]but et sans mobile \[Lab] \[Sib]
Je \[Dom]marche dans la ville \[Lab] \[Sib]
Tout \[Dom]seul et anonyme \[Lab] \[Sib]
\endverse

\beginverse
La \[Dom]ville et \[Sib]ses \[Dom]pièges
ce \[Lab]sont mes privi\[Sib]lèges
Je \[Dom]suis \[Sib]riche de \[Dom]ça
mais ça ne \[Lab]s'achète \[Sib]pas
\endverse


\beginchorus
Et j'm'en \[Do]fous, j'm'en \[Sol]fous, de \[Lam]tout
De ces \[Fa]chaînes qui pendent à nos \[Do]cous
J'm'en en\[Sol]fuis, j'oub\[La]lie
J'm'offre \[Fa]une parenthèse, un sur\[Do]sis \[Sol]
\[La]Je marche \[Ré]seul \[La] \[Sim]
Dans les \[Sol]rues qui \[La]se \[Ré]donnent \[La] \[Sim]
Et la \[Sol]nuit me \[La]par\[Ré]donne, je \[La]marche \[Sim]seul \[La] \[Sol]
En oubliant les \[La]heures
\[La]Je marche \[Ré]seul \[La] \[Sim]
Sans \[Sol]témoin sans \[La]per\[Ré]sonne \[La] \[Sim]
Que mes \[Sol]pas qui \[La]ré\[Ré]sonnent, je \[La]marche \[Sim]seul \[La] \[Sol]
Acteur \[La]et voyageur \[Ré]
\endchorus

\beginverse
Se rencontrer, séduire
Quand la nuit fait des siennes
Promettre sans le dire
Juste des yeux qui traînent
Oh quand la vie s'obstine
En ces heures assassines
Je suis riche de ça
Mais ça ne s'achète pas
\endverse

\beginchorus
Refrain
\endchorus

\beginverse
Je marche \[Ré]seul \[La] \[Sim]
Quand ma \[Sol]vie dérai\[Ré]sonne \[La] \[Sim]
Quand l'envie m'\[La]aban\[Ré]donne
Je \[La]marche \[Sim]seul \[La] \[Sol]
Pour me no\[La]yer d'ail\[Rém]leurs
\endverse

\beginverse
\[La]Je marche \[Ré]seul \[La] \[Sim]
Dans les \[Sol]rues qui \[La]se \[Ré]donnent \[La] \[Sim]
Et la \[Sol]nuit me \[La]pardonne, je \[La]marche \[Sim]seul \[La] \[Sol]
En oubl\[La]iant les \[Ré]heures
Je marche seul
Sans témoin sans personne
Que mes pas qui résonnent, je marche seul
Acteur et voyageur
Je marche seul
Dans les rues qui se donnent
Et la nuit me pardonne, je marche seul
En oubliant les heures
\endverse

\endsong
\beginsong{Je n'aurai pas le temps}[by={Michel Fugain (1969)},sr={Capo V}]

\beginverse
\[Fa]Je n'aurai pas le te\[Do]mps
p\[Sol]as le t\[Do]emps
\endverse

\beginverse
\[Do]Même \[Sol]en cou\[Do]rant
Plus vi\[Sol]te que le \[Do]vent
Plus vi\[Fa]te que le \[Sol]temps \[Sol7]
\[Mi]Même en vo\[Lam]lant
Je n'au\[Fa]rai pas le \[Do]temps
\[Sol]Pas le \[Do]temps
\endverse

\beginverse
De visiter toute l'immensité
D'un si grand univers
Même en cent ans
Je n'aurai pas le temps
De tout faire
\endverse


\beginchorus
\[Do]J'ou\[Si7]vre tout grand mon \[Mim]cœur
\[Sol]J'ai\[Ré7]me de tous mes \[Sol]yeux
\[Fa]C'est trop \[Do]peu
\[Sol]Pour tant de \[Ré]coeurs
\[La]Et tant de fl\[Ré]eurs
\[Do]Des \[Sol]mil\[Fa]liers de \[Do]jours
\[Fa]C'est bien trop \[Do]court
\[Sol]C'est bien trop \[Do]court
\endchorus

\beginverse
Et pour aimer
Comme l'on doit aimer
Quand on aime vraiment
Même en cent ans
Je n'aurai pas le temps
Pas le temps
\endverse

\endsong
\beginsong{Je veux}[by={Zaz (2010)},sr={Capo V}]

\beginchorus
\[Lam] Donnez-moi une suite au Ritz, je n'en veux \[Sol]pas
Des bijoux de chez Chanel, je n'en veux \[Fa]pas
Donnez-moi une limousine, j'en ferais \[Rém]quoi ? Papalap\[Mi]apapala

\[Lam] Offrez-moi du personnel, j'en ferais \[Sol]quoi ?
Un manoir à Neuchâtel, ce n'est pas pour \[Fa]moi
Offrez-moi la Tour Eiffel, j'en ferais \[Rém]quoi ? Papalap\[Mi]apapala
\endchorus

\beginchorus
Je \[Lam]veux d'l'amour, d'la \[Fa]joie, de la bonne hu\[Sol]meur
C'n'est pas votre arg\[Mim]ent qui f'ra mon bonh\[Fa]eur
Moi j'veux cre\[Rém]ver la main sur le \[Mi]cœur
Al\[Lam]lons ensemble, dé\[Fa]couvrir ma liber\[Sol]té
Ou\[Mim]bliez donc tous vos cli\[Fa]chés
Bienvenue dans \[Rém]ma réali\[Mi]té
\endchorus

\beginverse
J'en ai marre d'vos bonnes manières, c'est trop pour moi
Moi je mange avec les mains et j'suis comme ça
J'parle fort et je suis franche, excusez-moi
\endverse

\beginverse
Finie l'hypocrisie, moi, j'me casse de là
J'en ai marre des langues de bois
Regardez-moi, d'toute manière j'vous en veux pas
Et j'suis comme ça
\endverse

\beginchorus
\bis{Refrain}{3}
\endchorus

\endsong
\beginsong{Je voudrais déjà être roi}[by={Elton John, Time rice \- Le Roi Lion (1994)},sr={Capo IV}]

\beginverse
C'est \[Ré]moi Simba, c'est moi le roi
Du royaume animal
C'est la \[Sol]première fois qu'on voit un roi
Avec si peu de \[Ré]poils
\endverse

\beginverse
Je \[Ré]vais faire dans la cour des grands
Une entrée triomphale
En \[Sol]poussant, très royalement
Un rugissement besti\[Ré]al
\endverse

\beginverse
Maj\[Mim]esté, tu ne te mouches \[Sol]pas du \[La]coude
\endverse

\beginverse
Je vou\[Sol]drais dé\[La]jà être \[Ré]roi
\endverse

\beginverse
Tu as encore un long chemin à faire
Votre altesse, tu peux me croire
\endverse

\beginverse
Au roi, on ne \[Sol]dit pas
Tiens ta langue et \[Mim]tais-toi
Surtout ne fais \[La]pas ça
Reste ici, a\[Ré]ssieds-toi
Restez ici
\endverse

\beginverse
\[Sol]Sans jamais dire \[Mim]où \[Sol]je \[La]vais
\endverse

\beginverse
\[Sol]Je veux faire ce \[La]qui me \[Ré]plaît
\endverse

\beginverse
Il \[Ré]est grand temps votre Grandeur
Qu'on par le de cœur à cœur
\endverse

\beginverse
\[Sol]Le roi n'a que faire
Des conseils \[Ré]d'une vieille corneille
\endverse

\beginverse
Si tu confonds la monarchie avec la tyrannie
Vive \[Sol]la république
Adieu l'Afrique
Je fer\[Ré]me la boutique
Oh \[Mim]prends garde, lion, ne te \[Sol]trompe \[La]pas de voie
\endverse

\beginverse
Je vou\[Sol]drais dé\[La]jà être \[Ré]roi !
\endverse

\beginverse
Regardez \[Sol]bien à l'ouest
Regardez \[Mim]bien à l'est
Mon pouvoir, \[La]sans conteste
Est sans \[Ré]frontière
\endverse

(Pas encore !)

\beginverse
C'est une rumeur qui \[Mim]monte jusqu'au \[La]ciel
Les \[Sol]animaux ré\[Mim]pandent la nou\[La]velle
\[Sol]Simba sera le \[Mim]nouveau roi sol\[La]eil
\endverse

\beginverse
\bis{Je vou\[Sol]drais dé\[La]jà être \[Ré]roi !}{3}
\endverse

\endsong
\beginsong{Jeune et con}[by={Saez (1999)}]

\beginverse
\[Mim]Encore un \[Do]jour se lève \[Sol]sur la planète \[Ré]France
Et je \[Mim]sors dou\[Do]cement de mes rêves, je \[Sol]rentre dans la \[Ré]danse
Comme tou\[Mim]jours il est \[Do]huit heures du soir j'ai \[Sol]dormi tout le \[Ré]jour
Je me \[Mim]suis encore \[Do]couché trop tard je \[Sol]me suis rendu \[Ré]sourd
\endverse

\beginverse
En\[Mim]core, en\[Do]core une soirée \[Sol]où la jeunesse \[Ré]France
En\[Mim]core elle va \[Do]bien s'amuser puisq\[Sol]u'ici rien a de \[Ré]sens
Alors on \[Mim]va dan\[Do]ser faire semblant \[Sol]d'être heu\[Ré]reux
Pour al\[Mim]ler gen\[Do]timent se coucher mais \[Sol]demain rien n'ira \[Ré]mieux
\endverse


\beginchorus
Puisqu'on est \[Mim]jeune et \[Do]con
Puisqu'ils sont \[Sol]vieux et \[Ré]fous
Puisque des \[Mim]hommes \[Do]crèvent sous les ponts
Mais \[Sol]ce monde s'en \[Ré]fout
Puisqu'\[Mim]on est que des \[Do]pions
Content d'être \[Sol]à ge\[Ré]noux
Puisque je \[Mim]sais qu'un jour nous \[Do]gagnerons \MultiwordChords\[Sol]à devenir \[Ré]fous
\bis{Devenir \[Mim]fous \[Do] \[Sol]\[Ré]}{2}

\endchorus

\beginverse
Encore un jour se lève sur la planète France
Mais j'ai depuis longtemps perdu mes rêves je connais trop la danse
Comme toujours il est huit heures du soir j'ai dormi tout le jour
Mais je sais qu'on est quelques milliards à chercher l'amour
\endverse

\beginverse
Encore, encore une soirée où la jeunesse France
Encore, elle va bien s'amuser dans cet état d'urgence
Alors elle va danser, faire semblant d'exister
Qui sait si l'on ferme les yeux on vivra vieux ?
\endverse

\beginchorus
Refrain
\endchorus

\beginverse
Comme des \[Mim]fous, \[Do]fous, \[Sol]fous, \[Ré]fous
\endverse

\beginverse
Encore un jour se lève sur la planète France
Et j'ai depuis longtemps perdu mes rêves je connais trop la danse
Comme toujours il est huit heures du soir j'ai dormi tout le jour
Mais je \[Mim]sais qu'on \[Do]est quelques milliards \[Sol] \[Ré]
\[Mim] A chercher l'amour
\endverse

\endsong
\beginsong{Jeunesse, lève-toi*}[by={Saez (2008)}]

\beginverse
\[Mim]Comme un éclat de rire vient consoler tristesse
Comme un souffle avenir vient raviver les bra\[Do]ises
Comme un parfum de souffre qui fait naître la fla\[Ré]mme
\[Sol]Jeunesse \[Ré]lève-\[Mim]toi
\endverse

\beginverse
Contre la vie qui va qui vient puis qui s'éteint
Contre l'amour qu'on prend, qu'on tient mais qui tient pas
Contre la trace qui s'efface au derrière de soi
Jeunesse lève-toi
\endverse

\beginverse
Moi contre ton épaule je repars à la lutte
Contre les gravités qui nous mènent à la chute
Pour faire du bruit encore à réveiller les morts
Pour redonner éclat à l'émeraude en toi \[Rém]
\endverse

\beginverse
\[Rém]Pour rendre au crépuscule la beauté des aurores
Dis-moi qu'on brûle encore dis-moi que brûle encore
Cet espoir que tu tiens parce que tu n'en sais rien
De la fougue et du feu que je vois dans tes yeux
Jeunesse lève-toi
\endverse

\beginverse
Quand tu vois comme on pleure à chaque rue sa peine
Comment on nous écœure perfusion dans la veine
A l'ombre du faisceau mon vieux tu m'auras plus
Ami, dis, quand viendra la crue ?
\endverse

\beginverse
Contre courant toujours sont les contre-cultures
Au gré des émissions leurs gueules de vide-ordures
Puisque c'en est sonné la mort du politique
L'heure est aux rêves aux utopiques
\endverse

\beginverse
Pour faire nos ADN un peu plus équitables
Pour faire de la poussière un peu plus que du sable
Dans ce triste pays tu sais un jour ou l'autre
Faudra tuer le père faire entendre ta voix
Jeunesse lève-toi
\endverse

\beginverse
Au clair de lune indien toujours surfer la vague
A l'âme au creux des reins faut aiguiser la lame
Puisqu'ici, il n'y a qu'au combat qu'on est libre
De ton triste coma, je t'en prie, libère-toi
\endverse

\beginverse
Puisqu'ici il faut faire des bilans et du chiffre
Sont nos amours toujours au bord du précipice
N'entends-tu pas ce soir chanter le chant des morts
Ne vois-tu pas le ciel à la portée des doigts ?
Jeunesse lève-toi
\endverse

\beginverse
Comme un éclat de rire vient consoler tristesse
Comme un souffle avenir vient raviver les braises
Comme un parfum de souffre qui fait naître la flamme
Quand, plongé dans le gouffre, on sait plus où est l'âme
Jeunesse lève-toi
\endverse

\beginverse
Contre la vie qui va, qui vient puis qui nous perd
Contre l'amour qu'on prend qu'on tient puis qu'on enterre
Contre la trace qui s'efface au derrière de soi
Jeunesse lève-toi
Jeunesse lève-toi
\endverse

\beginverse
Au clair de lune indien toujours surfer la vague
A l'âme au creux des reins faut aiguiser la lame
Puisqu'ici il n'y a qu'au combat qu'on est libre
De ton triste coma, je t'en prie libère-toi
\endverse

\beginverse
Puisqu'ici, il faut faire des bilans et du chiffre
Sont nos amours toujours au bord du précipice
N'entends-tu pas ce soir chanter le chant des morts
A la mémoire de ceux qui sont tombés pour toi ?
Jeunesse lève-toi
\endverse

\endsong
\beginsong{Jolene}[by={Dolly Parton (1973)},sr={Capo IV}]


\beginchorus
Jo\[Lam]lene, Jo\[Do]lene, Jo\[Sol]lene, Jo\[Lam]lene
I'm \[Sol]begging of you, please don't take \[Lam]my man
Jo\[Lam]lene, Jo\[Do]lene, Jo\[Sol]lene, Jo\[Lam]lene
\[Sol]Please don't take him just because \[Lam]you can
\endchorus

\beginverse
Your \[Lam]beauty is be\[Do]yond compare
With \[Sol]flaming locks of \[Lam]auburn hair
With \[Sol]ivory skin and eyes of eme\[Lam]rald green
\endverse

\beginverse
Your smile is like a breath of spring
Your voice is soft like summer rain
I cannot compete with you, Jolene
\endverse

\beginverse
He talks about you in his sleep
There's nothing I can do to keep
From crying when he calls your name, Jolene
\endverse

\beginverse
And I can easily understand
How you could easily take my man
But you don't know what he means to me, Jolene
\endverse

\beginchorus
Refrain
\endchorus

\beginverse
Now you could have your choice of men
But I could never love again
He's the only one for me, Jolene
\endverse

\beginverse
I had to have this talk with you
My happiness depends on you
And whatever you decide to do, Jolene
\endverse

\beginchorus
Refrain
\endchorus

\beginverse
\bis{Jolene}{2}
\endverse

\endsong
\beginsong{La jument de Michao}[by={Tri Yann (1976), Nolwenn Leroy (2010)}]

\beginverse
\bis{\[Mim]C'est dans dix ans je m'en irai
    J'entends le loup et le re\[Ré]nard chan\[Mim]ter
}{2}
\bis{J'entends le loup, le re\[Sol]nard et la be\[Ré]lette
    \[Mim]J'entends le \[Ré]loup et le renard chan\[Mim]ter
}{2}
\endverse

\beginverse
\bis{C'est \[Lam]dans neuf ans je m'en \[Do]irai
    La jument de M\[Mim]ichao a \[Ré]passé dans le \MultiwordChords\[Mim]pré}{2}
\bis{La jument de Michao et \[Sol]son petit pou\[Ré]lain
    A \[Mim]passé dans le \[Ré]pré et mangé tout le \[Mim]foin}{2}
\bis{L'hiver viendra, les gars, l'hi\[Sol]ver vien\[Ré]dra
    La \[Mim]jument de Mic\[Do]hao, elle \[Ré]s'en repenti\[Lam]ra}{2}
\endverse

\beginverse
C'est dans huit ans
\endverse

\endsong
\beginsong{Junge}[by={Die Ärtze (2007)},sr={Capo I}]

\beginverse
\[Fa]Junge, \[Sol]warum hast du nichts \[Lam]gelernt ?
Guck dir den \[Fa]Dieter an, \[Sol]der hat sogar ein \[Lam]Auto
Warum \[Fa]gehst du nicht zu Onkel Werner in die \[Sol]Werkstatt ?
Der gibt dir 'ne \[Lam]Festanstellung
Wenn du ihn darum bittest
\endverse

\beginverse
\[Fa]Jun\[Sol]ge, und wie du wie\[Fa]der aussiehst ?
Löcher in der \[Rém]Hose
Und ständig dieser \[Lam]Lärm
Was sollen die \[Do]Nachbarn sagen ?
Und dann noch deine \[Fa]Haare
Da fehlen mir die \[Rém]Worte
Musst du die denn \[Lam]färben ?
Was sollen die \[Do]Nachbarn sagen ?
Nie kommst du nach \[Fa]Hause
Wir wissen nicht mehr \[Ré]weiter
\endverse

\beginverse
Junge, brich deiner Mutter nicht das Herz
Es ist noch nicht zu spät, dich an der Uni einzuschreiben
Du hast dich doch früher so für Tiere interessiert
Wäre das nichts für dich, eine eigene Praxis ?
\endverse

\beginverse
Junge, und wie du wieder aussiehst
Löcher in der Nase
Und ständig dieser Lärm
Was sollen die Nachbarn sagen ?
Elektrische Gitarren und immer diese Texte
Das will doch keiner hören
Was sollen die Nachbarn sagen ?
Nie kommst du nach Hause
Soviel schlechter Umgang
Wir werden dich enterben
Was soll das Finanzamt sagen ?
Wo soll das alles enden ?
Wir machen uns doch Sorgen
\endverse

\beginverse
\[Fa]Und du warst \[Rém]so ein süsses \[Lam]Kind
Und du warst \[Do]so ein süsses \[Fa]Kind
Und du warst \[Ré]so ein süsses \[Lam]Kind
Du warst so \[Do]süss
Und immer deine Freunde
Ihr nehmt doch alle Drogen
Und ständig dieser Lärm
Was sollen die Nachbarn sagen ?
Denk an deine Zukunft
Denk an deine Eltern
Willst du dass wir sterben ?
\endverse

\endsong
\beginsong{Kumbaya}[by={Traditionnel}]


\beginchorus
Kumba\[La]ya, my Lord, \[Ré]Kumba\[La]ya
Kumbaya, my Lord, Kumba\[Mi]ya
Kumba\[La]ya, my Lord, \[Ré]Kumba\[La]ya
\[Ré]Oh \[La]Lord, \[Mi]Kumba\[La]ya
\endchorus

\beginverse
Someone's singing, Lord, Kumbaya
Someone's sleeping, Lord, Kumbaya
Someone's praying, Lord, Kumbaya
Someone's crying, Lord, Kumbaya
Someone's thinking, Lord, Kumbaya
Kumbaya, my Lord, Kumbaya
\endverse

\endsong
