\section*{Avant-propos}
\small Tu tiens entre tes mains la toute nouvelle édition du Kumbaya : le chansonnier de l'Association du Scoutisme Jurassien ! Pensé pour toutes celles et ceux qui aiment pousser la chansonnette autour d'un feu de camp, guitare en bandoulière, ou encore micro à la main pour chanter en mode karaoké grâce à notre playlist Spotify “Kumbaya”. Ce recueil t'accompagnera dans tes moments musicaux et festifs.

Au cœur de sa conception, plusieurs objectifs ont guidé notre équipe. D'abord, nous avons souhaité rafraîchir le répertoire musical, en gardant nos classiques tout en faisant la part belle aux années 1990-2020 et à la découverte de nouvelles et nouveaux artistes. Plusieurs reprises contemporaines sont ainsi mentionnées aux côtés des artistes d'origine.

Ensuite, nous avons souhaité offrir un chansonnier accessible pour encourager la pratique musicale, en particulier à la guitare. Les accords figurent dans ce recueil accompagnés d'une méthodologie pratique pour que les grands rêveurs ainsi que les débutant·e·s osent enfin se lancer à la gratte !
Par ailleurs, notre chansonnier vise à intégrer davantage les branches Louveteaux et Route dans son répertoire. Des chansons pour les plus petit·e·s et autres bans figurent ainsi dans notre table des matières. D'autre part, plusieurs chansons adressées à un public plus majeur dont les paroles portent sur la consommation de substances addictives ainsi que des idées potentiellement violentes, politiques ou morales subjectives de leurs artistes sont marquées du signe *. Notre équipe souhaite laisser à chaque adulte la réflexion du choix des chansons et le soin d'une prévention ou d'une remise en contexte adaptée.

Née d'une idée lancée à l'automne 2023 au large de la mer Égée, cette édition du Kumbaya est le fruit d'un travail collectif, nourri par les propositions des douze groupes de l'ASJ, de longues discussions entre passionné·e·s de chant, et d'un élan enthousiaste printanier pour le faire aboutir à temps pour le camp JUBACA. Son contenu trouve ses origines dans le Kumbaya de l'ASJ (1997), la première édition du Petit Romand (1998), le Faramaz du Groupe Perceval (2013) et la nouvelle édition du P'tit Romand du scoutisme genevois (2023).

Alors que s'élève la première note de la farandole, que crépitent les braises ou que la sono s'allume… nous te souhaitons de belles veillées sous les étoiles!

L'équipe Kumbaya de l'Association du Scoutisme Jurassien (ASJ)

Juillet 2025
